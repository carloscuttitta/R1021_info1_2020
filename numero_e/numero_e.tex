\documentclass{beamer}
\usepackage[latin1]{inputenc}
\usepackage[spanish]{babel}
\usepackage{multicol}
\usepackage{fancybox}
\usepackage{beamerthemeshadow}
\usepackage{times} %font times
\usepackage[T1]{fontenc} %para cuando seleccione texto con letras acentuadad y las � Usar la codificaci�n T1
\usepackage{enumerate}
\usepackage{listings}
\usepackage{calligra} 
\usepackage{graphicx}
\usepackage{array}
\usepackage{caption}
\usepackage{tikz}
\usepackage{verbatim}
\usepackage{mdwlist}
\usepackage{yhmath}  % para numeros periodicos
\usetikzlibrary{chains,fit,shapes,arrows,calc,shapes,decorations.pathreplacing}
\usefonttheme{professionalfonts}

\newtheorem{defi}{Definici�n} 
\hypersetup{pdfpagemode=FullScreen}

\mode<presentation>{
\usetheme{Warsaw}
\setbeamercovered{transparent}
}

\lstset{
	frame=Ltb,
	framerule=0pt,
	aboveskip=0.5cm,
	framextopmargin=3pt,
	framexbottommargin=3pt,
	framexleftmargin=0.4cm,
	framesep=0pt,
	rulesep=.4pt,
%	backgroundcolor=\color{gray!20},
	rulesepcolor=\color{black},
	language=C,
	captionpos=b,
	tabsize=3,
	frame=lines,
	keywordstyle=\color{blue},
	commentstyle=\color{gray},
	stringstyle=\color{red},
	numbers=left,
	numberstyle=\tiny,
	numbersep=5pt,
	breaklines=true,
	showstringspaces=false,
	basicstyle=\small,
	emph={label},
	framerule=0pt,
}

\title{\em INFORMATICA I}
\subtitle{Serie del n�mero de {\color{yellow}Euler} en {\color{yellow}"C"}}
\author{\em Ing.Juan Carlos Cuttitta}
\institute{\calligra{\fontsize{16pt}{7pt}\selectfont{Universidad Tecnol�gica Nacional\\ Facultad Regional Buenos Aires \\ Departamento de {Ingenier�a} {Electr�nica}}}}
\date{\today}

%portada

\begin{document}

\begin{figure}[ht!]
  \centering
  \includegraphics [width=0.25\textwidth]{informacion.jpg}
\end{figure}
\vspace{-1.2cm} % para subir el titulo 
\titlepage

%%%%%%%%%%%%%%%%%%%%%%%%%%%%%%%%%%%%%%%%%%%%%%%%
%Template de aqui en mas ;)
%%%%%%%%%%%%%%%%%%%%%%%%%%%%%%%%%%%%%%%%%%%%%%%%
\pgfdeclareimage[height=0.9cm]{left-logo}{arania.png}
\setbeamertemplate{sidebar left}
{
\logo{\pgfuseimage{left-logo}}
\vfill%
\rlap{\hskip0.15cm\insertlogo}%
\vskip10pt%
}
%%%%%%%%%%%%%%%%%%%%%%%%%%%%%%%%%%%%%%%%%%%%%%%%
\tikzstyle{every picture}+=[remember picture]
\tikzstyle{na} = [baseline=-.5ex]

\begin{frame}[fragile]
\fontsize{13pt}{12pt}\selectfont
\frametitle{Enunciado del problema}
El valor aproximado del n�mero de Euler ({\color{red}e}) se puede obtener con la siguente serie.\\
\begin{center}
{\color{blue}$e=\sum_{n=0}^{\infty}\frac{1}{!n}$}
\end{center}
\begin{center}
{\color{blue}$e=\frac{1}{!0}+\frac{1}{!1}+\frac{1}{!2}+\frac{1}{!3}+\frac{1}{!4}+\frac{1}{!5}+ \cdot\cdot\cdot$}
\end{center}
Escribir un programa que calcule el valor aproximado de {\color{red}e} mediante un ciclo repetitivo que termine cuando la diferencia entre dos aproximaciones sucesivas difiera en menos de {\color{red}$10^{-9}$} \\
\end{frame}

\begin{frame}[fragile]
\fontsize{6.5pt}{10pt}\selectfont
\frametitle{Declaraci�n y disposici�n en memoria}
\begin{columns}[c]
\column{0.5\textwidth}
\begin{tikzpicture}
	\begin{scope}[every node/.style={draw,anchor=text,rectangle split,rectangle split parts=10, 
rectangle split part fill={green!0,blue!0,blue!0,blue!0,blue!0,blue!0,blue!0,blue!0,blue!0,red!0,red!0},minimum width=2.6cm}]
		\node (R) at (2,6)
  		{
  			\nodepart{one}{}
  	 		\nodepart{two}{\color{red}1}
  	 		\nodepart{three}{\color{red}1}	
  	 		\nodepart{four}{\color{red}0x??}
  	 		\nodepart{five}{\color{red}0}
  	 		\nodepart{six}{}
  	 		\nodepart{seven}{\color{red}-1.00}
  	 		\nodepart{eight}{}
  	 		\nodepart{nine}{$\vdots$}
  	 		\nodepart{ten}{}
  	 	};
	\end{scope}
	\draw [ultra thick,color=white](0.8, 4.6) -- (3.37, 4.6);
	\draw [ultra thick,color=white](0.8, 3.92) -- (3.37, 3.92);
	\fill[color=green] (5.2,7.8) -- (10,7.8) -- (10,7.3) -- (5.2,7.3) -- cycle;
  	\node at (2.1,9) {\normalsize Arquitectura X86-\textcolor{red}{32} bits };
  	\node at (2.1,7.9) {\small $Disposici\acute{o}n$ $de$ $las$ $variables$};
  	\node at (2.1,7.5) {\small        $en$ $la$ $memoria$  };
  	\node at (8,9.7) {\normalsize {\color{red!50}$C\acute{o}digo$ $del$ $programa$ $fuente$}};
	\node at (4,5.75) {\textcolor{blue}{factorial}};
  	\node at (4,5.4){\textcolor{blue}{contador}};
  	\node at (3.6,5.05) {\textcolor{blue}{i}};
  	\node at (3.9,4.6) {\textcolor{blue}{nEuler}};
  	\node at (4.05,4) {\textcolor{blue}{nEulerOld}};
	\node at (0.1,6.1) {0x3F00480\textcolor{red}{0}};
  	\node at (0.1,5.75) {0x3F00480\textcolor{red}{4}};
  	\node at (0.1,5.4) {0x3F00480\textcolor{red}{8}};
  	\node at (0.1,5.05) {0x3F00480\textcolor{red}{C}};
  	\node at (0.1,4.7) {0x3F0048\textcolor{red}{10}};
  	\node at (0.1,4.37) {0x3F0048\textcolor{red}{14}};
  	\node at (0.1,4.03) {0x3F0048\textcolor{red}{18}};
  	\node at (0.1,3.67) {0x3F0048\textcolor{red}{1C}};
  	\draw[decorate,decoration={brace,mirror,raise=6pt,amplitude=8pt}, thick]
    (3.3,6.1) -- (0.9,6.1)node [black,midway,xshift=0.0cm,yshift=0.7cm]{\footnotesize {\color{red}$4$ $bytes$}};
\end{tikzpicture}
\column{0.55\textwidth}
\lstset{basicstyle=\tiny}
\begin{lstlisting}[escapechar=\|,label=noint]
#include <stdio.h>
#include <math.h>

#define LIMITE   pow(10,-9)
int  main(void)
{
int		factorial = 1,contador =1,i;    
double	nEuler=0,nEulerOld=-1;

for(i=0;(nEuler-nEulerOld)>= LIMITE;i++)
  }
     nEulerOld=nEuler;
     while(contador <= i)
     { 
     	factorial = factorial * contador;
     	contador++;
     }
     nEuler = nEuler + (1/(double)factorial); 
        
     factorial = 1;
     contador =1;
  }
  printf("e es %0.10f \n",nEuler);
  printf("e(lib math.h)es %0.10f\n",M_E);
  return 0;
}
\end{lstlisting}
\end{columns}
\end{frame}

\begin{frame}[fragile]
\fontsize{6.5pt}{10pt}\selectfont
\frametitle{Declaraci�n y disposici�n en memoria}
\begin{columns}[c]
\column{0.5\textwidth}
\begin{tikzpicture}
	\begin{scope}[every node/.style={draw,anchor=text,rectangle split,rectangle split parts=10, 
rectangle split part fill={green!0,blue!0,blue!0,blue!0,blue!0,blue!0,blue!0,blue!0,blue!0,red!0,red!0},minimum width=2.6cm}]
    	\node (R) at (2,6)
    	{
    		\nodepart{one}{}
    		\nodepart{two}{1}
    		\nodepart{three}{1}	
    		\nodepart{four}{\color{red}0}
    		\nodepart{five}{0}
    		\nodepart{six}{}
    		\nodepart{seven}{-1.00}
    		\nodepart{eight}{}
    		\nodepart{nine}{$\vdots$}
    		\nodepart{ten}{}
    	};
	\end{scope}
	\draw [ultra thick,color=white](0.8, 4.6) -- (3.37, 4.6);
	\draw [ultra thick,color=white](0.8, 3.92) -- (3.37, 3.92);
	\fill[color=green] (5.7,7.2) -- (6.2,7.2) -- (6.2,6.8) -- (5.7,6.8) -- cycle;
	\node at (2.1,9) {\normalsize Arquitectura X86-\textcolor{red}{32} bits };
	\node at (2.1,7.9) {\small $Declara$ $la$ $variable$ $\color{red}i$};
	\node at (8,9.7) {\normalsize {\color{red!50}$C\acute{o}digo$ $del$ $programa$ $fuente$}};
	\node at (4,5.75) {\textcolor{blue}{factorial}};
	\node at (4,5.4){\textcolor{blue}{contador}};
	\node at (3.6,5.05) {\textcolor{blue}{i}};
	\node at (3.9,4.6) {\textcolor{blue}{nEuler}};
	\node at (4.05,4) {\textcolor{blue}{nEulerOld}};
	\node at (0.1,6.1) {0x3F00480\textcolor{red}{0}};
	\node at (0.1,5.75) {0x3F00480\textcolor{red}{4}};
	\node at (0.1,5.4) {0x3F00480\textcolor{red}{8}};
	\node at (0.1,5.05) {0x3F00480\textcolor{red}{C}};
	\node at (0.1,4.7) {0x3F0048\textcolor{red}{10}};
	\node at (0.1,4.37) {0x3F0048\textcolor{red}{14}};
	\node at (0.1,4.03) {0x3F0048\textcolor{red}{18}};
	\node at (0.1,3.67) {0x3F0048\textcolor{red}{1C}};
	\draw[decorate,decoration={brace,mirror,raise=6pt,amplitude=8pt}, thick]
    (3.3,6.1) -- (0.9,6.1)node [black,midway,xshift=0.0cm,yshift=0.7cm]{\footnotesize {\color{red}$4$ $bytes$}};
\end{tikzpicture}
\column{0.55\textwidth}
\lstset{basicstyle=\tiny}
\begin{lstlisting}[escapechar=\|,label=noint]
#include <stdio.h>
#include <math.h>

#define LIMITE   pow(10,-9)
int  main(void)
{
int		factorial = 1,contador =1,i;    
double	nEuler=0,nEulerOld=-1;

for(i=0;(nEuler-nEulerOld)>= LIMITE;i++)
  }
     nEulerOld=nEuler;
     while(contador <= i)
     { 
     	factorial = factorial * contador;
     	contador++;
     }
     nEuler = nEuler + (1/(double)factorial); 
        
     factorial = 1;
     contador =1;
  }
  printf("e es %0.10f \n",nEuler);
  printf("e(lib math.h)es %0.10f\n",M_E);
  return 0;
}
\end{lstlisting}
\end{columns}
\end{frame}

\begin{frame}[fragile]
\fontsize{6.5pt}{10pt}\selectfont
\frametitle{Recorriendo el programa {\color{yellow}paso} a {\color{yellow}paso}}
\begin{columns}[c]
 \column{0.5\textwidth}
\begin{tikzpicture}
	\begin{scope}[	every node/.style={draw,anchor=text,rectangle split,rectangle split parts=10, 
rectangle split part fill={green!0,blue!0,blue!0,blue!0,blue!0,blue!0,blue!0,blue!0,blue!0,red!0,red!0},minimum width=2.6cm}]
    	\node (R) at (2,6)
    	{
    		\nodepart{one}{}
    		\nodepart{two}{1}
    		\nodepart{three}{1}	
    		\nodepart{four}{0}
    		\nodepart{five}{0}
    		\nodepart{six}{}
    		\nodepart{seven}{-1.00}
    		\nodepart{eight}{}
    		\nodepart{nine}{$\vdots$}
    		\nodepart{ten}{}
    	};
	\end{scope}
	\draw [ultra thick,color=white](0.8, 4.6) -- (3.37, 4.6);
	\draw [ultra thick,color=white](0.8, 3.92) -- (3.37, 3.92);
	\fill[color=green] (6.4,7.2) -- (9.7,7.2) -- (9.7,6.8) -- (6.4,6.8) -- cycle;
	\node at (2.1,9) {\normalsize Arquitectura X86-\textcolor{red}{32} bits };
  	\node at (2.1,7.9) {\small $Pregunta$ $condici\acute{o}n$ $del$ $\color{red}for$};
  	\node at (8,9.7) {\normalsize {\color{red!50}$C\acute{o}digo$ $del$ $programa$ $fuente$}};
  	\node at (4,5.75) {\textcolor{blue}{factorial}};
  	\node at (4,5.4){\textcolor{blue}{contador}};
  	\node at (3.6,5.05) {\textcolor{blue}{i}};
  	\node at (3.9,4.6) {\textcolor{blue}{nEuler}};
  	\node at (4.05,4) {\textcolor{blue}{nEulerOld}};
  	\node at (0.1,6.1) {0x3F00480\textcolor{red}{0}};
  	\node at (0.1,5.75) {0x3F00480\textcolor{red}{4}};
  	\node at (0.1,5.4) {0x3F00480\textcolor{red}{8}};
  	\node at (0.1,5.05) {0x3F00480\textcolor{red}{C}};
  	\node at (0.1,4.7) {0x3F0048\textcolor{red}{10}};
  	\node at (0.1,4.37) {0x3F0048\textcolor{red}{14}};
  	\node at (0.1,4.03) {0x3F0048\textcolor{red}{18}};
  	\node at (0.1,3.67) {0x3F0048\textcolor{red}{1C}};
  	\draw[decorate,decoration={brace,mirror,raise=6pt,amplitude=8pt}, thick]
    (3.3,6.1) -- (0.9,6.1)node [black,midway,xshift=0.0cm,yshift=0.7cm]{\footnotesize {\color{red}$4$ $bytes$}};
  	\draw[decorate,decoration={brace,raise=4pt,amplitude=4pt}, thick]
    (7.35,8.4) -- (8.7,8.4)node[black,midway,xshift=1.0cm,yshift=0.4cm]{\footnotesize{\color{red!60}$10^{-9}$}};
  	\draw[red,->](9.3,7.1)to [out=10,in=0]node[right,midway]{} ++(-0.2,1.3) ; 
\end{tikzpicture}
\column{0.55\textwidth}
\lstset{basicstyle=\tiny}
\begin{lstlisting}[escapechar=\|,label=noint]
#include <stdio.h>
#include <math.h>

#define LIMITE   pow(10,-9)
int  main(void)
{
int		factorial = 1,contador =1,i;    
double	nEuler=0,nEulerOld=-1;

for(i=0;(nEuler-nEulerOld)>= LIMITE;i++)
  }
     nEulerOld=nEuler;
     while(contador <= i)
     { 
     	factorial = factorial * contador;
     	contador++;
     }
     nEuler = nEuler + (1/(double)factorial); 
        
     factorial = 1;
     contador =1;
  }
  printf("e es %0.10f \n",nEuler);
  printf("e(lib math.h)es %0.10f\n",M_E);
  return 0;
}
\end{lstlisting}
\end{columns}
\end{frame}

\begin{frame}[fragile]
\fontsize{6.5pt}{10pt}\selectfont
\frametitle{Recorriendo el programa {\color{yellow}paso} a {\color{yellow}paso}}
\begin{columns}[c]
\column{0.5\textwidth}
\begin{tikzpicture}
	\begin{scope}[every node/.style={draw,anchor=text,rectangle split,rectangle split parts=10, 
rectangle split part fill={green!0,blue!0,blue!0,blue!0,blue!0,blue!0,blue!0,blue!0,blue!0,red!0,red!0},minimum width=2.6cm}]
    	\node (R) at (2,6)
    	{
    		\nodepart{one}{}
    		\nodepart{two}{1}
    		\nodepart{three}{1}	
    		\nodepart{four}{0}
    		\nodepart{five}{0}
    		\nodepart{six}{}
    		\nodepart{seven}{\color{red}0}
    		\nodepart{eight}{}
    		\nodepart{nine}{$\vdots$}
    		\nodepart{ten}{}
    	};
	\end{scope}
	\draw [ultra thick,color=white](0.8, 4.6) -- (3.37, 4.6);
	\draw [ultra thick,color=white](0.8, 3.92) -- (3.37, 3.92);
	\fill[color=green] (5.8,6.65) -- (8,6.65) -- (8,6.35) -- (5.8,6.35) -- cycle;
  	\node at (2.1,9) {\normalsize Arquitectura X86-\textcolor{red}{32} bits };
  	\node at (2.1,7.9) {\small $Para$ $conservar$ $el$ $\color{red}e$ $anterior$};
  	\node at (8,9.7) {\normalsize {\color{red!50}$C\acute{o}digo$ $del$ $programa$ $fuente$}};
  	\node at (4,5.75) {\textcolor{blue}{factorial}};
  	\node at (4,5.4){\textcolor{blue}{contador}};
  	\node at (3.6,5.05) {\textcolor{blue}{i}};
  	\node at (3.9,4.6) {\textcolor{blue}{nEuler}};
  	\node at (4.05,4) {\textcolor{blue}{nEulerOld}};
  	\node at (0.1,6.1) {0x3F00480\textcolor{red}{0}};
  	\node at (0.1,5.75) {0x3F00480\textcolor{red}{4}};
  	\node at (0.1,5.4) {0x3F00480\textcolor{red}{8}};
  	\node at (0.1,5.05) {0x3F00480\textcolor{red}{C}};
  	\node at (0.1,4.7) {0x3F0048\textcolor{red}{10}};
  	\node at (0.1,4.37) {0x3F0048\textcolor{red}{14}};
  	\node at (0.1,4.03) {0x3F0048\textcolor{red}{18}};
  	\node at (0.1,3.67) {0x3F0048\textcolor{red}{1C}};
  	\draw[decorate,decoration={brace,mirror,raise=6pt,amplitude=8pt}, thick]
    (3.3,6.1) -- (0.9,6.1)node [black,midway,xshift=0.0cm,yshift=0.7cm]{\footnotesize {\color{red}$4$ $bytes$}};
\end{tikzpicture}
\column{0.55\textwidth}
\lstset{basicstyle=\tiny}
\begin{lstlisting}[escapechar=\|,label=noint]
#include <stdio.h>
#include <math.h>

#define LIMITE   pow(10,-9)
int  main(void)
{
int		factorial = 1,contador =1,i;    
double	nEuler=0,nEulerOld=-1;

for(i=0;(nEuler-nEulerOld)>= LIMITE;i++)
  }
     nEulerOld=nEuler;
     while(contador <= i)
     { 
     	factorial = factorial * contador;
     	contador++;
     }
     nEuler = nEuler + (1/(double)factorial); 
        
     factorial = 1;
     contador =1;
  }
  printf("e es %0.10f \n",nEuler);
  printf("e(lib math.h)es %0.10f\n",M_E);
  return 0;
}
\end{lstlisting}
\end{columns}
\end{frame}

\begin{frame}[fragile]
\fontsize{6.5pt}{10pt}\selectfont
\frametitle{Recorriendo el programa {\color{yellow}paso} a {\color{yellow}paso}}
\begin{columns}[c]
\column{0.5\textwidth}
\begin{tikzpicture}
	\begin{scope}[every node/.style={draw,anchor=text,rectangle split,rectangle split parts=10, 
rectangle split part fill={green!0,blue!0,blue!0,blue!0,blue!0,blue!0,blue!0,blue!0,blue!0,red!0,red!0},minimum width=2.6cm}]
    	\node (R) at (2,6)
    	{
    		\nodepart{one}{}
    		\nodepart{two}{1}
    		\nodepart{three}{1}	
    		\nodepart{four}{0}
    		\nodepart{five}{0}
    		\nodepart{six}{}
    		\nodepart{seven}{0}
    		\nodepart{eight}{}
    		\nodepart{nine}{$\vdots$}
    		\nodepart{ten}{}
    	};
	\end{scope}
	\draw [ultra thick,color=white](0.8, 4.6) -- (3.37, 4.6);
	\draw [ultra thick,color=white](0.8, 3.92) -- (3.37, 3.92);
	\fill[color=green] (5.8,6.4) -- (8.4,6.4) -- (8.4,6) -- (5.8,6) -- cycle;
  	\node at (2.1,9) {\normalsize Arquitectura X86-\textcolor{red}{32} bits };
  	\node at (2.1,7.9) {\small $Como$ $\color{red}contador$ $no$ $es$ $\color{red}<=$ $a$ $\color{red}i$};
  	\node at (2.1,7.5) {\small $no$ $entra$ $dentro$ $del$ $\color{red}while$};
  	\node at (8,9.7) {\normalsize {\color{red!50}$C\acute{o}digo$ $del$ $programa$ $fuente$}};
  	\node at (4,5.75) {\textcolor{blue}{factorial}};
  	\node at (4,5.4){\textcolor{blue}{contador}};
  	\node at (3.6,5.05) {\textcolor{blue}{i}};
  	\node at (3.9,4.6) {\textcolor{blue}{nEuler}};
  	\node at (4.05,4) {\textcolor{blue}{nEulerOld}};
  	\node at (0.1,6.1) {0x3F00480\textcolor{red}{0}};
  	\node at (0.1,5.75) {0x3F00480\textcolor{red}{4}};
  	\node at (0.1,5.4) {0x3F00480\textcolor{red}{8}};
  	\node at (0.1,5.05) {0x3F00480\textcolor{red}{C}};
  	\node at (0.1,4.7) {0x3F0048\textcolor{red}{10}};
  	\node at (0.1,4.37) {0x3F0048\textcolor{red}{14}};
  	\node at (0.1,4.03) {0x3F0048\textcolor{red}{18}};
  	\node at (0.1,3.67) {0x3F0048\textcolor{red}{1C}};
  	\draw[decorate,decoration={brace,mirror,raise=6pt,amplitude=8pt}, thick]
    (3.3,6.1) -- (0.9,6.1)node [black,midway,xshift=0.0cm,yshift=0.7cm]{\footnotesize {\color{red}$4$ $bytes$}};
\end{tikzpicture}
\column{0.55\textwidth}
\lstset{basicstyle=\tiny}
\begin{lstlisting}[escapechar=\|,label=noint]
#include <stdio.h>
#include <math.h>

#define LIMITE   pow(10,-9)
int  main(void)
{
int		factorial = 1,contador =1,i;    
double	nEuler=0,nEulerOld=-1;

for(i=0;(nEuler-nEulerOld)>= LIMITE;i++)
  }
     nEulerOld=nEuler;
     while(contador <= i)
     { 
     	factorial = factorial * contador;
     	contador++;
     }
     nEuler = nEuler + (1/(double)factorial); 
        
     factorial = 1;
     contador =1;
  }
  printf("e es %0.10f \n",nEuler);
  printf("e(lib math.h)es %0.10f\n",M_E);
  return 0;
}
\end{lstlisting}
\end{columns}
\end{frame}

\begin{frame}[fragile]
\fontsize{6.5pt}{10pt}\selectfont
\frametitle{Recorriendo el programa {\color{yellow}paso} a {\color{yellow}paso}}
\begin{columns}[c]
\column{0.5\textwidth}
\begin{tikzpicture}
	\begin{scope}[every node/.style={draw,anchor=text,rectangle split,rectangle split parts=10, 
rectangle split part fill={green!0,blue!0,blue!0,blue!0,blue!0,blue!0,blue!0,blue!0,blue!0,red!0,red!0},minimum width=2.6cm}]
    	\node (R) at (2,6)
    	{
    		\nodepart{one}{}
    		\nodepart{two}{1}
    		\nodepart{three}{1}	
    		\nodepart{four}{0}
    		\nodepart{five}{0}
    		\nodepart{six}{}
    		\nodepart{seven}{0}
    		\nodepart{eight}{}
    		\nodepart{nine}{$\vdots$}
    		\nodepart{ten}{}
    	};
	\end{scope}
	\draw [ultra thick,color=white](0.8, 4.6) -- (3.37, 4.6);
	\draw [ultra thick,color=white](0.8, 3.92) -- (3.37, 3.92);
	\fill[color=green] (8.6,5.2) -- (9.5,5.2) -- (9.5,4.8) -- (8.6,4.8) -- cycle;
  	\node at (2.1,9) {\normalsize Arquitectura X86-\textcolor{red}{32} bits };
  	\node at (2.1,7.9) {\small $Se$ $debe$ $\color{red}castear$ $la$ $variable$};
  	\node at (2.1,7.5) {\small $para$ $que$ $ambas$ $sean$ $\color{red}double$};
  	\node at (8,9.7) {\normalsize {\color{red!50}$C\acute{o}digo$ $del$ $programa$ $fuente$}};
  	\node at (4,5.75) {\textcolor{blue}{factorial}};
  	\node at (4,5.4){\textcolor{blue}{contador}};
  	\node at (3.6,5.05) {\textcolor{blue}{i}};
  	\node at (3.9,4.6) {\textcolor{blue}{nEuler}};
  	\node at (4.05,4) {\textcolor{blue}{nEulerOld}};
  	\node at (0.1,6.1) {0x3F00480\textcolor{red}{0}};
  	\node at (0.1,5.75) {0x3F00480\textcolor{red}{4}};
  	\node at (0.1,5.4) {0x3F00480\textcolor{red}{8}};
  	\node at (0.1,5.05) {0x3F00480\textcolor{red}{C}};
  	\node at (0.1,4.7) {0x3F0048\textcolor{red}{10}};
  	\node at (0.1,4.37) {0x3F0048\textcolor{red}{14}};
  	\node at (0.1,4.03) {0x3F0048\textcolor{red}{18}};
  	\node at (0.1,3.67) {0x3F0048\textcolor{red}{1C}};
  	\draw[decorate,decoration={brace,mirror,raise=6pt,amplitude=8pt}, thick]
    (3.3,6.1) -- (0.9,6.1)node [black,midway,xshift=0.0cm,yshift=0.7cm]{\footnotesize {\color{red}$4$ $bytes$}};
\end{tikzpicture}
\column{0.55\textwidth}
\lstset{basicstyle=\tiny}
\begin{lstlisting}[escapechar=\|,label=noint]
#include <stdio.h>
#include <math.h>

#define LIMITE   pow(10,-9)
int  main(void)
{
int		factorial = 1,contador =1,i;    
double	nEuler=0,nEulerOld=-1;

for(i=0;(nEuler-nEulerOld)>= LIMITE;i++)
  }
     nEulerOld=nEuler;
     while(contador <= i)
     { 
     	factorial = factorial * contador;
     	contador++;
     }
     nEuler = nEuler + (1/(double)factorial); 
        
     factorial = 1;
     contador =1;
  }
  printf("e es %0.10f \n",nEuler);
  printf("e(lib math.h)es %0.10f\n",M_E);
  return 0;
}
\end{lstlisting}
\end{columns}
\end{frame}

\begin{frame}[fragile]
\fontsize{6.5pt}{10pt}\selectfont
\frametitle{Recorriendo el programa {\color{yellow}paso} a {\color{yellow}paso}}
\begin{columns}[c]
\column{0.5\textwidth}
\begin{tikzpicture}
	\begin{scope}[every node/.style={draw,anchor=text,rectangle split,rectangle split parts=10, 
rectangle split part fill={green!0,blue!0,blue!0,blue!0,blue!0,blue!0,blue!0,blue!0,blue!0,red!0,red!0},minimum width=2.6cm}]
    	\node (R) at (2,6)
    	{
    		\nodepart{one}{}	
    		\nodepart{two}{1}
    		\nodepart{three}{1}	
    		\nodepart{four}{0}
    		\nodepart{five}{\color{red}1.00}
    		\nodepart{six}{}
    		\nodepart{seven}{0}
    		\nodepart{eight}{}
    		\nodepart{nine}{$\vdots$}
    		\nodepart{ten}{}
    	};
	\end{scope}
	\draw [ultra thick,color=white](0.8, 4.6) -- (3.37, 4.6);
	\draw [ultra thick,color=white](0.8, 3.92) -- (3.37, 3.92);
	\fill[color=green] (5.6,5.2) -- (11,5.2) -- (11,4.8) -- (5.6,4.8) -- cycle;
  	\node at (2.1,9) {\normalsize Arquitectura X86-\textcolor{red}{32} bits };
  	\node at (2.1,7.9) {\small $almacena$ $el$ $resultado$ $en$ $la$};
  	\node at (2.1,7.5) {\small $variable$ $\color{red}nEuler$};
  	\node at (8,9.7) {\normalsize {\color{red!50}$C\acute{o}digo$ $del$ $programa$ $fuente$}};
  	\node at (4,5.75) {\textcolor{blue}{factorial}};
  	\node at (4,5.4){\textcolor{blue}{contador}};
  	\node at (3.6,5.05) {\textcolor{blue}{i}};
  	\node at (3.9,4.6) {\textcolor{blue}{nEuler}};
  	\node at (4.05,4) {\textcolor{blue}{nEulerOld}};
  	\node at (0.1,6.1) {0x3F00480\textcolor{red}{0}};
  	\node at (0.1,5.75) {0x3F00480\textcolor{red}{4}};
  	\node at (0.1,5.4) {0x3F00480\textcolor{red}{8}};
  	\node at (0.1,5.05) {0x3F00480\textcolor{red}{C}};
  	\node at (0.1,4.7) {0x3F0048\textcolor{red}{10}};
  	\node at (0.1,4.37) {0x3F0048\textcolor{red}{14}};
  	\node at (0.1,4.03) {0x3F0048\textcolor{red}{18}};
  	\node at (0.1,3.67) {0x3F0048\textcolor{red}{1C}};
  	\draw[decorate,decoration={brace,mirror,raise=6pt,amplitude=8pt}, thick]
    (3.3,6.1) -- (0.9,6.1)node [black,midway,xshift=0.0cm,yshift=0.7cm]{\footnotesize {\color{red}$4$ $bytes$}};
  	\draw[decorate,decoration={brace,raise=3pt,amplitude=3pt}, thick]
    (7,5.1)--(10.7,5.1)node [black,midway,xshift=1.5cm,yshift=0.4cm]{\scriptsize{\color{red}$0+\frac{1}{1.00}=1.00$ }};
\end{tikzpicture}
\column{0.55\textwidth}
\lstset{basicstyle=\tiny}
\begin{lstlisting}[escapechar=\|,label=noint]
#include <stdio.h>
#include <math.h>

#define LIMITE   pow(10,-9)
int  main(void)
{
int		factorial = 1,contador =1,i;    
double	nEuler=0,nEulerOld=-1;

for(i=0;(nEuler-nEulerOld)>= LIMITE;i++)
  }
     nEulerOld=nEuler;
     while(contador <= i)
     { 
     	factorial = factorial * contador;
     	contador++;
     }
     nEuler = nEuler + (1/(double)factorial); 
        
     factorial = 1;
     contador =1;
  }
  printf("e es %0.10f \n",nEuler);
  printf("e(lib math.h)es %0.10f\n",M_E);
  return 0;
}
\end{lstlisting}
\end{columns}
\end{frame}

\begin{frame}[fragile]
\fontsize{6.5pt}{10pt}\selectfont
\frametitle{Recorriendo el programa {\color{yellow}paso} a {\color{yellow}paso}}
\begin{columns}[c]
\column{0.5\textwidth}
\begin{tikzpicture}
	\begin{scope}[every node/.style={draw,anchor=text,rectangle split,rectangle split parts=10, 
rectangle split part fill={green!0,blue!0,blue!0,blue!0,blue!0,blue!0,blue!0,blue!0,blue!0,red!0,red!0},minimum width=2.6cm}]
    	\node (R) at (2,6)
    	{
    		\nodepart{one}{}
    		\nodepart{two}{\color{red}1}
    		\nodepart{three}{\color{red}1}	
    		\nodepart{four}{0}
    		\nodepart{five}{1.00}
    		\nodepart{six}{}
    		\nodepart{seven}{0}
    		\nodepart{eight}{}
    		\nodepart{nine}{$\vdots$}
    		\nodepart{ten}{}
    	};
	\end{scope}
	\draw [ultra thick,color=white](0.8, 4.6) -- (3.37, 4.6);
	\draw [ultra thick,color=white](0.8, 3.92) -- (3.37, 3.92);
	\fill[color=green] (5.6,4.6) -- (8,4.6) -- (8,4.1) -- (5.6,4.1) -- cycle;
  	\node at (2.1,9) {\normalsize Arquitectura X86-\textcolor{red}{32} bits };
  	\node at (2.1,7.9) {\small $inicializo$ $nuevamente$ $las$};
  	\node at (2.1,7.5) {\small $variables$ $\color{red}factorial$ y $\color{red}contador$};
  	\node at (8,9.7) {\normalsize {\color{red!50}$C\acute{o}digo$ $del$ $programa$ $fuente$}};
  	\node at (4,5.75) {\textcolor{blue}{factorial}};
  	\node at (4,5.4){\textcolor{blue}{contador}};
  	\node at (3.6,5.05) {\textcolor{blue}{i}};
  	\node at (3.9,4.6) {\textcolor{blue}{nEuler}};
  	\node at (4.05,4) {\textcolor{blue}{nEulerOld}};
  	\node at (0.1,6.1) {0x3F00480\textcolor{red}{0}};
  	\node at (0.1,5.75) {0x3F00480\textcolor{red}{4}};
  	\node at (0.1,5.4) {0x3F00480\textcolor{red}{8}};
  	\node at (0.1,5.05) {0x3F00480\textcolor{red}{C}};
  	\node at (0.1,4.7) {0x3F0048\textcolor{red}{10}};
  	\node at (0.1,4.37) {0x3F0048\textcolor{red}{14}};
  	\node at (0.1,4.03) {0x3F0048\textcolor{red}{18}};
  	\node at (0.1,3.67) {0x3F0048\textcolor{red}{1C}};
  	\draw[decorate,decoration={brace,mirror,raise=6pt,amplitude=8pt}, thick]
    (3.3,6.1) -- (0.9,6.1)node [black,midway,xshift=0.0cm,yshift=0.7cm]{\footnotesize {\color{red}$4$ $bytes$}};
\end{tikzpicture}
\column{0.55\textwidth}
\lstset{basicstyle=\tiny}
\begin{lstlisting}[escapechar=\|,label=noint]
#include <stdio.h>
#include <math.h>

#define LIMITE   pow(10,-9)
int  main(void)
{
int		factorial = 1,contador =1,i;    
double	nEuler=0,nEulerOld=-1;

for(i=0;(nEuler-nEulerOld)>= LIMITE;i++)
  }
     nEulerOld=nEuler;
     while(contador <= i)
     { 
     	factorial = factorial * contador;
     	contador++;
     }
     nEuler = nEuler + (1/(double)factorial); 
        
     factorial = 1;
     contador =1;
  }
  printf("e es %0.10f \n",nEuler);
  printf("e(lib math.h)es %0.10f\n",M_E);
  return 0;
}
\end{lstlisting}
\end{columns}
\end{frame}

\begin{frame}[fragile]
\fontsize{6.5pt}{10pt}\selectfont
\frametitle{Recorriendo el programa {\color{yellow}paso} a {\color{yellow}paso}}
\begin{columns}[c]
\column{0.5\textwidth}
\begin{tikzpicture}
	\begin{scope}[every node/.style={draw,anchor=text,rectangle split,rectangle split parts=10, 
rectangle split part fill={green!0,blue!0,blue!0,blue!0,blue!0,blue!0,blue!0,blue!0,blue!0,red!0,red!0},minimum width=2.6cm}]
    	\node (R) at (2,6)
    	{
    		\nodepart{one}{}
    		\nodepart{two}{1}
    		\nodepart{three}{1}	
    		\nodepart{four}{\color{red}1}
    		\nodepart{five}{1.00}
    		\nodepart{six}{}
    		\nodepart{seven}{0}
    		\nodepart{eight}{}
    		\nodepart{nine}{$\vdots$}
    		\nodepart{ten}{}
    	};
	\end{scope}
	\draw [ultra thick,color=white](0.8, 4.6) -- (3.37, 4.6);
	\draw [ultra thick,color=white](0.8, 3.92) -- (3.37, 3.92);
	\fill[color=green] (9.7,7.2) -- (10.3,7.2) -- (10.3,6.8) -- (9.7,6.8) -- cycle;
  	\node at (2.1,9) {\normalsize Arquitectura X86-\textcolor{red}{32} bits };
  	\node at (2.1,7.9) {\small $incremento$ $la$ $variable$ $\color{red}i$};
  	\node at (8,9.7) {\normalsize {\color{red!50}$C\acute{o}digo$ $del$ $programa$ $fuente$}};
  	\node at (4,5.75) {\textcolor{blue}{factorial}};
  	\node at (4,5.4){\textcolor{blue}{contador}};
  	\node at (3.6,5.05) {\textcolor{blue}{i}};
  	\node at (3.9,4.6) {\textcolor{blue}{nEuler}};
  	\node at (4.05,4) {\textcolor{blue}{nEulerOld}};
  	\node at (0.1,6.1) {0x3F00480\textcolor{red}{0}};
  	\node at (0.1,5.75) {0x3F00480\textcolor{red}{4}};
  	\node at (0.1,5.4) {0x3F00480\textcolor{red}{8}};
  	\node at (0.1,5.05) {0x3F00480\textcolor{red}{C}};
  	\node at (0.1,4.7) {0x3F0048\textcolor{red}{10}};
  	\node at (0.1,4.37) {0x3F0048\textcolor{red}{14}};
  	\node at (0.1,4.03) {0x3F0048\textcolor{red}{18}};
  	\node at (0.1,3.67) {0x3F0048\textcolor{red}{1C}};
  	\draw[decorate,decoration={brace,mirror,raise=6pt,amplitude=8pt}, thick]
    (3.3,6.1) -- (0.9,6.1)node [black,midway,xshift=0.0cm,yshift=0.7cm]{\footnotesize {\color{red}$4$ $bytes$}};
\end{tikzpicture}
\column{0.55\textwidth}
\lstset{basicstyle=\tiny}
\begin{lstlisting}[escapechar=\|,label=noint]
#include <stdio.h>
#include <math.h>

#define LIMITE   pow(10,-9)
int  main(void)
{
int		factorial = 1,contador =1,i;    
double	nEuler=0,nEulerOld=-1;

for(i=0;(nEuler-nEulerOld)>= LIMITE;i++)
  }
     nEulerOld=nEuler;
     while(contador <= i)
     { 
     	factorial = factorial * contador;
     	contador++;
     }
     nEuler = nEuler + (1/(double)factorial); 
        
     factorial = 1;
     contador =1;
  }
  printf("e es %0.10f \n",nEuler);
  printf("e(lib math.h)es %0.10f\n",M_E);
  return 0;
}
   \end{lstlisting}
 \end{columns}
\end{frame}

\begin{frame}[fragile]
\fontsize{6.5pt}{10pt}\selectfont
\frametitle{Recorriendo el programa {\color{yellow}paso} a {\color{yellow}paso}}
\begin{columns}[c]
\column{0.5\textwidth}
\begin{tikzpicture}
	\begin{scope}[every node/.style={draw,anchor=text,rectangle split,rectangle split parts=10, 
rectangle split part fill={green!0,blue!0,blue!0,blue!0,blue!0,blue!0,blue!0,blue!0,blue!0,red!0,red!0},minimum width=2.6cm}]
    	\node (R) at (2,6)
    	{
    		\nodepart{one}{}
    		\nodepart{two}{1}
    		\nodepart{three}{1}	
    		\nodepart{four}{0}
    		\nodepart{five}{1.00}
    		\nodepart{six}{}
    		\nodepart{seven}{0}
    		\nodepart{eight}{}
    		\nodepart{nine}{$\vdots$}
    		\nodepart{ten}{}
    	};
	\end{scope}
	\draw [ultra thick,color=white](0.8, 4.6) -- (3.37, 4.6);
	\draw [ultra thick,color=white](0.8, 3.92) -- (3.37, 3.92);
	\fill[color=green] (6.4,7.2) -- (9.7,7.2) -- (9.7,6.8) -- (6.4,6.8) -- cycle;
  	\node at (2.1,9) {\normalsize Arquitectura X86-\textcolor{red}{32} bits };
  	\node at (2.1,7.9) {\small $Pregunta$ $condici\acute{o}n$ $del$ $\color{red}for$};
  	\node at (8,9.7) {\normalsize {\color{red!50}$C\acute{o}digo$ $del$ $programa$ $fuente$}};
  	\node at (4,5.75) {\textcolor{blue}{factorial}};
  	\node at (4,5.4){\textcolor{blue}{contador}};
  	\node at (3.6,5.05) {\textcolor{blue}{i}};
  	\node at (3.9,4.6) {\textcolor{blue}{nEuler}};
  	\node at (4.05,4) {\textcolor{blue}{nEulerOld}};
  	\node at (0.1,6.1) {0x3F00480\textcolor{red}{0}};
  	\node at (0.1,5.75) {0x3F00480\textcolor{red}{4}};
  	\node at (0.1,5.4) {0x3F00480\textcolor{red}{8}};
  	\node at (0.1,5.05) {0x3F00480\textcolor{red}{C}};
  	\node at (0.1,4.7) {0x3F0048\textcolor{red}{10}};
  	\node at (0.1,4.37) {0x3F0048\textcolor{red}{14}};
  	\node at (0.1,4.03) {0x3F0048\textcolor{red}{18}};
  	\node at (0.1,3.67) {0x3F0048\textcolor{red}{1C}};
  	\draw[decorate,decoration={brace,mirror,raise=6pt,amplitude=8pt}, thick]
    (3.3,6.1) -- (0.9,6.1)node [black,midway,xshift=0.0cm,yshift=0.7cm]{\footnotesize {\color{red}$4$ $bytes$}};
  	\draw[decorate,decoration={brace,raise=4pt,amplitude=4pt}, thick]
    (7.35,8.4)--(8.7,8.4)node [black,midway,xshift=1.0cm,yshift=0.4cm]{\footnotesize {\color{red!60}$10^{-9}$}};
  	\draw[red,->](9.3,7.1)to [out=10,in=0]node[right,midway]{} ++(-0.2,1.3) ; 
\end{tikzpicture}
\column{0.55\textwidth}
\lstset{basicstyle=\tiny}
\begin{lstlisting}[escapechar=\|,label=noint]
#include <stdio.h>
#include <math.h>

#define LIMITE   pow(10,-9)
int  main(void)
{
int		factorial = 1,contador =1,i;    
double	nEuler=0,nEulerOld=-1;

for(i=0;(nEuler-nEulerOld)>= LIMITE;i++)
  }
     nEulerOld=nEuler;
     while(contador <= i)
     { 
     	factorial = factorial * contador;
     	contador++;
     }
     nEuler = nEuler + (1/(double)factorial); 
        
     factorial = 1;
     contador =1;
  }
  printf("e es %0.10f \n",nEuler);
  printf("e(lib math.h)es %0.10f\n",M_E);
  return 0;
}
\end{lstlisting}
\end{columns}
\end{frame}

\begin{frame}[fragile]
\fontsize{6.5pt}{10pt}\selectfont
\frametitle{Recorriendo el programa {\color{yellow}paso} a {\color{yellow}paso}}
\begin{columns}[c]
\column{0.5\textwidth}
\begin{tikzpicture}
	\begin{scope}[every node/.style={draw,anchor=text,rectangle split,rectangle split parts=10, 
rectangle split part fill={green!0,blue!0,blue!0,blue!0,blue!0,blue!0,blue!0,blue!0,blue!0,red!0,red!0},minimum width=2.6cm}]
    	\node (R) at (2,6)
    	{
    		\nodepart{one}{}
    		\nodepart{two}{1}
    		\nodepart{three}{1}	
    		\nodepart{four}{1}
    		\nodepart{five}{1.00}
    		\nodepart{six}{}
    		\nodepart{seven}{\color{red}1.00}
    		\nodepart{eight}{}
    		\nodepart{nine}{$\vdots$}
    		\nodepart{ten}{}
    	};
	\end{scope}
	\draw [ultra thick,color=white](0.8, 4.6) -- (3.37, 4.6);
	\draw [ultra thick,color=white](0.8, 3.92) -- (3.37, 3.92);
	\fill[color=green] (5.8,6.65) -- (8,6.65) -- (8,6.35) -- (5.8,6.35) -- cycle;
	\node at (2.1,9) {\normalsize Arquitectura X86-\textcolor{red}{32} bits };
  	\node at (2.1,7.9) {\small $Para$ $conservar$ $el$ $\color{red}e$ $anterior$};
  	\node at (8,9.7) {\normalsize {\color{red!50}$C\acute{o}digo$ $del$ $programa$ $fuente$}};
  	\node at (4,5.75) {\textcolor{blue}{factorial}};
  	\node at (4,5.4){\textcolor{blue}{contador}};
  	\node at (3.6,5.05) {\textcolor{blue}{i}};
  	\node at (3.9,4.6) {\textcolor{blue}{nEuler}};
  	\node at (4.05,4) {\textcolor{blue}{nEulerOld}};
  	\node at (0.1,6.1) {0x3F00480\textcolor{red}{0}};
  	\node at (0.1,5.75) {0x3F00480\textcolor{red}{4}};
  	\node at (0.1,5.4) {0x3F00480\textcolor{red}{8}};
  	\node at (0.1,5.05) {0x3F00480\textcolor{red}{C}};
  	\node at (0.1,4.7) {0x3F0048\textcolor{red}{10}};
  	\node at (0.1,4.37) {0x3F0048\textcolor{red}{14}};
  	\node at (0.1,4.03) {0x3F0048\textcolor{red}{18}};
  	\node at (0.1,3.67) {0x3F0048\textcolor{red}{1C}};
  	\draw[decorate,decoration={brace,mirror,raise=6pt,amplitude=8pt}, thick]
    (3.3,6.1) -- (0.9,6.1)node [black,midway,xshift=0.0cm,yshift=0.7cm]{\footnotesize {\color{red}$4$ $bytes$}};
\end{tikzpicture}
\column{0.55\textwidth}
\lstset{basicstyle=\tiny}
\begin{lstlisting}[escapechar=\|,label=noint]
#include <stdio.h>
#include <math.h>

#define LIMITE   pow(10,-9)
int  main(void)
{
int		factorial = 1,contador =1,i;    
double	nEuler=0,nEulerOld=-1;

for(i=0;(nEuler-nEulerOld)>= LIMITE;i++)
  }
     nEulerOld=nEuler;
     while(contador <= i)
     { 
     	factorial = factorial * contador;
     	contador++;
     }
     nEuler = nEuler + (1/(double)factorial); 
        
     factorial = 1;
     contador =1;
  }
  printf("e es %0.10f \n",nEuler);
  printf("e(lib math.h)es %0.10f\n",M_E);
  return 0;
}
\end{lstlisting}
\end{columns}
\end{frame}

\begin{frame}[fragile]
\fontsize{6.5pt}{10pt}\selectfont
\frametitle{Recorriendo el programa {\color{yellow}paso} a {\color{yellow}paso}}
\begin{columns}[c]
\column{0.5\textwidth}
\begin{tikzpicture}
	\begin{scope}[every node/.style={draw,anchor=text,rectangle split,rectangle split parts=10, 
rectangle split part fill={green!0,blue!0,blue!0,blue!0,blue!0,blue!0,blue!0,blue!0,blue!0,red!0,red!0},minimum width=2.6cm}]
    	\node (R) at (2,6)
    	{
    		\nodepart{one}{}
    		\nodepart{two}{1}
    		\nodepart{three}{1}	
    		\nodepart{four}{1}
    		\nodepart{five}{1.00}
    		\nodepart{six}{}
    		\nodepart{seven}{1.00}
    		\nodepart{eight}{}
    		\nodepart{nine}{$\vdots$}
    		\nodepart{ten}{}
    	};
	\end{scope}
	\draw [ultra thick,color=white](0.8, 4.6) -- (3.37, 4.6);
	\draw [ultra thick,color=white](0.8, 3.92) -- (3.37, 3.92);
	\fill[color=green] (5.8,6.4) -- (8.4,6.4) -- (8.4,6) -- (5.8,6) -- cycle;
  	\node at (2.1,9) {\normalsize Arquitectura X86-\textcolor{red}{32} bits };
  	\node at (2.1,7.9) {\small $Como$ $\color{red}contador$ $es$ $\color{red}<=$ $a$ $\color{red}i$};
  	\node at (2.1,7.5) {\small $entra$ $dentro$ $del$ $\color{red}while$};
  	\node at (8,9.7) {\normalsize {\color{red!50}$C\acute{o}digo$ $del$ $programa$ $fuente$}};
  	\node at (4,5.75) {\textcolor{blue}{factorial}};
  	\node at (4,5.4){\textcolor{blue}{contador}};
  	\node at (3.6,5.05) {\textcolor{blue}{i}};
  	\node at (3.9,4.6) {\textcolor{blue}{nEuler}};
  	\node at (4.05,4) {\textcolor{blue}{nEulerOld}};
  	\node at (0.1,6.1) {0x3F00480\textcolor{red}{0}};
  	\node at (0.1,5.75) {0x3F00480\textcolor{red}{4}};
  	\node at (0.1,5.4) {0x3F00480\textcolor{red}{8}};
  	\node at (0.1,5.05) {0x3F00480\textcolor{red}{C}};
  	\node at (0.1,4.7) {0x3F0048\textcolor{red}{10}};
  	\node at (0.1,4.37) {0x3F0048\textcolor{red}{14}};
  	\node at (0.1,4.03) {0x3F0048\textcolor{red}{18}};
  	\node at (0.1,3.67) {0x3F0048\textcolor{red}{1C}};
  	\draw[decorate,decoration={brace,mirror,raise=6pt,amplitude=8pt}, thick]
    (3.3,6.1) -- (0.9,6.1)node [black,midway,xshift=0.0cm,yshift=0.7cm]{\footnotesize {\color{red}$4$ $bytes$}};
\end{tikzpicture}
\column{0.55\textwidth}
\lstset{basicstyle=\tiny}
\begin{lstlisting}[escapechar=\|,label=noint]
#include <stdio.h>
#include <math.h>

#define LIMITE   pow(10,-9)
int  main(void)
{
int		factorial = 1,contador =1,i;    
double	nEuler=0,nEulerOld=-1;

for(i=0;(nEuler-nEulerOld)>= LIMITE;i++)
  }
     nEulerOld=nEuler;
     while(contador <= i)
     { 
     	factorial = factorial * contador;
     	contador++;
     }
     nEuler = nEuler + (1/(double)factorial); 
        
     factorial = 1;
     contador =1;
  }
  printf("e es %0.10f \n",nEuler);
  printf("e(lib math.h)es %0.10f\n",M_E);
  return 0;
}
\end{lstlisting}
\end{columns}
\end{frame}

\begin{frame}[fragile]
\fontsize{6.5pt}{10pt}\selectfont
\frametitle{Recorriendo el programa {\color{yellow}paso} a {\color{yellow}paso}}
\begin{columns}[c]
\column{0.5\textwidth}
\begin{tikzpicture}
	\begin{scope}[every node/.style={draw,anchor=text,rectangle split,rectangle split parts=10, 
rectangle split part fill={green!0,blue!0,blue!0,blue!0,blue!0,blue!0,blue!0,blue!0,blue!0,red!0,red!0},minimum width=2.6cm}]
    	\node (R) at (2,6)
    	{
    		\nodepart{one}{}
    		\nodepart{two}{\color{red}1}
    		\nodepart{three}{1}	
    		\nodepart{four}{1}
    		\nodepart{five}{1.00}
    		\nodepart{six}{}
    		\nodepart{seven}{1.00}
    		\nodepart{eight}{}
    		\nodepart{nine}{$\vdots$}
    		\nodepart{ten}{}
    	};
	\end{scope}
	\draw [ultra thick,color=white](0.8, 4.6) -- (3.37, 4.6);
	\draw [ultra thick,color=white](0.8, 3.92) -- (3.37, 3.92);
	\fill[color=green] (5.8,5.9) -- (10.5,5.9) -- (10.5,5.6) -- (5.8,5.6) -- cycle;
  	\node at (2.1,9) {\normalsize Arquitectura X86-\textcolor{red}{32} bits };
  	\node at (2.1,7.9) {\small $almacena$ $el$ $resultado$ $en$ $la$};
  	\node at (2.1,7.5) {\small $variable$ $\color{red}factorial$};
  	\node at (8,9.7) {\normalsize {\color{red!50}$C\acute{o}digo$ $del$ $programa$ $fuente$}};
  	\node at (4,5.75) {\textcolor{blue}{factorial}};
  	\node at (4,5.4){\textcolor{blue}{contador}};
  	\node at (3.6,5.05) {\textcolor{blue}{i}};
  	\node at (3.9,4.6) {\textcolor{blue}{nEuler}};
  	\node at (4.05,4) {\textcolor{blue}{nEulerOld}};
  	\node at (0.1,6.1) {0x3F00480\textcolor{red}{0}};
  	\node at (0.1,5.75) {0x3F00480\textcolor{red}{4}};
  	\node at (0.1,5.4) {0x3F00480\textcolor{red}{8}};
  	\node at (0.1,5.05) {0x3F00480\textcolor{red}{C}};
  	\node at (0.1,4.7) {0x3F0048\textcolor{red}{10}};
  	\node at (0.1,4.37) {0x3F0048\textcolor{red}{14}};
  	\node at (0.1,4.03) {0x3F0048\textcolor{red}{18}};
  	\node at (0.1,3.67) {0x3F0048\textcolor{red}{1C}};
  	\draw[decorate,decoration={brace,mirror,raise=6pt,amplitude=8pt}, thick]
    (3.3,6.1) -- (0.9,6.1)node [black,midway,xshift=0.0cm,yshift=0.7cm]{\footnotesize {\color{red}$4$ $bytes$}};
  	\draw[decorate,decoration={brace,raise=3pt,amplitude=3pt}, thick]
    (7.5,5.8)--(10.2,5.8)node [black,midway,xshift=1.5cm,yshift=0.4cm]{\footnotesize{\color{red}$1\cdot 1=1$ }};
\end{tikzpicture}
\column{0.55\textwidth}
\lstset{basicstyle=\tiny}
\begin{lstlisting}[escapechar=\|,label=noint]
#include <stdio.h>
#include <math.h>

#define LIMITE   pow(10,-9)
int  main(void)
{
int		factorial = 1,contador =1,i;    
double	nEuler=0,nEulerOld=-1;

for(i=0;(nEuler-nEulerOld)>= LIMITE;i++)
  }
     nEulerOld=nEuler;
     while(contador <= i)
     { 
     	factorial = factorial * contador;
     	contador++;
     }
     nEuler = nEuler + (1/(double)factorial); 
        
     factorial = 1;
     contador =1;
  }
  printf("e es %0.10f \n",nEuler);
  printf("e(lib math.h)es %0.10f\n",M_E);
  return 0;
}
\end{lstlisting}
\end{columns}
\end{frame}

\begin{frame}[fragile]
\fontsize{6.5pt}{10pt}\selectfont
\frametitle{Recorriendo el programa {\color{yellow}paso} a {\color{yellow}paso}}
\begin{columns}[c]
\column{0.5\textwidth}
\begin{tikzpicture}
	\begin{scope}[every node/.style={draw,anchor=text,rectangle split,rectangle split parts=10, 
rectangle split part fill={green!0,blue!0,blue!0,blue!0,blue!0,blue!0,blue!0,blue!0,blue!0,red!0,red!0},minimum width=2.6cm}]
    	\node (R) at (2,6)
    	{
    		\nodepart{one}{}
    		\nodepart{two}{1}
    		\nodepart{three}{\color{red}2}	
    		\nodepart{four}{1}
    		\nodepart{five}{1.00}
    		\nodepart{six}{}
    		\nodepart{seven}{1.00}
    		\nodepart{eight}{}
    		\nodepart{nine}{$\vdots$}
    		\nodepart{ten}{}
    	};
	\end{scope}
	\draw [ultra thick,color=white](0.8, 4.6) -- (3.37, 4.6);
	\draw [ultra thick,color=white](0.8, 3.92) -- (3.37, 3.92);
	\fill[color=green] (5.8,5.6) -- (8,5.6) -- (8,5.3) -- (5.8,5.3) -- cycle;
  	\node at (2.1,9) {\normalsize Arquitectura X86-\textcolor{red}{32} bits };
  	\node at (2.1,7.9) {\small $incremento$ $la$ $variable$ $\color{red}contador$};
  	\node at (8,9.7) {\normalsize {\color{red!50}$C\acute{o}digo$ $del$ $programa$ $fuente$}};
  	\node at (4,5.75) {\textcolor{blue}{factorial}};
  	\node at (4,5.4){\textcolor{blue}{contador}};
  	\node at (3.6,5.05) {\textcolor{blue}{i}};
  	\node at (3.9,4.6) {\textcolor{blue}{nEuler}};
  	\node at (4.05,4) {\textcolor{blue}{nEulerOld}};
  	\node at (0.1,6.1) {0x3F00480\textcolor{red}{0}};
  	\node at (0.1,5.75) {0x3F00480\textcolor{red}{4}};
  	\node at (0.1,5.4) {0x3F00480\textcolor{red}{8}};
  	\node at (0.1,5.05) {0x3F00480\textcolor{red}{C}};
  	\node at (0.1,4.7) {0x3F0048\textcolor{red}{10}};
  	\node at (0.1,4.37) {0x3F0048\textcolor{red}{14}};
  	\node at (0.1,4.03) {0x3F0048\textcolor{red}{18}};
  	\node at (0.1,3.67) {0x3F0048\textcolor{red}{1C}};
  	\draw[decorate,decoration={brace,mirror,raise=6pt,amplitude=8pt}, thick]
    (3.3,6.1) -- (0.9,6.1)node [black,midway,xshift=0.0cm,yshift=0.7cm]{\footnotesize {\color{red}$4$ $bytes$}};
\end{tikzpicture}
\column{0.55\textwidth}
\lstset{basicstyle=\tiny}
\begin{lstlisting}[escapechar=\|,label=noint]
#include <stdio.h>
#include <math.h>

#define LIMITE   pow(10,-9)
int  main(void)
{
int		factorial = 1,contador =1,i;    
double	nEuler=0,nEulerOld=-1;

for(i=0;(nEuler-nEulerOld)>= LIMITE;i++)
  }
     nEulerOld=nEuler;
     while(contador <= i)
     { 
     	factorial = factorial * contador;
     	contador++;
     }
     nEuler = nEuler + (1/(double)factorial); 
        
     factorial = 1;
     contador =1;
  }
  printf("e es %0.10f \n",nEuler);
  printf("e(lib math.h)es %0.10f\n",M_E);
  return 0;
}
\end{lstlisting}
\end{columns}
\end{frame}

\begin{frame}[fragile]
\fontsize{6.5pt}{10pt}\selectfont
\frametitle{Recorriendo el programa {\color{yellow}paso} a {\color{yellow}paso}}
\begin{columns}[c]
\column{0.5\textwidth}
\begin{tikzpicture}
	\begin{scope}[every node/.style={draw,anchor=text,rectangle split,rectangle split parts=10, 
rectangle split part fill={green!0,blue!0,blue!0,blue!0,blue!0,blue!0,blue!0,blue!0,blue!0,red!0,red!0},minimum width=2.6cm}]
    	\node (R) at (2,6)
    	{
    		\nodepart{one}{}
    		\nodepart{two}{1}
    		\nodepart{three}{2}	
    		\nodepart{four}{1}
    		\nodepart{five}{1.00}
    		\nodepart{six}{}
    		\nodepart{seven}{1.00}
    		\nodepart{eight}{}
    		\nodepart{nine}{$\vdots$}
    		\nodepart{ten}{}
    	};
	\end{scope}
	\draw [ultra thick,color=white](0.8, 4.6) -- (3.37, 4.6);
	\draw [ultra thick,color=white](0.8, 3.92) -- (3.37, 3.92);
	\fill[color=green] (5.8,6.4) -- (8.4,6.4) -- (8.4,6) -- (5.8,6) -- cycle;
  	\node at (2.1,9) {\normalsize Arquitectura X86-\textcolor{red}{32} bits };
  	\node at (2.1,7.9) {\small $Como$ $\color{red}contador$ $no$ $es$ $\color{red}<=$ $a$ $\color{red}i$};
  	\node at (2.1,7.5) {\small $sale$ $del$ $loop$ $\color{red}while$};
  	\node at (8,9.7) {\normalsize {\color{red!50}$C\acute{o}digo$ $del$ $programa$ $fuente$}};
  	\node at (4,5.75) {\textcolor{blue}{factorial}};
  	\node at (4,5.4){\textcolor{blue}{contador}};
  	\node at (3.6,5.05) {\textcolor{blue}{i}};
  	\node at (3.9,4.6) {\textcolor{blue}{nEuler}};
  	\node at (4.05,4) {\textcolor{blue}{nEulerOld}};
  	\node at (0.1,6.1) {0x3F00480\textcolor{red}{0}};
  	\node at (0.1,5.75) {0x3F00480\textcolor{red}{4}};
  	\node at (0.1,5.4) {0x3F00480\textcolor{red}{8}};
  	\node at (0.1,5.05) {0x3F00480\textcolor{red}{C}};
  	\node at (0.1,4.7) {0x3F0048\textcolor{red}{10}};
  	\node at (0.1,4.37) {0x3F0048\textcolor{red}{14}};
  	\node at (0.1,4.03) {0x3F0048\textcolor{red}{18}};
  	\node at (0.1,3.67) {0x3F0048\textcolor{red}{1C}};
  	\draw[decorate,decoration={brace,mirror,raise=6pt,amplitude=8pt}, thick]
    (3.3,6.1) -- (0.9,6.1)node [black,midway,xshift=0.0cm,yshift=0.7cm]{\footnotesize {\color{red}$4$ $bytes$}};
\end{tikzpicture}
\column{0.55\textwidth}
\lstset{basicstyle=\tiny}
\begin{lstlisting}[escapechar=\|,label=noint]
#include <stdio.h>
#include <math.h>

#define LIMITE   pow(10,-9)
int  main(void)
{
int		factorial = 1,contador =1,i;    
double	nEuler=0,nEulerOld=-1;

for(i=0;(nEuler-nEulerOld)>= LIMITE;i++)
  }
     nEulerOld=nEuler;
     while(contador <= i)
     { 
     	factorial = factorial * contador;
     	contador++;
     }
     nEuler = nEuler + (1/(double)factorial); 
        
     factorial = 1;
     contador =1;
  }
  printf("e es %0.10f \n",nEuler);
  printf("e(lib math.h)es %0.10f\n",M_E);
  return 0;
}
\end{lstlisting}
\end{columns}
\end{frame}

\begin{frame}[fragile]
\fontsize{6.5pt}{10pt}\selectfont
\frametitle{Recorriendo el programa {\color{yellow}paso} a {\color{yellow}paso}}
\begin{columns}[c]
\column{0.5\textwidth}
\begin{tikzpicture}
	\begin{scope}[every node/.style={draw,anchor=text,rectangle split,rectangle split parts=10, 
rectangle split part fill={green!0,blue!0,blue!0,blue!0,blue!0,blue!0,blue!0,blue!0,blue!0,red!0,red!0},minimum width=2.6cm}]
    	\node (R) at (2,6)
    	{
    		\nodepart{one}{}
    		\nodepart{two}{1}
    		\nodepart{three}{2}	
    		\nodepart{four}{1}
    		\nodepart{five}{\color{red}2.00}
    		\nodepart{six}{}
    		\nodepart{seven}{1.00}
    		\nodepart{eight}{}
    		\nodepart{nine}{$\vdots$}
    		\nodepart{ten}{}
    	};
	\end{scope}
	\draw [ultra thick,color=white](0.8, 4.6) -- (3.37, 4.6);
	\draw [ultra thick,color=white](0.8, 3.92) -- (3.37, 3.92);
	\fill[color=green] (5.6,5.2) -- (11,5.2) -- (11,4.8) -- (5.6,4.8) -- cycle;
  	\node at (2.1,7.9) {\small $almacena$ $el$ $resultado$ $en$ $la$};
  	\node at (2.1,7.5) {\small $variable$ $\color{red}nEuler$};
  	\node at (2.1,9) {\normalsize Arquitectura X86-\textcolor{red}{32} bits };
  	\node at (8,9.7) {\normalsize {\color{red!50}$C\acute{o}digo$ $del$ $programa$ $fuente$}};
  	\node at (4,5.75) {\textcolor{blue}{factorial}};
  	\node at (4,5.4){\textcolor{blue}{contador}};
  	\node at (3.6,5.05) {\textcolor{blue}{i}};
  	\node at (3.9,4.6) {\textcolor{blue}{nEuler}};
  	\node at (4.05,4) {\textcolor{blue}{nEulerOld}};
  	\node at (0.1,6.1) {0x3F00480\textcolor{red}{0}};
  	\node at (0.1,5.75) {0x3F00480\textcolor{red}{4}};
  	\node at (0.1,5.4) {0x3F00480\textcolor{red}{8}};
  	\node at (0.1,5.05) {0x3F00480\textcolor{red}{C}};
  	\node at (0.1,4.7) {0x3F0048\textcolor{red}{10}};
  	\node at (0.1,4.37) {0x3F0048\textcolor{red}{14}};
  	\node at (0.1,4.03) {0x3F0048\textcolor{red}{18}};
  	\node at (0.1,3.67) {0x3F0048\textcolor{red}{1C}};
  	\draw[decorate,decoration={brace,mirror,raise=6pt,amplitude=8pt}, thick]
    (3.3,6.1) -- (0.9,6.1)node [black,midway,xshift=0.0cm,yshift=0.7cm]{\footnotesize {\color{red}$4$ $bytes$}};
  	\draw[decorate,decoration={brace,raise=3pt,amplitude=3pt}, thick]
    (7,5.1) -- (10.7,5.1)node [black,midway,xshift=1.5cm,yshift=0.4cm]{\scriptsize {\color{red}$1.00+\frac{1}{1.00}=2.00$ }};
\end{tikzpicture}
\column{0.55\textwidth}
\lstset{basicstyle=\tiny}
\begin{lstlisting}[escapechar=\|,label=noint]
#include <stdio.h>
#include <math.h>

#define LIMITE   pow(10,-9)
int  main(void)
{
int		factorial = 1,contador =1,i;    
double	nEuler=0,nEulerOld=-1;

for(i=0;(nEuler-nEulerOld)>= LIMITE;i++)
  }
     nEulerOld=nEuler;
     while(contador <= i)
     { 
     	factorial = factorial * contador;
     	contador++;
     }
     nEuler = nEuler + (1/(double)factorial); 
        
     factorial = 1;
     contador =1;
  }
  printf("e es %0.10f \n",nEuler);
  printf("e(lib math.h)es %0.10f\n",M_E);
  return 0;
}
\end{lstlisting}
\end{columns}
\end{frame}

\begin{frame}[fragile]
\fontsize{6.5pt}{10pt}\selectfont
\frametitle{Recorriendo el programa {\color{yellow}paso} a {\color{yellow}paso}}
\begin{columns}[c]
\column{0.5\textwidth}
\begin{tikzpicture}
	\begin{scope}[every node/.style={draw,anchor=text,rectangle split,rectangle split parts=10, 
rectangle split part fill={green!0,blue!0,blue!0,blue!0,blue!0,blue!0,blue!0,blue!0,blue!0,red!0,red!0},minimum width=2.6cm}]
    	\node (R) at (2,6)
    	{
    		\nodepart{one}{}
    		\nodepart{two}{\color{red}1}
    		\nodepart{three}{\color{red}1}	
    		\nodepart{four}{1}
    		\nodepart{five}{2.00}
    		\nodepart{six}{}
    		\nodepart{seven}{1.00}
    		\nodepart{eight}{}
    		\nodepart{nine}{$\vdots$}
    		\nodepart{ten}{}
    	};
	\end{scope}
	\draw [ultra thick,color=white](0.8, 4.6) -- (3.37, 4.6);
	\draw [ultra thick,color=white](0.8, 3.92) -- (3.37, 3.92);
	\fill[color=green] (5.6,4.6) -- (8,4.6) -- (8,4.1) -- (5.6,4.1) -- cycle;
  	\node at (2.1,9) {\normalsize Arquitectura X86-\textcolor{red}{32} bits };
  	\node at (2.1,7.9) {\small $inicializo$ $nuevamente$ $las$};
  	\node at (2.1,7.5) {\small $variables$ $\color{red}factorial$ y $\color{red}contador$};
  	\node at (8,9.7) {\normalsize {\color{red!50}$C\acute{o}digo$ $del$ $programa$ $fuente$}};
  	\node at (4,5.75) {\textcolor{blue}{factorial}};
  	\node at (4,5.4){\textcolor{blue}{contador}};
  	\node at (3.6,5.05) {\textcolor{blue}{i}};
  	\node at (3.9,4.6) {\textcolor{blue}{nEuler}};
  	\node at (4.05,4) {\textcolor{blue}{nEulerOld}};
  	\node at (0.1,6.1) {0x3F00480\textcolor{red}{0}};
  	\node at (0.1,5.75) {0x3F00480\textcolor{red}{4}};
  	\node at (0.1,5.4) {0x3F00480\textcolor{red}{8}};
  	\node at (0.1,5.05) {0x3F00480\textcolor{red}{C}};
  	\node at (0.1,4.7) {0x3F0048\textcolor{red}{10}};
  	\node at (0.1,4.37) {0x3F0048\textcolor{red}{14}};
  	\node at (0.1,4.03) {0x3F0048\textcolor{red}{18}};
  	\node at (0.1,3.67) {0x3F0048\textcolor{red}{1C}};
  	\draw[decorate,decoration={brace,mirror,raise=6pt,amplitude=8pt}, thick]
    (3.3,6.1) -- (0.9,6.1)node [black,midway,xshift=0.0cm,yshift=0.7cm]{\footnotesize {\color{red}$4$ $bytes$}};
\end{tikzpicture}
\column{0.55\textwidth}
\lstset{basicstyle=\tiny}
\begin{lstlisting}[escapechar=\|,label=noint]
#include <stdio.h>
#include <math.h>

#define LIMITE   pow(10,-9)
int  main(void)
{
int		factorial = 1,contador =1,i;    
double	nEuler=0,nEulerOld=-1;

for(i=0;(nEuler-nEulerOld)>= LIMITE;i++)
  }
     nEulerOld=nEuler;
     while(contador <= i)
     { 
     	factorial = factorial * contador;
     	contador++;
     }
     nEuler = nEuler + (1/(double)factorial); 
        
     factorial = 1;
     contador =1;
  }
  printf("e es %0.10f \n",nEuler);
  printf("e(lib math.h)es %0.10f\n",M_E);
  return 0;
}
\end{lstlisting}
\end{columns}
\end{frame}

\begin{frame}[fragile]
\fontsize{6.5pt}{10pt}\selectfont
\frametitle{Recorriendo el programa {\color{yellow}paso} a {\color{yellow}paso}}
\begin{columns}[c]
\column{0.5\textwidth}
\begin{tikzpicture}
	\begin{scope}[every node/.style={draw,anchor=text,rectangle split,rectangle split parts=10, 
rectangle split part fill={green!0,blue!0,blue!0,blue!0,blue!0,blue!0,blue!0,blue!0,blue!0,red!0,red!0},minimum width=2.6cm}]
    	\node (R) at (2,6)
    	{
    		\nodepart{one}{}
    		\nodepart{two}{1}
    		\nodepart{three}{1}	
    		\nodepart{four}{\color{red}2}
    		\nodepart{five}{2.00}
    		\nodepart{six}{}
    		\nodepart{seven}{1.00}
    		\nodepart{eight}{}
    		\nodepart{nine}{$\vdots$}
    		\nodepart{ten}{}
    	};
	\end{scope}
	\draw [ultra thick,color=white](0.8, 4.6) -- (3.37, 4.6);
	\draw [ultra thick,color=white](0.8, 3.92) -- (3.37, 3.92);
	\fill[color=green] (9.7,7.2) -- (10.3,7.2) -- (10.3,6.8) -- (9.7,6.8) -- cycle;
  	\node at (2.1,9) {\normalsize Arquitectura X86-\textcolor{red}{32} bits };
  	\node at (2.1,7.9) {\small $incremento$ $la$ $variable$ $\color{red}i$};
  	\node at (8,9.7) {\normalsize {\color{red!50}$C\acute{o}digo$ $del$ $programa$ $fuente$}};
  	\node at (4,5.75) {\textcolor{blue}{factorial}};
  	\node at (4,5.4){\textcolor{blue}{contador}};
  	\node at (3.6,5.05) {\textcolor{blue}{i}};
  	\node at (3.9,4.6) {\textcolor{blue}{nEuler}};
  	\node at (4.05,4) {\textcolor{blue}{nEulerOld}};
  	\node at (0.1,6.1) {0x3F00480\textcolor{red}{0}};
  	\node at (0.1,5.75) {0x3F00480\textcolor{red}{4}};
  	\node at (0.1,5.4) {0x3F00480\textcolor{red}{8}};
  	\node at (0.1,5.05) {0x3F00480\textcolor{red}{C}};
  	\node at (0.1,4.7) {0x3F0048\textcolor{red}{10}};
  	\node at (0.1,4.37) {0x3F0048\textcolor{red}{14}};
  	\node at (0.1,4.03) {0x3F0048\textcolor{red}{18}};
  	\node at (0.1,3.67) {0x3F0048\textcolor{red}{1C}};
  	\draw[decorate,decoration={brace,mirror,raise=6pt,amplitude=8pt}, thick]
    (3.3,6.1) -- (0.9,6.1)node [black,midway,xshift=0.0cm,yshift=0.7cm]{\footnotesize {\color{red}$4$ $bytes$}};
\end{tikzpicture}
\column{0.55\textwidth}
\lstset{basicstyle=\tiny}
\begin{lstlisting}[escapechar=\|,label=noint]
#include <stdio.h>
#include <math.h>

#define LIMITE   pow(10,-9)
int  main(void)
{
int		factorial = 1,contador =1,i;    
double	nEuler=0,nEulerOld=-1;

for(i=0;(nEuler-nEulerOld)>= LIMITE;i++)
  }
     nEulerOld=nEuler;
     while(contador <= i)
     { 
     	factorial = factorial * contador;
     	contador++;
     }
     nEuler = nEuler + (1/(double)factorial); 
        
     factorial = 1;
     contador =1;
  }
  printf("e es %0.10f \n",nEuler);
  printf("e(lib math.h)es %0.10f\n",M_E);
  return 0;
}
\end{lstlisting}
\end{columns}
\end{frame}

\begin{frame}[fragile]
\fontsize{6.5pt}{10pt}\selectfont
\frametitle{Recorriendo el programa {\color{yellow}paso} a {\color{yellow}paso}}
\begin{columns}[c]
\column{0.5\textwidth}
\begin{tikzpicture}
	\begin{scope}[every node/.style={draw,anchor=text,rectangle split,rectangle split parts=10, 
rectangle split part fill={green!0,blue!0,blue!0,blue!0,blue!0,blue!0,blue!0,blue!0,blue!0,red!0,red!0},minimum width=2.6cm}]
    	\node (R) at (2,6)
    	{
    		\nodepart{one}{}
    		\nodepart{two}{1}
    		\nodepart{three}{1}	
    		\nodepart{four}{2}
    		\nodepart{five}{2.00}
    		\nodepart{six}{}
    		\nodepart{seven}{1.00}
    		\nodepart{eight}{}
    		\nodepart{nine}{$\vdots$}
    		\nodepart{ten}{}
    	};
	\end{scope}
	\draw [ultra thick,color=white](0.8, 4.6) -- (3.37, 4.6);
	\draw [ultra thick,color=white](0.8, 3.92) -- (3.37, 3.92);
	\fill[color=green] (6.4,7.2) -- (9.7,7.2) -- (9.7,6.8) -- (6.4,6.8) -- cycle;
  	\node at (2.1,9) {\normalsize Arquitectura X86-\textcolor{red}{32} bits };
  	\node at (2.1,7.9) {\small $Pregunta$ $condici\acute{o}n$ $del$ $\color{red}for$};
  	\node at (8,9.7) {\normalsize {\color{red!50}$C\acute{o}digo$ $del$ $programa$ $fuente$}};
  	\node at (4,5.75) {\textcolor{blue}{factorial}};
  	\node at (4,5.4){\textcolor{blue}{contador}};
  	\node at (3.6,5.05) {\textcolor{blue}{i}};
  	\node at (3.9,4.6) {\textcolor{blue}{nEuler}};
  	\node at (4.05,4) {\textcolor{blue}{nEulerOld}};
	\node at (0.1,6.1) {0x3F00480\textcolor{red}{0}};
  	\node at (0.1,5.75) {0x3F00480\textcolor{red}{4}};
  	\node at (0.1,5.4) {0x3F00480\textcolor{red}{8}};
  	\node at (0.1,5.05) {0x3F00480\textcolor{red}{C}};
  	\node at (0.1,4.7) {0x3F0048\textcolor{red}{10}};
  	\node at (0.1,4.37) {0x3F0048\textcolor{red}{14}};
  	\node at (0.1,4.03) {0x3F0048\textcolor{red}{18}};
  	\node at (0.1,3.67) {0x3F0048\textcolor{red}{1C}};
  	\draw[decorate,decoration={brace,mirror,raise=6pt,amplitude=8pt}, thick]
    (3.3,6.1) -- (0.9,6.1)node [black,midway,xshift=0.0cm,yshift=0.7cm]{\footnotesize {\color{red}$4$ $bytes$}};
  	\draw[decorate,decoration={brace,raise=4pt,amplitude=4pt}, thick]
    (7.35,8.4)--(8.7,8.4)node [black,midway,xshift=1.0cm,yshift=0.4cm]{\footnotesize {\color{red!60}$10^{-9}$}};
  	\draw[red,->](9.3,7.1)to [out=10,in=0]node[right,midway]{} ++(-0.2,1.3) ; 
\end{tikzpicture}
\column{0.55\textwidth}
\lstset{basicstyle=\tiny}
\begin{lstlisting}[escapechar=\|,label=noint]
#include <stdio.h>
#include <math.h>

#define LIMITE   pow(10,-9)
int  main(void)
{
int		factorial = 1,contador =1,i;    
double	nEuler=0,nEulerOld=-1;

for(i=0;(nEuler-nEulerOld)>= LIMITE;i++)
  }
     nEulerOld=nEuler;
     while(contador <= i)
     { 
     	factorial = factorial * contador;
     	contador++;
     }
     nEuler = nEuler + (1/(double)factorial); 
        
     factorial = 1;
     contador =1;
  }
  printf("e es %0.10f \n",nEuler);
  printf("e(lib math.h)es %0.10f\n",M_E);
  return 0;
}
\end{lstlisting}
\end{columns}
\end{frame}

\begin{frame}[fragile]
\fontsize{6.5pt}{10pt}\selectfont
\frametitle{Recorriendo el programa {\color{yellow}paso} a {\color{yellow}paso}}
\begin{columns}[c]
\column{0.5\textwidth}
\begin{tikzpicture}
	\begin{scope}[every node/.style={draw,anchor=text,rectangle split,rectangle split parts=10, 
rectangle split part fill={green!0,blue!0,blue!0,blue!0,blue!0,blue!0,blue!0,blue!0,blue!0,red!0,red!0},minimum width=2.6cm}]
    	\node (R) at (2,6)
    	{
    		\nodepart{one}{}
    		\nodepart{two}{1}
    		\nodepart{three}{1}	
    		\nodepart{four}{2}
    		\nodepart{five}{2.00}
    		\nodepart{six}{}
    		\nodepart{seven}{\color{red}2.00}
    		\nodepart{eight}{}
    		\nodepart{nine}{$\vdots$}
    		\nodepart{ten}{}
    	};
	\end{scope}
	\draw [ultra thick,color=white](0.8, 4.6) -- (3.37, 4.6);
	\draw [ultra thick,color=white](0.8, 3.92) -- (3.37, 3.92);
	\fill[color=green] (5.8,6.65) -- (8,6.65) -- (8,6.35) -- (5.8,6.35) -- cycle;
	\node at (2.1,9) {\normalsize Arquitectura X86-\textcolor{red}{32} bits };
  	\node at (2.1,7.9) {\small $Para$ $conservar$ $el$ $\color{red}e$ $anterior$};
  	\node at (8,9.7) {\normalsize {\color{red!50}$C\acute{o}digo$ $del$ $programa$ $fuente$}};
  	\node at (4,5.75) {\textcolor{blue}{factorial}};
  	\node at (4,5.4){\textcolor{blue}{contador}};
  	\node at (3.6,5.05) {\textcolor{blue}{i}};
  	\node at (3.9,4.6) {\textcolor{blue}{nEuler}};
  	\node at (4.05,4) {\textcolor{blue}{nEulerOld}};
  	\node at (0.1,6.1) {0x3F00480\textcolor{red}{0}};
  	\node at (0.1,5.75) {0x3F00480\textcolor{red}{4}};
  	\node at (0.1,5.4) {0x3F00480\textcolor{red}{8}};
  	\node at (0.1,5.05) {0x3F00480\textcolor{red}{C}};
  	\node at (0.1,4.7) {0x3F0048\textcolor{red}{10}};
  	\node at (0.1,4.37) {0x3F0048\textcolor{red}{14}};
  	\node at (0.1,4.03) {0x3F0048\textcolor{red}{18}};
  	\node at (0.1,3.67) {0x3F0048\textcolor{red}{1C}};
  	\draw[decorate,decoration={brace,mirror,raise=6pt,amplitude=8pt}, thick]
    (3.3,6.1) -- (0.9,6.1)node [black,midway,xshift=0.0cm,yshift=0.7cm]{\footnotesize {\color{red}$4$ $bytes$}};
\end{tikzpicture}
\column{0.55\textwidth}
\lstset{basicstyle=\tiny}
\begin{lstlisting}[escapechar=\|,label=noint]
#include <stdio.h>
#include <math.h>

#define LIMITE   pow(10,-9)
int  main(void)
{
int		factorial = 1,contador =1,i;    
double	nEuler=0,nEulerOld=-1;

for(i=0;(nEuler-nEulerOld)>= LIMITE;i++)
  }
     nEulerOld=nEuler;
     while(contador <= i)
     { 
     	factorial = factorial * contador;
     	contador++;
     }
     nEuler = nEuler + (1/(double)factorial); 
        
     factorial = 1;
     contador =1;
  }
  printf("e es %0.10f \n",nEuler);
  printf("e(lib math.h)es %0.10f\n",M_E);
  return 0;
}
\end{lstlisting}
\end{columns}
\end{frame}

\begin{frame}[fragile]
\fontsize{6.5pt}{10pt}\selectfont
\frametitle{Recorriendo el programa {\color{yellow}paso} a {\color{yellow}paso}}
\begin{columns}[c]
\column{0.5\textwidth}
\begin{tikzpicture}
	\begin{scope}[every node/.style={draw,anchor=text,rectangle split,rectangle split parts=10, 
rectangle split part fill={green!0,blue!0,blue!0,blue!0,blue!0,blue!0,blue!0,blue!0,blue!0,red!0,red!0},minimum width=2.6cm}]
    	\node (R) at (2,6)
    	{
    		\nodepart{one}{}
    		\nodepart{two}{1}
    		\nodepart{three}{1}	
    		\nodepart{four}{2}
    		\nodepart{five}{2.00}
    		\nodepart{six}{}
    		\nodepart{seven}{2.00}
    		\nodepart{eight}{}
    		\nodepart{nine}{$\vdots$}
    		\nodepart{ten}{}
    	};
	\end{scope}
	\draw [ultra thick,color=white](0.8, 4.6) -- (3.37, 4.6);
	\draw [ultra thick,color=white](0.8, 3.92) -- (3.37, 3.92);
	\fill[color=green] (5.8,6.4) -- (8.4,6.4) -- (8.4,6) -- (5.8,6) -- cycle;
	\node at (2.1,7.9) {\small $Como$ $\color{red}contador$ $es$ $\color{red}<=$ $a$ $\color{red}i$};
	\node at (2.1,7.5) {\small $entra$ $dentro$ $del$ $\color{red}while$};
	\node at (2.1,9) {\normalsize Arquitectura X86-\textcolor{red}{32} bits };
    \node at (8,9.7) {\normalsize {\color{red!50}$C\acute{o}digo$ $del$ $programa$ $fuente$}};
  	\node at (4,5.75) {\textcolor{blue}{factorial}};
  	\node at (4,5.4){\textcolor{blue}{contador}};
  	\node at (3.6,5.05) {\textcolor{blue}{i}};
  	\node at (3.9,4.6) {\textcolor{blue}{nEuler}};
  	\node at (4.05,4) {\textcolor{blue}{nEulerOld}};
  	\node at (0.1,6.1) {0x3F00480\textcolor{red}{0}};
  	\node at (0.1,5.75) {0x3F00480\textcolor{red}{4}};
  	\node at (0.1,5.4) {0x3F00480\textcolor{red}{8}};
  	\node at (0.1,5.05) {0x3F00480\textcolor{red}{C}};
  	\node at (0.1,4.7) {0x3F0048\textcolor{red}{10}};
  	\node at (0.1,4.37) {0x3F0048\textcolor{red}{14}};
  	\node at (0.1,4.03) {0x3F0048\textcolor{red}{18}};
  	\node at (0.1,3.67) {0x3F0048\textcolor{red}{1C}};
	\draw[decorate,decoration={brace,mirror,raise=6pt,amplitude=8pt}, thick]
    (3.3,6.1) -- (0.9,6.1)node [black,midway,xshift=0.0cm,yshift=0.7cm]{\footnotesize {\color{red}$4$ $bytes$}};
\end{tikzpicture}
\column{0.55\textwidth}
\lstset{basicstyle=\tiny}
\begin{lstlisting}[escapechar=\|,label=noint]
#include <stdio.h>
#include <math.h>

#define LIMITE   pow(10,-9)
int  main(void)
{
int		factorial = 1,contador =1,i;    
double	nEuler=0,nEulerOld=-1;

for(i=0;(nEuler-nEulerOld)>= LIMITE;i++)
  }
     nEulerOld=nEuler;
     while(contador <= i)
     { 
     	factorial = factorial * contador;
     	contador++;
     }
     nEuler = nEuler + (1/(double)factorial); 
        
     factorial = 1;
     contador =1;
  }
  printf("e es %0.10f \n",nEuler);
  printf("e(lib math.h)es %0.10f\n",M_E);
  return 0;
}
\end{lstlisting}
\end{columns}
\end{frame}

\begin{frame}[fragile]
\fontsize{6.5pt}{10pt}\selectfont
\frametitle{Recorriendo el programa {\color{yellow}paso} a {\color{yellow}paso}}
\begin{columns}[c]
\column{0.5\textwidth}
\begin{tikzpicture}
	\begin{scope}[every node/.style={draw,anchor=text,rectangle split,rectangle split parts=10, 
rectangle split part fill={green!0,blue!0,blue!0,blue!0,blue!0,blue!0,blue!0,blue!0,blue!0,red!0,red!0},minimum width=2.6cm}]
    	\node (R) at (2,6)
    	{
    		\nodepart{one}{}
    		\nodepart{two}{\color{red}1}
    		\nodepart{three}{1}	
    		\nodepart{four}{2}
    		\nodepart{five}{2.00}
    		\nodepart{six}{}
    		\nodepart{seven}{2.00}
    		\nodepart{eight}{}
    		\nodepart{nine}{$\vdots$}
    		\nodepart{ten}{}
    	};
	\end{scope}
	\draw [ultra thick,color=white](0.8, 4.6) -- (3.37, 4.6);
	\draw [ultra thick,color=white](0.8, 3.92) -- (3.37, 3.92);
	\fill[color=green] (5.8,5.9) -- (10.5,5.9) -- (10.5,5.6) -- (5.8,5.6) -- cycle;
  	\node at (2.1,7.9) {\small $almacena$ $el$ $resultado$ $en$ $la$};
  	\node at (2.1,7.5) {\small $variable$ $\color{red}factorial$};
  	\node at (2.1,9) {\normalsize Arquitectura X86-\textcolor{red}{32} bits };
  	\node at (8,9.7) {\normalsize {\color{red!50}$C\acute{o}digo$ $del$ $programa$ $fuente$}};
  	\node at (4,5.75) {\textcolor{blue}{factorial}};
  	\node at (4,5.4){\textcolor{blue}{contador}};
  	\node at (3.6,5.05) {\textcolor{blue}{i}};
  	\node at (3.9,4.6) {\textcolor{blue}{nEuler}};
  	\node at (4.05,4) {\textcolor{blue}{nEulerOld}};
  	\node at (0.1,6.1) {0x3F00480\textcolor{red}{0}};
  	\node at (0.1,5.75) {0x3F00480\textcolor{red}{4}};
  	\node at (0.1,5.4) {0x3F00480\textcolor{red}{8}};
  	\node at (0.1,5.05) {0x3F00480\textcolor{red}{C}};
  	\node at (0.1,4.7) {0x3F0048\textcolor{red}{10}};
  	\node at (0.1,4.37) {0x3F0048\textcolor{red}{14}};
  	\node at (0.1,4.03) {0x3F0048\textcolor{red}{18}};
  	\node at (0.1,3.67) {0x3F0048\textcolor{red}{1C}};
  	\draw[decorate,decoration={brace,mirror,raise=6pt,amplitude=8pt}, thick]
    (3.3,6.1) -- (0.9,6.1)node [black,midway,xshift=0.0cm,yshift=0.7cm]{\footnotesize {\color{red}$4$ $bytes$}};
  	\draw[decorate,decoration={brace,raise=3pt,amplitude=3pt}, thick]
    (7.5,5.8)--(10.2,5.8)node [black,midway,xshift=1.5cm,yshift=0.4cm]{\footnotesize{\color{red}$1\cdot 1=1$ }};
\end{tikzpicture}
\column{0.55\textwidth}
\lstset{basicstyle=\tiny}
\begin{lstlisting}[escapechar=\|,label=noint]
#include <stdio.h>
#include <math.h>

#define LIMITE   pow(10,-9)
int  main(void)
{
int		factorial = 1,contador =1,i;    
double	nEuler=0,nEulerOld=-1;

for(i=0;(nEuler-nEulerOld)>= LIMITE;i++)
  }
     nEulerOld=nEuler;
     while(contador <= i)
     { 
     	factorial = factorial * contador;
     	contador++;
     }
     nEuler = nEuler + (1/(double)factorial); 
        
     factorial = 1;
     contador =1;
  }
  printf("e es %0.10f \n",nEuler);
  printf("e(lib math.h)es %0.10f\n",M_E);
  return 0;
}
\end{lstlisting}
\end{columns}
\end{frame}

\begin{frame}[fragile]
\fontsize{6.5pt}{10pt}\selectfont
\frametitle{Recorriendo el programa {\color{yellow}paso} a {\color{yellow}paso}}
\begin{columns}[c]
\column{0.5\textwidth}
\begin{tikzpicture}
	\begin{scope}[every node/.style={draw,anchor=text,rectangle split,rectangle split parts=10, 
rectangle split part fill={green!0,blue!0,blue!0,blue!0,blue!0,blue!0,blue!0,blue!0,blue!0,red!0,red!0},minimum width=2.6cm}]
    	\node (R) at (2,6)
    	{
    		\nodepart{one}{}
    		\nodepart{two}{1}
    		\nodepart{three}{\color{red}2}	
    		\nodepart{four}{2}
    		\nodepart{five}{2.00}
    		\nodepart{six}{}
    		\nodepart{seven}{2.00}
    		\nodepart{eight}{}
    		\nodepart{nine}{$\vdots$}
    		\nodepart{ten}{}
    	};
	\end{scope}
	\draw [ultra thick,color=white](0.8, 4.6) -- (3.37, 4.6);
	\draw [ultra thick,color=white](0.8, 3.92) -- (3.37, 3.92);
	\fill[color=green] (5.8,5.6) -- (8,5.6) -- (8,5.3) -- (5.8,5.3) -- cycle;
  	\node at (2.1,7.9) {\small $incremento$ $la$ $variable$ $\color{red}contador$};
  	\node at (2.1,9) {\normalsize Arquitectura X86-\textcolor{red}{32} bits };
  	\node at (8,9.7) {\normalsize {\color{red!50}$C\acute{o}digo$ $del$ $programa$ $fuente$}};
  	\node at (4,5.75) {\textcolor{blue}{factorial}};
  	\node at (4,5.4){\textcolor{blue}{contador}};
  	\node at (3.6,5.05) {\textcolor{blue}{i}};
  	\node at (3.9,4.6) {\textcolor{blue}{nEuler}};
  	\node at (4.05,4) {\textcolor{blue}{nEulerOld}};
  	\node at (0.1,6.1) {0x3F00480\textcolor{red}{0}};
  	\node at (0.1,5.75) {0x3F00480\textcolor{red}{4}};
  	\node at (0.1,5.4) {0x3F00480\textcolor{red}{8}};
  	\node at (0.1,5.05) {0x3F00480\textcolor{red}{C}};
  	\node at (0.1,4.7) {0x3F0048\textcolor{red}{10}};
  	\node at (0.1,4.37) {0x3F0048\textcolor{red}{14}};
  	\node at (0.1,4.03) {0x3F0048\textcolor{red}{18}};
  	\node at (0.1,3.67) {0x3F0048\textcolor{red}{1C}};
  	\draw[decorate,decoration={brace,mirror,raise=6pt,amplitude=8pt}, thick]
    (3.3,6.1) -- (0.9,6.1)node [black,midway,xshift=0.0cm,yshift=0.7cm]{\footnotesize {\color{red}$4$ $bytes$}};
\end{tikzpicture}
\column{0.55\textwidth}
\lstset{basicstyle=\tiny}
\begin{lstlisting}[escapechar=\|,label=noint]
#include <stdio.h>
#include <math.h>

#define LIMITE   pow(10,-9)
int  main(void)
{
int		factorial = 1,contador =1,i;    
double	nEuler=0,nEulerOld=-1;

for(i=0;(nEuler-nEulerOld)>= LIMITE;i++)
  }
     nEulerOld=nEuler;
     while(contador <= i)
     { 
     	factorial = factorial * contador;
     	contador++;
     }
     nEuler = nEuler + (1/(double)factorial); 
        
     factorial = 1;
     contador =1;
  }
  printf("e es %0.10f \n",nEuler);
  printf("e(lib math.h)es %0.10f\n",M_E);
  return 0;
}
\end{lstlisting}
\end{columns}
\end{frame}

\begin{frame}[fragile]
\fontsize{6.5pt}{10pt}\selectfont
\frametitle{Recorriendo el programa {\color{yellow}paso} a {\color{yellow}paso}}
\begin{columns}[c]
\column{0.5\textwidth}
\begin{tikzpicture}
	\begin{scope}[every node/.style={draw,anchor=text,rectangle split,rectangle split parts=10, 
rectangle split part fill={green!0,blue!0,blue!0,blue!0,blue!0,blue!0,blue!0,blue!0,blue!0,red!0,red!0},minimum width=2.6cm}]
    	\node (R) at (2,6)
    	{
    		\nodepart{one}{}
    		\nodepart{two}{1}
    		\nodepart{three}{2}	
    		\nodepart{four}{2}
    		\nodepart{five}{2.00}
    		\nodepart{six}{}
    		\nodepart{seven}{2.00}
    		\nodepart{eight}{}
    		\nodepart{nine}{$\vdots$}
    		\nodepart{ten}{}
    	};
	\end{scope}
	\draw [ultra thick,color=white](0.8, 4.6) -- (3.37, 4.6);
	\draw [ultra thick,color=white](0.8, 3.92) -- (3.37, 3.92);
	\fill[color=green] (5.8,6.4) -- (8.4,6.4) -- (8.4,6) -- (5.8,6) -- cycle;
  	\node at (2.1,7.9) {\small $Como$ $\color{red}contador$ $es$ $\color{red}<=$ $a$ $\color{red}i$};
  	\node at (2.1,7.5) {\small $entra$ $dentro$ $del$ $\color{red}while$};
  	\node at (2.1,9) {\normalsize Arquitectura X86-\textcolor{red}{32} bits };
  	\node at (8,9.7) {\normalsize {\color{red!50}$C\acute{o}digo$ $del$ $programa$ $fuente$}};
  	\node at (4,5.75) {\textcolor{blue}{factorial}};
  	\node at (4,5.4){\textcolor{blue}{contador}};
  	\node at (3.6,5.05) {\textcolor{blue}{i}};
  	\node at (3.9,4.6) {\textcolor{blue}{nEuler}};
  	\node at (4.05,4) {\textcolor{blue}{nEulerOld}};
  	\node at (0.1,6.1) {0x3F00480\textcolor{red}{0}};
  	\node at (0.1,5.75) {0x3F00480\textcolor{red}{4}};
  	\node at (0.1,5.4) {0x3F00480\textcolor{red}{8}};
  	\node at (0.1,5.05) {0x3F00480\textcolor{red}{C}};
  	\node at (0.1,4.7) {0x3F0048\textcolor{red}{10}};
  	\node at (0.1,4.37) {0x3F0048\textcolor{red}{14}};
  	\node at (0.1,4.03) {0x3F0048\textcolor{red}{18}};
  	\node at (0.1,3.67) {0x3F0048\textcolor{red}{1C}};
  	\draw[decorate,decoration={brace,mirror,raise=6pt,amplitude=8pt}, thick]
    (3.3,6.1) -- (0.9,6.1)node [black,midway,xshift=0.0cm,yshift=0.7cm]{\footnotesize {\color{red}$4$ $bytes$}};
\end{tikzpicture}
\column{0.55\textwidth}
\lstset{basicstyle=\tiny}
\begin{lstlisting}[escapechar=\|,label=noint]
#include <stdio.h>
#include <math.h>

#define LIMITE   pow(10,-9)
int  main(void)
{
int		factorial = 1,contador =1,i;    
double	nEuler=0,nEulerOld=-1;

for(i=0;(nEuler-nEulerOld)>= LIMITE;i++)
  }
     nEulerOld=nEuler;
     while(contador <= i)
     { 
     	factorial = factorial * contador;
     	contador++;
     }
     nEuler = nEuler + (1/(double)factorial); 
        
     factorial = 1;
     contador =1;
  }
  printf("e es %0.10f \n",nEuler);
  printf("e(lib math.h)es %0.10f\n",M_E);
  return 0;
}
\end{lstlisting}
\end{columns}
\end{frame}

\begin{frame}[fragile]
\fontsize{6.5pt}{10pt}\selectfont
\frametitle{Recorriendo el programa {\color{yellow}paso} a {\color{yellow}paso}}
\begin{columns}[c]
\column{0.5\textwidth}
\begin{tikzpicture}
	\begin{scope}[every node/.style={draw,anchor=text,rectangle split,rectangle split parts=10, 
rectangle split part fill={green!0,blue!0,blue!0,blue!0,blue!0,blue!0,blue!0,blue!0,blue!0,red!0,red!0},minimum width=2.6cm}]
    	\node (R) at (2,6)
    	{
    		\nodepart{one}{}
    		\nodepart{two}{\color{red}2}
    		\nodepart{three}{2}	
    		\nodepart{four}{2}
    		\nodepart{five}{2.00}
    		\nodepart{six}{}
    		\nodepart{seven}{2.00}
    		\nodepart{eight}{}
    		\nodepart{nine}{$\vdots$}
    		\nodepart{ten}{}
    	};
	\end{scope}
	\draw [ultra thick,color=white](0.8, 4.6) -- (3.37, 4.6);
	\draw [ultra thick,color=white](0.8, 3.92) -- (3.37, 3.92);
	\fill[color=green] (5.8,5.9) -- (10.5,5.9) -- (10.5,5.6) -- (5.8,5.6) -- cycle;
  	\node at (2.1,7.9) {\small $almacena$ $el$ $resultado$ $en$ $la$};
  	\node at (2.1,7.5) {\small $variable$ $\color{red}factorial$};
  	\node at (2.1,9) {\normalsize Arquitectura X86-\textcolor{red}{32} bits };
  	\node at (8,9.7) {\normalsize {\color{red!50}$C\acute{o}digo$ $del$ $programa$ $fuente$}};
  	\node at (4,5.75) {\textcolor{blue}{factorial}};
  	\node at (4,5.4){\textcolor{blue}{contador}};
  	\node at (3.6,5.05) {\textcolor{blue}{i}};
  	\node at (3.9,4.6) {\textcolor{blue}{nEuler}};
  	\node at (4.05,4) {\textcolor{blue}{nEulerOld}};
  	\node at (0.1,6.1) {0x3F00480\textcolor{red}{0}};
  	\node at (0.1,5.75) {0x3F00480\textcolor{red}{4}};
  	\node at (0.1,5.4) {0x3F00480\textcolor{red}{8}};
  	\node at (0.1,5.05) {0x3F00480\textcolor{red}{C}};
  	\node at (0.1,4.7) {0x3F0048\textcolor{red}{10}};
  	\node at (0.1,4.37) {0x3F0048\textcolor{red}{14}};
  	\node at (0.1,4.03) {0x3F0048\textcolor{red}{18}};
  	\node at (0.1,3.67) {0x3F0048\textcolor{red}{1C}};
  	\draw[decorate,decoration={brace,mirror,raise=6pt,amplitude=8pt}, thick]
    (3.3,6.1) -- (0.9,6.1)node [black,midway,xshift=0.0cm,yshift=0.7cm]{\footnotesize {\color{red}$4$ $bytes$}};
  	\draw[decorate,decoration={brace,raise=3pt,amplitude=3pt}, thick]
    (7.5,5.8)--(10.2,5.8)node [black,midway,xshift=1.5cm,yshift=0.4cm]{\footnotesize{\color{red}$1\cdot 2=2$ }};
\end{tikzpicture}
\column{0.55\textwidth}
\lstset{basicstyle=\tiny}
\begin{lstlisting}[escapechar=\|,label=noint]
#include <stdio.h>
#include <math.h>

#define LIMITE   pow(10,-9)
int  main(void)
{
int		factorial = 1,contador =1,i;    
double	nEuler=0,nEulerOld=-1;

for(i=0;(nEuler-nEulerOld)>= LIMITE;i++)
  }
     nEulerOld=nEuler;
     while(contador <= i)
     { 
     	factorial = factorial * contador;
     	contador++;
     }
     nEuler = nEuler + (1/(double)factorial); 
        
     factorial = 1;
     contador =1;
  }
  printf("e es %0.10f \n",nEuler);
  printf("e(lib math.h)es %0.10f\n",M_E);
  return 0;
}
\end{lstlisting}
\end{columns}
\end{frame}

\begin{frame}[fragile]
\fontsize{6.5pt}{10pt}\selectfont
\frametitle{Recorriendo el programa {\color{yellow}paso} a {\color{yellow}paso}}
\begin{columns}[c]
\column{0.5\textwidth}
\begin{tikzpicture}
	\begin{scope}[every node/.style={draw,anchor=text,rectangle split,rectangle split parts=10, 
rectangle split part fill={green!0,blue!0,blue!0,blue!0,blue!0,blue!0,blue!0,blue!0,blue!0,red!0,red!0},minimum width=2.6cm}]
    	\node (R) at (2,6)
    	{
    		\nodepart{one}{}
    		\nodepart{two}{2}
    		\nodepart{three}{\color{red}3}	
    		\nodepart{four}{2}
    		\nodepart{five}{2.00}
    		\nodepart{six}{}
    		\nodepart{seven}{2.00}
    		\nodepart{eight}{}
    		\nodepart{nine}{$\vdots$}
    		\nodepart{ten}{}
    	};
	\end{scope}
	\draw [ultra thick,color=white](0.8, 4.6) -- (3.37, 4.6);
	\draw [ultra thick,color=white](0.8, 3.92) -- (3.37, 3.92);
	\fill[color=green] (5.8,5.6) -- (8,5.6) -- (8,5.3) -- (5.8,5.3) -- cycle;
  	\node at (2.1,7.9) {\small $incremento$ $la$ $variable$ $\color{red}contador$};
  	\node at (2.1,9) {\normalsize Arquitectura X86-\textcolor{red}{32} bits };
  	\node at (8,9.7) {\normalsize {\color{red!50}$C\acute{o}digo$ $del$ $programa$ $fuente$}};
  	\node at (4,5.75) {\textcolor{blue}{factorial}};
  	\node at (4,5.4){\textcolor{blue}{contador}};
  	\node at (3.6,5.05) {\textcolor{blue}{i}};
  	\node at (3.9,4.6) {\textcolor{blue}{nEuler}};
  	\node at (4.05,4) {\textcolor{blue}{nEulerOld}};
  	\node at (0.1,6.1) {0x3F00480\textcolor{red}{0}};
  	\node at (0.1,5.75) {0x3F00480\textcolor{red}{4}};
  	\node at (0.1,5.4) {0x3F00480\textcolor{red}{8}};
  	\node at (0.1,5.05) {0x3F00480\textcolor{red}{C}};
  	\node at (0.1,4.7) {0x3F0048\textcolor{red}{10}};
  	\node at (0.1,4.37) {0x3F0048\textcolor{red}{14}};
  	\node at (0.1,4.03) {0x3F0048\textcolor{red}{18}};
  	\node at (0.1,3.67) {0x3F0048\textcolor{red}{1C}};
  	\draw[decorate,decoration={brace,mirror,raise=6pt,amplitude=8pt}, thick]
    (3.3,6.1) -- (0.9,6.1)node [black,midway,xshift=0.0cm,yshift=0.7cm]{\footnotesize {\color{red}$4$ $bytes$}};
\end{tikzpicture}
\column{0.55\textwidth}
\lstset{basicstyle=\tiny}
\begin{lstlisting}[escapechar=\|,label=noint]
#include <stdio.h>
#include <math.h>

#define LIMITE   pow(10,-9)
int  main(void)
{
int		factorial = 1,contador =1,i;    
double	nEuler=0,nEulerOld=-1;

for(i=0;(nEuler-nEulerOld)>= LIMITE;i++)
  }
     nEulerOld=nEuler;
     while(contador <= i)
     { 
     	factorial = factorial * contador;
     	contador++;
     }
     nEuler = nEuler + (1/(double)factorial); 
        
     factorial = 1;
     contador =1;
  }
  printf("e es %0.10f \n",nEuler);
  printf("e(lib math.h)es %0.10f\n",M_E);
  return 0;
}
\end{lstlisting}
\end{columns}
\end{frame}

\begin{frame}[fragile]
\fontsize{6.5pt}{10pt}\selectfont
\frametitle{Recorriendo el programa {\color{yellow}paso} a {\color{yellow}paso}}
\begin{columns}[c]
\column{0.5\textwidth}
\begin{tikzpicture}
	\begin{scope}[every node/.style={draw,anchor=text,rectangle split,rectangle split parts=10, 
rectangle split part fill={green!0,blue!0,blue!0,blue!0,blue!0,blue!0,blue!0,blue!0,blue!0,red!0,red!0},minimum width=2.6cm}]
    	\node (R) at (2,6)
    	{
    		\nodepart{one}{}
    		\nodepart{two}{2}
    		\nodepart{three}{3}	
    		\nodepart{four}{2}
    		\nodepart{five}{2.00}
    		\nodepart{six}{}
    		\nodepart{seven}{2.00}
    		\nodepart{eight}{}
    		\nodepart{nine}{$\vdots$}
    		\nodepart{ten}{}
    	};
	\end{scope}
	\draw [ultra thick,color=white](0.8, 4.6) -- (3.37, 4.6);
	\draw [ultra thick,color=white](0.8, 3.92) -- (3.37, 3.92);
	\fill[color=green] (5.8,6.4) -- (8.4,6.4) -- (8.4,6) -- (5.8,6) -- cycle;
  	\node at (2.1,7.9) {\small $Como$ $\color{red}contador$ $no$ $es$ $\color{red}<=$ $a$ $\color{red}i$};
  	\node at (2.1,7.5) {\small $sale$ $del$ $loop$ $\color{red}while$};
  	\node at (2.1,9) {\normalsize Arquitectura X86-\textcolor{red}{32} bits };
  	\node at (8,9.7) {\normalsize {\color{red!50}$C\acute{o}digo$ $del$ $programa$ $fuente$}};
  	\node at (4,5.75) {\textcolor{blue}{factorial}};
  	\node at (4,5.4){\textcolor{blue}{contador}};
  	\node at (3.6,5.05) {\textcolor{blue}{i}};
  	\node at (3.9,4.6) {\textcolor{blue}{nEuler}};
  	\node at (4.05,4) {\textcolor{blue}{nEulerOld}};
  	\node at (0.1,6.1) {0x3F00480\textcolor{red}{0}};
  	\node at (0.1,5.75) {0x3F00480\textcolor{red}{4}};
  	\node at (0.1,5.4) {0x3F00480\textcolor{red}{8}};
  	\node at (0.1,5.05) {0x3F00480\textcolor{red}{C}};
  	\node at (0.1,4.7) {0x3F0048\textcolor{red}{10}};
  	\node at (0.1,4.37) {0x3F0048\textcolor{red}{14}};
  	\node at (0.1,4.03) {0x3F0048\textcolor{red}{18}};
  	\node at (0.1,3.67) {0x3F0048\textcolor{red}{1C}};
  	\draw[decorate,decoration={brace,mirror,raise=6pt,amplitude=8pt}, thick]
    (3.3,6.1) -- (0.9,6.1)node [black,midway,xshift=0.0cm,yshift=0.7cm]{\footnotesize {\color{red}$4$ $bytes$}};
\end{tikzpicture}
\column{0.55\textwidth}
\lstset{basicstyle=\tiny}
\begin{lstlisting}[escapechar=\|,label=noint]
#include <stdio.h>
#include <math.h>

#define LIMITE   pow(10,-9)
int  main(void)
{
int		factorial = 1,contador =1,i;    
double	nEuler=0,nEulerOld=-1;

for(i=0;(nEuler-nEulerOld)>= LIMITE;i++)
  }
     nEulerOld=nEuler;
     while(contador <= i)
     { 
     	factorial = factorial * contador;
     	contador++;
     }
     nEuler = nEuler + (1/(double)factorial); 
        
     factorial = 1;
     contador =1;
  }
  printf("e es %0.10f \n",nEuler);
  printf("e(lib math.h)es %0.10f\n",M_E);
  return 0;
}
\end{lstlisting}
\end{columns}
\end{frame}

\begin{frame}[fragile]
\fontsize{6.5pt}{10pt}\selectfont
\frametitle{Recorriendo el programa {\color{yellow}paso} a {\color{yellow}paso}}
\begin{columns}[c]
\column{0.5\textwidth}
\begin{tikzpicture}
	\begin{scope}[every node/.style={draw,anchor=text,rectangle split,rectangle split parts=10, 
rectangle split part fill={green!0,blue!0,blue!0,blue!0,blue!0,blue!0,blue!0,blue!0,blue!0,red!0,red!0},minimum width=2.6cm}]
    	\node (R) at (2,6)
    	{
    		\nodepart{one}{}
    		\nodepart{two}{2}
    		\nodepart{three}{3}	
    		\nodepart{four}{2}
    		\nodepart{five}{
    		\color{red}2.50}
    		\nodepart{six}{}
    		\nodepart{seven}{2.00}
    		\nodepart{eight}{}
    		\nodepart{nine}{$\vdots$}
    		\nodepart{ten}{}
    	};
	\end{scope}
	\draw [ultra thick,color=white](0.8, 4.6) -- (3.37, 4.6);
	\draw [ultra thick,color=white](0.8, 3.92) -- (3.37, 3.92);
	\fill[color=green] (5.6,5.2) -- (11,5.2) -- (11,4.8) -- (5.6,4.8) -- cycle;
  	\node at (2.1,7.9) {\small $almacena$ $el$ $resultado$ $en$ $la$};
  	\node at (2.1,7.5) {\small $variable$ $\color{red}nEuler$};
  	\node at (2.1,9) {\normalsize Arquitectura X86-\textcolor{red}{32} bits };
  	\node at (8,9.7) {\normalsize {\color{red!50}$C\acute{o}digo$ $del$ $programa$ $fuente$}};
  	\node at (4,5.75) {\textcolor{blue}{factorial}};
  	\node at (4,5.4){\textcolor{blue}{contador}};
  	\node at (3.6,5.05) {\textcolor{blue}{i}};
  	\node at (3.9,4.6) {\textcolor{blue}{nEuler}};
  	\node at (4.05,4) {\textcolor{blue}{nEulerOld}};
  	\node at (0.1,6.1) {0x3F00480\textcolor{red}{0}};
  	\node at (0.1,5.75) {0x3F00480\textcolor{red}{4}};
  	\node at (0.1,5.4) {0x3F00480\textcolor{red}{8}};
  	\node at (0.1,5.05) {0x3F00480\textcolor{red}{C}};
  	\node at (0.1,4.7) {0x3F0048\textcolor{red}{10}};
  	\node at (0.1,4.37) {0x3F0048\textcolor{red}{14}};
  	\node at (0.1,4.03) {0x3F0048\textcolor{red}{18}};
  	\node at (0.1,3.67) {0x3F0048\textcolor{red}{1C}};
  	\draw[decorate,decoration={brace,mirror,raise=6pt,amplitude=8pt}, thick]
    (3.3,6.1) -- (0.9,6.1)node [black,midway,xshift=0.0cm,yshift=0.7cm]{\footnotesize {\color{red}$4$ $bytes$}};
  	\draw[decorate,decoration={brace,raise=3pt,amplitude=3pt}, thick]
    (7,5.1)--(10.7,5.1)node [black,midway,xshift=1.5cm,yshift=0.4cm]{\tiny{\color{red}$2.00+\frac{1}{2.00}=2.50$ }};
\end{tikzpicture}
\column{0.55\textwidth}
\lstset{basicstyle=\tiny}
\begin{lstlisting}[escapechar=\|,label=noint]
#include <stdio.h>
#include <math.h>

#define LIMITE   pow(10,-9)
int  main(void)
{
int		factorial = 1,contador =1,i;    
double	nEuler=0,nEulerOld=-1;

for(i=0;(nEuler-nEulerOld)>= LIMITE;i++)
  }
     nEulerOld=nEuler;
     while(contador <= i)
     { 
     	factorial = factorial * contador;
     	contador++;
     }
     nEuler = nEuler + (1/(double)factorial); 
        
     factorial = 1;
     contador =1;
  }
  printf("e es %0.10f \n",nEuler);
  printf("e(lib math.h)es %0.10f\n",M_E);
  return 0;
}
\end{lstlisting}
\end{columns}
\end{frame}

\begin{frame}[fragile]
\fontsize{6.5pt}{10pt}\selectfont
\frametitle{Recorriendo el programa {\color{yellow}paso} a {\color{yellow}paso}}
\begin{columns}[c]
\column{0.5\textwidth}
\begin{tikzpicture}
	\begin{scope}[every node/.style={draw,anchor=text,rectangle split,rectangle split parts=10, 
rectangle split part fill={green!0,blue!0,blue!0,blue!0,blue!0,blue!0,blue!0,blue!0,blue!0,red!0,red!0},minimum width=2.6cm}]
    	\node (R) at (2,6)
    	{
    		\nodepart{one}{}
    		\nodepart{two}{\color{red}1}
    		\nodepart{three}{\color{red}1}	
    		\nodepart{four}{2}
    		\nodepart{five}{2.50}
    		\nodepart{six}{}
    		\nodepart{seven}{2.00}
    		\nodepart{eight}{}
    		\nodepart{nine}{$\vdots$}
    		\nodepart{ten}{}
    	};
	\end{scope}
	\draw [ultra thick,color=white](0.8, 4.6) -- (3.37, 4.6);
	\draw [ultra thick,color=white](0.8, 3.92) -- (3.37, 3.92);
	\fill[color=green] (5.6,4.6) -- (8,4.6) -- (8,4.1) -- (5.6,4.1) -- cycle;
  	\node at (2.1,7.9) {\small $inicializo$ $nuevamente$ $las$};
  	\node at (2.1,7.5) {\small $variables$ $\color{red}factorial$ y $\color{red}contador$};
  	\node at (2.1,9) {\normalsize Arquitectura X86-\textcolor{red}{32} bits };
  	\node at (8,9.7) {\normalsize {\color{red!50}$C\acute{o}digo$ $del$ $programa$ $fuente$}};
  	\node at (4,5.75) {\textcolor{blue}{factorial}};
  	\node at (4,5.4){\textcolor{blue}{contador}};
  	\node at (3.6,5.05) {\textcolor{blue}{i}};
  	\node at (3.9,4.6) {\textcolor{blue}{nEuler}};
  	\node at (4.05,4) {\textcolor{blue}{nEulerOld}};
  	\node at (0.1,6.1) {0x3F00480\textcolor{red}{0}};
  	\node at (0.1,5.75) {0x3F00480\textcolor{red}{4}};
  	\node at (0.1,5.4) {0x3F00480\textcolor{red}{8}};
  	\node at (0.1,5.05) {0x3F00480\textcolor{red}{C}};
  	\node at (0.1,4.7) {0x3F0048\textcolor{red}{10}};
  	\node at (0.1,4.37) {0x3F0048\textcolor{red}{14}};
  	\node at (0.1,4.03) {0x3F0048\textcolor{red}{18}};
  	\node at (0.1,3.67) {0x3F0048\textcolor{red}{1C}};
  	\draw[decorate,decoration={brace,mirror,raise=6pt,amplitude=8pt}, thick]
    (3.3,6.1) -- (0.9,6.1)node [black,midway,xshift=0.0cm,yshift=0.7cm]{\footnotesize {\color{red}$4$ $bytes$}};
\end{tikzpicture}
\column{0.55\textwidth}
\lstset{basicstyle=\tiny}
\begin{lstlisting}[escapechar=\|,label=noint]
#include <stdio.h>
#include <math.h>

#define LIMITE   pow(10,-9)
int  main(void)
{
int		factorial = 1,contador =1,i;    
double	nEuler=0,nEulerOld=-1;

for(i=0;(nEuler-nEulerOld)>= LIMITE;i++)
  }
     nEulerOld=nEuler;
     while(contador <= i)
     { 
     	factorial = factorial * contador;
     	contador++;
     }
     nEuler = nEuler + (1/(double)factorial); 
        
     factorial = 1;
     contador =1;
  }
  printf("e es %0.10f \n",nEuler);
  printf("e(lib math.h)es %0.10f\n",M_E);
  return 0;
}
\end{lstlisting}
\end{columns}
\end{frame}

\begin{frame}[fragile]
\fontsize{6.5pt}{10pt}\selectfont
\frametitle{Recorriendo el programa {\color{yellow}paso} a {\color{yellow}paso}}
\begin{columns}[c]
\column{0.5\textwidth}
\begin{tikzpicture}
	\begin{scope}[every node/.style={draw,anchor=text,rectangle split,rectangle split parts=10, 
rectangle split part fill={green!0,blue!0,blue!0,blue!0,blue!0,blue!0,blue!0,blue!0,blue!0,red!0,red!0},minimum width=2.6cm}]
    	\node (R) at (2,6)
    	{
    		\nodepart{one}{}
    		\nodepart{two}{1}
    		\nodepart{three}{1}	
    		\nodepart{four}{\color{red}3}
    		\nodepart{five}{2.50}
    		\nodepart{six}{}
    		\nodepart{seven}{2.00}
    		\nodepart{eight}{}
    		\nodepart{nine}{$\vdots$}
    		\nodepart{ten}{}
    	};
	\end{scope}
	\draw [ultra thick,color=white](0.8, 4.6) -- (3.37, 4.6);
	\draw [ultra thick,color=white](0.8, 3.92) -- (3.37, 3.92);
	\fill[color=green] (9.7,7.2) -- (10.3,7.2) -- (10.3,6.8) -- (9.7,6.8) -- cycle;
  	\node at (2.1,7.9) {\small $incremento$ $la$ $variable$ $\color{red}i$};
  	\node at (2.1,9) {\normalsize Arquitectura X86-\textcolor{red}{32} bits };
  	\node at (8,9.7) {\normalsize {\color{red!50}$C\acute{o}digo$ $del$ $programa$ $fuente$}};
  	\node at (4,5.75) {\textcolor{blue}{factorial}};
  	\node at (4,5.4){\textcolor{blue}{contador}};
  	\node at (3.6,5.05) {\textcolor{blue}{i}};
  	\node at (3.9,4.6) {\textcolor{blue}{nEuler}};
  	\node at (4.05,4) {\textcolor{blue}{nEulerOld}};
  	\node at (0.1,6.1) {0x3F00480\textcolor{red}{0}};
  	\node at (0.1,5.75) {0x3F00480\textcolor{red}{4}};
  	\node at (0.1,5.4) {0x3F00480\textcolor{red}{8}};
  	\node at (0.1,5.05) {0x3F00480\textcolor{red}{C}};
  	\node at (0.1,4.7) {0x3F0048\textcolor{red}{10}};
  	\node at (0.1,4.37) {0x3F0048\textcolor{red}{14}};
  	\node at (0.1,4.03) {0x3F0048\textcolor{red}{18}};
  	\node at (0.1,3.67) {0x3F0048\textcolor{red}{1C}};
  	\draw[decorate,decoration={brace,mirror,raise=6pt,amplitude=8pt}, thick]
    (3.3,6.1) -- (0.9,6.1)node [black,midway,xshift=0.0cm,yshift=0.7cm]{\footnotesize {\color{red}$4$ $bytes$}};
\end{tikzpicture}
\column{0.55\textwidth}
\lstset{basicstyle=\tiny}
\begin{lstlisting}[escapechar=\|,label=noint]
#include <stdio.h>
#include <math.h>

#define LIMITE   pow(10,-9)
int  main(void)
{
int		factorial = 1,contador =1,i;    
double	nEuler=0,nEulerOld=-1;

for(i=0;(nEuler-nEulerOld)>= LIMITE;i++)
  }
     nEulerOld=nEuler;
     while(contador <= i)
     { 
     	factorial = factorial * contador;
     	contador++;
     }
     nEuler = nEuler + (1/(double)factorial); 
        
     factorial = 1;
     contador =1;
  }
  printf("e es %0.10f \n",nEuler);
  printf("e(lib math.h)es %0.10f\n",M_E);
  return 0;
}
\end{lstlisting}
\end{columns}
\end{frame}

\begin{frame}[fragile]
\fontsize{6.5pt}{10pt}\selectfont
\frametitle{Recorriendo el programa {\color{yellow}paso} a {\color{yellow}paso}}
\begin{columns}[c]
\column{0.5\textwidth}
\begin{tikzpicture}
	\begin{scope}[every node/.style={draw,anchor=text,rectangle split,rectangle split parts=10, 
rectangle split part fill={green!0,blue!0,blue!0,blue!0,blue!0,blue!0,blue!0,blue!0,blue!0,red!0,red!0},minimum width=2.6cm}]
    	\node (R) at (2,6)
    	{
    		\nodepart{one}{}
    		\nodepart{two}{1}
    		\nodepart{three}{1}	
    		\nodepart{four}{3}
    		\nodepart{five}{2.50}
    		\nodepart{six}{}
    		\nodepart{seven}{2.00}
    		\nodepart{eight}{}
    		\nodepart{nine}{$\vdots$}
    		\nodepart{ten}{}
    	};
	\end{scope}
	\draw [ultra thick,color=white](0.8, 4.6) -- (3.37, 4.6);
	\draw [ultra thick,color=white](0.8, 3.92) -- (3.37, 3.92);
	\fill[color=green] (6.4,7.2) -- (9.7,7.2) -- (9.7,6.8) -- (6.4,6.8) -- cycle;
  	\node at (2.1,9) {\normalsize Arquitectura X86-\textcolor{red}{32} bits };
  	\node at (2.1,7.9) {\small $Pregunta$ $condici\acute{o}n$ $del$ $\color{red}for$};
  	\node at (2.1,9) {\normalsize Arquitectura X86-\textcolor{red}{32} bits };
  	\node at (8,9.7) {\normalsize {\color{red!50}$C\acute{o}digo$ $del$ $programa$ $fuente$}};
  	\node at (4,5.75) {\textcolor{blue}{factorial}};
  	\node at (4,5.4){\textcolor{blue}{contador}};
  	\node at (3.6,5.05) {\textcolor{blue}{i}};
  	\node at (3.9,4.6) {\textcolor{blue}{nEuler}};
  	\node at (4.05,4) {\textcolor{blue}{nEulerOld}};
  	\node at (0.1,6.1) {0x3F00480\textcolor{red}{0}};
  	\node at (0.1,5.75) {0x3F00480\textcolor{red}{4}};
  	\node at (0.1,5.4) {0x3F00480\textcolor{red}{8}};
  	\node at (0.1,5.05) {0x3F00480\textcolor{red}{C}};
  	\node at (0.1,4.7) {0x3F0048\textcolor{red}{10}};
  	\node at (0.1,4.37) {0x3F0048\textcolor{red}{14}};
  	\node at (0.1,4.03) {0x3F0048\textcolor{red}{18}};
  	\node at (0.1,3.67) {0x3F0048\textcolor{red}{1C}};
  	\draw[decorate,decoration={brace,mirror,raise=6pt,amplitude=8pt}, thick]
    (3.3,6.1) -- (0.9,6.1)node [black,midway,xshift=0.0cm,yshift=0.7cm]{\footnotesize {\color{red}$4$ $bytes$}};
  	\draw[decorate,decoration={brace,raise=4pt,amplitude=4pt}, thick]
    (7.35,8.4)--(8.7,8.4)node [black,midway,xshift=1.0cm,yshift=0.4cm]{\footnotesize {\color{red!60}$10^{-9}$}};
  	\draw[red,->](9.3,7.1)to [out=10,in=0]node[right,midway]{} ++(-0.2,1.3) ; 
\end{tikzpicture}
\column{0.55\textwidth}
\lstset{basicstyle=\tiny}
\begin{lstlisting}[escapechar=\|,label=noint]
#include <stdio.h>
#include <math.h>

#define LIMITE   pow(10,-9)
int  main(void)
{
int		factorial = 1,contador =1,i;    
double	nEuler=0,nEulerOld=-1;

for(i=0;(nEuler-nEulerOld)>= LIMITE;i++)
  }
     nEulerOld=nEuler;
     while(contador <= i)
     { 
     	factorial = factorial * contador;
     	contador++;
     }
     nEuler = nEuler + (1/(double)factorial); 
        
     factorial = 1;
     contador =1;
  }
  printf("e es %0.10f \n",nEuler);
  printf("e(lib math.h)es %0.10f\n",M_E);
  return 0;
}
\end{lstlisting}
\end{columns}
\end{frame}

\begin{frame}[fragile]
\fontsize{6.5pt}{10pt}\selectfont
\frametitle{Recorriendo el programa {\color{yellow}paso} a {\color{yellow}paso}}
\begin{columns}[c]
\column{0.5\textwidth}
\begin{tikzpicture}
	\begin{scope}[every node/.style={draw,anchor=text,rectangle split,rectangle split parts=10, 
rectangle split part fill={green!0,blue!0,blue!0,blue!0,blue!0,blue!0,blue!0,blue!0,blue!0,red!0,red!0},minimum width=2.6cm}]
    	\node (R) at (2,6)
    	{
    		\nodepart{one}{}
    		\nodepart{two}{1}
    		\nodepart{three}{1}	
    		\nodepart{four}{3}
    		\nodepart{five}{2.50}
    		\nodepart{six}{}
    		\nodepart{seven}{\color{red}2.50}
    		\nodepart{eight}{}
    		\nodepart{nine}{$\vdots$}
    		\nodepart{ten}{}
    	};
	\end{scope}
	\draw [ultra thick,color=white](0.8, 4.6) -- (3.37, 4.6);
	\draw [ultra thick,color=white](0.8, 3.92) -- (3.37, 3.92);
	\fill[color=green] (5.8,6.65) -- (8,6.65) -- (8,6.35) -- (5.8,6.35) -- cycle;
  	\node at (2.1,7.9) {\small $Para$ $conservar$ $el$ $\color{red}e$ $anterior$};
  	\node at (2.1,9) {\normalsize Arquitectura X86-\textcolor{red}{32} bits };
  	\node at (8,9.7) {\normalsize {\color{red!50}$C\acute{o}digo$ $del$ $programa$ $fuente$}};
  	\node at (4,5.75) {\textcolor{blue}{factorial}};
  	\node at (4,5.4){\textcolor{blue}{contador}};
  	\node at (3.6,5.05) {\textcolor{blue}{i}};
  	\node at (3.9,4.6) {\textcolor{blue}{nEuler}};
  	\node at (4.05,4) {\textcolor{blue}{nEulerOld}};
  	\node at (0.1,6.1) {0x3F00480\textcolor{red}{0}};
  	\node at (0.1,5.75) {0x3F00480\textcolor{red}{4}};
  	\node at (0.1,5.4) {0x3F00480\textcolor{red}{8}};
  	\node at (0.1,5.05) {0x3F00480\textcolor{red}{C}};
  	\node at (0.1,4.7) {0x3F0048\textcolor{red}{10}};
  	\node at (0.1,4.37) {0x3F0048\textcolor{red}{14}};
  	\node at (0.1,4.03) {0x3F0048\textcolor{red}{18}};
  	\node at (0.1,3.67) {0x3F0048\textcolor{red}{1C}};
  	\draw[decorate,decoration={brace,mirror,raise=6pt,amplitude=8pt}, thick]
    (3.3,6.1) -- (0.9,6.1)node [black,midway,xshift=0.0cm,yshift=0.7cm]{\footnotesize {\color{red}$4$ $bytes$}};
\end{tikzpicture}
\column{0.55\textwidth}
\lstset{basicstyle=\tiny}
\begin{lstlisting}[escapechar=\|,label=noint]
#include <stdio.h>
#include <math.h>

#define LIMITE   pow(10,-9)
int  main(void)
{
int		factorial = 1,contador =1,i;    
double	nEuler=0,nEulerOld=-1;

for(i=0;(nEuler-nEulerOld)>= LIMITE;i++)
  }
     nEulerOld=nEuler;
     while(contador <= i)
     { 
     	factorial = factorial * contador;
     	contador++;
     }
     nEuler = nEuler + (1/(double)factorial); 
        
     factorial = 1;
     contador =1;
  }
  printf("e es %0.10f \n",nEuler);
  printf("e(lib math.h)es %0.10f\n",M_E);
  return 0;
}
\end{lstlisting}
\end{columns}
\end{frame}

\begin{frame}[fragile]
\fontsize{6.5pt}{10pt}\selectfont
\frametitle{Recorriendo el programa {\color{yellow}paso} a {\color{yellow}paso}}
\begin{columns}[c]
\column{0.5\textwidth}
\begin{tikzpicture}
	\begin{scope}[every node/.style={draw,anchor=text,rectangle split,rectangle split parts=10, 
rectangle split part fill={green!0,blue!0,blue!0,blue!0,blue!0,blue!0,blue!0,blue!0,blue!0,red!0,red!0},minimum width=2.6cm}]
    	\node (R) at (2,6)
    	{
    		\nodepart{one}{}
    		\nodepart{two}{1}
    		\nodepart{three}{1}	
    		\nodepart{four}{3}
    		\nodepart{five}{2.50}
    		\nodepart{six}{}
    		\nodepart{seven}{2.50}
    		\nodepart{eight}{}
    		\nodepart{nine}{$\vdots$}
    		\nodepart{ten}{}
    	};
	\end{scope}
	\draw [ultra thick,color=white](0.8, 4.6) -- (3.37, 4.6);
	\draw [ultra thick,color=white](0.8, 3.92) -- (3.37, 3.92);
	\fill[color=green] (5.8,6.4) -- (8.4,6.4) -- (8.4,6) -- (5.8,6) -- cycle;
  	\node at (2.1,7.9) {\small $Como$ $\color{red}contador$ $es$ $\color{red}<=$ $a$ $\color{red}i$};
  	\node at (2.1,7.5) {\small $entra$ $dentro$ $del$ $\color{red}while$};
  	\node at (2.1,9) {\normalsize Arquitectura X86-\textcolor{red}{32} bits };
  	\node at (8,9.7) {\normalsize {\color{red!50}$C\acute{o}digo$ $del$ $programa$ $fuente$}};
  	\node at (4,5.75) {\textcolor{blue}{factorial}};
  	\node at (4,5.4){\textcolor{blue}{contador}};
  	\node at (3.6,5.05) {\textcolor{blue}{i}};
  	\node at (3.9,4.6) {\textcolor{blue}{nEuler}};
  	\node at (4.05,4) {\textcolor{blue}{nEulerOld}};
  	\node at (0.1,6.1) {0x3F00480\textcolor{red}{0}};
  	\node at (0.1,5.75) {0x3F00480\textcolor{red}{4}};
  	\node at (0.1,5.4) {0x3F00480\textcolor{red}{8}};
  	\node at (0.1,5.05) {0x3F00480\textcolor{red}{C}};
  	\node at (0.1,4.7) {0x3F0048\textcolor{red}{10}};
  	\node at (0.1,4.37) {0x3F0048\textcolor{red}{14}};
  	\node at (0.1,4.03) {0x3F0048\textcolor{red}{18}};
  	\node at (0.1,3.67) {0x3F0048\textcolor{red}{1C}};
  	\draw[decorate,decoration={brace,mirror,raise=6pt,amplitude=8pt}, thick]
    (3.3,6.1) -- (0.9,6.1)node [black,midway,xshift=0.0cm,yshift=0.7cm]{\footnotesize {\color{red}$4$ $bytes$}};
\end{tikzpicture}
\column{0.55\textwidth}
\lstset{basicstyle=\tiny}
\begin{lstlisting}[escapechar=\|,label=noint]
#include <stdio.h>
#include <math.h>

#define LIMITE   pow(10,-9)
int  main(void)
{
int		factorial = 1,contador =1,i;    
double	nEuler=0,nEulerOld=-1;

for(i=0;(nEuler-nEulerOld)>= LIMITE;i++)
  }
     nEulerOld=nEuler;
     while(contador <= i)
     { 
     	factorial = factorial * contador;
     	contador++;
     }
     nEuler = nEuler + (1/(double)factorial); 
        
     factorial = 1;
     contador =1;
  }
  printf("e es %0.10f \n",nEuler);
  printf("e(lib math.h)es %0.10f\n",M_E);
  return 0;
}
\end{lstlisting}
\end{columns}
\end{frame}

\begin{frame}[fragile]
\fontsize{6.5pt}{10pt}\selectfont
\frametitle{Recorriendo el programa {\color{yellow}paso} a {\color{yellow}paso}}
\begin{columns}[c]
\column{0.5\textwidth}
\begin{tikzpicture}
	\begin{scope}[every node/.style={draw,anchor=text,rectangle split,rectangle split parts=10, 
rectangle split part fill={green!0,blue!0,blue!0,blue!0,blue!0,blue!0,blue!0,blue!0,blue!0,red!0,red!0},minimum width=2.6cm}]
    	\node (R) at (2,6)
    	{
    		\nodepart{one}{}
    		\nodepart{two}{\color{red}1}
    		\nodepart{three}{1}	
    		\nodepart{four}{3}
    		\nodepart{five}{2.50}
    		\nodepart{six}{}
    		\nodepart{seven}{2.50}
    		\nodepart{eight}{}
    		\nodepart{nine}{$\vdots$}
    		\nodepart{ten}{}
    	};
	\end{scope}
	\draw [ultra thick,color=white](0.8, 4.6) -- (3.37, 4.6);
	\draw [ultra thick,color=white](0.8, 3.92) -- (3.37, 3.92);
	\fill[color=green] (5.8,5.9) -- (10.5,5.9) -- (10.5,5.6) -- (5.8,5.6) -- cycle;
  	\node at (2.1,7.9) {\small $almacena$ $el$ $resultado$ $en$ $la$};
  	\node at (2.1,7.5) {\small $variable$ $\color{red}factorial$};
  	\node at (2.1,9) {\normalsize Arquitectura X86-\textcolor{red}{32} bits };
  	\node at (8,9.7) {\normalsize {\color{red!50}$C\acute{o}digo$ $del$ $programa$ $fuente$}};
  	\node at (4,5.75) {\textcolor{blue}{factorial}};
  	\node at (4,5.4){\textcolor{blue}{contador}};
  	\node at (3.6,5.05) {\textcolor{blue}{i}};
  	\node at (3.9,4.6) {\textcolor{blue}{nEuler}};
  	\node at (4.05,4) {\textcolor{blue}{nEulerOld}};
  	\node at (0.1,6.1) {0x3F00480\textcolor{red}{0}};
  	\node at (0.1,5.75) {0x3F00480\textcolor{red}{4}};
  	\node at (0.1,5.4) {0x3F00480\textcolor{red}{8}};
  	\node at (0.1,5.05) {0x3F00480\textcolor{red}{C}};
  	\node at (0.1,4.7) {0x3F0048\textcolor{red}{10}};
  	\node at (0.1,4.37) {0x3F0048\textcolor{red}{14}};
  	\node at (0.1,4.03) {0x3F0048\textcolor{red}{18}};
  	\node at (0.1,3.67) {0x3F0048\textcolor{red}{1C}};
  	\draw[decorate,decoration={brace,mirror,raise=6pt,amplitude=8pt}, thick]
    (3.3,6.1) -- (0.9,6.1)node [black,midway,xshift=0.0cm,yshift=0.7cm]{\footnotesize {\color{red}$4$ $bytes$}};
  	\draw[decorate,decoration={brace,raise=3pt,amplitude=3pt}, thick]
    (7.5,5.8)--(10.2,5.8)node [black,midway,xshift=1.5cm,yshift=0.4cm]{\footnotesize{\color{red}$1\cdot 1=1$}};
\end{tikzpicture}
\column{0.55\textwidth}
\lstset{basicstyle=\tiny}
\begin{lstlisting}[escapechar=\|,label=noint]
#include <stdio.h>
#include <math.h>

#define LIMITE   pow(10,-9)
int  main(void)
{
int		factorial = 1,contador =1,i;    
double	nEuler=0,nEulerOld=-1;

for(i=0;(nEuler-nEulerOld)>= LIMITE;i++)
  }
     nEulerOld=nEuler;
     while(contador <= i)
     { 
     	factorial = factorial * contador;
     	contador++;
     }
     nEuler = nEuler + (1/(double)factorial); 
        
     factorial = 1;
     contador =1;
  }
  printf("e es %0.10f \n",nEuler);
  printf("e(lib math.h)es %0.10f\n",M_E);
  return 0;
}
\end{lstlisting}
\end{columns}
\end{frame}

\begin{frame}[fragile]
\fontsize{6.5pt}{10pt}\selectfont
\frametitle{Recorriendo el programa {\color{yellow}paso} a {\color{yellow}paso}}
\begin{columns}[c]
\column{0.5\textwidth}
\begin{tikzpicture}
	\begin{scope}[every node/.style={draw,anchor=text,rectangle split,rectangle split parts=10, 
rectangle split part fill={green!0,blue!0,blue!0,blue!0,blue!0,blue!0,blue!0,blue!0,blue!0,red!0,red!0},minimum width=2.6cm}]
    	\node (R) at (2,6)
    	{
    		\nodepart{one}{}
    		\nodepart{two}{1}
    		\nodepart{three}{\color{red}2}	
    		\nodepart{four}{3}
    		\nodepart{five}{2.50}
    		\nodepart{six}{}
    		\nodepart{seven}{2.50}
    		\nodepart{eight}{}
    		\nodepart{nine}{$\vdots$}
    		\nodepart{ten}{}
    	};
	\end{scope}
	\draw [ultra thick,color=white](0.8, 4.6) -- (3.37, 4.6);
	\draw [ultra thick,color=white](0.8, 3.92) -- (3.37, 3.92);
	\fill[color=green] (5.8,5.6) -- (8,5.6) -- (8,5.3) -- (5.8,5.3) -- cycle;
  	\node at (2.1,7.9) {\small $incremento$ $la$ $variable$ $\color{red}contador$};
  	\node at (2.1,9) {\normalsize Arquitectura X86-\textcolor{red}{32} bits };
  	\node at (8,9.7) {\normalsize {\color{red!50}$C\acute{o}digo$ $del$ $programa$ $fuente$}};
  	\node at (4,5.75) {\textcolor{blue}{factorial}};
  	\node at (4,5.4){\textcolor{blue}{contador}};
  	\node at (3.6,5.05) {\textcolor{blue}{i}};
  	\node at (3.9,4.6) {\textcolor{blue}{nEuler}};
  	\node at (4.05,4) {\textcolor{blue}{nEulerOld}};
  	\node at (0.1,6.1) {0x3F00480\textcolor{red}{0}};
  	\node at (0.1,5.75) {0x3F00480\textcolor{red}{4}};
  	\node at (0.1,5.4) {0x3F00480\textcolor{red}{8}};
  	\node at (0.1,5.05) {0x3F00480\textcolor{red}{C}};
  	\node at (0.1,4.7) {0x3F0048\textcolor{red}{10}};
  	\node at (0.1,4.37) {0x3F0048\textcolor{red}{14}};
  	\node at (0.1,4.03) {0x3F0048\textcolor{red}{18}};
  	\node at (0.1,3.67) {0x3F0048\textcolor{red}{1C}};
  	\draw[decorate,decoration={brace,mirror,raise=6pt,amplitude=8pt}, thick]
    (3.3,6.1) -- (0.9,6.1)node [black,midway,xshift=0.0cm,yshift=0.7cm]{\footnotesize {\color{red}$4$ $bytes$}};
\end{tikzpicture}
\column{0.55\textwidth}
\lstset{basicstyle=\tiny}
\begin{lstlisting}[escapechar=\|,label=noint]
#include <stdio.h>
#include <math.h>

#define LIMITE   pow(10,-9)
int  main(void)
{
int		factorial = 1,contador =1,i;    
double	nEuler=0,nEulerOld=-1;

for(i=0;(nEuler-nEulerOld)>= LIMITE;i++)
  }
     nEulerOld=nEuler;
     while(contador <= i)
     { 
     	factorial = factorial * contador;
     	contador++;
     }
     nEuler = nEuler + (1/(double)factorial); 
        
     factorial = 1;
     contador =1;
  }
  printf("e es %0.10f \n",nEuler);
  printf("e(lib math.h)es %0.10f\n",M_E);
  return 0;
}
\end{lstlisting}
\end{columns}
\end{frame}

\begin{frame}[fragile]
\fontsize{6.5pt}{10pt}\selectfont
\frametitle{Recorriendo el programa {\color{yellow}paso} a {\color{yellow}paso}}
\begin{columns}[c]
\column{0.5\textwidth}
\begin{tikzpicture}
	\begin{scope}[every node/.style={draw,anchor=text,rectangle split,rectangle split parts=10, 
rectangle split part fill={green!0,blue!0,blue!0,blue!0,blue!0,blue!0,blue!0,blue!0,blue!0,red!0,red!0},minimum width=2.6cm}]
    	\node (R) at (2,6)
    	{
    		\nodepart{one}{}
    		\nodepart{two}{1}
    		\nodepart{three}{2}	
    		\nodepart{four}{3}
    		\nodepart{five}{2.50}
    		\nodepart{six}{}
    		\nodepart{seven}{2.50}
    		\nodepart{eight}{}
    		\nodepart{nine}{$\vdots$}
    		\nodepart{ten}{}
    	};
	\end{scope}
	\draw [ultra thick,color=white](0.8, 4.6) -- (3.37, 4.6);
	\draw [ultra thick,color=white](0.8, 3.92) -- (3.37, 3.92);
	\fill[color=green] (5.8,6.4) -- (8.4,6.4) -- (8.4,6) -- (5.8,6) -- cycle;
  	\node at (2.1,7.9) {\small $Como$ $\color{red}contador$ $es$ $\color{red}<=$ $a$ $\color{red}i$};
  	\node at (2.1,7.5) {\small $entra$ $dentro$ $del$ $\color{red}while$};
  	\node at (2.1,9) {\normalsize Arquitectura X86-\textcolor{red}{32} bits };
  	\node at (8,9.7) {\normalsize {\color{red!50}$C\acute{o}digo$ $del$ $programa$ $fuente$}};
  	\node at (4,5.75) {\textcolor{blue}{factorial}};
  	\node at (4,5.4){\textcolor{blue}{contador}};
  	\node at (3.6,5.05) {\textcolor{blue}{i}};
  	\node at (3.9,4.6) {\textcolor{blue}{nEuler}};
  	\node at (4.05,4) {\textcolor{blue}{nEulerOld}};
  	\node at (0.1,6.1) {0x3F00480\textcolor{red}{0}};
  	\node at (0.1,5.75) {0x3F00480\textcolor{red}{4}};
  	\node at (0.1,5.4) {0x3F00480\textcolor{red}{8}};
  	\node at (0.1,5.05) {0x3F00480\textcolor{red}{C}};
  	\node at (0.1,4.7) {0x3F0048\textcolor{red}{10}};
  	\node at (0.1,4.37) {0x3F0048\textcolor{red}{14}};
  	\node at (0.1,4.03) {0x3F0048\textcolor{red}{18}};
  	\node at (0.1,3.67) {0x3F0048\textcolor{red}{1C}};
  	\draw[decorate,decoration={brace,mirror,raise=6pt,amplitude=8pt}, thick]
    (3.3,6.1) -- (0.9,6.1)node [black,midway,xshift=0.0cm,yshift=0.7cm]{\footnotesize {\color{red}$4$ $bytes$}};
\end{tikzpicture}
\column{0.55\textwidth}
\lstset{basicstyle=\tiny}
\begin{lstlisting}[escapechar=\|,label=noint]
#include <stdio.h>
#include <math.h>

#define LIMITE   pow(10,-9)
int  main(void)
{
int		factorial = 1,contador =1,i;    
double	nEuler=0,nEulerOld=-1;

for(i=0;(nEuler-nEulerOld)>= LIMITE;i++)
  }
     nEulerOld=nEuler;
     while(contador <= i)
     { 
     	factorial = factorial * contador;
     	contador++;
     }
     nEuler = nEuler + (1/(double)factorial); 
        
     factorial = 1;
     contador =1;
  }
  printf("e es %0.10f \n",nEuler);
  printf("e(lib math.h)es %0.10f\n",M_E);
  return 0;
}
\end{lstlisting}
\end{columns}
\end{frame}

\begin{frame}[fragile]
\fontsize{6.5pt}{10pt}\selectfont
\frametitle{Recorriendo el programa {\color{yellow}paso} a {\color{yellow}paso}}
\begin{columns}[c]
\column{0.5\textwidth}
\begin{tikzpicture}
	\begin{scope}[every node/.style={draw,anchor=text,rectangle split,rectangle split parts=10, 
rectangle split part fill={green!0,blue!0,blue!0,blue!0,blue!0,blue!0,blue!0,blue!0,blue!0,red!0,red!0},minimum width=2.6cm}]
    	\node (R) at (2,6)
    	{
    		\nodepart{one}{}
    		\nodepart{two}{\color{red}2}
    		\nodepart{three}{2}	
    		\nodepart{four}{3}
    		\nodepart{five}{2.50}
    		\nodepart{six}{}
    		\nodepart{seven}{2.50}
    		\nodepart{eight}{}
    		\nodepart{nine}{$\vdots$}
    		\nodepart{ten}{}
    	};
	\end{scope}
	\draw [ultra thick,color=white](0.8, 4.6) -- (3.37, 4.6);
	\draw [ultra thick,color=white](0.8, 3.92) -- (3.37, 3.92);
	\fill[color=green] (5.8,5.9) -- (10.5,5.9) -- (10.5,5.6) -- (5.8,5.6) -- cycle;
  	\node at (2.1,7.9) {\small $almacena$ $el$ $resultado$ $en$ $la$};
  	\node at (2.1,7.5) {\small $variable$ $\color{red}factorial$};
  	\node at (2.1,9) {\normalsize Arquitectura X86-\textcolor{red}{32} bits };
  	\node at (8,9.7) {\normalsize {\color{red!50}$C\acute{o}digo$ $del$ $programa$ $fuente$}};
  	\node at (4,5.75) {\textcolor{blue}{factorial}};
  	\node at (4,5.4){\textcolor{blue}{contador}};
  	\node at (3.6,5.05) {\textcolor{blue}{i}};
  	\node at (3.9,4.6) {\textcolor{blue}{nEuler}};
  	\node at (4.05,4) {\textcolor{blue}{nEulerOld}};
  	\node at (0.1,6.1) {0x3F00480\textcolor{red}{0}};
  	\node at (0.1,5.75) {0x3F00480\textcolor{red}{4}};
  	\node at (0.1,5.4) {0x3F00480\textcolor{red}{8}};
  	\node at (0.1,5.05) {0x3F00480\textcolor{red}{C}};
  	\node at (0.1,4.7) {0x3F0048\textcolor{red}{10}};
  	\node at (0.1,4.37) {0x3F0048\textcolor{red}{14}};
  	\node at (0.1,4.03) {0x3F0048\textcolor{red}{18}};
  	\node at (0.1,3.67) {0x3F0048\textcolor{red}{1C}};
  	\draw[decorate,decoration={brace,mirror,raise=6pt,amplitude=8pt}, thick]
    (3.3,6.1) -- (0.9,6.1)node [black,midway,xshift=0.0cm,yshift=0.7cm]{\footnotesize {\color{red}$4$ $bytes$}};
  	\draw[decorate,decoration={brace,raise=3pt,amplitude=3pt}, thick]
    (7.5,5.8)--(10.2,5.8)node [black,midway,xshift=1.5cm,yshift=0.4cm]{\footnotesize{\color{red}$1\cdot 2=2$ }};
\end{tikzpicture}
\column{0.55\textwidth}
\lstset{basicstyle=\tiny}
\begin{lstlisting}[escapechar=\|,label=noint]
#include <stdio.h>
#include <math.h>

#define LIMITE   pow(10,-9)
int  main(void)
{
int		factorial = 1,contador =1,i;    
double	nEuler=0,nEulerOld=-1;

for(i=0;(nEuler-nEulerOld)>= LIMITE;i++)
  }
     nEulerOld=nEuler;
     while(contador <= i)
     { 
     	factorial = factorial * contador;
     	contador++;
     }
     nEuler = nEuler + (1/(double)factorial); 
        
     factorial = 1;
     contador =1;
  }
  printf("e es %0.10f \n",nEuler);
  printf("e(lib math.h)es %0.10f\n",M_E);
  return 0;
}
\end{lstlisting}
\end{columns}
\end{frame}

\begin{frame}[fragile]
\fontsize{6.5pt}{10pt}\selectfont
\frametitle{Recorriendo el programa {\color{yellow}paso} a {\color{yellow}paso}}
\begin{columns}[c]
\column{0.5\textwidth}
\begin{tikzpicture}
	\begin{scope}[every node/.style={draw,anchor=text,rectangle split,rectangle split parts=10, 
rectangle split part fill={green!0,blue!0,blue!0,blue!0,blue!0,blue!0,blue!0,blue!0,blue!0,red!0,red!0},minimum width=2.6cm}]
    	\node (R) at (2,6)
    	{
    		\nodepart{one}{}
    		\nodepart{two}{2}
    		\nodepart{three}{\color{red}3}	
    		\nodepart{four}{3}
    		\nodepart{five}{2.50}
    		\nodepart{six}{}
    		\nodepart{seven}{2.50}
    		\nodepart{eight}{}
    		\nodepart{nine}{$\vdots$}
    		\nodepart{ten}{}
    	};
	\end{scope}
	\draw [ultra thick,color=white](0.8, 4.6) -- (3.37, 4.6);
	\draw [ultra thick,color=white](0.8, 3.92) -- (3.37, 3.92);
	\fill[color=green] (5.8,5.6) -- (8,5.6) -- (8,5.3) -- (5.8,5.3) -- cycle;
  	\node at (2.1,7.9) {\small $incremento$ $la$ $variable$ $\color{red}contador$};
  	\node at (2.1,9) {\normalsize Arquitectura X86-\textcolor{red}{32} bits };
  	\node at (8,9.7) {\normalsize {\color{red!50}$C\acute{o}digo$ $del$ $programa$ $fuente$}};
  	\node at (4,5.75) {\textcolor{blue}{factorial}};
  	\node at (4,5.4){\textcolor{blue}{contador}};
  	\node at (3.6,5.05) {\textcolor{blue}{i}};
  	\node at (3.9,4.6) {\textcolor{blue}{nEuler}};
  	\node at (4.05,4) {\textcolor{blue}{nEulerOld}};
  	\node at (0.1,6.1) {0x3F00480\textcolor{red}{0}};
  	\node at (0.1,5.75) {0x3F00480\textcolor{red}{4}};
  	\node at (0.1,5.4) {0x3F00480\textcolor{red}{8}};
  	\node at (0.1,5.05) {0x3F00480\textcolor{red}{C}};
  	\node at (0.1,4.7) {0x3F0048\textcolor{red}{10}};
  	\node at (0.1,4.37) {0x3F0048\textcolor{red}{14}};
  	\node at (0.1,4.03) {0x3F0048\textcolor{red}{18}};
  	\node at (0.1,3.67) {0x3F0048\textcolor{red}{1C}};
  	\draw[decorate,decoration={brace,mirror,raise=6pt,amplitude=8pt}, thick]
    (3.3,6.1) -- (0.9,6.1)node [black,midway,xshift=0.0cm,yshift=0.7cm]{\footnotesize {\color{red}$4$ $bytes$}};
\end{tikzpicture}
\column{0.55\textwidth}
\lstset{basicstyle=\tiny}
\begin{lstlisting}[escapechar=\|,label=noint]
#include <stdio.h>
#include <math.h>

#define LIMITE   pow(10,-9)
int  main(void)
{
int		factorial = 1,contador =1,i;    
double	nEuler=0,nEulerOld=-1;

for(i=0;(nEuler-nEulerOld)>= LIMITE;i++)
  }
     nEulerOld=nEuler;
     while(contador <= i)
     { 
     	factorial = factorial * contador;
     	contador++;
     }
     nEuler = nEuler + (1/(double)factorial); 
        
     factorial = 1;
     contador =1;
  }
  printf("e es %0.10f \n",nEuler);
  printf("e(lib math.h)es %0.10f\n",M_E);
  return 0;
}
\end{lstlisting}
\end{columns}
\end{frame}

\begin{frame}[fragile]
\fontsize{6.5pt}{10pt}\selectfont
\frametitle{Recorriendo el programa {\color{yellow}paso} a {\color{yellow}paso}}
\begin{columns}[c]
\column{0.5\textwidth}
\begin{tikzpicture}
	\begin{scope}[every node/.style={draw,anchor=text,rectangle split,rectangle split parts=10, 
rectangle split part fill={green!0,blue!0,blue!0,blue!0,blue!0,blue!0,blue!0,blue!0,blue!0,red!0,red!0},minimum width=2.6cm}]
    	\node (R) at (2,6)
    	{
    		\nodepart{one}{}
    		\nodepart{two}{2}
    		\nodepart{three}{3}	
    		\nodepart{four}{3}
    		\nodepart{five}{2.50}
    		\nodepart{six}{}
    		\nodepart{seven}{2.50}
    		\nodepart{eight}{}
    		\nodepart{nine}{$\vdots$}
    		\nodepart{ten}{}
    	};
	\end{scope}
	\draw [ultra thick,color=white](0.8, 4.6) -- (3.37, 4.6);
	\draw [ultra thick,color=white](0.8, 3.92) -- (3.37, 3.92);
	\fill[color=green] (5.8,6.4) -- (8.4,6.4) -- (8.4,6) -- (5.8,6) -- cycle;
  	\node at (2.1,7.9) {\small $Como$ $\color{red}contador$ $es$ $\color{red}<=$ $a$ $\color{red}i$};
  	\node at (2.1,7.5) {\small $entra$ $dentro$ $del$ $\color{red}while$};
  	\node at (2.1,9) {\normalsize Arquitectura X86-\textcolor{red}{32} bits };
  	\node at (8,9.7) {\normalsize {\color{red!50}$C\acute{o}digo$ $del$ $programa$ $fuente$}};
  	\node at (4,5.75) {\textcolor{blue}{factorial}};
  	\node at (4,5.4){\textcolor{blue}{contador}};
  	\node at (3.6,5.05) {\textcolor{blue}{i}};
  	\node at (3.9,4.6) {\textcolor{blue}{nEuler}};
  	\node at (4.05,4) {\textcolor{blue}{nEulerOld}};
  	\node at (0.1,6.1) {0x3F00480\textcolor{red}{0}};
  	\node at (0.1,5.75) {0x3F00480\textcolor{red}{4}};
  	\node at (0.1,5.4) {0x3F00480\textcolor{red}{8}};
  	\node at (0.1,5.05) {0x3F00480\textcolor{red}{C}};
  	\node at (0.1,4.7) {0x3F0048\textcolor{red}{10}};
  	\node at (0.1,4.37) {0x3F0048\textcolor{red}{14}};
  	\node at (0.1,4.03) {0x3F0048\textcolor{red}{18}};
  	\node at (0.1,3.67) {0x3F0048\textcolor{red}{1C}};
  	\draw[decorate,decoration={brace,mirror,raise=6pt,amplitude=8pt}, thick]
    (3.3,6.1) -- (0.9,6.1)node [black,midway,xshift=0.0cm,yshift=0.7cm]{\footnotesize {\color{red}$4$ $bytes$}};
\end{tikzpicture}
\column{0.55\textwidth}
\lstset{basicstyle=\tiny}
\begin{lstlisting}[escapechar=\|,label=noint]
#include <stdio.h>
#include <math.h>

#define LIMITE   pow(10,-9)
int  main(void)
{
int		factorial = 1,contador =1,i;    
double	nEuler=0,nEulerOld=-1;

for(i=0;(nEuler-nEulerOld)>= LIMITE;i++)
  }
     nEulerOld=nEuler;
     while(contador <= i)
     { 
     	factorial = factorial * contador;
     	contador++;
     }
     nEuler = nEuler + (1/(double)factorial); 
        
     factorial = 1;
     contador =1;
  }
  printf("e es %0.10f \n",nEuler);
  printf("e(lib math.h)es %0.10f\n",M_E);
  return 0;
}
\end{lstlisting}
\end{columns}
\end{frame}

\begin{frame}[fragile]
\fontsize{6.5pt}{10pt}\selectfont
\frametitle{Recorriendo el programa {\color{yellow}paso} a {\color{yellow}paso}}
\begin{columns}[c]
\column{0.5\textwidth}
\begin{tikzpicture}
	\begin{scope}[every node/.style={draw,anchor=text,rectangle split,rectangle split parts=10, 
rectangle split part fill={green!0,blue!0,blue!0,blue!0,blue!0,blue!0,blue!0,blue!0,blue!0,red!0,red!0},minimum width=2.6cm}]
    	\node (R) at (2,6)
    	{
    		\nodepart{one}{}
    		\nodepart{two}{\color{red}6}
    		\nodepart{three}{3}	
    		\nodepart{four}{3}
    		\nodepart{five}{2.50}
    		\nodepart{six}{}
    		\nodepart{seven}{2.50}
    		\nodepart{eight}{}
    		\nodepart{nine}{$\vdots$}
    		\nodepart{ten}{}
    	};
	\end{scope}
	\draw [ultra thick,color=white](0.8, 4.6) -- (3.37, 4.6);
	\draw [ultra thick,color=white](0.8, 3.92) -- (3.37, 3.92);
	\fill[color=green] (5.8,5.9) -- (10.5,5.9) -- (10.5,5.6) -- (5.8,5.6) -- cycle;
  	\node at (2.1,7.9) {\small $almacena$ $el$ $resultado$ $en$ $la$};
  	\node at (2.1,7.5) {\small $variable$ $\color{red}factorial$};
  	\node at (2.1,9) {\normalsize Arquitectura X86-\textcolor{red}{32} bits };
  	\node at (8,9.7) {\normalsize {\color{red!50}$C\acute{o}digo$ $del$ $programa$ $fuente$}};
  	\node at (4,5.75) {\textcolor{blue}{factorial}};
  	\node at (4,5.4){\textcolor{blue}{contador}};
  	\node at (3.6,5.05) {\textcolor{blue}{i}};
  	\node at (3.9,4.6) {\textcolor{blue}{nEuler}};
  	\node at (4.05,4) {\textcolor{blue}{nEulerOld}};
  	\node at (0.1,6.1) {0x3F00480\textcolor{red}{0}};
  	\node at (0.1,5.75) {0x3F00480\textcolor{red}{4}};
  	\node at (0.1,5.4) {0x3F00480\textcolor{red}{8}};
  	\node at (0.1,5.05) {0x3F00480\textcolor{red}{C}};
  	\node at (0.1,4.7) {0x3F0048\textcolor{red}{10}};
  	\node at (0.1,4.37) {0x3F0048\textcolor{red}{14}};
  	\node at (0.1,4.03) {0x3F0048\textcolor{red}{18}};
  	\node at (0.1,3.67) {0x3F0048\textcolor{red}{1C}};
  	\draw[decorate,decoration={brace,mirror,raise=6pt,amplitude=8pt}, thick]
    (3.3,6.1) -- (0.9,6.1)node [black,midway,xshift=0.0cm,yshift=0.7cm]{\footnotesize {\color{red}$4$ $bytes$}};
  	\draw[decorate,decoration={brace,raise=3pt,amplitude=3pt}, thick]
    (7.5,5.8)--(10.2,5.8)node [black,midway,xshift=1.5cm,yshift=0.4cm]{\footnotesize{\color{red}$2\cdot 3=6$ }};
\end{tikzpicture}
\column{0.55\textwidth}
\lstset{basicstyle=\tiny}
\begin{lstlisting}[escapechar=\|,label=noint]
#include <stdio.h>
#include <math.h>

#define LIMITE   pow(10,-9)
int  main(void)
{
int		factorial = 1,contador =1,i;    
double	nEuler=0,nEulerOld=-1;

for(i=0;(nEuler-nEulerOld)>= LIMITE;i++)
  }
     nEulerOld=nEuler;
     while(contador <= i)
     { 
     	factorial = factorial * contador;
     	contador++;
     }
     nEuler = nEuler + (1/(double)factorial); 
        
     factorial = 1;
     contador =1;
  }
  printf("e es %0.10f \n",nEuler);
  printf("e(lib math.h)es %0.10f\n",M_E);
  return 0;
}
\end{lstlisting}
\end{columns}
\end{frame}

\begin{frame}[fragile]
\fontsize{6.5pt}{10pt}\selectfont
\frametitle{Recorriendo el programa {\color{yellow}paso} a {\color{yellow}paso}}
\begin{columns}[c]
\column{0.5\textwidth}
\begin{tikzpicture}
	\begin{scope}[every node/.style={draw,anchor=text,rectangle split,rectangle split parts=10, 
rectangle split part fill={green!0,blue!0,blue!0,blue!0,blue!0,blue!0,blue!0,blue!0,blue!0,red!0,red!0},minimum width=2.6cm}]
    	\node (R) at (2,6)
    	{
    		\nodepart{one}{}
    		\nodepart{two}{6}
    		\nodepart{three}{\color{red}4}	
    		\nodepart{four}{3}
    		\nodepart{five}{2.50}
    		\nodepart{six}{}
    		\nodepart{seven}{2.50}
    		\nodepart{eight}{}
    		\nodepart{nine}{$\vdots$}
    		\nodepart{ten}{}
    	};
	\end{scope}
	\draw [ultra thick,color=white](0.8, 4.6) -- (3.37, 4.6);
	\draw [ultra thick,color=white](0.8, 3.92) -- (3.37, 3.92);
	\fill[color=green] (5.8,5.6) -- (8,5.6) -- (8,5.3) -- (5.8,5.3) -- cycle;
  	\node at (2.1,7.9) {\small $incremento$ $la$ $variable$ $\color{red}contador$};
  	\node at (2.1,9) {\normalsize Arquitectura X86-\textcolor{red}{32} bits };
  	\node at (8,9.7) {\normalsize {\color{red!50}$C\acute{o}digo$ $del$ $programa$ $fuente$}};
  	\node at (4,5.75) {\textcolor{blue}{factorial}};
  	\node at (4,5.4){\textcolor{blue}{contador}};
  	\node at (3.6,5.05) {\textcolor{blue}{i}};
  	\node at (3.9,4.6) {\textcolor{blue}{nEuler}};
  	\node at (4.05,4) {\textcolor{blue}{nEulerOld}};
  	\node at (0.1,6.1) {0x3F00480\textcolor{red}{0}};
  	\node at (0.1,5.75) {0x3F00480\textcolor{red}{4}};
  	\node at (0.1,5.4) {0x3F00480\textcolor{red}{8}};
  	\node at (0.1,5.05) {0x3F00480\textcolor{red}{C}};
  	\node at (0.1,4.7) {0x3F0048\textcolor{red}{10}};
  	\node at (0.1,4.37) {0x3F0048\textcolor{red}{14}};
  	\node at (0.1,4.03) {0x3F0048\textcolor{red}{18}};
  	\node at (0.1,3.67) {0x3F0048\textcolor{red}{1C}};
  	\draw[decorate,decoration={brace,mirror,raise=6pt,amplitude=8pt}, thick]
    (3.3,6.1) -- (0.9,6.1)node [black,midway,xshift=0.0cm,yshift=0.7cm]{\footnotesize {\color{red}$4$ $bytes$}};
\end{tikzpicture}
\column{0.55\textwidth}
\lstset{basicstyle=\tiny}
\begin{lstlisting}[escapechar=\|,label=noint]
#include <stdio.h>
#include <math.h>

#define LIMITE   pow(10,-9)
int  main(void)
{
int		factorial = 1,contador =1,i;    
double	nEuler=0,nEulerOld=-1;

for(i=0;(nEuler-nEulerOld)>= LIMITE;i++)
  }
     nEulerOld=nEuler;
     while(contador <= i)
     { 
     	factorial = factorial * contador;
     	contador++;
     }
     nEuler = nEuler + (1/(double)factorial); 
        
     factorial = 1;
     contador =1;
  }
  printf("e es %0.10f \n",nEuler);
  printf("e(lib math.h)es %0.10f\n",M_E);
  return 0;
}
\end{lstlisting}
\end{columns}
\end{frame}

\begin{frame}[fragile]
\fontsize{6.5pt}{10pt}\selectfont
\frametitle{Recorriendo el programa {\color{yellow}paso} a {\color{yellow}paso}}
\begin{columns}[c]
\column{0.5\textwidth}
\begin{tikzpicture}
	\begin{scope}[every node/.style={draw,anchor=text,rectangle split,rectangle split parts=10, 
rectangle split part fill={green!0,blue!0,blue!0,blue!0,blue!0,blue!0,blue!0,blue!0,blue!0,red!0,red!0},minimum width=2.6cm}]
    	\node (R) at (2,6)
    	{
    		\nodepart{one}{}
    		\nodepart{two}{6}
    		\nodepart{three}{4}	
    		\nodepart{four}{3}
    		\nodepart{five}{2.50}
    		\nodepart{six}{}
    		\nodepart{seven}{2.50}
    		\nodepart{eight}{}
    		\nodepart{nine}{$\vdots$}
    		\nodepart{ten}{}
    	};
	\end{scope}
	\draw [ultra thick,color=white](0.8, 4.6) -- (3.37, 4.6);
	\draw [ultra thick,color=white](0.8, 3.92) -- (3.37, 3.92);
	\fill[color=green] (5.8,6.4) -- (8.4,6.4) -- (8.4,6) -- (5.8,6) -- cycle;
  	\node at (2.1,9) {\normalsize Arquitectura X86-\textcolor{red}{32} bits };
  	\node at (2.1,7.9) {\small $Como$ $\color{red}contador$ $no$ $es$ $\color{red}<=$ $a$ $\color{red}i$};
  	\node at (2.1,7.5) {\small $sale$ $del$ $loop$ $\color{red}while$};
  	\node at (2.1,9) {\normalsize Arquitectura X86-\textcolor{red}{32} bits };
  	\node at (8,9.7) {\normalsize {\color{red!50}$C\acute{o}digo$ $del$ $programa$ $fuente$}};
  	\node at (4,5.75) {\textcolor{blue}{factorial}};
  	\node at (4,5.4){\textcolor{blue}{contador}};
  	\node at (3.6,5.05) {\textcolor{blue}{i}};
  	\node at (3.9,4.6) {\textcolor{blue}{nEuler}};
  	\node at (4.05,4) {\textcolor{blue}{nEulerOld}};
  	\node at (0.1,6.1) {0x3F00480\textcolor{red}{0}};
  	\node at (0.1,5.75) {0x3F00480\textcolor{red}{4}};
  	\node at (0.1,5.4) {0x3F00480\textcolor{red}{8}};
  	\node at (0.1,5.05) {0x3F00480\textcolor{red}{C}};
  	\node at (0.1,4.7) {0x3F0048\textcolor{red}{10}};
  	\node at (0.1,4.37) {0x3F0048\textcolor{red}{14}};
  	\node at (0.1,4.03) {0x3F0048\textcolor{red}{18}};
  	\node at (0.1,3.67) {0x3F0048\textcolor{red}{1C}};
  	\draw[decorate,decoration={brace,mirror,raise=6pt,amplitude=8pt}, thick]
    (3.3,6.1) -- (0.9,6.1)node [black,midway,xshift=0.0cm,yshift=0.7cm]{\footnotesize {\color{red}$4$ $bytes$}};
\end{tikzpicture}
\column{0.55\textwidth}
\lstset{basicstyle=\tiny}
\begin{lstlisting}[escapechar=\|,label=noint]
#include <stdio.h>
#include <math.h>

#define LIMITE   pow(10,-9)
int  main(void)
{
int		factorial = 1,contador =1,i;    
double	nEuler=0,nEulerOld=-1;

for(i=0;(nEuler-nEulerOld)>= LIMITE;i++)
  }
     nEulerOld=nEuler;
     while(contador <= i)
     { 
     	factorial = factorial * contador;
     	contador++;
     }
     nEuler = nEuler + (1/(double)factorial); 
        
     factorial = 1;
     contador =1;
  }
  printf("e es %0.10f \n",nEuler);
  printf("e(lib math.h)es %0.10f\n",M_E);
  return 0;
}
\end{lstlisting}
\end{columns}
\end{frame}

\begin{frame}[fragile]
\fontsize{6.5pt}{10pt}\selectfont
\frametitle{Recorriendo el programa {\color{yellow}paso} a {\color{yellow}paso}}
\begin{columns}[c]
\column{0.5\textwidth}
\begin{tikzpicture}
	\begin{scope}[every node/.style={draw,anchor=text,rectangle split,rectangle split parts=10, 
rectangle split part fill={green!0,blue!0,blue!0,blue!0,blue!0,blue!0,blue!0,blue!0,blue!0,red!0,red!0},minimum width=2.6cm}]
    	\node (R) at (2,6)
    	{
    		\nodepart{one}{}
    		\nodepart{two}{6}
    		\nodepart{three}{4}	
    		\nodepart{four}{3}
    		\nodepart{five}{\color{red}2.666....}
    		\nodepart{six}{}
    		\nodepart{seven}{2.50}
    		\nodepart{eight}{}
    		\nodepart{nine}{$\vdots$}
    		\nodepart{ten}{}
    	};
	\end{scope}
	\draw [ultra thick,color=white](0.8, 4.6) -- (3.37, 4.6);
	\draw [ultra thick,color=white](0.8, 3.92) -- (3.37, 3.92);
	\fill[color=green] (5.6,5.2) -- (11,5.2) -- (11,4.8) -- (5.6,4.8) -- cycle;
  	\node at (2.1,7.9) {\small $almacena$ $el$ $resultado$ $en$ $la$};
  	\node at (2.1,7.5) {\small $variable$ $\color{red}nEuler$};
  	\node at (2.1,9) {\normalsize Arquitectura X86-\textcolor{red}{32} bits };
  	\node at (8,9.7) {\normalsize {\color{red!50}$C\acute{o}digo$ $del$ $programa$ $fuente$}};
  	\node at (4,5.75) {\textcolor{blue}{factorial}};
  	\node at (4,5.4){\textcolor{blue}{contador}};
  	\node at (3.6,5.05) {\textcolor{blue}{i}};
  	\node at (3.9,4.6) {\textcolor{blue}{nEuler}};
  	\node at (4.05,4) {\textcolor{blue}{nEulerOld}};
  	\node at (0.1,6.1) {0x3F00480\textcolor{red}{0}};
  	\node at (0.1,5.75) {0x3F00480\textcolor{red}{4}};
  	\node at (0.1,5.4) {0x3F00480\textcolor{red}{8}};
  	\node at (0.1,5.05) {0x3F00480\textcolor{red}{C}};
  	\node at (0.1,4.7) {0x3F0048\textcolor{red}{10}};
  	\node at (0.1,4.37) {0x3F0048\textcolor{red}{14}};
  	\node at (0.1,4.03) {0x3F0048\textcolor{red}{18}};
  	\node at (0.1,3.67) {0x3F0048\textcolor{red}{1C}};
  	\draw[decorate,decoration={brace,mirror,raise=6pt,amplitude=8pt}, thick]
    (3.3,6.1) -- (0.9,6.1)node [black,midway,xshift=0.0cm,yshift=0.7cm]{\footnotesize {\color{red}$4$ $bytes$}};
  	\draw[decorate,decoration={brace,raise=3pt,amplitude=3pt}, thick]
    (7,5.1) -- (10.7,5.1)node [black,midway,xshift=1.5cm,yshift=0.4cm]{\scriptsize {\color{red}$2.50+\frac{1}{6.00}=2.6\wideparen{6}$ }};
\end{tikzpicture}
\column{0.55\textwidth}
\lstset{basicstyle=\tiny}
\begin{lstlisting}[escapechar=\|,label=noint]
#include <stdio.h>
#include <math.h>

#define LIMITE   pow(10,-9)
int  main(void)
{
int		factorial = 1,contador =1,i;    
double	nEuler=0,nEulerOld=-1;

for(i=0;(nEuler-nEulerOld)>= LIMITE;i++)
  }
     nEulerOld=nEuler;
     while(contador <= i)
     { 
     	factorial = factorial * contador;
     	contador++;
     }
     nEuler = nEuler + (1/(double)factorial); 
        
     factorial = 1;
     contador =1;
  }
  printf("e es %0.10f \n",nEuler);
  printf("e(lib math.h)es %0.10f\n",M_E);
  return 0;
}
\end{lstlisting}
\end{columns}
\end{frame}

\begin{frame}[fragile]
\fontsize{6.5pt}{10pt}\selectfont
\frametitle{Recorriendo el programa {\color{yellow}paso} a {\color{yellow}paso}}
\begin{columns}[c]
\column{0.5\textwidth}
\begin{tikzpicture}
	\begin{scope}[every node/.style={draw,anchor=text,rectangle split,rectangle split parts=10, 
rectangle split part fill={green!0,blue!0,blue!0,blue!0,blue!0,blue!0,blue!0,blue!0,blue!0,red!0,red!0},minimum width=2.6cm}]
    	\node (R) at (2,6)
    	{
    		\nodepart{one}{}
    		\nodepart{two}{\color{red}1}
    		\nodepart{three}{\color{red}1}	
    		\nodepart{four}{3}
    		\nodepart{five}{2.666....}
    		\nodepart{six}{}
    		\nodepart{seven}{2.50}
    		\nodepart{eight}{}
    		\nodepart{nine}{$\vdots$}
    		\nodepart{ten}{}
    	};
	\end{scope}
	\draw [ultra thick,color=white](0.8, 4.6) -- (3.37, 4.6);
	\draw [ultra thick,color=white](0.8, 3.92) -- (3.37, 3.92);
	\fill[color=green] (5.6,4.6) -- (8,4.6) -- (8,4.1) -- (5.6,4.1) -- cycle;
  	\node at (2.1,7.9) {\small $inicializo$ $nuevamente$ $las$};
  	\node at (2.1,7.5) {\small $variables$ $\color{red}factorial$ y $\color{red}contador$};
   	\node at (2.1,9) {\normalsize Arquitectura X86-\textcolor{red}{32} bits };
  	\node at (8,9.7) {\normalsize {\color{red!50}$C\acute{o}digo$ $del$ $programa$ $fuente$}};
  	\node at (4,5.75) {\textcolor{blue}{factorial}};
  	\node at (4,5.4){\textcolor{blue}{contador}};
  	\node at (3.6,5.05) {\textcolor{blue}{i}};
  	\node at (3.9,4.6) {\textcolor{blue}{nEuler}};
  	\node at (4.05,4) {\textcolor{blue}{nEulerOld}};
  	\node at (0.1,6.1) {0x3F00480\textcolor{red}{0}};
  	\node at (0.1,5.75) {0x3F00480\textcolor{red}{4}};
  	\node at (0.1,5.4) {0x3F00480\textcolor{red}{8}};
  	\node at (0.1,5.05) {0x3F00480\textcolor{red}{C}};
  	\node at (0.1,4.7) {0x3F0048\textcolor{red}{10}};
  	\node at (0.1,4.37) {0x3F0048\textcolor{red}{14}};
  	\node at (0.1,4.03) {0x3F0048\textcolor{red}{18}};
  	\node at (0.1,3.67) {0x3F0048\textcolor{red}{1C}};
  	\draw[decorate,decoration={brace,mirror,raise=6pt,amplitude=8pt}, thick]
    (3.3,6.1) -- (0.9,6.1)node [black,midway,xshift=0.0cm,yshift=0.7cm]{\footnotesize {\color{red}$4$ $bytes$}};
\end{tikzpicture}
\column{0.55\textwidth}
\lstset{basicstyle=\tiny}
\begin{lstlisting}[escapechar=\|,label=noint]
#include <stdio.h>
#include <math.h>

#define LIMITE   pow(10,-9)
int  main(void)
{
int		factorial = 1,contador =1,i;    
double	nEuler=0,nEulerOld=-1;

for(i=0;(nEuler-nEulerOld)>= LIMITE;i++)
  }
     nEulerOld=nEuler;
     while(contador <= i)
     { 
     	factorial = factorial * contador;
     	contador++;
     }
     nEuler = nEuler + (1/(double)factorial); 
        
     factorial = 1;
     contador =1;
  }
  printf("e es %0.10f \n",nEuler);
  printf("e(lib math.h)es %0.10f\n",M_E);
  return 0;
}
\end{lstlisting}
\end{columns}
\end{frame}

\begin{frame}[fragile]
\fontsize{6.5pt}{10pt}\selectfont
\frametitle{Recorriendo el programa {\color{yellow}paso} a {\color{yellow}paso}}
\begin{columns}[c]
\column{0.5\textwidth}
\begin{tikzpicture}
	\begin{scope}[every node/.style={draw,anchor=text,rectangle split,rectangle split parts=10, 
rectangle split part fill={green!0,blue!0,blue!0,blue!0,blue!0,blue!0,blue!0,blue!0,blue!0,red!0,red!0},minimum width=2.6cm}]
    	\node (R) at (2,6)
    	{
    		\nodepart{one}{}
    		\nodepart{two}{1}
    		\nodepart{three}{1}	
    		\nodepart{four}{\color{red}4}
    		\nodepart{five}{2.666....}
    		\nodepart{six}{}
    		\nodepart{seven}{2.50}
    		\nodepart{eight}{}
    		\nodepart{nine}{$\vdots$}
    		\nodepart{ten}{}
    	};
	\end{scope}
	\draw [ultra thick,color=white](0.8, 4.6) -- (3.37, 4.6);
	\draw [ultra thick,color=white](0.8, 3.92) -- (3.37, 3.92);
	\fill[color=green] (9.7,7.2) -- (10.3,7.2) -- (10.3,6.8) -- (9.7,6.8) -- cycle;
  	\node at (2.1,7.9) {\small $incremento$ $la$ $variable$ $\color{red}i$};
  	\node at (2.1,9) {\normalsize Arquitectura X86-\textcolor{red}{32} bits };
  	\node at (8,9.7) {\normalsize {\color{red!50}$C\acute{o}digo$ $del$ $programa$ $fuente$}};
  	\node at (4,5.75) {\textcolor{blue}{factorial}};
  	\node at (4,5.4){\textcolor{blue}{contador}};
  	\node at (3.6,5.05) {\textcolor{blue}{i}};
  	\node at (3.9,4.6) {\textcolor{blue}{nEuler}};
  	\node at (4.05,4) {\textcolor{blue}{nEulerOld}};
  	\node at (0.1,6.1) {0x3F00480\textcolor{red}{0}};
  	\node at (0.1,5.75) {0x3F00480\textcolor{red}{4}};
  	\node at (0.1,5.4) {0x3F00480\textcolor{red}{8}};
  	\node at (0.1,5.05) {0x3F00480\textcolor{red}{C}};
  	\node at (0.1,4.7) {0x3F0048\textcolor{red}{10}};
  	\node at (0.1,4.37) {0x3F0048\textcolor{red}{14}};
  	\node at (0.1,4.03) {0x3F0048\textcolor{red}{18}};
  	\node at (0.1,3.67) {0x3F0048\textcolor{red}{1C}};
  	\draw[decorate,decoration={brace,mirror,raise=6pt,amplitude=8pt}, thick]
    (3.3,6.1) -- (0.9,6.1)node [black,midway,xshift=0.0cm,yshift=0.7cm]{\footnotesize {\color{red}$4$ $bytes$}};
\end{tikzpicture}
\column{0.55\textwidth}
\lstset{basicstyle=\tiny}
\begin{lstlisting}[escapechar=\|,label=noint]
#include <stdio.h>
#include <math.h>

#define LIMITE   pow(10,-9)
int  main(void)
{
int		factorial = 1,contador =1,i;    
double	nEuler=0,nEulerOld=-1;

for(i=0;(nEuler-nEulerOld)>= LIMITE;i++)
  }
     nEulerOld=nEuler;
     while(contador <= i)
     { 
     	factorial = factorial * contador;
     	contador++;
     }
     nEuler = nEuler + (1/(double)factorial); 
        
     factorial = 1;
     contador =1;
  }
  printf("e es %0.10f \n",nEuler);
  printf("e(lib math.h)es %0.10f\n",M_E);
  return 0;
}
\end{lstlisting}
\end{columns}
\end{frame}

\begin{frame}[fragile]
\fontsize{6.5pt}{10pt}\selectfont
\frametitle{Recorriendo el programa {\color{yellow}paso} a {\color{yellow}paso}}
\begin{columns}[c]
\column{0.5\textwidth}
\begin{tikzpicture}
	\begin{scope}[every node/.style={draw,anchor=text,rectangle split,rectangle split parts=10, 
rectangle split part fill={green!0,blue!0,blue!0,blue!0,blue!0,blue!0,blue!0,blue!0,blue!0,red!0,red!0},minimum width=2.6cm}]
    	\node (R) at (2,6)
    	{
    		\nodepart{one}{}
    		\nodepart{two}{1}
    		\nodepart{three}{1}	
    		\nodepart{four}{4}
    		\nodepart{five}{2.666....}
    		\nodepart{six}{}
    		\nodepart{seven}{2.50}
    		\nodepart{eight}{}
    		\nodepart{nine}{$\vdots$}
    		\nodepart{ten}{}
    	};
	\end{scope}
	\draw [ultra thick,color=white](0.8, 4.6) -- (3.37, 4.6);
	\draw [ultra thick,color=white](0.8, 3.92) -- (3.37, 3.92);
	\fill[color=green] (6.4,7.2) -- (9.7,7.2) -- (9.7,6.8) -- (6.4,6.8) -- cycle;
  	\node at (2.1,7.9) {\small $Pregunta$ $condici\acute{o}n$ $del$ $\color{red}for$};
   	\node at (2.1,9) {\normalsize Arquitectura X86-\textcolor{red}{32} bits };
  	\node at (8,9.7) {\normalsize {\color{red!50}$C\acute{o}digo$ $del$ $programa$ $fuente$}};
  	\node at (4,5.75) {\textcolor{blue}{factorial}};
  	\node at (4,5.4){\textcolor{blue}{contador}};
  	\node at (3.6,5.05) {\textcolor{blue}{i}};
  	\node at (3.9,4.6) {\textcolor{blue}{nEuler}};
  	\node at (4.05,4) {\textcolor{blue}{nEulerOld}};
  	\node at (0.1,6.1) {0x3F00480\textcolor{red}{0}};
  	\node at (0.1,5.75) {0x3F00480\textcolor{red}{4}};
  	\node at (0.1,5.4) {0x3F00480\textcolor{red}{8}};
  	\node at (0.1,5.05) {0x3F00480\textcolor{red}{C}};
  	\node at (0.1,4.7) {0x3F0048\textcolor{red}{10}};
  	\node at (0.1,4.37) {0x3F0048\textcolor{red}{14}};
  	\node at (0.1,4.03) {0x3F0048\textcolor{red}{18}};
  	\node at (0.1,3.67) {0x3F0048\textcolor{red}{1C}};
  	\draw[decorate,decoration={brace,mirror,raise=6pt,amplitude=8pt}, thick]
    (3.3,6.1) -- (0.9,6.1)node [black,midway,xshift=0.0cm,yshift=0.7cm]{\footnotesize {\color{red}$4$ $bytes$}};
  	\draw[decorate,decoration={brace,raise=4pt,amplitude=4pt}, thick]
    (7.35,8.4)--(8.7,8.4)node [black,midway,xshift=1.0cm,yshift=0.4cm]{\footnotesize {\color{red!60}$10^{-9}$}};
  	\draw[red,->](9.3,7.1)to [out=10,in=0]node[right,midway]{} ++(-0.2,1.3) ; 
\end{tikzpicture}
\column{0.55\textwidth}
\lstset{basicstyle=\tiny}
\begin{lstlisting}[escapechar=\|,label=noint]
#include <stdio.h>
#include <math.h>

#define LIMITE   pow(10,-9)
int  main(void)
{
int		factorial = 1,contador =1,i;    
double	nEuler=0,nEulerOld=-1;

for(i=0;(nEuler-nEulerOld)>= LIMITE;i++)
  }
     nEulerOld=nEuler;
     while(contador <= i)
     { 
     	factorial = factorial * contador;
     	contador++;
     }
     nEuler = nEuler + (1/(double)factorial); 
        
     factorial = 1;
     contador =1;
  }
  printf("e es %0.10f \n",nEuler);
  printf("e(lib math.h)es %0.10f\n",M_E);
  return 0;
}
\end{lstlisting}
\end{columns}
\end{frame}

\begin{frame}[fragile]
\fontsize{6.5pt}{10pt}\selectfont
\frametitle{Recorriendo el programa {\color{yellow}paso} a {\color{yellow}paso}}
\begin{columns}[c]
\column{0.5\textwidth}
\begin{tikzpicture}
	\begin{scope}[every node/.style={draw,anchor=text,rectangle split,rectangle split parts=10, 
rectangle split part fill={green!0,blue!0,blue!0,blue!0,blue!0,blue!0,blue!0,blue!0,blue!0,red!0,red!0},minimum width=2.6cm}]
    	\node (R) at (2,6)
    	{
    		\nodepart{one}{}
    		\nodepart{two}{1}
    		\nodepart{three}{1}	
    		\nodepart{four}{4}
    		\nodepart{five}{2.666....}
    		\nodepart{six}{}
    		\nodepart{seven}{\color{red}2.666....}
    		\nodepart{eight}{}
    		\nodepart{nine}{$\vdots$}
    		\nodepart{ten}{}
    	};
	\end{scope}
	\draw [ultra thick,color=white](0.8, 4.6) -- (3.37, 4.6);
	\draw [ultra thick,color=white](0.8, 3.92) -- (3.37, 3.92);
	\fill[color=green] (5.8,6.65) -- (8,6.65) -- (8,6.35) -- (5.8,6.35) -- cycle;
  	\node at (2.1,7.9) {\small $Para$ $conservar$ $el$ $\color{red}e$ $anterior$};
   	\node at (2.1,9) {\normalsize Arquitectura X86-\textcolor{red}{32} bits };
  	\node at (8,9.7) {\normalsize {\color{red!50}$C\acute{o}digo$ $del$ $programa$ $fuente$}};
  	\node at (4,5.75) {\textcolor{blue}{factorial}};
  	\node at (4,5.4){\textcolor{blue}{contador}};
  	\node at (3.6,5.05) {\textcolor{blue}{i}};
  	\node at (3.9,4.6) {\textcolor{blue}{nEuler}};
  	\node at (4.05,4) {\textcolor{blue}{nEulerOld}};
  	\node at (0.1,6.1) {0x3F00480\textcolor{red}{0}};
  	\node at (0.1,5.75) {0x3F00480\textcolor{red}{4}};
  	\node at (0.1,5.4) {0x3F00480\textcolor{red}{8}};
  	\node at (0.1,5.05) {0x3F00480\textcolor{red}{C}};
  	\node at (0.1,4.7) {0x3F0048\textcolor{red}{10}};
  	\node at (0.1,4.37) {0x3F0048\textcolor{red}{14}};
  	\node at (0.1,4.03) {0x3F0048\textcolor{red}{18}};
  	\node at (0.1,3.67) {0x3F0048\textcolor{red}{1C}};
  	\draw[decorate,decoration={brace,mirror,raise=6pt,amplitude=8pt}, thick]
    (3.3,6.1) -- (0.9,6.1)node [black,midway,xshift=0.0cm,yshift=0.7cm]{\footnotesize {\color{red}$4$ $bytes$}};
\end{tikzpicture}
\column{0.55\textwidth}
\lstset{basicstyle=\tiny}
\begin{lstlisting}[escapechar=\|,label=noint]
#include <stdio.h>
#include <math.h>

#define LIMITE   pow(10,-9)
int  main(void)
{
int		factorial = 1,contador =1,i;    
double	nEuler=0,nEulerOld=-1;

for(i=0;(nEuler-nEulerOld)>= LIMITE;i++)
  }
     nEulerOld=nEuler;
     while(contador <= i)
     { 
     	factorial = factorial * contador;
     	contador++;
     }
     nEuler = nEuler + (1/(double)factorial); 
        
     factorial = 1;
     contador =1;
  }
  printf("e es %0.10f \n",nEuler);
  printf("e(lib math.h)es %0.10f\n",M_E);
  return 0;
}
\end{lstlisting}
\end{columns}
\end{frame}

\begin{frame}[fragile]
\fontsize{6.5pt}{10pt}\selectfont
\frametitle{Recorriendo el programa {\color{yellow}paso} a {\color{yellow}paso}}
\begin{columns}[c]
\column{0.5\textwidth}
\begin{tikzpicture}
	\begin{scope}[every node/.style={draw,anchor=text,rectangle split,rectangle split parts=10, 
rectangle split part fill={green!0,blue!0,blue!0,blue!0,blue!0,blue!0,blue!0,blue!0,blue!0,red!0,red!0},minimum width=2.6cm}]
    	\node (R) at (2,6)
    	{
    		\nodepart{one}{}
    		\nodepart{two}{1}
    		\nodepart{three}{1}	
    		\nodepart{four}{4}
    		\nodepart{five}{2.666....}
    		\nodepart{six}{}
    		\nodepart{seven}{2.666....}
    		\nodepart{eight}{}
    		\nodepart{nine}{$\vdots$}
    		\nodepart{ten}{}
    	};
	\end{scope}
	\draw [ultra thick,color=white](0.8, 4.6) -- (3.37, 4.6);
	\draw [ultra thick,color=white](0.8, 3.92) -- (3.37, 3.92);
	\fill[color=green] (5.8,6.4) -- (8.4,6.4) -- (8.4,6) -- (5.8,6) -- cycle;
  	\node at (2.1,7.9) {\small $Como$ $\color{red}contador$ $es$ $\color{red}<=$ $a$ $\color{red}i$};
  	\node at (2.1,7.5) {\small $entra$ $dentro$ $del$ $\color{red}while$}; 
  	\node at (2.1,9) {\normalsize Arquitectura X86-\textcolor{red}{32} bits };
  	\node at (8,9.7) {\normalsize {\color{red!50}$C\acute{o}digo$ $del$ $programa$ $fuente$}};
  	\node at (4,5.75) {\textcolor{blue}{factorial}};
  	\node at (4,5.4){\textcolor{blue}{contador}};
  	\node at (3.6,5.05) {\textcolor{blue}{i}};
  	\node at (3.9,4.6) {\textcolor{blue}{nEuler}};
  	\node at (4.05,4) {\textcolor{blue}{nEulerOld}};
  	\node at (0.1,6.1) {0x3F00480\textcolor{red}{0}};
  	\node at (0.1,5.75) {0x3F00480\textcolor{red}{4}};
  	\node at (0.1,5.4) {0x3F00480\textcolor{red}{8}};
  	\node at (0.1,5.05) {0x3F00480\textcolor{red}{C}};
  	\node at (0.1,4.7) {0x3F0048\textcolor{red}{10}};
  	\node at (0.1,4.37) {0x3F0048\textcolor{red}{14}};
  	\node at (0.1,4.03) {0x3F0048\textcolor{red}{18}};
  	\node at (0.1,3.67) {0x3F0048\textcolor{red}{1C}};
  	\draw[decorate,decoration={brace,mirror,raise=6pt,amplitude=8pt}, thick]
    (3.3,6.1) -- (0.9,6.1)node [black,midway,xshift=0.0cm,yshift=0.7cm]{\footnotesize {\color{red}$4$ $bytes$}};
\end{tikzpicture}
\column{0.55\textwidth}
\lstset{basicstyle=\tiny}
\begin{lstlisting}[escapechar=\|,label=noint]
#include <stdio.h>
#include <math.h>

#define LIMITE   pow(10,-9)
int  main(void)
{
int		factorial = 1,contador =1,i;    
double	nEuler=0,nEulerOld=-1;

for(i=0;(nEuler-nEulerOld)>= LIMITE;i++)
  }
     nEulerOld=nEuler;
     while(contador <= i)
     { 
     	factorial = factorial * contador;
     	contador++;
     }
     nEuler = nEuler + (1/(double)factorial); 
        
     factorial = 1;
     contador =1;
  }
  printf("e es %0.10f \n",nEuler);
  printf("e(lib math.h)es %0.10f\n",M_E);
  return 0;
}
\end{lstlisting}
\end{columns}
\end{frame}

\begin{frame}[fragile]
\fontsize{6.5pt}{10pt}\selectfont
\frametitle{Recorriendo el programa {\color{yellow}paso} a {\color{yellow}paso}}
\begin{columns}[c]
\column{0.5\textwidth}
\begin{tikzpicture}
	\begin{scope}[every node/.style={draw,anchor=text,rectangle split,rectangle split parts=10, 
rectangle split part fill={green!0,blue!0,blue!0,blue!0,blue!0,blue!0,blue!0,blue!0,blue!0,red!0,red!0},minimum width=2.6cm}]
    	\node (R) at (2,6)
    	{
    		\nodepart{one}{}
    		\nodepart{two}{\color{red}1}
    		\nodepart{three}{1}	
    		\nodepart{four}{4}
    		\nodepart{five}{2.666....}
    		\nodepart{six}{}
    		\nodepart{seven}{2.666....}
    		\nodepart{eight}{}
    		\nodepart{nine}{$\vdots$}
    		\nodepart{ten}{}
    	};
	\end{scope}
	\draw [ultra thick,color=white](0.8, 4.6) -- (3.37, 4.6);
	\draw [ultra thick,color=white](0.8, 3.92) -- (3.37, 3.92);
	\fill[color=green] (5.8,5.9) -- (10.5,5.9) -- (10.5,5.6) -- (5.8,5.6) -- cycle;
  	\node at (2.1,7.9) {\small $almacena$ $el$ $resultado$ $en$ $la$};
  	\node at (2.1,7.5) {\small $variable$ $\color{red}factorial$};
  	\node at (2.1,9) {\normalsize Arquitectura X86-\textcolor{red}{32} bits };
  	\node at (8,9.7) {\normalsize {\color{red!50}$C\acute{o}digo$ $del$ $programa$ $fuente$}};
  	\node at (4,5.75) {\textcolor{blue}{factorial}};
  	\node at (4,5.4){\textcolor{blue}{contador}};
  	\node at (3.6,5.05) {\textcolor{blue}{i}};
  	\node at (3.9,4.6) {\textcolor{blue}{nEuler}};
  	\node at (4.05,4) {\textcolor{blue}{nEulerOld}};
  	\node at (0.1,6.1) {0x3F00480\textcolor{red}{0}};
  	\node at (0.1,5.75) {0x3F00480\textcolor{red}{4}};
  	\node at (0.1,5.4) {0x3F00480\textcolor{red}{8}};
  	\node at (0.1,5.05) {0x3F00480\textcolor{red}{C}};
  	\node at (0.1,4.7) {0x3F0048\textcolor{red}{10}};
  	\node at (0.1,4.37) {0x3F0048\textcolor{red}{14}};
  	\node at (0.1,4.03) {0x3F0048\textcolor{red}{18}};
  	\node at (0.1,3.67) {0x3F0048\textcolor{red}{1C}};
  	\draw[decorate,decoration={brace,mirror,raise=6pt,amplitude=8pt}, thick]
    (3.3,6.1) -- (0.9,6.1)node [black,midway,xshift=0.0cm,yshift=0.7cm]{\footnotesize {\color{red}$4$ $bytes$}};
  	\draw[decorate,decoration={brace,raise=3pt,amplitude=3pt}, thick]
    (7.5,5.8)--(10.2,5.8)node [black,midway,xshift=1.5cm,yshift=0.4cm]{\footnotesize{\color{red}$1\cdot 1=1$ }};
\end{tikzpicture}
\column{0.55\textwidth}
\lstset{basicstyle=\tiny}
\begin{lstlisting}[escapechar=\|,label=noint]
#include <stdio.h>
#include <math.h>

#define LIMITE   pow(10,-9)
int  main(void)
{
int		factorial = 1,contador =1,i;    
double	nEuler=0,nEulerOld=-1;

for(i=0;(nEuler-nEulerOld)>= LIMITE;i++)
  }
     nEulerOld=nEuler;
     while(contador <= i)
     { 
     	factorial = factorial * contador;
     	contador++;
     }
     nEuler = nEuler + (1/(double)factorial); 
        
     factorial = 1;
     contador =1;
  }
  printf("e es %0.10f \n",nEuler);
  printf("e(lib math.h)es %0.10f\n",M_E);
  return 0;
}
\end{lstlisting}
\end{columns}
\end{frame}

\begin{frame}[fragile]
\fontsize{6.5pt}{10pt}\selectfont
\frametitle{Recorriendo el programa {\color{yellow}paso} a {\color{yellow}paso}}
\begin{columns}[c]
\column{0.5\textwidth}
\begin{tikzpicture}
	\begin{scope}[every node/.style={draw,anchor=text,rectangle split,rectangle split parts=10, 
rectangle split part fill={green!0,blue!0,blue!0,blue!0,blue!0,blue!0,blue!0,blue!0,blue!0,red!0,red!0},minimum width=2.6cm}]
    	\node (R) at (2,6)
    	{
    		\nodepart{one}{}
    		\nodepart{two}{1}
    		\nodepart{three}{\color{red}2}	
    		\nodepart{four}{4}
    		\nodepart{five}{2.666....}
    		\nodepart{six}{}
    		\nodepart{seven}{2.666....}
    		\nodepart{eight}{}
    		\nodepart{nine}{$\vdots$}
    		\nodepart{ten}{}
    	};
	\end{scope}
	\draw [ultra thick,color=white](0.8, 4.6) -- (3.37, 4.6);
	\draw [ultra thick,color=white](0.8, 3.92) -- (3.37, 3.92);
	\fill[color=green] (5.8,5.6) -- (8,5.6) -- (8,5.3) -- (5.8,5.3) -- cycle;
  	\node at (2.1,7.9) {\small $incremento$ $la$ $variable$ $\color{red}contador$};
  	\node at (2.1,9) {\normalsize Arquitectura X86-\textcolor{red}{32} bits };
  	\node at (8,9.7) {\normalsize {\color{red!50}$C\acute{o}digo$ $del$ $programa$ $fuente$}};
  	\node at (4,5.75) {\textcolor{blue}{factorial}};
  	\node at (4,5.4){\textcolor{blue}{contador}};
  	\node at (3.6,5.05) {\textcolor{blue}{i}};
  	\node at (3.9,4.6) {\textcolor{blue}{nEuler}};
  	\node at (4.05,4) {\textcolor{blue}{nEulerOld}};
  	\node at (0.1,6.1) {0x3F00480\textcolor{red}{0}};
  	\node at (0.1,5.75) {0x3F00480\textcolor{red}{4}};
  	\node at (0.1,5.4) {0x3F00480\textcolor{red}{8}};
  	\node at (0.1,5.05) {0x3F00480\textcolor{red}{C}};
  	\node at (0.1,4.7) {0x3F0048\textcolor{red}{10}};
  	\node at (0.1,4.37) {0x3F0048\textcolor{red}{14}};
  	\node at (0.1,4.03) {0x3F0048\textcolor{red}{18}};
  	\node at (0.1,3.67) {0x3F0048\textcolor{red}{1C}};
  	\draw[decorate,decoration={brace,mirror,raise=6pt,amplitude=8pt}, thick]
    (3.3,6.1) -- (0.9,6.1)node [black,midway,xshift=0.0cm,yshift=0.7cm]{\footnotesize {\color{red}$4$ $bytes$}};
\end{tikzpicture}
\column{0.55\textwidth}
\lstset{basicstyle=\tiny}
\begin{lstlisting}[escapechar=\|,label=noint]
#include <stdio.h>
#include <math.h>

#define LIMITE   pow(10,-9)
int  main(void)
{
int		factorial = 1,contador =1,i;    
double	nEuler=0,nEulerOld=-1;

for(i=0;(nEuler-nEulerOld)>= LIMITE;i++)
  }
     nEulerOld=nEuler;
     while(contador <= i)
     { 
     	factorial = factorial * contador;
     	contador++;
     }
     nEuler = nEuler + (1/(double)factorial); 
        
     factorial = 1;
     contador =1;
  }
  printf("e es %0.10f \n",nEuler);
  printf("e(lib math.h)es %0.10f\n",M_E);
  return 0;
}
\end{lstlisting}
\end{columns}
\end{frame}

\begin{frame}[fragile]
\fontsize{6.5pt}{10pt}\selectfont
\frametitle{Recorriendo el programa {\color{yellow}paso} a {\color{yellow}paso}}
\begin{columns}[c]
\column{0.5\textwidth}
\begin{tikzpicture}
	\begin{scope}[every node/.style={draw,anchor=text,rectangle split,rectangle split parts=10, 
rectangle split part fill={green!0,blue!0,blue!0,blue!0,blue!0,blue!0,blue!0,blue!0,blue!0,red!0,red!0},minimum width=2.6cm}]
    	\node (R) at (2,6)
    	{
    		\nodepart{one}{}
    		\nodepart{two}{1}
    		\nodepart{three}{2}	
    		\nodepart{four}{4}
    		\nodepart{five}{2.666....}
    		\nodepart{six}{}
    		\nodepart{seven}{2.666....}
    		\nodepart{eight}{}
    		\nodepart{nine}{$\vdots$}
    		\nodepart{ten}{}
    	};
	\end{scope}
	\draw [ultra thick,color=white](0.8, 4.6) -- (3.37, 4.6);
	\draw [ultra thick,color=white](0.8, 3.92) -- (3.37, 3.92);
	\fill[color=green] (5.8,6.4) -- (8.4,6.4) -- (8.4,6) -- (5.8,6) -- cycle;
  	\node at (2.1,7.9) {\small $Como$ $\color{red}contador$ $es$ $\color{red}<=$ $a$ $\color{red}i$};
  	\node at (2.1,7.5) {\small $entra$ $dentro$ $del$ $\color{red}while$}; 
  	\node at (2.1,9) {\normalsize Arquitectura X86-\textcolor{red}{32} bits };
  	\node at (8,9.7) {\normalsize {\color{red!50}$C\acute{o}digo$ $del$ $programa$ $fuente$}};
  	\node at (4,5.75) {\textcolor{blue}{factorial}};
  	\node at (4,5.4){\textcolor{blue}{contador}};
  	\node at (3.6,5.05) {\textcolor{blue}{i}};
  	\node at (3.9,4.6) {\textcolor{blue}{nEuler}};
  	\node at (4.05,4) {\textcolor{blue}{nEulerOld}};
  	\node at (0.1,6.1) {0x3F00480\textcolor{red}{0}};
  	\node at (0.1,5.75) {0x3F00480\textcolor{red}{4}};
  	\node at (0.1,5.4) {0x3F00480\textcolor{red}{8}};
  	\node at (0.1,5.05) {0x3F00480\textcolor{red}{C}};
  	\node at (0.1,4.7) {0x3F0048\textcolor{red}{10}};
  	\node at (0.1,4.37) {0x3F0048\textcolor{red}{14}};
  	\node at (0.1,4.03) {0x3F0048\textcolor{red}{18}};
  	\node at (0.1,3.67) {0x3F0048\textcolor{red}{1C}};
  	\draw[decorate,decoration={brace,mirror,raise=6pt,amplitude=8pt}, thick]
    (3.3,6.1) -- (0.9,6.1)node [black,midway,xshift=0.0cm,yshift=0.7cm]{\footnotesize {\color{red}$4$ $bytes$}};
\end{tikzpicture}
\column{0.55\textwidth}
\lstset{basicstyle=\tiny}
\begin{lstlisting}[escapechar=\|,label=noint]
#include <stdio.h>
#include <math.h>

#define LIMITE   pow(10,-9)
int  main(void)
{
int		factorial = 1,contador =1,i;    
double	nEuler=0,nEulerOld=-1;

for(i=0;(nEuler-nEulerOld)>= LIMITE;i++)
  }
     nEulerOld=nEuler;
     while(contador <= i)
     { 
     	factorial = factorial * contador;
     	contador++;
     }
     nEuler = nEuler + (1/(double)factorial); 
        
     factorial = 1;
     contador =1;
  }
  printf("e es %0.10f \n",nEuler);
  printf("e(lib math.h)es %0.10f\n",M_E);
  return 0;
}
\end{lstlisting}
\end{columns}
\end{frame}

\begin{frame}[fragile]
\fontsize{6.5pt}{10pt}\selectfont
\frametitle{Recorriendo el programa {\color{yellow}paso} a {\color{yellow}paso}}
\begin{columns}[c]
\column{0.5\textwidth}
\begin{tikzpicture}
	\begin{scope}[every node/.style={draw,anchor=text,rectangle split,rectangle split parts=10, 
rectangle split part fill={green!0,blue!0,blue!0,blue!0,blue!0,blue!0,blue!0,blue!0,blue!0,red!0,red!0},minimum width=2.6cm}]
    	\node (R) at (2,6)
    	{
    		\nodepart{one}{}
    		\nodepart{two}{\color{red}2}
    		\nodepart{three}{2}	
    		\nodepart{four}{4}
    		\nodepart{five}{2.666....}
    		\nodepart{six}{}
    		\nodepart{seven}{2.666....}
    		\nodepart{eight}{}
    		\nodepart{nine}{$\vdots$}
    		\nodepart{ten}{}
    	};
	\end{scope}
	\draw [ultra thick,color=white](0.8, 4.6) -- (3.37, 4.6);
	\draw [ultra thick,color=white](0.8, 3.92) -- (3.37, 3.92);
	\fill[color=green] (5.8,5.9) -- (10.5,5.9) -- (10.5,5.6) -- (5.8,5.6) -- cycle;
	\node at (2.1,7.9) {\small $almacena$ $el$ $resultado$ $en$ $la$};
  	\node at (2.1,7.5) {\small $variable$ $\color{red}factorial$};
  	\node at (2.1,9) {\normalsize Arquitectura X86-\textcolor{red}{32} bits };
  	\node at (8,9.7) {\normalsize {\color{red!50}$C\acute{o}digo$ $del$ $programa$ $fuente$}};
  	\node at (4,5.75) {\textcolor{blue}{factorial}};
  	\node at (4,5.4){\textcolor{blue}{contador}};
  	\node at (3.6,5.05) {\textcolor{blue}{i}};
  	\node at (3.9,4.6) {\textcolor{blue}{nEuler}};
  	\node at (4.05,4) {\textcolor{blue}{nEulerOld}};
  	\node at (0.1,6.1) {0x3F00480\textcolor{red}{0}};
  	\node at (0.1,5.75) {0x3F00480\textcolor{red}{4}};
  	\node at (0.1,5.4) {0x3F00480\textcolor{red}{8}};
  	\node at (0.1,5.05) {0x3F00480\textcolor{red}{C}};
  	\node at (0.1,4.7) {0x3F0048\textcolor{red}{10}};
  	\node at (0.1,4.37) {0x3F0048\textcolor{red}{14}};
  	\node at (0.1,4.03) {0x3F0048\textcolor{red}{18}};
  	\node at (0.1,3.67) {0x3F0048\textcolor{red}{1C}};
  	\draw[decorate,decoration={brace,mirror,raise=6pt,amplitude=8pt}, thick]
    (3.3,6.1) -- (0.9,6.1)node [black,midway,xshift=0.0cm,yshift=0.7cm]{\footnotesize {\color{red}$4$ $bytes$}};
  	\draw[decorate,decoration={brace,raise=3pt,amplitude=3pt}, thick]
    (7.5,5.8)--(10.2,5.8)node [black,midway,xshift=1.5cm,yshift=0.4cm]{\footnotesize{\color{red}$1\cdot 2=2$ }};
\end{tikzpicture}
\column{0.55\textwidth}
\lstset{basicstyle=\tiny}
\begin{lstlisting}[escapechar=\|,label=noint]
#include <stdio.h>
#include <math.h>

#define LIMITE   pow(10,-9)
int  main(void)
{
int		factorial = 1,contador =1,i;    
double	nEuler=0,nEulerOld=-1;

for(i=0;(nEuler-nEulerOld)>= LIMITE;i++)
  }
     nEulerOld=nEuler;
     while(contador <= i)
     { 
     	factorial = factorial * contador;
     	contador++;
     }
     nEuler = nEuler + (1/(double)factorial); 
        
     factorial = 1;
     contador =1;
  }
  printf("e es %0.10f \n",nEuler);
  printf("e(lib math.h)es %0.10f\n",M_E);
  return 0;
}
\end{lstlisting}
\end{columns}
\end{frame}

\begin{frame}[fragile]
\fontsize{6.5pt}{10pt}\selectfont
\frametitle{Recorriendo el programa {\color{yellow}paso} a {\color{yellow}paso}}
\begin{columns}[c]
\column{0.5\textwidth}
\begin{tikzpicture}
	\begin{scope}[every node/.style={draw,anchor=text,rectangle split,rectangle split parts=10, 
rectangle split part fill={green!0,blue!0,blue!0,blue!0,blue!0,blue!0,blue!0,blue!0,blue!0,red!0,red!0},minimum width=2.6cm}]
    	\node (R) at (2,6)
    	{
    		\nodepart{one}{}
    		\nodepart{two}{2}
    		\nodepart{three}{\color{red}3}	
    		\nodepart{four}{4}
    		\nodepart{five}{2.666....}
    		\nodepart{six}{}
    		\nodepart{seven}{2.666....}
    		\nodepart{eight}{}
    		\nodepart{nine}{$\vdots$}
    		\nodepart{ten}{}
    	};
	\end{scope}
	\draw [ultra thick,color=white](0.8, 4.6) -- (3.37, 4.6);
	\draw [ultra thick,color=white](0.8, 3.92) -- (3.37, 3.92);
	\fill[color=green] (5.8,5.6) -- (8,5.6) -- (8,5.3) -- (5.8,5.3) -- cycle;
	\node at (2.1,7.9) {\small $incremento$ $la$ $variable$ $\color{red}contador$};
  	\node at (2.1,9) {\normalsize Arquitectura X86-\textcolor{red}{32} bits };
  	\node at (8,9.7) {\normalsize {\color{red!50}$C\acute{o}digo$ $del$ $programa$ $fuente$}};
  	\node at (4,5.75) {\textcolor{blue}{factorial}};
  	\node at (4,5.4){\textcolor{blue}{contador}};
  	\node at (3.6,5.05) {\textcolor{blue}{i}};
  	\node at (3.9,4.6) {\textcolor{blue}{nEuler}};
  	\node at (4.05,4) {\textcolor{blue}{nEulerOld}};
  	\node at (0.1,6.1) {0x3F00480\textcolor{red}{0}};
  	\node at (0.1,5.75) {0x3F00480\textcolor{red}{4}};
  	\node at (0.1,5.4) {0x3F00480\textcolor{red}{8}};
  	\node at (0.1,5.05) {0x3F00480\textcolor{red}{C}};
  	\node at (0.1,4.7) {0x3F0048\textcolor{red}{10}};
  	\node at (0.1,4.37) {0x3F0048\textcolor{red}{14}};
  	\node at (0.1,4.03) {0x3F0048\textcolor{red}{18}};
  	\node at (0.1,3.67) {0x3F0048\textcolor{red}{1C}};
  	\draw[decorate,decoration={brace,mirror,raise=6pt,amplitude=8pt}, thick]
    (3.3,6.1) -- (0.9,6.1)node [black,midway,xshift=0.0cm,yshift=0.7cm]{\footnotesize {\color{red}$4$ $bytes$}};
\end{tikzpicture}
\column{0.55\textwidth}
\lstset{basicstyle=\tiny}
\begin{lstlisting}[escapechar=\|,label=noint]
#include <stdio.h>
#include <math.h>

#define LIMITE   pow(10,-9)
int  main(void)
{
int		factorial = 1,contador =1,i;    
double	nEuler=0,nEulerOld=-1;

for(i=0;(nEuler-nEulerOld)>= LIMITE;i++)
  }
     nEulerOld=nEuler;
     while(contador <= i)
     { 
     	factorial = factorial * contador;
     	contador++;
     }
     nEuler = nEuler + (1/(double)factorial); 
        
     factorial = 1;
     contador =1;
  }
  printf("e es %0.10f \n",nEuler);
  printf("e(lib math.h)es %0.10f\n",M_E);
  return 0;
}
\end{lstlisting}
\end{columns}
\end{frame}

\begin{frame}[fragile]
\fontsize{6.5pt}{10pt}\selectfont
\frametitle{Recorriendo el programa {\color{yellow}paso} a {\color{yellow}paso}}
\begin{columns}[c]
\column{0.5\textwidth}
\begin{tikzpicture}
	\begin{scope}[every node/.style={draw,anchor=text,rectangle split,rectangle split parts=10, 
rectangle split part fill={green!0,blue!0,blue!0,blue!0,blue!0,blue!0,blue!0,blue!0,blue!0,red!0,red!0},minimum width=2.6cm}]
    	\node (R) at (2,6)
    	{
    		\nodepart{one}{}
    		\nodepart{two}{2}
    		\nodepart{three}{3}	
    		\nodepart{four}{4}
    		\nodepart{five}{2.666....}
    		\nodepart{six}{}
    		\nodepart{seven}{2.666....}
    		\nodepart{eight}{}
    		\nodepart{nine}{$\vdots$}
    		\nodepart{ten}{}
    	};
	\end{scope}
	\draw [ultra thick,color=white](0.8, 4.6) -- (3.37, 4.6);
	\draw [ultra thick,color=white](0.8, 3.92) -- (3.37, 3.92);
	\fill[color=green] (5.8,6.4) -- (8.4,6.4) -- (8.4,6) -- (5.8,6) -- cycle;
  	\node at (2.1,7.9) {\small $Como$ $\color{red}contador$ $es$ $\color{red}<=$ $a$ $\color{red}i$};
  	\node at (2.1,7.5) {\small $entra$ $dentro$ $del$ $\color{red}while$}; 
  	\node at (2.1,9) {\normalsize Arquitectura X86-\textcolor{red}{32} bits };
  	\node at (8,9.7) {\normalsize {\color{red!50}$C\acute{o}digo$ $del$ $programa$ $fuente$}};
  	\node at (4,5.75) {\textcolor{blue}{factorial}};
  	\node at (4,5.4){\textcolor{blue}{contador}};
  	\node at (3.6,5.05) {\textcolor{blue}{i}};
  	\node at (3.9,4.6) {\textcolor{blue}{nEuler}};
  	\node at (4.05,4) {\textcolor{blue}{nEulerOld}};
  	\node at (0.1,6.1) {0x3F00480\textcolor{red}{0}};
  	\node at (0.1,5.75) {0x3F00480\textcolor{red}{4}};
  	\node at (0.1,5.4) {0x3F00480\textcolor{red}{8}};
  	\node at (0.1,5.05) {0x3F00480\textcolor{red}{C}};
  	\node at (0.1,4.7) {0x3F0048\textcolor{red}{10}};
  	\node at (0.1,4.37) {0x3F0048\textcolor{red}{14}};
  	\node at (0.1,4.03) {0x3F0048\textcolor{red}{18}};
  	\node at (0.1,3.67) {0x3F0048\textcolor{red}{1C}};
  	\draw[decorate,decoration={brace,mirror,raise=6pt,amplitude=8pt}, thick]
    (3.3,6.1) -- (0.9,6.1)node [black,midway,xshift=0.0cm,yshift=0.7cm]{\footnotesize {\color{red}$4$ $bytes$}};
\end{tikzpicture}
\column{0.55\textwidth}
\lstset{basicstyle=\tiny}
\begin{lstlisting}[escapechar=\|,label=noint]
#include <stdio.h>
#include <math.h>

#define LIMITE   pow(10,-9)
int  main(void)
{
int		factorial = 1,contador =1,i;    
double	nEuler=0,nEulerOld=-1;

for(i=0;(nEuler-nEulerOld)>= LIMITE;i++)
  }
     nEulerOld=nEuler;
     while(contador <= i)
     { 
     	factorial = factorial * contador;
     	contador++;
     }
     nEuler = nEuler + (1/(double)factorial); 
        
     factorial = 1;
     contador =1;
  }
  printf("e es %0.10f \n",nEuler);
  printf("e(lib math.h)es %0.10f\n",M_E);
  return 0;
}
\end{lstlisting}
\end{columns}
\end{frame}

\begin{frame}[fragile]
\fontsize{6.5pt}{10pt}\selectfont
\frametitle{Recorriendo el programa {\color{yellow}paso} a {\color{yellow}paso}}
\begin{columns}[c]
\column{0.5\textwidth}
\begin{tikzpicture}
	\begin{scope}[every node/.style={draw,anchor=text,rectangle split,rectangle split parts=10, 
rectangle split part fill={green!0,blue!0,blue!0,blue!0,blue!0,blue!0,blue!0,blue!0,blue!0,red!0,red!0},minimum width=2.6cm}]
    	\node (R) at (2,6)
    	{
    		\nodepart{one}{}
    		\nodepart{two}{\color{red}6}
    		\nodepart{three}{3}	
    		\nodepart{four}{4}
    		\nodepart{five}{2.666....}
    		\nodepart{six}{}
    		\nodepart{seven}{2.666....}
    		\nodepart{eight}{}
    		\nodepart{nine}{$\vdots$}
    		\nodepart{ten}{}
    	};
	\end{scope}
	\draw [ultra thick,color=white](0.8, 4.6) -- (3.37, 4.6);
	\draw [ultra thick,color=white](0.8, 3.92) -- (3.37, 3.92);
	\fill[color=green] (5.8,5.9) -- (10.5,5.9) -- (10.5,5.6) -- (5.8,5.6) -- cycle;
  	\node at (2.1,7.9) {\small $almacena$ $el$ $resultado$ $en$ $la$};
  	\node at (2.1,7.5) {\small $variable$ $\color{red}factorial$};
  	\node at (2.1,9) {\normalsize Arquitectura X86-\textcolor{red}{32} bits };
  	\node at (8,9.7) {\normalsize {\color{red!50}$C\acute{o}digo$ $del$ $programa$ $fuente$}};
  	\node at (4,5.75) {\textcolor{blue}{factorial}};
  	\node at (4,5.4){\textcolor{blue}{contador}};
  	\node at (3.6,5.05) {\textcolor{blue}{i}};
  	\node at (3.9,4.6) {\textcolor{blue}{nEuler}};
  	\node at (4.05,4) {\textcolor{blue}{nEulerOld}};
  	\node at (0.1,6.1) {0x3F00480\textcolor{red}{0}};
  	\node at (0.1,5.75) {0x3F00480\textcolor{red}{4}};
  	\node at (0.1,5.4) {0x3F00480\textcolor{red}{8}};
  	\node at (0.1,5.05) {0x3F00480\textcolor{red}{C}};
  	\node at (0.1,4.7) {0x3F0048\textcolor{red}{10}};
  	\node at (0.1,4.37) {0x3F0048\textcolor{red}{14}};
  	\node at (0.1,4.03) {0x3F0048\textcolor{red}{18}};
  	\node at (0.1,3.67) {0x3F0048\textcolor{red}{1C}};
  	\draw[decorate,decoration={brace,mirror,raise=6pt,amplitude=8pt}, thick]
    (3.3,6.1) -- (0.9,6.1)node [black,midway,xshift=0.0cm,yshift=0.7cm]{\footnotesize {\color{red}$4$ $bytes$}};
  	\draw[decorate,decoration={brace,raise=3pt,amplitude=3pt}, thick]
    (7.5,5.8)--(10.2,5.8)node [black,midway,xshift=1.5cm,yshift=0.4cm]{\footnotesize{\color{red}$2\cdot 3=6$ }};
\end{tikzpicture}
\column{0.55\textwidth}
\lstset{basicstyle=\tiny}
\begin{lstlisting}[escapechar=\|,label=noint]
#include <stdio.h>
#include <math.h>

#define LIMITE   pow(10,-9)
int  main(void)
{
int		factorial = 1,contador =1,i;    
double	nEuler=0,nEulerOld=-1;

for(i=0;(nEuler-nEulerOld)>= LIMITE;i++)
  }
     nEulerOld=nEuler;
     while(contador <= i)
     { 
     	factorial = factorial * contador;
     	contador++;
     }
     nEuler = nEuler + (1/(double)factorial); 
        
     factorial = 1;
     contador =1;
  }
  printf("e es %0.10f \n",nEuler);
  printf("e(lib math.h)es %0.10f\n",M_E);
  return 0;
}
\end{lstlisting}
\end{columns}
\end{frame}

\begin{frame}[fragile]
\fontsize{6.5pt}{10pt}\selectfont
\frametitle{Recorriendo el programa {\color{yellow}paso} a {\color{yellow}paso}}
\begin{columns}[c]
\column{0.5\textwidth}
\begin{tikzpicture}
	\begin{scope}[every node/.style={draw,anchor=text,rectangle split,rectangle split parts=10, 
rectangle split part fill={green!0,blue!0,blue!0,blue!0,blue!0,blue!0,blue!0,blue!0,blue!0,red!0,red!0},minimum width=2.6cm}]
    	\node (R) at (2,6)
    	{
    		\nodepart{one}{}
    		\nodepart{two}{6}
    		\nodepart{three}{\color{red}4}	
    		\nodepart{four}{4}
    		\nodepart{five}{2.666....}
    		\nodepart{six}{}
    		\nodepart{seven}{2.666....}
    		\nodepart{eight}{}
    		\nodepart{nine}{$\vdots$}
    		\nodepart{ten}{}
    	};
	\end{scope}
	\draw [ultra thick,color=white](0.8, 4.6) -- (3.37, 4.6);
	\draw [ultra thick,color=white](0.8, 3.92) -- (3.37, 3.92);
	\fill[color=green] (5.8,5.6) -- (8,5.6) -- (8,5.3) -- (5.8,5.3) -- cycle;
  	\node at (2.1,7.9) {\small $incremento$ $la$ $variable$ $\color{red}contador$};
  	\node at (2.1,9) {\normalsize Arquitectura X86-\textcolor{red}{32} bits };
  	\node at (8,9.7) {\normalsize {\color{red!50}$C\acute{o}digo$ $del$ $programa$ $fuente$}};
  	\node at (4,5.75) {\textcolor{blue}{factorial}};
  	\node at (4,5.4){\textcolor{blue}{contador}};
  	\node at (3.6,5.05) {\textcolor{blue}{i}};
  	\node at (3.9,4.6) {\textcolor{blue}{nEuler}};
  	\node at (4.05,4) {\textcolor{blue}{nEulerOld}};
  	\node at (0.1,6.1) {0x3F00480\textcolor{red}{0}};
  	\node at (0.1,5.75) {0x3F00480\textcolor{red}{4}};
  	\node at (0.1,5.4) {0x3F00480\textcolor{red}{8}};
  	\node at (0.1,5.05) {0x3F00480\textcolor{red}{C}};
  	\node at (0.1,4.7) {0x3F0048\textcolor{red}{10}};
  	\node at (0.1,4.37) {0x3F0048\textcolor{red}{14}};
  	\node at (0.1,4.03) {0x3F0048\textcolor{red}{18}};
  	\node at (0.1,3.67) {0x3F0048\textcolor{red}{1C}};
  	\draw[decorate,decoration={brace,mirror,raise=6pt,amplitude=8pt}, thick]
    (3.3,6.1) -- (0.9,6.1)node [black,midway,xshift=0.0cm,yshift=0.7cm]{\footnotesize {\color{red}$4$ $bytes$}};
\end{tikzpicture}
\column{0.55\textwidth}
\lstset{basicstyle=\tiny}
\begin{lstlisting}[escapechar=\|,label=noint]
#include <stdio.h>
#include <math.h>

#define LIMITE   pow(10,-9)
int  main(void)
{
int		factorial = 1,contador =1,i;    
double	nEuler=0,nEulerOld=-1;

for(i=0;(nEuler-nEulerOld)>= LIMITE;i++)
  }
     nEulerOld=nEuler;
     while(contador <= i)
     { 
     	factorial = factorial * contador;
     	contador++;
     }
     nEuler = nEuler + (1/(double)factorial); 
        
     factorial = 1;
     contador =1;
  }
  printf("e es %0.10f \n",nEuler);
  printf("e(lib math.h)es %0.10f\n",M_E);
  return 0;
}
\end{lstlisting}
\end{columns}
\end{frame}

\begin{frame}[fragile]
\fontsize{6.5pt}{10pt}\selectfont
\frametitle{Recorriendo el programa {\color{yellow}paso} a {\color{yellow}paso}}
\begin{columns}[c]
\column{0.5\textwidth}
\begin{tikzpicture}
	\begin{scope}[every node/.style={draw,anchor=text,rectangle split,rectangle split parts=10, 
rectangle split part fill={green!0,blue!0,blue!0,blue!0,blue!0,blue!0,blue!0,blue!0,blue!0,red!0,red!0},minimum width=2.6cm}]
    	\node (R) at (2,6)
    	{
    		\nodepart{one}{}
    		\nodepart{two}{6}
    		\nodepart{three}{4}	
    		\nodepart{four}{4}
    		\nodepart{five}{2.666....}
    		\nodepart{six}{}
    		\nodepart{seven}{2.666....}
    		\nodepart{eight}{}
    		\nodepart{nine}{$\vdots$}
    		\nodepart{ten}{}
    	};
	\end{scope}
	\draw [ultra thick,color=white](0.8, 4.6) -- (3.37, 4.6);
	\draw [ultra thick,color=white](0.8, 3.92) -- (3.37, 3.92);
	\fill[color=green] (5.8,6.4) -- (8.4,6.4) -- (8.4,6) -- (5.8,6) -- cycle;
  	\node at (2.1,7.9) {\small $Como$ $\color{red}contador$ $es$ $\color{red}<=$ $a$ $\color{red}i$};
  	\node at (2.1,7.5) {\small $entra$ $dentro$ $del$ $\color{red}while$}; 
  	\node at (2.1,9) {\normalsize Arquitectura X86-\textcolor{red}{32} bits };
  	\node at (8,9.7) {\normalsize {\color{red!50}$C\acute{o}digo$ $del$ $programa$ $fuente$}};
  	\node at (4,5.75) {\textcolor{blue}{factorial}};
  	\node at (4,5.4){\textcolor{blue}{contador}};
  	\node at (3.6,5.05) {\textcolor{blue}{i}};
  	\node at (3.9,4.6) {\textcolor{blue}{nEuler}};
  	\node at (4.05,4) {\textcolor{blue}{nEulerOld}};
  	\node at (0.1,6.1) {0x3F00480\textcolor{red}{0}};
  	\node at (0.1,5.75) {0x3F00480\textcolor{red}{4}};
  	\node at (0.1,5.4) {0x3F00480\textcolor{red}{8}};
  	\node at (0.1,5.05) {0x3F00480\textcolor{red}{C}};
  	\node at (0.1,4.7) {0x3F0048\textcolor{red}{10}};
  	\node at (0.1,4.37) {0x3F0048\textcolor{red}{14}};
  	\node at (0.1,4.03) {0x3F0048\textcolor{red}{18}};
  	\node at (0.1,3.67) {0x3F0048\textcolor{red}{1C}};
  	\draw[decorate,decoration={brace,mirror,raise=6pt,amplitude=8pt}, thick]
    (3.3,6.1) -- (0.9,6.1)node [black,midway,xshift=0.0cm,yshift=0.7cm]{\footnotesize {\color{red}$4$ $bytes$}};
\end{tikzpicture}
\column{0.55\textwidth}
\lstset{basicstyle=\tiny}
\begin{lstlisting}[escapechar=\|,label=noint]
#include <stdio.h>
#include <math.h>

#define LIMITE   pow(10,-9)
int  main(void)
{
int		factorial = 1,contador =1,i;    
double	nEuler=0,nEulerOld=-1;

for(i=0;(nEuler-nEulerOld)>= LIMITE;i++)
  }
     nEulerOld=nEuler;
     while(contador <= i)
     { 
     	factorial = factorial * contador;
     	contador++;
     }
     nEuler = nEuler + (1/(double)factorial); 
        
     factorial = 1;
     contador =1;
  }
  printf("e es %0.10f \n",nEuler);
  printf("e(lib math.h)es %0.10f\n",M_E);
  return 0;
}
\end{lstlisting}
\end{columns}
\end{frame}

\begin{frame}[fragile]
\fontsize{6.5pt}{10pt}\selectfont
\frametitle{Recorriendo el programa {\color{yellow}paso} a {\color{yellow}paso}}
\begin{columns}[c]
\column{0.5\textwidth}
\begin{tikzpicture}
	\begin{scope}[every node/.style={draw,anchor=text,rectangle split,rectangle split parts=10, 
rectangle split part fill={green!0,blue!0,blue!0,blue!0,blue!0,blue!0,blue!0,blue!0,blue!0,red!0,red!0},minimum width=2.6cm}]
    	\node (R) at (2,6)
    	{
    		\nodepart{one}{}
    		\nodepart{two}{\color{red}24}
    		\nodepart{three}{4}	
    		\nodepart{four}{4}
    		\nodepart{five}{2.666....}
    		\nodepart{six}{}
    		\nodepart{seven}{2.666....}
    		\nodepart{eight}{}
    		\nodepart{nine}{$\vdots$}
    		\nodepart{ten}{}
    	};
	\end{scope}
	\draw [ultra thick,color=white](0.8, 4.6) -- (3.37, 4.6);
	\draw [ultra thick,color=white](0.8, 3.92) -- (3.37, 3.92);
	\fill[color=green] (5.8,5.9) -- (10.5,5.9) -- (10.5,5.6) -- (5.8,5.6) -- cycle;
  	\node at (2.1,7.9) {\small $almacena$ $el$ $resultado$ $en$ $la$};
  	\node at (2.1,7.5) {\small $variable$ $\color{red}factorial$};
  	\node at (2.1,9) {\normalsize Arquitectura X86-\textcolor{red}{32} bits };
  	\node at (8,9.7) {\normalsize {\color{red!50}$C\acute{o}digo$ $del$ $programa$ $fuente$}};
  	\node at (4,5.75) {\textcolor{blue}{factorial}};
  	\node at (4,5.4){\textcolor{blue}{contador}};
  	\node at (3.6,5.05) {\textcolor{blue}{i}};
  	\node at (3.9,4.6) {\textcolor{blue}{nEuler}};
  	\node at (4.05,4) {\textcolor{blue}{nEulerOld}};
  	\node at (0.1,6.1) {0x3F00480\textcolor{red}{0}};
  	\node at (0.1,5.75) {0x3F00480\textcolor{red}{4}};
  	\node at (0.1,5.4) {0x3F00480\textcolor{red}{8}};
  	\node at (0.1,5.05) {0x3F00480\textcolor{red}{C}};
  	\node at (0.1,4.7) {0x3F0048\textcolor{red}{10}};
  	\node at (0.1,4.37) {0x3F0048\textcolor{red}{14}};
  	\node at (0.1,4.03) {0x3F0048\textcolor{red}{18}};
  	\node at (0.1,3.67) {0x3F0048\textcolor{red}{1C}};
  	\draw[decorate,decoration={brace,mirror,raise=6pt,amplitude=8pt}, thick]
    (3.3,6.1) -- (0.9,6.1)node [black,midway,xshift=0.0cm,yshift=0.7cm]{\footnotesize {\color{red}$4$ $bytes$}};
  	\draw[decorate,decoration={brace,raise=3pt,amplitude=3pt}, thick]
    (7.5,5.8)--(10.2,5.8)node [black,midway,xshift=1.5cm,yshift=0.4cm]{\footnotesize{\color{red}$6\cdot4=24$ }};
\end{tikzpicture}
\column{0.55\textwidth}
\lstset{basicstyle=\tiny}
\begin{lstlisting}[escapechar=\|,label=noint]
#include <stdio.h>
#include <math.h>

#define LIMITE   pow(10,-9)
int  main(void)
{
int		factorial = 1,contador =1,i;    
double	nEuler=0,nEulerOld=-1;

for(i=0;(nEuler-nEulerOld)>= LIMITE;i++)
  }
     nEulerOld=nEuler;
     while(contador <= i)
     { 
     	factorial = factorial * contador;
     	contador++;
     }
     nEuler = nEuler + (1/(double)factorial); 
        
     factorial = 1;
     contador =1;
  }
  printf("e es %0.10f \n",nEuler);
  printf("e(lib math.h)es %0.10f\n",M_E);
  return 0;
}
\end{lstlisting}
\end{columns}
\end{frame}

\begin{frame}[fragile]
\fontsize{6.5pt}{10pt}\selectfont
\frametitle{Recorriendo el programa {\color{yellow}paso} a {\color{yellow}paso}}
\begin{columns}[c]
\column{0.5\textwidth}
\begin{tikzpicture}
	\begin{scope}[every node/.style={draw,anchor=text,rectangle split,rectangle split parts=10, 
rectangle split part fill={green!0,blue!0,blue!0,blue!0,blue!0,blue!0,blue!0,blue!0,blue!0,red!0,red!0},minimum width=2.6cm}]
    	\node (R) at (2,6)
    	{
    		\nodepart{one}{}
    		\nodepart{two}{24}
    		\nodepart{three}{\color{red}5}	
    		\nodepart{four}{4}
    		\nodepart{five}{2.666....}
    		\nodepart{six}{}
    		\nodepart{seven}{2.666....}
    		\nodepart{eight}{}
    		\nodepart{nine}{$\vdots$}
    		\nodepart{ten}{}
    	};
	\end{scope}
	\draw [ultra thick,color=white](0.8, 4.6) -- (3.37, 4.6);
	\draw [ultra thick,color=white](0.8, 3.92) -- (3.37, 3.92);
	\fill[color=green] (5.8,5.6) -- (8,5.6) -- (8,5.3) -- (5.8,5.3) -- cycle;
  	\node at (2.1,7.9) {\small $incremento$ $la$ $variable$ $\color{red}contador$};
  	\node at (2.1,9) {\normalsize Arquitectura X86-\textcolor{red}{32} bits };
  	\node at (8,9.7) {\normalsize {\color{red!50}$C\acute{o}digo$ $del$ $programa$ $fuente$}};
  	\node at (4,5.75) {\textcolor{blue}{factorial}};
  	\node at (4,5.4){\textcolor{blue}{contador}};
  	\node at (3.6,5.05) {\textcolor{blue}{i}};
  	\node at (3.9,4.6) {\textcolor{blue}{nEuler}};
  	\node at (4.05,4) {\textcolor{blue}{nEulerOld}};
  	\node at (0.1,6.1) {0x3F00480\textcolor{red}{0}};
  	\node at (0.1,5.75) {0x3F00480\textcolor{red}{4}};
  	\node at (0.1,5.4) {0x3F00480\textcolor{red}{8}};
  	\node at (0.1,5.05) {0x3F00480\textcolor{red}{C}};
  	\node at (0.1,4.7) {0x3F0048\textcolor{red}{10}};
  	\node at (0.1,4.37) {0x3F0048\textcolor{red}{14}};
  	\node at (0.1,4.03) {0x3F0048\textcolor{red}{18}};
  	\node at (0.1,3.67) {0x3F0048\textcolor{red}{1C}};
  	\draw[decorate,decoration={brace,mirror,raise=6pt,amplitude=8pt}, thick]
    (3.3,6.1) -- (0.9,6.1)node [black,midway,xshift=0.0cm,yshift=0.7cm]{\footnotesize {\color{red}$4$ $bytes$}};
\end{tikzpicture}
\column{0.55\textwidth}
\lstset{basicstyle=\tiny}
\begin{lstlisting}[escapechar=\|,label=noint]
#include <stdio.h>
#include <math.h>

#define LIMITE   pow(10,-9)
int  main(void)
{
int		factorial = 1,contador =1,i;    
double	nEuler=0,nEulerOld=-1;

for(i=0;(nEuler-nEulerOld)>= LIMITE;i++)
  }
     nEulerOld=nEuler;
     while(contador <= i)
     { 
     	factorial = factorial * contador;
     	contador++;
     }
     nEuler = nEuler + (1/(double)factorial); 
        
     factorial = 1;
     contador =1;
  }
  printf("e es %0.10f \n",nEuler);
  printf("e(lib math.h)es %0.10f\n",M_E);
  return 0;
}
\end{lstlisting}
\end{columns}
\end{frame}

\begin{frame}[fragile]
\fontsize{6.5pt}{10pt}\selectfont
\frametitle{Recorriendo el programa {\color{yellow}paso} a {\color{yellow}paso}}
\begin{columns}[c]
\column{0.5\textwidth}
\begin{tikzpicture}
	\begin{scope}[every node/.style={draw,anchor=text,rectangle split,rectangle split parts=10, 
rectangle split part fill={green!0,blue!0,blue!0,blue!0,blue!0,blue!0,blue!0,blue!0,blue!0,red!0,red!0},minimum width=2.6cm}]
    	\node (R) at (2,6)
    	{
    		\nodepart{one}{}
    		\nodepart{two}{24}
    		\nodepart{three}{5}	
    		\nodepart{four}{4}
    		\nodepart{five}{2.666....}
    		\nodepart{six}{}
    		\nodepart{seven}{2.666....}
    		\nodepart{eight}{}
    		\nodepart{nine}{$\vdots$}
    		\nodepart{ten}{}
    	};
	\end{scope}
	\draw [ultra thick,color=white](0.8, 4.6) -- (3.37, 4.6);
	\draw [ultra thick,color=white](0.8, 3.92) -- (3.37, 3.92);
	\fill[color=green] (5.8,6.4) -- (8.4,6.4) -- (8.4,6) -- (5.8,6) -- cycle;
  	\node at (2.1,7.9) {\small $Como$ $\color{red}contador$ $no$ $es$ $\color{red}<=$ $a$ $\color{red}i$};
  	\node at (2.1,7.5) {\small $sale$ $del$ $loop$ $\color{red}while$};
  	\node at (2.1,9) {\normalsize Arquitectura X86-\textcolor{red}{32} bits };
  	\node at (8,9.7) {\normalsize {\color{red!50}$C\acute{o}digo$ $del$ $programa$ $fuente$}};
  	\node at (4,5.75) {\textcolor{blue}{factorial}};
  	\node at (4,5.4){\textcolor{blue}{contador}};
  	\node at (3.6,5.05) {\textcolor{blue}{i}};
  	\node at (3.9,4.6) {\textcolor{blue}{nEuler}};
  	\node at (4.05,4) {\textcolor{blue}{nEulerOld}};
  	\node at (0.1,6.1) {0x3F00480\textcolor{red}{0}};
  	\node at (0.1,5.75) {0x3F00480\textcolor{red}{4}};
  	\node at (0.1,5.4) {0x3F00480\textcolor{red}{8}};
  	\node at (0.1,5.05) {0x3F00480\textcolor{red}{C}};
  	\node at (0.1,4.7) {0x3F0048\textcolor{red}{10}};
  	\node at (0.1,4.37) {0x3F0048\textcolor{red}{14}};
  	\node at (0.1,4.03) {0x3F0048\textcolor{red}{18}};
  	\node at (0.1,3.67) {0x3F0048\textcolor{red}{1C}};
  	\draw[decorate,decoration={brace,mirror,raise=6pt,amplitude=8pt}, thick]
    (3.3,6.1) -- (0.9,6.1)node [black,midway,xshift=0.0cm,yshift=0.7cm]{\footnotesize {\color{red}$4$ $bytes$}};
\end{tikzpicture}
\column{0.55\textwidth}
\lstset{basicstyle=\tiny}
\begin{lstlisting}[escapechar=\|,label=noint]
#include <stdio.h>
#include <math.h>

#define LIMITE   pow(10,-9)
int  main(void)
{
int		factorial = 1,contador =1,i;    
double	nEuler=0,nEulerOld=-1;

for(i=0;(nEuler-nEulerOld)>= LIMITE;i++)
  }
     nEulerOld=nEuler;
     while(contador <= i)
     { 
     	factorial = factorial * contador;
     	contador++;
     }
     nEuler = nEuler + (1/(double)factorial); 
        
     factorial = 1;
     contador =1;
  }
  printf("e es %0.10f \n",nEuler);
  printf("e(lib math.h)es %0.10f\n",M_E);
  return 0;
}
\end{lstlisting}
\end{columns}
\end{frame}

\begin{frame}[fragile]
\fontsize{6.5pt}{10pt}\selectfont
\frametitle{Recorriendo el programa {\color{yellow}paso} a {\color{yellow}paso}}
\begin{columns}[c]
\column{0.5\textwidth}
\begin{tikzpicture}
	\begin{scope}[every node/.style={draw,anchor=text,rectangle split,rectangle split parts=10, 
rectangle split part fill={green!0,blue!0,blue!0,blue!0,blue!0,blue!0,blue!0,blue!0,blue!0,red!0,red!0},minimum width=2.6cm}]
    	\node (R) at (2,6)
    	{
    		\nodepart{one}{}
    		\nodepart{two}{24}
    		\nodepart{three}{5}	
    		\nodepart{four}{4}
    		\nodepart{five}{\color{red}2.7083....}
    		\nodepart{six}{}
    		\nodepart{seven}{2.666....}
    		\nodepart{eight}{}
    		\nodepart{nine}{$\vdots$}
    		\nodepart{ten}{}
    	};
	\end{scope}
	\draw [ultra thick,color=white](0.8, 4.6) -- (3.37, 4.6);
	\draw [ultra thick,color=white](0.8, 3.92) -- (3.37, 3.92);
	\fill[color=green] (5.6,5.2) -- (11,5.2) -- (11,4.8) -- (5.6,4.8) -- cycle;
  	\node at (2.1,7.9) {\small $almacena$ $el$ $resultado$ $en$ $la$};
  	\node at (2.1,7.5) {\small $variable$ $\color{red}nEuler$};
   	\node at (2.1,9) {\normalsize Arquitectura X86-\textcolor{red}{32} bits };
  	\node at (8,9.7) {\normalsize {\color{red!50}$C\acute{o}digo$ $del$ $programa$ $fuente$}};
  	\node at (4,5.75) {\textcolor{blue}{factorial}};
  	\node at (4,5.4){\textcolor{blue}{contador}};
  	\node at (3.6,5.05) {\textcolor{blue}{i}};
  	\node at (3.9,4.6) {\textcolor{blue}{nEuler}};
  	\node at (4.05,4) {\textcolor{blue}{nEulerOld}};
  	\node at (0.1,6.1) {0x3F00480\textcolor{red}{0}};
  	\node at (0.1,5.75) {0x3F00480\textcolor{red}{4}};
  	\node at (0.1,5.4) {0x3F00480\textcolor{red}{8}};
  	\node at (0.1,5.05) {0x3F00480\textcolor{red}{C}};
  	\node at (0.1,4.7) {0x3F0048\textcolor{red}{10}};
  	\node at (0.1,4.37) {0x3F0048\textcolor{red}{14}};
  	\node at (0.1,4.03) {0x3F0048\textcolor{red}{18}};
  	\node at (0.1,3.67) {0x3F0048\textcolor{red}{1C}};
  	\draw[decorate,decoration={brace,mirror,raise=6pt,amplitude=8pt}, thick]
    (3.3,6.1) -- (0.9,6.1)node [black,midway,xshift=0.0cm,yshift=0.7cm]{\footnotesize {\color{red}$4$ $bytes$}};
  	\draw[decorate,decoration={brace,raise=3pt,amplitude=3pt}, thick]
    (7,5.1) -- (10.7,5.1)node [black,midway,xshift=1.2cm,yshift=0.4cm]{\tiny {\color{red}$2.66+\frac{1}{24.00}=2.708\wideparen{3}$ }};
\end{tikzpicture}
\column{0.55\textwidth}
\lstset{basicstyle=\tiny}
\begin{lstlisting}[escapechar=\|,label=noint]
#include <stdio.h>
#include <math.h>

#define LIMITE   pow(10,-9)
int  main(void)
{
int		factorial = 1,contador =1,i;    
double	nEuler=0,nEulerOld=-1;

for(i=0;(nEuler-nEulerOld)>= LIMITE;i++)
  }
     nEulerOld=nEuler;
     while(contador <= i)
     { 
     	factorial = factorial * contador;
     	contador++;
     }
     nEuler = nEuler + (1/(double)factorial); 
        
     factorial = 1;
     contador =1;
  }
  printf("e es %0.10f \n",nEuler);
  printf("e(lib math.h)es %0.10f\n",M_E);
  return 0;
}
\end{lstlisting}
\end{columns}
\end{frame}

\begin{frame}[fragile]
\fontsize{6.5pt}{10pt}\selectfont
\frametitle{Recorriendo el programa {\color{yellow}paso} a {\color{yellow}paso}}
\begin{columns}[c]
\column{0.5\textwidth}
\begin{tikzpicture}
	\begin{scope}[every node/.style={draw,anchor=text,rectangle split,rectangle split parts=10, 
rectangle split part fill={green!0,blue!0,blue!0,blue!0,blue!0,blue!0,blue!0,blue!0,blue!0,red!0,red!0},minimum width=2.6cm}]
    	\node (R) at (2,6)
    	{
    		\nodepart{one}{}
    		\nodepart{two}{\color{red}1}
    		\nodepart{three}{\color{red}1}	
    		\nodepart{four}{4}
    		\nodepart{five}{2.7083....}
    		\nodepart{six}{}
    		\nodepart{seven}{2.666....}
    		\nodepart{eight}{}
    		\nodepart{nine}{$\vdots$}
    		\nodepart{ten}{}
    	};
	\end{scope}
	\draw [ultra thick,color=white](0.8, 4.6) -- (3.37, 4.6);
	\draw [ultra thick,color=white](0.8, 3.92) -- (3.37, 3.92);
	\fill[color=green] (5.6,4.6) -- (8,4.6) -- (8,4.1) -- (5.6,4.1) -- cycle;
	\node at (2.1,7.9) {\small $inicializo$ $nuevamente$ $las$};
  	\node at (2.1,7.5) {\small $variables$ $\color{red}factorial$ y $\color{red}contador$};
   	\node at (2.1,9) {\normalsize Arquitectura X86-\textcolor{red}{32} bits };
  	\node at (8,9.7) {\normalsize {\color{red!50}$C\acute{o}digo$ $del$ $programa$ $fuente$}};
  	\node at (4,5.75) {\textcolor{blue}{factorial}};
  	\node at (4,5.4){\textcolor{blue}{contador}};
  	\node at (3.6,5.05) {\textcolor{blue}{i}};
  	\node at (3.9,4.6) {\textcolor{blue}{nEuler}};
  	\node at (4.05,4) {\textcolor{blue}{nEulerOld}};
  	\node at (0.1,6.1) {0x3F00480\textcolor{red}{0}};
  	\node at (0.1,5.75) {0x3F00480\textcolor{red}{4}};
  	\node at (0.1,5.4) {0x3F00480\textcolor{red}{8}};
  	\node at (0.1,5.05) {0x3F00480\textcolor{red}{C}};
  	\node at (0.1,4.7) {0x3F0048\textcolor{red}{10}};
  	\node at (0.1,4.37) {0x3F0048\textcolor{red}{14}};
  	\node at (0.1,4.03) {0x3F0048\textcolor{red}{18}};
  	\node at (0.1,3.67) {0x3F0048\textcolor{red}{1C}};
  	\draw[decorate,decoration={brace,mirror,raise=6pt,amplitude=8pt}, thick]
    (3.3,6.1) -- (0.9,6.1)node [black,midway,xshift=0.0cm,yshift=0.7cm]{\footnotesize {\color{red}$4$ $bytes$}};
  	\draw[decorate,decoration={brace,raise=3pt,amplitude=3pt}, thick]
    (7,5.1) -- (10.7,5.1)node [black,midway,xshift=1.2cm,yshift=0.4cm]{\tiny {\color{red}$2.66+\frac{1}{24.00}=2.708\wideparen{3}$ }};
\end{tikzpicture}
\column{0.55\textwidth}
\lstset{basicstyle=\tiny}
\begin{lstlisting}[escapechar=\|,label=noint]
#include <stdio.h>
#include <math.h>

#define LIMITE   pow(10,-9)
int  main(void)
{
int		factorial = 1,contador =1,i;    
double	nEuler=0,nEulerOld=-1;

for(i=0;(nEuler-nEulerOld)>= LIMITE;i++)
  }
     nEulerOld=nEuler;
     while(contador <= i)
     { 
     	factorial = factorial * contador;
     	contador++;
     }
     nEuler = nEuler + (1/(double)factorial); 
        
     factorial = 1;
     contador =1;
  }
  printf("e es %0.10f \n",nEuler);
  printf("e(lib math.h)es %0.10f\n",M_E);
  return 0;
}
\end{lstlisting}
\end{columns}
\end{frame}

\begin{frame}[fragile]
\fontsize{6.5pt}{10pt}\selectfont
\frametitle{Recorriendo el programa {\color{yellow}paso} a {\color{yellow}paso}}
\begin{columns}[c]
\column{0.5\textwidth}
\begin{tikzpicture}
	\begin{scope}[every node/.style={draw,anchor=text,rectangle split,rectangle split parts=10, 
rectangle split part fill={green!0,blue!0,blue!0,blue!0,blue!0,blue!0,blue!0,blue!0,blue!0,red!0,red!0},minimum width=2.6cm}]
    	\node (R) at (2,6)
    	{
    		\nodepart{one}{}
    		\nodepart{two}{1}
    		\nodepart{three}{1}	
    		\nodepart{four}{\color{red}5}
    		\nodepart{five}{2.7083....}
    		\nodepart{six}{}
    		\nodepart{seven}{2.666....}
    		\nodepart{eight}{}
    		\nodepart{nine}{$\vdots$}
    		\nodepart{ten}{}
    	};
	\end{scope}
	\draw [ultra thick,color=white](0.8, 4.6) -- (3.37, 4.6);
	\draw [ultra thick,color=white](0.8, 3.92) -- (3.37, 3.92);
	\fill[color=green] (9.7,7.2) -- (10.3,7.2) -- (10.3,6.8) -- (9.7,6.8) -- cycle;
  	\node at (2.1,7.9) {\small $incremento$ $la$ $variable$ $\color{red}i$};
  	\node at (2.1,9) {\normalsize Arquitectura X86-\textcolor{red}{32} bits };
  	\node at (8,9.7) {\normalsize {\color{red!50}$C\acute{o}digo$ $del$ $programa$ $fuente$}};
  	\node at (4,5.75) {\textcolor{blue}{factorial}};
  	\node at (4,5.4){\textcolor{blue}{contador}};
  	\node at (3.6,5.05) {\textcolor{blue}{i}};
  	\node at (3.9,4.6) {\textcolor{blue}{nEuler}};
  	\node at (4.05,4) {\textcolor{blue}{nEulerOld}};
  	\node at (0.1,6.1) {0x3F00480\textcolor{red}{0}};
  	\node at (0.1,5.75) {0x3F00480\textcolor{red}{4}};
  	\node at (0.1,5.4) {0x3F00480\textcolor{red}{8}};
  	\node at (0.1,5.05) {0x3F00480\textcolor{red}{C}};
  	\node at (0.1,4.7) {0x3F0048\textcolor{red}{10}};
  	\node at (0.1,4.37) {0x3F0048\textcolor{red}{14}};
  	\node at (0.1,4.03) {0x3F0048\textcolor{red}{18}};
  	\node at (0.1,3.67) {0x3F0048\textcolor{red}{1C}};
  	\draw[decorate,decoration={brace,mirror,raise=6pt,amplitude=8pt}, thick]
    (3.3,6.1) -- (0.9,6.1)node [black,midway,xshift=0.0cm,yshift=0.7cm]{\footnotesize {\color{red}$4$ $bytes$}};
\end{tikzpicture}
\column{0.55\textwidth}
\lstset{basicstyle=\tiny}
\begin{lstlisting}[escapechar=\|,label=noint]
#include <stdio.h>
#include <math.h>

#define LIMITE   pow(10,-9)
int  main(void)
{
int		factorial = 1,contador =1,i;    
double	nEuler=0,nEulerOld=-1;

for(i=0;(nEuler-nEulerOld)>= LIMITE;i++)
  }
     nEulerOld=nEuler;
     while(contador <= i)
     { 
     	factorial = factorial * contador;
     	contador++;
     }
     nEuler = nEuler + (1/(double)factorial); 
        
     factorial = 1;
     contador =1;
  }
  printf("e es %0.10f \n",nEuler);
  printf("e(lib math.h)es %0.10f\n",M_E);
  return 0;
}
\end{lstlisting}
\end{columns}
\end{frame}

\begin{frame}[fragile]
\fontsize{6.5pt}{10pt}\selectfont
\frametitle{Recorriendo el programa {\color{yellow}paso} a {\color{yellow}paso}}
\begin{columns}[c]
\column{0.5\textwidth}
\begin{tikzpicture}
	\begin{scope}[every node/.style={draw,anchor=text,rectangle split,rectangle split parts=10, 
rectangle split part fill={green!0,blue!0,blue!0,blue!0,blue!0,blue!0,blue!0,blue!0,blue!0,red!0,red!0},minimum width=2.6cm}]
    	\node (R) at (2,6)
    	{
    		\nodepart{one}{}
    		\nodepart{two}{1}
    		\nodepart{three}{1}	
    		\nodepart{four}{5}
    		\nodepart{five}{2.7083....}
    		\nodepart{six}{}
    		\nodepart{seven}{2.666....}
    		\nodepart{eight}{}
    		\nodepart{nine}{$\vdots$}
    		\nodepart{ten}{}
    	};
	\end{scope}
	\draw [ultra thick,color=white](0.8, 4.6) -- (3.37, 4.6);
	\draw [ultra thick,color=white](0.8, 3.92) -- (3.37, 3.92);
	\fill[color=green] (6.4,7.2) -- (9.7,7.2) -- (9.7,6.8) -- (6.4,6.8) -- cycle;
  	\node at (2.1,9) {\normalsize Arquitectura X86-\textcolor{red}{32} bits };
  	\node at (2.1,7.9) {\small $Pregunta$ $condici\acute{o}n$ $del$ $\color{red}for$};
  	\node at (2.1,9) {\normalsize Arquitectura X86-\textcolor{red}{32} bits };
  	\node at (8,9.7) {\normalsize {\color{red!50}$C\acute{o}digo$ $del$ $programa$ $fuente$}};
  	\node at (4,5.75) {\textcolor{blue}{factorial}};
  	\node at (4,5.4){\textcolor{blue}{contador}};
  	\node at (3.6,5.05) {\textcolor{blue}{i}};
  	\node at (3.9,4.6) {\textcolor{blue}{nEuler}};
  	\node at (4.05,4) {\textcolor{blue}{nEulerOld}};
  	\node at (0.1,6.1) {0x3F00480\textcolor{red}{0}};
  	\node at (0.1,5.75) {0x3F00480\textcolor{red}{4}};
  	\node at (0.1,5.4) {0x3F00480\textcolor{red}{8}};
  	\node at (0.1,5.05) {0x3F00480\textcolor{red}{C}};
  	\node at (0.1,4.7) {0x3F0048\textcolor{red}{10}};
  	\node at (0.1,4.37) {0x3F0048\textcolor{red}{14}};
  	\node at (0.1,4.03) {0x3F0048\textcolor{red}{18}};
  	\node at (0.1,3.67) {0x3F0048\textcolor{red}{1C}};
  	\draw[decorate,decoration={brace,mirror,raise=6pt,amplitude=8pt}, thick]
    (3.3,6.1) -- (0.9,6.1)node [black,midway,xshift=0.0cm,yshift=0.7cm]{\footnotesize {\color{red}$4$ $bytes$}};
  	\draw[decorate,decoration={brace,raise=4pt,amplitude=4pt}, thick]
    (7.35,8.4)--(8.7,8.4)node [black,midway,xshift=1.0cm,yshift=0.4cm]{\footnotesize {\color{red!60}$10^{-9}$}};
  	\draw[red,->](9.3,7.1)to [out=10,in=0]node[right,midway]{} ++(-0.2,1.3) ; 
\end{tikzpicture}
\column{0.55\textwidth}
\lstset{basicstyle=\tiny}
\begin{lstlisting}[escapechar=\|,label=noint]
#include <stdio.h>
#include <math.h>

#define LIMITE   pow(10,-9)
int  main(void)
{
int		factorial = 1,contador =1,i;    
double	nEuler=0,nEulerOld=-1;

for(i=0;(nEuler-nEulerOld)>= LIMITE;i++)
  }
     nEulerOld=nEuler;
     while(contador <= i)
     { 
     	factorial = factorial * contador;
     	contador++;
     }
     nEuler = nEuler + (1/(double)factorial); 
        
     factorial = 1;
     contador =1;
  }
  printf("e es %0.10f \n",nEuler);
  printf("e(lib math.h)es %0.10f\n",M_E);
  return 0;
}
\end{lstlisting}
\end{columns}
\end{frame}

\begin{frame}[fragile]
\fontsize{13pt}{20pt}\selectfont
\frametitle{Terminando el ciclo \color{yellow}for}
Si seguimos observando paso a paso como se va ejecutando el programa y como van cambiando los valores de las variables de memoria, llegaremos al punto donde en el ciclo del {\color{red}for} no se va a cumplir la condici�n
\begin{center}
({\color{red}$nEuler$ $-$ $nEulerOld$}) >=  {\color{red}$LIMITE$}
\end{center}

y por lo tanto los valores de memoria tendr�n los siguientes valores
\end{frame}

\begin{frame}[fragile]
\fontsize{6.5pt}{10pt}\selectfont
\frametitle{Recorriendo el programa {\color{yellow}paso} a {\color{yellow}paso}}
\begin{columns}[c]
\column{0.5\textwidth}
\begin{tikzpicture}
	\begin{scope}[every node/.style={draw,anchor=text,rectangle split,rectangle split parts=10, 
rectangle split part fill={green!0,blue!0,blue!0,blue!0,blue!0,blue!0,blue!0,blue!0,blue!0,red!0,red!0},minimum width=2.6cm}]
    	\node (R) at (2,6)
    	{
    		\nodepart{one}{}
    		\nodepart{two}{1}
    		\nodepart{three}{1}	
    		\nodepart{four}{13}
    		\nodepart{five}{2.718281828\color{red}8}
    		\nodepart{six}{}
    		\nodepart{seven}{2.718281828\color{red}3}
    		\nodepart{eight}{}
    		\nodepart{nine}{$\vdots$}
    		\nodepart{ten}{}
    	};
	\end{scope}
	\draw [ultra thick,color=white](0.8, 4.6) -- (3.37, 4.6);
	\draw [ultra thick,color=white](0.8, 3.92) -- (3.37, 3.92);
	\fill[color=green] (6.4,7.2) -- (9.7,7.2) -- (9.7,6.8) -- (6.4,6.8) -- cycle;
  	\node at (2.1,9) {\normalsize Arquitectura X86-\textcolor{red}{32} bits };
  	\node at (2.1,7.9) {\small $\color{red}No$ $cumple$ $condici\acute{o}n$ $del$ $\color{red}for$};
  	\node at (2.1,7.5) {\scriptsize ($nEuler-nEulerOld$) $es$ $\color{red}0.0000000005$};
  	\node at (2.1,9) {\normalsize Arquitectura X86-\textcolor{red}{32} bits };
  	\node at (8,9.7) {\normalsize {\color{red!50}$C\acute{o}digo$ $del$ $programa$ $fuente$}};
  	\node at (4,5.75) {\textcolor{blue}{factorial}};
  	\node at (4,5.4){\textcolor{blue}{contador}};
  	\node at (3.6,5.05) {\textcolor{blue}{i}};
  	\node at (3.9,4.6) {\textcolor{blue}{nEuler}};
  	\node at (4.05,4) {\textcolor{blue}{nEulerOld}};
  	\node at (0.1,6.1) {0x3F00480\textcolor{red}{0}};
  	\node at (0.1,5.75) {0x3F00480\textcolor{red}{4}};
  	\node at (0.1,5.4) {0x3F00480\textcolor{red}{8}};
  	\node at (0.1,5.05) {0x3F00480\textcolor{red}{C}};
  	\node at (0.1,4.7) {0x3F0048\textcolor{red}{10}};
  	\node at (0.1,4.37) {0x3F0048\textcolor{red}{14}};
  	\node at (0.1,4.03) {0x3F0048\textcolor{red}{18}};
  	\node at (0.1,3.67) {0x3F0048\textcolor{red}{1C}};
  	\draw[decorate,decoration={brace,mirror,raise=6pt,amplitude=8pt}, thick]
    (3.3,6.1) -- (0.9,6.1)node [black,midway,xshift=0.0cm,yshift=0.7cm]{\footnotesize {\color{red}$4$ $bytes$}};
  	\draw[decorate,decoration={brace,raise=4pt,amplitude=4pt}, thick]
    (7.35,8.4)--(8.7,8.4)node [black,midway,xshift=1.0cm,yshift=0.4cm]{\footnotesize {\color{red!60}$10^{-9}$}};
  	\draw[red,->](9.3,7.1)to [out=10,in=0]node[right,midway]{} ++(-0.2,1.3) ; 
\end{tikzpicture}
\column{0.55\textwidth}
\lstset{basicstyle=\tiny}
\begin{lstlisting}[escapechar=\|,label=noint]
#include <stdio.h>
#include <math.h>

#define LIMITE   pow(10,-9)
int  main(void)
{
int		factorial = 1,contador =1,i;    
double	nEuler=0,nEulerOld=-1;

for(i=0;(nEuler-nEulerOld)>= LIMITE;i++)
  }
     nEulerOld=nEuler;
     while(contador <= i)
     { 
     	factorial = factorial * contador;
     	contador++;
     }
     nEuler = nEuler + (1/(double)factorial); 
        
     factorial = 1;
     contador =1;
  }
  printf("e es %0.10f \n",nEuler);
  printf("e(lib math.h)es %0.10f\n",M_E);
  return 0;
}
\end{lstlisting}
\end{columns}
\end{frame}

\begin{frame}[fragile]
\fontsize{6.5pt}{10pt}\selectfont
\frametitle{Fin}
\begin{columns}[c]
\column{0.5\textwidth}
\begin{tikzpicture}
	\begin{scope}[every node/.style={draw,anchor=text,rectangle split,rectangle split parts=10, 
rectangle split part fill={green!0,blue!0,blue!0,blue!0,blue!0,blue!0,blue!0,blue!0,blue!0,red!0,red!0},minimum width=2.6cm}]
    	\node (R) at (2,6)
    	{
    		\nodepart{one}{}
    		\nodepart{two}{1}
    		\nodepart{three}{1}	
    		\nodepart{four}{13}
    		\nodepart{five}{2.718281828\color{red}8}
    		\nodepart{six}{}
    		\nodepart{seven}{2.718281828\color{red}3}
    		\nodepart{eight}{}
    		\nodepart{nine}{$\vdots$}
    		\nodepart{ten}{}
    	};
	\end{scope}
	\draw [ultra thick,color=white](0.8, 4.6) -- (3.37, 4.6);
	\draw [ultra thick,color=white](0.8, 3.92) -- (3.37, 3.92);
	\fill[color=green] (5.5,3.9) -- (10.5,3.9) -- (10.5,3.3) -- (5.5,3.3) -- cycle;
  	\node at (2.1,9) {\normalsize Arquitectura X86-\textcolor{red}{32} bits };
  	\node at (2.1,7.9) {\small $\color{blue}e$ $es$ $\color{red}2.7182818288$};
  	\node at (2.1,7.5) {\small $\color{blue}e$ $(lib$ $math.h)$ $es$ $\color{red}2.7182818285$};
  	\node at (2.1,9) {\normalsize Arquitectura X86-\textcolor{red}{32} bits };
  	\node at (8,9.7) {\normalsize {\color{red!50}$C\acute{o}digo$ $del$ $programa$ $fuente$}};
  	\node at (4,5.75) {\textcolor{blue}{factorial}};
  	\node at (4,5.4){\textcolor{blue}{contador}};
  	\node at (3.6,5.05) {\textcolor{blue}{i}};
  	\node at (3.9,4.6) {\textcolor{blue}{nEuler}};
  	\node at (4.05,4) {\textcolor{blue}{nEulerOld}};
  	\node at (0.1,6.1) {0x3F00480\textcolor{red}{0}};
  	\node at (0.1,5.75) {0x3F00480\textcolor{red}{4}};
  	\node at (0.1,5.4) {0x3F00480\textcolor{red}{8}};
  	\node at (0.1,5.05) {0x3F00480\textcolor{red}{C}};
  	\node at (0.1,4.7) {0x3F0048\textcolor{red}{10}};
  	\node at (0.1,4.37) {0x3F0048\textcolor{red}{14}};
  	\node at (0.1,4.03) {0x3F0048\textcolor{red}{18}};
  	\node at (0.1,3.67) {0x3F0048\textcolor{red}{1C}};
  	\draw[decorate,decoration={brace,mirror,raise=6pt,amplitude=8pt}, thick]
    (3.3,6.1) -- (0.9,6.1)node [black,midway,xshift=0.0cm,yshift=0.7cm]{\footnotesize {\color{red}$4$ $bytes$}};
\end{tikzpicture}
\column{0.55\textwidth}
\lstset{basicstyle=\tiny}
\begin{lstlisting}[escapechar=\|,label=noint]
#include <stdio.h>
#include <math.h>

#define LIMITE   pow(10,-9)
int  main(void)
{
int		factorial = 1,contador =1,i;    
double	nEuler=0,nEulerOld=-1;

for(i=0;(nEuler-nEulerOld)>= LIMITE;i++)
  }
     nEulerOld=nEuler;
     while(contador <= i)
     { 
     	factorial = factorial * contador;
     	contador++;
     }
     nEuler = nEuler + (1/(double)factorial); 
        
     factorial = 1;
     contador =1;
  }
  printf("e es %0.10f \n",nEuler);
  printf("e(lib math.h)es %0.10f\n",M_E);
  return 0;
}
\end{lstlisting}
\end{columns}
\end{frame}

\end{document}


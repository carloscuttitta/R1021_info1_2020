\documentclass{beamer}
\usepackage[latin1]{inputenc}
\usepackage[spanish]{babel}
\usepackage{multicol}
\usepackage{fancybox}
\usepackage{beamerthemeshadow}
\usepackage{times} %font times
\usepackage[T1]{fontenc} %para que cuando se seleccione un texto las letras acentuadad y las � se copien bien Usar la codificaci�n T1
\usepackage{enumerate}
\usepackage{listings}
\usepackage{calligra} 

\usepackage{graphicx}
\usepackage{array}
\usepackage{caption}

\usepackage{tikz}
\usepackage{verbatim}
\usepackage{mdwlist}
\usetikzlibrary{chains,fit,shapes}

\usetikzlibrary{chains,fit,shapes,arrows,calc,shapes,decorations.pathreplacing}
%\usepackage[active,tightpage]{preview}
%\PreviewEnvironment{tikzpicture}

\usefonttheme{professionalfonts}

\newtheorem{defi}{Definici�n} 
\hypersetup{pdfpagemode=FullScreen}

\mode<presentation>{
\usetheme{Warsaw}
\setbeamercovered{transparent}
}


\lstset{
	frame=Ltb,
	framerule=0pt,
	aboveskip=0.5cm,
	framextopmargin=3pt,
	framexbottommargin=3pt,
	framexleftmargin=0.4cm,
	framesep=0pt,
	rulesep=.4pt,
%	backgroundcolor=\color{gray!20},
	rulesepcolor=\color{black},
	language=C,
	captionpos=b,
	tabsize=3,
	frame=lines,
	keywordstyle=\color{blue},
	commentstyle=\color{gray},
	stringstyle=\color{red},
	numbers=left,
	numberstyle=\tiny,
	numbersep=5pt,
	breaklines=true,
	showstringspaces=false,
	basicstyle=\small,
	emph={label},
	framerule=0pt,
}

\title{\em INFORMATICA I}
\subtitle{Introducci�n a {\color{yellow}funciones} en {\color{yellow}"C"}}
\author{\em Ing.Juan Carlos Cuttitta}
\institute{\calligra{\fontsize{16pt}{7pt}\selectfont{Universidad Tecnol�gica Nacional\\ Facultad Regional Buenos Aires \\ Departamento de {Ingenier�a} {Electr�nica}}}}
\date{\today}

%portada

\begin{document}

\begin{figure}[ht!]
  \centering
  \includegraphics [width=0.25\textwidth]{informacion.jpg}
\end{figure}
\vspace{-1.2cm} % para subir el titulo 
\titlepage

%%%%%%%%%%%%%%%%%%%%%%%%%%%%%%%%%%%%%%%%%%%%%%%%
%Template de aqui en mas ;)
%%%%%%%%%%%%%%%%%%%%%%%%%%%%%%%%%%%%%%%%%%%%%%%%
\pgfdeclareimage[height=0.9cm]{left-logo}{arania.png}
%\pgfdeclareimage[height=0.9cm]{right-logo}{arania.png}
%\logo{\pgfuseimage{right-logo}}
\setbeamertemplate{sidebar left}
{
\logo{\pgfuseimage{left-logo}}
\vfill%
\rlap{\hskip0.15cm\insertlogo}%
\vskip10pt%
}
%%%%%%%%%%%%%%%%%%%%%%%%%%%%%%%%%%%%%%%%%%%%%%%%

\tikzstyle{every picture}+=[remember picture]
\tikzstyle{na} = [baseline=-.5ex]

\begin{frame}[fragile]
\fontsize{6.5pt}{12pt}\selectfont
 \frametitle{Ejemplo de un programa que \textbf{\textit{\color{yellow}suma}}}
 \begin{columns}[c]
  \column{0.48\textwidth}
\begin{tikzpicture}
  \node at (6.5,7) {\normalsize C�digo en programa fuente};
  \draw[->](0.3,0.6)to [out=90,in=-90]node[right,midway]{} ++(0,0.5) ; 
\end{tikzpicture}
  \column{2.9\textwidth}
  \lstset{basicstyle=\tiny}
   \begin{lstlisting}[escapechar=\|,label=noint]
#include <stdio.h>

int main (void)
{
int a,b,c;

	a = 6;
	b = 9;
	c = a + b;
	printf("El resultado es % d \n",c);

	a = 3;
	b = 1;
	c = a + b;
	printf("El resultado es % d \n",c);

	a = 12;
	b = 4;
	c = a + b;
	printf("El resultado es % d \n",c);

	return (0);
}   
	\end{lstlisting}
  \end{columns}
\end{frame}

\begin{frame}[fragile]
\fontsize{6.5pt}{12pt}\selectfont
 \frametitle{Tareas \textbf{\textit{\color{yellow}repetidas}}}
 \begin{columns}[c]
  \column{0.48\textwidth}
\begin{tikzpicture}
  \node at (6.5,7) {\normalsize C�digo en programa fuente};
% usado para hacer sombreado de un sector
  \fill[color=green] (5.7,5.25) -- (7.7,5.25) -- (7.7,4.4) -- (5.7,4.4) -- cycle;
  \draw[->](0.3,0.6)to [out=90,in=-90]node[right,midway]{} ++(0,0.5) ; 
\end{tikzpicture}
  \column{6\textwidth}
  \lstset{basicstyle=\tiny}
   \begin{lstlisting}[escapechar=\|,label=noint]
#include <stdio.h>

int main (void)
{
int a,b,c;

	a = 6;
	b = 9;
	c = a + b;
	printf("El resultado es % d \n",c);

	a = 3;
	b = 1;
	c = a + b;
	printf("El resultado es % d \n",c);

	a = 12;
	b = 4;
	c = a + b;
	printf("El resultado es % d \n",c);

	return (0);
}   
   \end{lstlisting}
 \end{columns}
\end{frame}

\begin{frame}[fragile]
\fontsize{6.5pt}{12pt}\selectfont
 \frametitle{Tareas \textbf{\textit{\color{yellow}repetidas}}}
 \begin{columns}[c]
  \column{0.48\textwidth}
\begin{tikzpicture}
  \node at (6.5,7) {\normalsize C�digo en programa fuente};
% usado para hacer sombreado de un sector
  \fill[color=green] (5.7,5.25) -- (7.7,5.25) -- (7.7,4.4) -- (5.7,4.4) -- cycle;
  \fill[color=green] (5.7,4) -- (7.7,4) -- (7.7,3.17) -- (5.7,3.17) -- cycle;
  \draw[->](0.3,0.6)to [out=90,in=-90]node[right,midway]{} ++(0,0.5) ; 
\end{tikzpicture}
  \column{6\textwidth}
  \lstset{basicstyle=\tiny}
   \begin{lstlisting}[escapechar=\|,label=noint]
#include <stdio.h>

int main (void)
{
int a,b,c;

	a = 6;
	b = 9;
	c = a + b;
	printf("El resultado es % d \n",c);

	a = 3;
	b = 1;
	c = a + b;
	printf("El resultado es % d \n",c);

	a = 12;
	b = 4;
	c = a + b;
	printf("El resultado es % d \n",c);

	return (0);
}   
   \end{lstlisting}
 \end{columns}
\end{frame}

\begin{frame}[fragile]
\fontsize{6.5pt}{12pt}\selectfont
 \frametitle{Tareas \textbf{\textit{\color{yellow}repetidas}}}
 \begin{columns}[c]
  \column{0.48\textwidth}
\begin{tikzpicture}
  \node at (6.5,7) {\normalsize C�digo en programa fuente};
% usado para hacer sombreado de un sector
  \fill[color=green] (5.7,5.25) -- (7.7,5.25) -- (7.7,4.4) -- (5.7,4.4) -- cycle;
  \fill[color=green] (5.7,4) -- (7.7,4) -- (7.7,3.17) -- (5.7,3.17) -- cycle;
  \fill[color=green] (5.7,2.8) -- (7.7,2.8) -- (7.7,1.9) -- (5.7,1.9) -- cycle;
  \draw[->](0.3,0.6)to [out=90,in=-90]node[right,midway]{} ++(0,0.5) ; 
\end{tikzpicture}
  \column{6\textwidth}
  \lstset{basicstyle=\tiny}
   \begin{lstlisting}[escapechar=\|,label=noint]
#include <stdio.h>

int main (void)
{
int a,b,c;

	a = 6;
	b = 9;
	c = a + b;
	printf("El resultado es % d \n",c);

	a = 3;
	b = 1;
	c = a + b;
	printf("El resultado es % d \n",c);

	a = 12;
	b = 4;
	c = a + b;
	printf("El resultado es % d \n",c);

	return (0);
}   
   \end{lstlisting}
 \end{columns}
\end{frame}

\begin{frame}[fragile]
\fontsize{6.5pt}{12pt}\selectfont
 \frametitle{Tareas \textbf{\textit{\color{yellow}repetidas}}}
 \begin{columns}[c]
  \column{0.48\textwidth}
\begin{tikzpicture}
  \node at (6.5,7) {\normalsize C�digo en programa fuente};
  \node at (3,3.8) {\fbox{\LARGE{c = a + b;}}};
% usado para hacer sombreado de un sector
  \fill[color=green] (5.7,5.25) -- (7.7,5.25) -- (7.7,4.4) -- (5.7,4.4) -- cycle;
  \fill[color=green] (5.7,4) -- (7.7,4) -- (7.7,3.17) -- (5.7,3.17) -- cycle;
  \fill[color=green] (5.7,2.8) -- (7.7,2.8) -- (7.7,1.9) -- (5.7,1.9) -- cycle;
  \draw[->](0.3,0.6)to [out=90,in=-90]node[right,midway]{} ++(0,0.5) ; 
\end{tikzpicture}
  \column{6\textwidth}
  \lstset{basicstyle=\tiny}
   \begin{lstlisting}[escapechar=\|,label=noint]
#include <stdio.h>

int main (void)
{
int a,b,c;

	
	
	
	printf("El resultado es % d \n",c);

	
	
	
	printf("El resultado es % d \n",c);

	
	
	
	printf("El resultado es % d \n",c);

	return (0);
}   
   \end{lstlisting}
 \end{columns}
\end{frame}

\begin{frame}[fragile]
\fontsize{6.5pt}{12pt}\selectfont
 \frametitle{Tareas \textbf{\textit{\color{yellow}repetidas}}}
 \begin{columns}[c]
  \column{0.48\textwidth}
\begin{tikzpicture}
  \node at (6.5,7) {\normalsize C�digo en programa fuente};
  \node at (3,3.8) {\fbox{\LARGE{c = a + b;}}};
% usado para hacer sombreado de un sector
  \fill[color=green] (5.7,5.25) -- (7.7,5.25) -- (7.7,4.4) -- (5.7,4.4) -- cycle;
  \fill[color=green] (5.7,4) -- (7.7,4) -- (7.7,3.17) -- (5.7,3.17) -- cycle;
  \fill[color=green] (5.7,2.8) -- (7.7,2.8) -- (7.7,1.9) -- (5.7,1.9) -- cycle;
  \draw[->](0.3,0.6)to [out=90,in=-90]node[right,midway]{} ++(0,0.5) ; 
  \draw[red!60,very thick,->](6.0,5)to [out=180,in=90]node[right,midway]{} ++(-2.9,-0.8) ; 
\end{tikzpicture}
  \column{6\textwidth}
  \lstset{basicstyle=\tiny}
   \begin{lstlisting}[escapechar=\|,label=noint]
#include <stdio.h>

int main (void)
{
int a,b,c;

	
	
	
	printf("El resultado es % d \n",c);

	
	
	
	printf("El resultado es % d \n",c);

	
	
	
	printf("El resultado es % d \n",c);

	return (0);
}   
   \end{lstlisting}
 \end{columns}
\end{frame}

\begin{frame}[fragile]
\fontsize{6.5pt}{12pt}\selectfont
 \frametitle{Tareas \textbf{\textit{\color{yellow}repetidas}}}
 \begin{columns}[c]
  \column{0.48\textwidth}
\begin{tikzpicture}
  \node at (6.5,7) {\normalsize C�digo en programa fuente};
  \node at (3,3.8) {\fbox{\LARGE{c = a + b;}}};
% usado para hacer sombreado de un sector
  \fill[color=green] (5.7,5.25) -- (7.7,5.25) -- (7.7,4.4) -- (5.7,4.4) -- cycle;
  \fill[color=green] (5.7,4) -- (7.7,4) -- (7.7,3.17) -- (5.7,3.17) -- cycle;
  \fill[color=green] (5.7,2.8) -- (7.7,2.8) -- (7.7,1.9) -- (5.7,1.9) -- cycle;
  \draw[->](0.3,0.6)to [out=90,in=-90]node[right,midway]{} ++(0,0.5) ; 
  \draw[red!60,very thick,->](6.0,5)to [out=180,in=90]node[right,midway]{} ++(-2.9,-0.8) ; 
  \draw[red!60,very thick,->](3.1,3.4)to [out=-90,in=-90]node[right,midway]{} ++(2.9,1.2) ; 
\end{tikzpicture}
  \column{6\textwidth}
  \lstset{basicstyle=\tiny}
   \begin{lstlisting}[escapechar=\|,label=noint]
#include <stdio.h>

int main (void)
{
int a,b,c;

	
	
	
	printf("El resultado es % d \n",c);

	
	
	
	printf("El resultado es % d \n",c);

	
	
	
	printf("El resultado es % d \n",c);

	return (0);
}   
   \end{lstlisting}
 \end{columns}
\end{frame}

\begin{frame}[fragile]
\fontsize{6.5pt}{12pt}\selectfont
 \frametitle{Tareas \textbf{\textit{\color{yellow}repetidas}}}
 \begin{columns}[c]
  \column{0.48\textwidth}
\begin{tikzpicture}
  \node at (6.5,7) {\normalsize C�digo en programa fuente};
  \node at (3,3.8) {\fbox{\LARGE{c = a + b;}}};
% usado para hacer sombreado de un sector
  \fill[color=green] (5.7,5.25) -- (7.7,5.25) -- (7.7,4.4) -- (5.7,4.4) -- cycle;
  \fill[color=green] (5.7,4) -- (7.7,4) -- (7.7,3.17) -- (5.7,3.17) -- cycle;
  \fill[color=green] (5.7,2.8) -- (7.7,2.8) -- (7.7,1.9) -- (5.7,1.9) -- cycle;
  \draw[->](0.3,0.6)to [out=90,in=-90]node[right,midway]{} ++(0,0.5) ; 
  \draw[red!60,very thick,->](6.0,3.8)to [out=150,in=90]node[right,midway]{} ++(-2.9,0.5) ; 
\end{tikzpicture}
  \column{6\textwidth}
  \lstset{basicstyle=\tiny}
   \begin{lstlisting}[escapechar=\|,label=noint]
#include <stdio.h>

int main (void)
{
int a,b,c;

	
	
	
	printf("El resultado es % d \n",c);

	
	
	
	printf("El resultado es % d \n",c);

	
	
	
	printf("El resultado es % d \n",c);

	return (0);
}   
   \end{lstlisting}
 \end{columns}
\end{frame}

\begin{frame}[fragile]
\fontsize{6.5pt}{12pt}\selectfont
 \frametitle{Tareas \textbf{\textit{\color{yellow}repetidas}}}
 \begin{columns}[c]
  \column{0.48\textwidth}
\begin{tikzpicture}
  \node at (6.5,7) {\normalsize C�digo en programa fuente};
  \node at (3,3.8) {\fbox{\LARGE{c = a + b;}}};
% usado para hacer sombreado de un sector
  \fill[color=green] (5.7,5.25) -- (7.7,5.25) -- (7.7,4.4) -- (5.7,4.4) -- cycle;
  \fill[color=green] (5.7,4) -- (7.7,4) -- (7.7,3.17) -- (5.7,3.17) -- cycle;
  \fill[color=green] (5.7,2.8) -- (7.7,2.8) -- (7.7,1.9) -- (5.7,1.9) -- cycle;
  \draw[->](0.3,0.6)to [out=90,in=-90]node[right,midway]{} ++(0,0.5) ; 
  \draw[red!60,very thick,->](6.0,3.8)to [out=150,in=90]node[right,midway]{} ++(-2.9,0.5) ; 
  \draw[red!60,very thick,->](3.1,3.4)to [out=-90,in=180]node[right,midway]{} ++(2.9,0.2) ; 
\end{tikzpicture}
  \column{6\textwidth}
  \lstset{basicstyle=\tiny}
   \begin{lstlisting}[escapechar=\|,label=noint]
#include <stdio.h>

int main (void)
{
int a,b,c;

	
	
	
	printf("El resultado es % d \n",c);

	
	
	
	printf("El resultado es % d \n",c);

	
	
	
	printf("El resultado es % d \n",c);

	return (0);
}   
   \end{lstlisting}
 \end{columns}
\end{frame}

\begin{frame}[fragile]
\fontsize{6.5pt}{12pt}\selectfont
 \frametitle{Tareas \textbf{\textit{\color{yellow}repetidas}}}
 \begin{columns}[c]
  \column{0.48\textwidth}
\begin{tikzpicture}
  \node at (6.5,7) {\normalsize C�digo en programa fuente};
  \node at (3,3.8) {\fbox{\LARGE{c = a + b;}}};
% usado para hacer sombreado de un sector
  \fill[color=green] (5.7,5.25) -- (7.7,5.25) -- (7.7,4.4) -- (5.7,4.4) -- cycle;
  \fill[color=green] (5.7,4) -- (7.7,4) -- (7.7,3.17) -- (5.7,3.17) -- cycle;
  \fill[color=green] (5.7,2.8) -- (7.7,2.8) -- (7.7,1.9) -- (5.7,1.9) -- cycle;
  \draw[->](0.3,0.6)to [out=90,in=-90]node[right,midway]{} ++(0,0.5) ; 
  \draw[red!60,very thick,->](6.0,2.5)to [out=180,in=-90]node[right,midway]{} ++(-2.9,0.9) ; 
\end{tikzpicture}
  \column{6\textwidth}
  \lstset{basicstyle=\tiny}
   \begin{lstlisting}[escapechar=\|,label=noint]
#include <stdio.h>

int main (void)
{
int a,b,c;

	
	
	
	printf("El resultado es % d \n",c);

	
	
	
	printf("El resultado es % d \n",c);

	
	
	
	printf("El resultado es % d \n",c);

	return (0);
}   
   \end{lstlisting}
 \end{columns}
\end{frame}

\begin{frame}[fragile]
\fontsize{6.5pt}{12pt}\selectfont
 \frametitle{Tareas \textbf{\textit{\color{yellow}repetidas}}}
 \begin{columns}[c]
  \column{0.48\textwidth}
\begin{tikzpicture}
  \node at (6.5,7) {\normalsize C�digo en programa fuente};
  \node at (3,3.8) {\fbox{\LARGE{c = a + b;}}};
% usado para hacer sombreado de un sector
  \fill[color=green] (5.7,5.25) -- (7.7,5.25) -- (7.7,4.4) -- (5.7,4.4) -- cycle;
  \fill[color=green] (5.7,4) -- (7.7,4) -- (7.7,3.17) -- (5.7,3.17) -- cycle;
  \fill[color=green] (5.7,2.8) -- (7.7,2.8) -- (7.7,1.9) -- (5.7,1.9) -- cycle;
  \draw[->](0.3,0.6)to [out=90,in=-90]node[right,midway]{} ++(0,0.5) ; 
  \draw[red!60,very thick,->](6.0,2.5)to [out=180,in=-90]node[right,midway]{} ++(-2.9,0.9) ; 
  \draw[red!60,very thick,->](3.1,4.2)to [out= 90,in= 90]node[right,midway]{} ++(3.6,-1.6) ; 
\end{tikzpicture}
  \column{6\textwidth}
  \lstset{basicstyle=\tiny}
   \begin{lstlisting}[escapechar=\|,label=noint]
#include <stdio.h>

int main (void)
{
int a,b,c;

	
	
	
	printf("El resultado es % d \n",c);

	
	
	
	printf("El resultado es % d \n",c);

	
	
	
	printf("El resultado es % d \n",c);

	return (0);
}   
   \end{lstlisting}
 \end{columns}
\end{frame}

\begin{frame}[fragile]
\fontsize{6.5pt}{12pt}\selectfont
 \frametitle{\textbf{\textit{\color{yellow}Funciones}}}
 \begin{columns}[c]
  \column{0.48\textwidth}
\begin{tikzpicture}
  \node at (6.5,7) {\normalsize C�digo en programa fuente};
  \node at (2.8,3.8) {\fbox{
						\begin{minipage}{3.5cm}
							\fontsize{8pt}{8pt}\selectfont
		{\tiny{1}} \hspace{0.05cm} {\color{blue}int} Mi\_suma ( {\color{blue}int} x , {\color{blue}int} y ) \\
		{\tiny{2}} \hspace{0.05cm} \{ \\
		{\tiny{3}} \hspace{0.05cm} {\color{blue}int} z; \\
		{\tiny{4}} \hspace{0.5cm} z = x + y; \\
		{\tiny{5}} \hspace{0.5cm} {\color{blue}return} z; \\
		{\tiny{6}} \hspace{0.05cm} \{
						\end{minipage}}};						  
% usado para hacer sombreado de un sector
  \fill[color=green] (5.7,5.25) -- (7.7,5.25) -- (7.7,4.4) -- (5.7,4.4) -- cycle;
  \fill[color=green] (5.7,4) -- (7.7,4) -- (7.7,3.17) -- (5.7,3.17) -- cycle;
  \fill[color=green] (5.7,2.8) -- (7.7,2.8) -- (7.7,1.9) -- (5.7,1.9) -- cycle;
  \draw[->](0.3,0.6)to [out=90,in=-90]node[right,midway]{} ++(0,0.5) ; 
\end{tikzpicture}
  \column{6\textwidth}
  \lstset{basicstyle=\tiny}
   \begin{lstlisting}[escapechar=\|,label=noint]
#include <stdio.h>

int main (void)
{
int a,b,c;

	
	
	
	printf("El resultado es % d \n",c);

	
	
	
	printf("El resultado es % d \n",c);

	
	
	
	printf("El resultado es % d \n",c);

	return (0);
}   
   \end{lstlisting}
 \end{columns}
\end{frame}

\begin{frame}[fragile]
\fontsize{6.5pt}{12pt}\selectfont
 \frametitle{\textbf{\textit{\color{yellow}Funciones}}}
 \begin{columns}[c]
  \column{0.48\textwidth}
\begin{tikzpicture}
  \node at (6.5,7) {\normalsize C�digo en programa fuente};
%  \node at (3,3.8) {\fbox{\small{c = a + b;}}};
  \node at (2.8,3.8) {\fbox{
						\begin{minipage}{3.5cm}
							\fontsize{8pt}{8pt}\selectfont
		{\tiny{1}} \hspace{0.05cm} {\color{blue}int} Mi\_suma ( {\color{blue}int} x , {\color{blue}int} y ) \\
		{\tiny{2}} \hspace{0.05cm} \{ \\
		{\tiny{3}} \hspace{0.05cm} {\color{blue}int} z; \\
		{\tiny{4}} \hspace{0.5cm} z = x + y; \\
		{\tiny{5}} \hspace{0.5cm} {\color{blue}return} z; \\
		{\tiny{6}} \hspace{0.05cm} \{
						\end{minipage}}};						  
% usado para hacer sombreado de un sector
  \fill[color=green] (5.7,5.25) -- (7.7,5.25) -- (7.7,4.4) -- (5.7,4.4) -- cycle;
  \fill[color=green] (5.7,4) -- (7.7,4) -- (7.7,3.17) -- (5.7,3.17) -- cycle;
  \fill[color=green] (5.7,2.8) -- (7.7,2.8) -- (7.7,1.9) -- (5.7,1.9) -- cycle;
  \draw[->](0.3,0.6)to [out=90,in=-90]node[right,midway]{} ++(0,0.5) ; 
  \draw[red,decorate,decoration={brace,raise=6pt,amplitude=12pt},thick] (1.4,4.5) -- (4.6,4.5) node [black,midway,xshift=0.1cm,yshift=1cm]{\footnotesize $\acute{u}nico$ $punto$ $de$ $entrada$};
\end{tikzpicture}
  \column{6\textwidth}
  \lstset{basicstyle=\tiny}
   \begin{lstlisting}[escapechar=\|,label=noint]
#include <stdio.h>

int main (void)
{
int a,b,c;

	
	
	
	printf("El resultado es % d \n",c);

	
	
	
	printf("El resultado es % d \n",c);

	
	
	
	printf("El resultado es % d \n",c);

	return (0);
}   
   \end{lstlisting}
 \end{columns}
\end{frame}

\begin{frame}[fragile]
\fontsize{6.5pt}{12pt}\selectfont
 \frametitle{\textbf{\textit{\color{yellow}Funciones}}}
 \begin{columns}[c]
  \column{0.48\textwidth}
\begin{tikzpicture}
  \node at (6.5,7) {\normalsize C�digo en programa fuente};
%  \node at (3,3.8) {\fbox{\small{c = a + b;}}};
  \node at (2.8,3.8) {\fbox{
						\begin{minipage}{3.5cm}
							\fontsize{8pt}{8pt}\selectfont
		{\tiny{1}} \hspace{0.05cm} {\color{blue}int} Mi\_suma ( {\color{blue}int} x , {\color{blue}int} y ) \\
		{\tiny{2}} \hspace{0.05cm} \{ \\
		{\tiny{3}} \hspace{0.05cm} {\color{blue}int} z; \\
		{\tiny{4}} \hspace{0.5cm} z = x + y; \\
		{\tiny{5}} \hspace{0.5cm} {\color{blue}return} z; \\
		{\tiny{6}} \hspace{0.05cm} \{
						\end{minipage}}};						  
% usado para hacer sombreado de un sector
  \fill[color=green] (5.7,5.25) -- (7.7,5.25) -- (7.7,4.4) -- (5.7,4.4) -- cycle;
  \fill[color=green] (5.7,4) -- (7.7,4) -- (7.7,3.17) -- (5.7,3.17) -- cycle;
  \fill[color=green] (5.7,2.8) -- (7.7,2.8) -- (7.7,1.9) -- (5.7,1.9) -- cycle;
  \draw[->](0.3,0.6)to [out=90,in=-90]node[right,midway]{} ++(0,0.5) ; 
  \draw[red,decorate,decoration={brace,raise=6pt,amplitude=6pt},thick] (2.9,3.3) -- (1.8,3.3) node [black,midway,xshift=0.1cm,yshift=-1cm]{\footnotesize $\acute{u}nico$ $punto$ $de$ $salida$};
\end{tikzpicture}
  \column{6\textwidth}
  \lstset{basicstyle=\tiny}
   \begin{lstlisting}[escapechar=\|,label=noint]
#include <stdio.h>

int main (void)
{
int a,b,c;

	
	
	
	printf("El resultado es % d \n",c);

	
	
	
	printf("El resultado es % d \n",c);

	
	
	
	printf("El resultado es % d \n",c);

	return (0);
}   
   \end{lstlisting}
 \end{columns}
\end{frame}

\begin{frame}[fragile]
\fontsize{6.5pt}{12pt}\selectfont
 \frametitle{\textbf{\textit{\color{yellow}Funciones}}}
 \begin{columns}[c]
  \column{0.48\textwidth}
\begin{tikzpicture}
  \node at (6.5,7) {\normalsize C�digo en programa fuente};
  \node at (2.8,3.8) {\fbox{
						\begin{minipage}{3.5cm}
							\fontsize{8pt}{8pt}\selectfont
		{\tiny{1}} \hspace{0.05cm} {\color{blue}int} Mi\_suma ( {\color{blue}int} x , {\color{blue}int} y ) \\
		{\tiny{2}} \hspace{0.05cm} \{ \\
		{\tiny{3}} \hspace{0.05cm} {\color{blue}int} z; \\
		{\tiny{4}} \hspace{0.5cm} z = x + y; \\
		{\tiny{5}} \hspace{0.5cm} {\color{blue}return} z; \\
		{\tiny{6}} \hspace{0.05cm} \{
						\end{minipage}}};						  
% usado para hacer sombreado de un sector
  \fill[color=green] (5.7,5.25) -- (7.7,5.25) -- (7.7,4.4) -- (5.7,4.4) -- cycle;
  \fill[color=green] (5.7,4) -- (7.7,4) -- (7.7,3.17) -- (5.7,3.17) -- cycle;
  \fill[color=green] (5.7,2.8) -- (7.7,2.8) -- (7.7,1.9) -- (5.7,1.9) -- cycle;
  \draw[->](0.3,0.6)to [out=90,in=-90]node[right,midway]{} ++(0,0.5) ; 
  \draw[red,decorate,decoration={brace,raise=6pt,amplitude=6pt},thick] (1.7,4.5) -- (3,4.5) node [black,midway,xshift=0.1cm,yshift=1cm]{\footnotesize $nombre$ $de$ $la$ $funci\acute{o}n$};
\end{tikzpicture}
  \column{6\textwidth}
  \lstset{basicstyle=\tiny}
   \begin{lstlisting}[escapechar=\|,label=noint]
#include <stdio.h>

int main (void)
{
int a,b,c;

	
	
	
	printf("El resultado es % d \n",c);

	
	
	
	printf("El resultado es % d \n",c);

	
	
	
	printf("El resultado es % d \n",c);

	return (0);
}   
   \end{lstlisting}
 \end{columns}
\end{frame}

\begin{frame}[fragile]
\fontsize{6.5pt}{12pt}\selectfont
 \frametitle{\textbf{\textit{\color{yellow}Funciones}}}
 \begin{columns}[c]
  \column{0.48\textwidth}
\begin{tikzpicture}
  \node at (6.5,7) {\normalsize C�digo en programa fuente};
  \node at (2.8,3.8) {\fbox{
						\begin{minipage}{3.5cm}
							\fontsize{8pt}{8pt}\selectfont
		{\tiny{1}} \hspace{0.05cm} {\color{blue}int} Mi\_suma ( {\color{blue}int} x , {\color{blue}int} y ) \\
		{\tiny{2}} \hspace{0.05cm} \{ \\
		{\tiny{3}} \hspace{0.05cm} {\color{blue}int} z; \\
		{\tiny{4}} \hspace{0.5cm} z = x + y; \\
		{\tiny{5}} \hspace{0.5cm} {\color{blue}return} z; \\
		{\tiny{6}} \hspace{0.05cm} \{
						\end{minipage}}};						  
% usado para hacer sombreado de un sector
  \fill[color=green] (5.7,5.25) -- (7.7,5.25) -- (7.7,4.4) -- (5.7,4.4) -- cycle;
  \fill[color=green] (5.7,4) -- (7.7,4) -- (7.7,3.17) -- (5.7,3.17) -- cycle;
  \fill[color=green] (5.7,2.8) -- (7.7,2.8) -- (7.7,1.9) -- (5.7,1.9) -- cycle;
  \draw[->](0.3,0.6)to [out=90,in=-90]node[right,midway]{} ++(0,0.5) ; 
  \draw[red,decorate,decoration={brace,raise=6pt,amplitude=6pt},thick] (3,4.5) -- (4.5,4.5) node [black,midway,xshift=-1cm,yshift=1cm]{\footnotesize $par\acute{a}metros$ $o$ $argumentos$};
\end{tikzpicture}
  \column{6\textwidth}
  \lstset{basicstyle=\tiny}
   \begin{lstlisting}[escapechar=\|,label=noint]
#include <stdio.h>

int main (void)
{
int a,b,c;

	
	
	
	printf("El resultado es % d \n",c);

	
	
	
	printf("El resultado es % d \n",c);

	
	
	
	printf("El resultado es % d \n",c);

	return (0);
}   
   \end{lstlisting}
 \end{columns}
\end{frame}

\begin{frame}[fragile]
\fontsize{6.5pt}{12pt}\selectfont
 \frametitle{\textbf{\textit{\color{yellow}Funciones}}}
 \begin{columns}[c]
  \column{0.48\textwidth}
\begin{tikzpicture}
  \node at (6.5,7) {\normalsize C�digo en programa fuente};
%  \node at (3,3.8) {\fbox{\small{c = a + b;}}};
  \node at (2.8,3.8) {\fbox{
						\begin{minipage}{3.5cm}
							\fontsize{8pt}{8pt}\selectfont
		{\tiny{1}} \hspace{0.05cm} {\color{blue}int} Mi\_suma ( {\color{blue}int} x , {\color{blue}int} y ) \\
		{\tiny{2}} \hspace{0.05cm} \{ \\
		{\tiny{3}} \hspace{0.05cm} {\color{blue}int} z; \\
		{\tiny{4}} \hspace{0.5cm} z = x + y; \\
		{\tiny{5}} \hspace{0.5cm} {\color{blue}return} z; \\
		{\tiny{6}} \hspace{0.05cm} \{
						\end{minipage}}};						  
% usado para hacer sombreado de un sector
  \fill[color=green] (5.7,5.25) -- (7.7,5.25) -- (7.7,4.4) -- (5.7,4.4) -- cycle;
  \fill[color=green] (5.7,4) -- (7.7,4) -- (7.7,3.17) -- (5.7,3.17) -- cycle;
  \fill[color=green] (5.7,2.8) -- (7.7,2.8) -- (7.7,1.9) -- (5.7,1.9) -- cycle;
  \draw[->](0.3,0.6)to [out=90,in=-90]node[right,midway]{} ++(0,0.5) ; 
  \draw[red,decorate,decoration={brace,raise=8pt,amplitude=2pt},thick] (1.2,4.5) -- (1.8,4.5) node [black,midway,xshift=0.1cm,yshift=0.8cm]{\footnotesize $tipo$ $de$ $retorno$};
\end{tikzpicture}
  \column{6\textwidth}
  \lstset{basicstyle=\tiny}
   \begin{lstlisting}[escapechar=\|,label=noint]
#include <stdio.h>

int main (void)
{
int a,b,c;

	
	
	
	printf("El resultado es % d \n",c);

	
	
	
	printf("El resultado es % d \n",c);

	
	
	
	printf("El resultado es % d \n",c);

	return (0);
}   
   \end{lstlisting}
 \end{columns}
\end{frame}

\begin{frame}[fragile]
\fontsize{6.5pt}{12pt}\selectfont
 \frametitle{\textbf{\textit{\color{yellow}Funciones}}}
 \begin{columns}[c]
  \column{0.48\textwidth}
\begin{tikzpicture}
  \node at (6.5,7) {\normalsize C�digo en programa fuente};
%  \node at (3,3.8) {\fbox{\small{c = a + b;}}};
  \node at (2.8,3.8) {\fbox{
						\begin{minipage}{3.5cm}
							\fontsize{8pt}{8pt}\selectfont
		{\tiny{1}} \hspace{0.05cm} {\color{blue}int} Mi\_suma ( {\color{blue}int} x , {\color{blue}int} y ) \\
		{\tiny{2}} \hspace{0.05cm} \{ \\
		{\tiny{3}} \hspace{0.05cm} {\color{blue}int} z; \\
		{\tiny{4}} \hspace{0.5cm} z = x + y; \\
		{\tiny{5}} \hspace{0.5cm} {\color{blue}return} z; \\
		{\tiny{6}} \hspace{0.05cm} \{
						\end{minipage}}};						  
% usado para hacer sombreado de un sector
  \fill[color=green] (5.7,5.25) -- (7.7,5.25) -- (7.7,4.4) -- (5.7,4.4) -- cycle;
  \fill[color=green] (5.7,4) -- (7.7,4) -- (7.7,3.17) -- (5.7,3.17) -- cycle;
  \fill[color=green] (5.7,2.8) -- (7.7,2.8) -- (7.7,1.9) -- (5.7,1.9) -- cycle;
  \draw[->](0.3,0.6)to [out=90,in=-90]node[right,midway]{} ++(0,0.5) ; 
  \draw[red,decorate,decoration={brace,raise=8pt,amplitude=2pt},thick] (1.2,4.5) -- (1.8,4.5) node [black,midway,xshift=0.1cm,yshift=0.8cm]{\footnotesize $\color{blue}tipo$ $de$ $retorno$};
  \draw[red,decorate,decoration={brace,raise=6pt,amplitude=6pt},thick] (2.9,3.3) -- (1.8,3.3) 
  node [black,midway,xshift=0.1cm,yshift=-1cm]{\footnotesize $la$ $variable$ $de$ $retorno$}
  node [black,midway,xshift=0.1cm,yshift=-1.4cm]{\footnotesize $tiene$ $que$ $ser$ $del$ $mismo$ $\color{blue}tipo$};
\draw[red,thick] (2.75,3.4) circle (0.2cm);
\draw[red,thick] (1.55,4.5) circle (0.22cm);
\end{tikzpicture}
  \column{6\textwidth}
  \lstset{basicstyle=\tiny}
   \begin{lstlisting}[escapechar=\|,label=noint]
#include <stdio.h>

int main (void)
{
int a,b,c;

	
	
	
	printf("El resultado es % d \n",c);

	
	
	
	printf("El resultado es % d \n",c);

	
	
	
	printf("El resultado es % d \n",c);

	return (0);
}   
   \end{lstlisting}
 \end{columns}
\end{frame}

\begin{frame}[fragile]
\fontsize{6.5pt}{12pt}\selectfont
 \frametitle{Ejemplo de un programa con una \textbf{\textit{\color{yellow}funci�n}}}
 \begin{columns}[c]
  \column{0.48\textwidth}
\begin{tikzpicture}
  \node at (6.5,7.8) {\normalsize C�digo en programa fuente};
  \draw[->](0.3,0.6)to [out=90,in=-90]node[right,midway]{} ++(0,0.5) ; 
  \fill[color=green] (5.4,6.9) -- (8.8,6.9) -- (8.8,6.5) -- (5.4,6.5) -- cycle;
  \draw[red!60,very thick,->](5.4,6.75)to [out=180,in=90]node[right,midway]{} ++(-2.9,-0.8) ; 
  \node at (2.5,5.5) {\normalsize PROTOTIPO};
  \node at (2.5,5.1) {\normalsize $informa$ $como$ $tiene$};
  \node at (2.5,4.7) {\normalsize $que$ $ser$ $la$ $funci\acute{o}n$};
\end{tikzpicture}
  \column{2.9\textwidth}
  \lstset{basicstyle=\tiny}
   \begin{lstlisting}[escapechar=\|,label=noint]
#include <stdio.h>

int Mi_suma ( int , int );

int main (void)
{
int a;

	a = Mi_suma ( 6 , 9 );
	printf("El resultado es % d \n",a);

	a = Mi_suma ( 3 , 1 );
	printf("El resultado es % d \n",a);

	a = Mi_suma ( 12 , 4 );
	printf("El resultado es % d \n",a);

	return (0);
}   

int Mi_suma ( int x , int y )
{
int z;
    z = x + y;
    return z;
}    
	\end{lstlisting}
  \end{columns}
\end{frame}

\begin{frame}[fragile]
\fontsize{6.5pt}{12pt}\selectfont
 \frametitle{Ejemplo de un programa con una \textbf{\textit{\color{yellow}funci�n}}}
 \begin{columns}[c]
  \column{0.48\textwidth}
\begin{tikzpicture}
  \node at (6.5,7.8) {\normalsize C�digo en programa fuente};
  \draw[->](0.3,0.6)to [out=90,in=-90]node[right,midway]{} ++(0,0.5) ; 
  \fill[color=green] (6.3,5.15) -- (8.8,5.15) -- (8.8,5.45) -- (6.3,5.45) -- cycle;
  \draw[red!60,very thick,->](6.8,5.4)to [out=135,in=180]node[right,midway]{} ++(-1.7,-3) ; 
  \node at (2.5,5.1) {\normalsize $llama$ $a$ $la$ $funci\acute{o}n$};
\end{tikzpicture}
  \column{2.9\textwidth}
  \lstset{basicstyle=\tiny}
   \begin{lstlisting}[escapechar=\|,label=noint]
#include <stdio.h>

int Mi_suma ( int , int );

int main (void)
{
int a;

	a = Mi_suma ( 6 , 9 );
	printf("El resultado es % d \n",a);

	a = Mi_suma ( 3 , 1 );
	printf("El resultado es % d \n",a);

	a = Mi_suma ( 12 , 4 );
	printf("El resultado es % d \n",a);

	return (0);
}   

int Mi_suma ( int x , int y )
{
int z;
    z = x + y;
    return z;
}    
	\end{lstlisting}
  \end{columns}
\end{frame}

\begin{frame}[fragile]
\fontsize{6.5pt}{12pt}\selectfont
 \frametitle{Ejemplo de un programa con una \textbf{\textit{\color{yellow}funci�n}}}
 \begin{columns}[c]
  \column{0.48\textwidth}
\begin{tikzpicture}
  \node at (6.5,7.8) {\normalsize C�digo en programa fuente};
  \draw[->](0.3,0.6)to [out=90,in=-90]node[right,midway]{} ++(0,0.5) ; 
  \fill[color=green] (5.6,5.15) -- (6,5.15) -- (6,5.45) -- (5.6,5.45) -- cycle;
  \draw[red!60,very thick,<-](5.6,5.2)to [out=180,in=180]node[right,midway]{} ++(0.3,-3.9) ; 
  \node at (2.5,5.1) {\normalsize $retorna$ $de$ $la$ $funci\acute{o}n$};
  \node at (2.5,4.6) {\normalsize $y$ $guarda$ $el$ $valorn$ $de$};
  \node at (2.5,4.1) {\normalsize $\color{red}z$ $en$ $\color{red}a$};
\end{tikzpicture}
  \column{2.9\textwidth}
  \lstset{basicstyle=\tiny}
   \begin{lstlisting}[escapechar=\|,label=noint]
#include <stdio.h>

int Mi_suma ( int , int );

int main (void)
{
int a;

	a = Mi_suma ( 6 , 9 );
	printf("El resultado es % d \n",a);

	a = Mi_suma ( 3 , 1 );
	printf("El resultado es % d \n",a);

	a = Mi_suma ( 12 , 4 );
	printf("El resultado es % d \n",a);

	return (0);
}   

int Mi_suma ( int x , int y )
{
int z;
    z = x + y;
    return z;
}    
	\end{lstlisting}
  \end{columns}
\end{frame}

\end{document}



\documentclass[12pt]{article} % Clase de documento: artículo y tamaño de letra
 
% \usepackage[spanish]{babel} % Manejo de idiomas
\usepackage[utf8]{inputenc} % Escritura en castellano con acentos
\usepackage[T1]{fontenc} % Escritura en castellano con acentos
\usepackage{calligra} 
\usepackage{listings}
\usepackage{pslatex}  % Fuente de letras
\usepackage{graphicx}
\usepackage{array}
%\usepackage{times} % Fuente de letras
\usepackage[margin=2.5cm]{geometry}
%Coloca 2.5 cm de margenes superior, inferior,derecho e izquierdo.
\usepackage{caption}
\usepackage{txfonts}
\usepackage{xcolor}
\usepackage{fancyhdr}

\usepackage[spanish,activeacute]{babel}
\usepackage{float}
\usepackage{multicol}
\usepackage{color}
\usepackage{times}

\usepackage{tikz}
\usepackage{verbatim}
\usepackage{mdwlist}


\usetikzlibrary{chains,fit,shapes,arrows,calc,shapes,decorations.pathreplacing}
\usetikzlibrary{through}


\DeclareCaptionFont{white}{\color{white}}
\DeclareCaptionFormat{listing}{\colorbox[cmyk]{0.43, 0.35, 0.35,0.01}{\parbox{\textwidth}{\hspace{15pt}#1#2#3}}}

%Paquetes para hacer tablas lindas...
% \usepackage{amsmath,amssymb,amsfonts,latexsym,stmaryrd}
\usepackage{tabularx}
\usepackage{colortbl}
\usepackage{shadow}
\usepackage{fancybox}
\usepackage{url}
\usepackage[hidelinks]{hyperref}
\usepackage{subfigure}
\usepackage{multirow}


\definecolor{OliveGreen}{RGB}{2,80,1}
\definecolor{LigthOrange}{RGB}{255,255,200}
\definecolor{gray97}{gray}{.97}
\definecolor{Gray}{RGB}{171,174,178}

\newcommand{\instr}[1]{{\sffamily{\small{\textsl{\textbf{#1}}}}}}
\newcommand{\code}[1]{{\sffamily{\smal\textheight = 24cm \textwidth = 16cm \topmargin = -1cm \oddsidemargin= 0cm {\textsl{#1}}}}}

\lstdefinelanguage {x86nasm}
{morekeywords={resb,resw,resd,resq,endstruc,at,istruc,iend}}

\pagestyle{fancy}
\headheight=50pt %para cambiar el tamaño del encabezado
\fancyhead[L]    %la "L" indica a la izquierda
{	
 \begin{minipage}{3.3cm}
  \includegraphics[width=1.0\textwidth]{Logo-UTN-BA.jpeg}
 \end{minipage}	
 \begin{minipage}{7.7cm}
\fontsize{13.5pt}{12pt}\selectfont
%   \normalsize
   {
     \textsl 
     {
       \calligra{Universidad Tecnológica Nacional\\ Facultad Regional Buenos Aires \\ Departamento de {Ingeniería} {Electrónica}} 
     }
   }
 \end{minipage}
}

\fancyhead[R] %la "R" indica a la derecha
{
  \begin{minipage}{4.0cm}
   \small
   {
     \emph{\textbf{Informática I}} \\ \emph{14 de noviembre de 2020} \\ \emph{Practica $2^o$Parcial}\\ \emph{Curso R1021}  
   }
  \end{minipage}}

\definecolor{OliveGreen}{RGB}{2,80,1}
\definecolor{LigthOrange}{RGB}{255,255,200}
\definecolor{gray97}{gray}{.97}



\begin{document} % Inicio del documento
\newpage
%Cambia de página, el texto después de este comando aparecerá en la siguiente página en adelante.


\noindent
  \begin{center}
   \begin{tabular}{| c | c | c |}
    \hline
     Apellido y Nombres \hspace{8cm} &  Legajo & {Calificación} \\ \hline 
      &	& \\ \hline
   \end{tabular}	
  \end{center}

\noindent

%%%%%%%%%%%%%%%%%%%%%%%%%%%%%%%%%%%%%%%%%%%%%%%%%%%%%%%%%%%%%%%%%%%%%%%%%%%%%%%%%%%%%%%%%%%%%%%%%%%%%%%%%%%%%%%
%%%%%%%%%%%%%%%%%%%%%%%%%%%%%%%%%%%%%%%%%%%%%%%%%%%%%%%%%%%%%%%%%%%%%%%%%%%%%%%%%%%%%%%%%%%%%%%%%%%%%%%%%%%%%%%
% Comienzo del tema del examen
%%%%%%%%%%%%%%%%%%%%%%%%%%%%%%%%%%%%%%%%%%%%%%%%%%%%%%%%%%%%%%%%%%%%%%%%%%%%%%%%%%%%%%%%%%%%%%%%%%%%%%%%%%%%%%%
%%%%%%%%%%%%%%%%%%%%%%%%%%%%%%%%%%%%%%%%%%%%%%%%%%%%%%%%%%%%%%%%%%%%%%%%%%%%%%%%%%%%%%%%%%%%%%%%%%%%%%%%%%%%%%%

\lstset{
	frame=Ltb,
	framerule=0pt,
	aboveskip=0.5cm,
	framextopmargin=3pt,
	framexbottommargin=3pt,
	framexleftmargin=0.4cm,
	framesep=0pt,
	rulesep=.4pt,
	backgroundcolor=\color{gray97},
	rulesepcolor=\color{black},
 	language=C,
	captionpos=b,
	tabsize=3,
	frame=lines,
	keywordstyle=\color{blue},
	commentstyle=\color{Gray},
	stringstyle=\color{red},
	numbers=left,
	numberstyle=\tiny,
	numbersep=5pt,
	breaklines=true,
	showstringspaces=false,
	basicstyle=\ttfamily\scriptsize,
	emph={label},
	framerule=0pt,
}

El programa consiste en leer los datos que se obtuvieron de diferentes sensores y se encuentran almacenados en  un archivo con la siguiente estructura de datos:
\begin{lstlisting}[escapechar=\|,label=noint]
typedef struct{
    int id;
    int timestamp;
    float valor;
    char unidad[10];
} sensor_t;
\end{lstlisting}
Los datos de cada sensor son los siguientes:
\begin{itemize}
\item Id
\item timestamp del momento de la medición
\item valor de la medición 
\item unidad de medida
\end{itemize}
Se pide que el programa:
\begin{itemize}
\item cuente la cantidad de registros almacenados en el archivo 
\item Despliegue un menú para seleccionar con que unidad de medida se debe ordenar el archivo
\begin{enumerate}
\item TENSION (V)
\item CORRIENTE (A)
\item RESISTENCIA (OHM)
\item CAPACIDAD (uF)
\item FRECUENCIA (Hz)
\item PRESION (Pa)
\item FUERZA (N)
\item VELOCIDAD (m/seg)
\item ACELERACION (m/$seg^2$)
\item LUZ (Cd)
\end{enumerate} 
\item guardar el archivo ordenado conservando la estructura de los datos con el nuevo nombre [nombre del archivo]-salida.dat ( ejemplo: el archivo de datos se llama {\color{blue}datos.dat} y el de salida será {\color{blue}datos-salida.dat}
\item guardar el archivo ordenado y en formato texto con el nuevo nombre [nombre del archivo]-salida.txt ( ejemplo: el archivo de datos se llama {\color{blue}datos.dat} y el de salida será {\color{blue}datos-salida.txt}
\item Calcule y presente las 10 últimas mediciones para la unidad seleccionada
\end{itemize}  
{\bf Ejemplo de salida:}\\
\\
Unidad: A\\
Cantidad de Mediciones: 53\\
Promedio: 10.5 A\\
Máximo: 11.3 A\\
Mínimo: 10.232 A\\ \\
   \begin{tabular}{c c c c c}
    
     {\bf Posición}&{\bf ID}&{\bf Timestamp}&{\bf Valor}&{\bf Unidad} \\  
     $1$  &$4$&$03/10/2019 \hspace{0.2cm} 03:22:45$&$11.30$&$A$\\  
     $2$  &$4$&$03/10/2019 \hspace{0.2cm} 03:22:25$&$11.05$&$A$\\  
     $3$  &$4$&$03/10/2019 \hspace{0.2cm} 03:21:35$&$10.30$&$A$\\  
     $...$&   &                                    &       &   \\  
     $10$ &$4$&$03/10/2019 \hspace{0.2cm} 03:20:55$&$10.95$&$A$\\  
   \end{tabular}
\vspace{5mm}

{\bf Notas:}\\
\begin{itemize}
\item La memoria para cada estructura debe ser pedida dinámicamente (malloc)
\item Utilice el archivo “mediciones.dat” provisto para probar el programa
\item Para realizar el ordenamiento no debe realizar ningún tipo de copia de datos,
solamente hacer un swap de los punteros
\item No olvide liberar la memoria una vez finalizado el programa y cerrar los archivos que se encuentren abiertos
\item Recuerde que el timestamp es la cantidad de segundos que pasaron desde el 1
de enero de 1970 a las 0hs en el Meridiano de Greenwich. El header “time.h”
contiene los prototipos de las funciones para utilizarlo
\end{itemize}

 
\end{document} % Fin del documento.
 

  

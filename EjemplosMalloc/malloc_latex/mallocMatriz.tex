\documentclass{beamer}
\usepackage[latin1]{inputenc}
\usepackage[spanish]{babel}
\usepackage{multicol}
\usepackage{fancybox}
\usepackage{beamerthemeshadow}
\usepackage{times} %font times
\usepackage[T1]{fontenc} %para que cuando se seleccione un texto las letras acentuadad y las � se copien bien Usar la codificaci�n T1
\usepackage{enumerate}
\usepackage{listings}
\usepackage{calligra} 
\usepackage{graphicx}
\usepackage{array}
\usepackage{caption}
\usepackage{tikz}
\usepackage{verbatim}
\usepackage{mdwlist}

\usetikzlibrary{chains,fit,shapes,arrows,calc,decorations.pathreplacing}
\usetikzlibrary{shapes.geometric,matrix,fit,backgrounds}
\usetikzlibrary{shapes.arrows}

\usefonttheme{professionalfonts}

\newtheorem{defi}{Definici�n} 
\hypersetup{pdfpagemode=FullScreen}

\mode<presentation>{
\usetheme{Warsaw}
\setbeamercovered{transparent}

}
\lstset{
	frame=Ltb,
	framerule=0pt,
	aboveskip=0.5cm,
	framextopmargin=3pt,
	framexbottommargin=3pt,
	framexleftmargin=0.4cm,
	framesep=0pt,
	rulesep=.4pt,
%	backgroundcolor=\color{gray!20},
	rulesepcolor=\color{black},
	language=C,
	captionpos=b,
	tabsize=3,
	frame=lines,
	keywordstyle=\color{blue},
	commentstyle=\color{gray},
	stringstyle=\color{red},
	numbers=left,
	numberstyle=\tiny,
	numbersep=5pt,
breaklines=true,
	showstringspaces=false,
	basicstyle=\small,
	emph={label},
	framerule=0pt,
}

\title{\em INFORMATICA I}
\subtitle{{Matriz dinamica con \color{yellow}malloc}}
\author{\em Ing.Juan Carlos Cuttitta}
\institute{\Large \calligra{Universidad Tecnol�gica Nacional\\ Facultad Regional Buenos Aires \\ Departamento de {Ingenier�a} {Electr�nica}}}
\date{\today}

%portada

\setbeamertemplate{footline}[frame number]
%para sacar la barra que tiene beamer por defoult
\setbeamertemplate{navigation symbols}{} 

\begin{document}

\begin{figure}[ht!]
  \centering
  \includegraphics [width=0.2\textwidth]{informacion.jpg}
\end{figure}
\vspace{-1.0cm} % para subir el titulo 
\titlepage

%%%%%%%%%%%%%%%%%%%%%%%%%%%%%%%%%%%%%%%%%%%%%%%%

\tikzstyle{every picture}+=[remember picture]
\tikzstyle{na} = [baseline=-.5ex]


\begin{frame}[fragile]
\frametitle{Enunciado del problema}
\fontsize{16pt}{18pt}\selectfont
Ejemplo para armar una matriz (N x N) en forma dinamica utilizando malloc. \\
\vspace{1cm}
En �ste ejemplo la matriz ser� de (2 x 3) quedando de la siguente forma: \\
\vspace{1cm}
\begin{equation*}
\begin{bmatrix}
\color{blue}11  & \color{blue}12 & \color{blue}13 \\
\color{blue}21 & \color{blue}22 & \color{blue}23 
\end{bmatrix} 
\end{equation*}
\end{frame}





\begin{frame}[fragile]
\fontsize{6.5pt}{12pt}\selectfont
 \frametitle{Matriz {\color{yellow}NxN} con {\color{yellow}malloc} en Arquitectura \textcolor{yellow}{X86-32} bits }
 \begin{columns}[c]
  \column{0.4\textwidth}
\lstset{basicstyle=\tiny}
\begin{tikzpicture}
  \begin{scope}[
  	every node/.style={draw, 
  	anchor=text, 
  	rectangle split,
    rectangle split parts=19, 
    rectangle split part fill={green!60,green!60,green!60,green!60,green!60,red!0,red!0,red!0,red!0,red!0,red!0,red!0,blue!0,red!0,red!0,red!0,red!0,red!0,red!0,red!0},
    minimum width=0.6cm}]
    \node (R) at (-0.5,8){ 
    		\nodepart{one}\tiny 0
    		\nodepart{two}\tiny 0
    		\nodepart{three}\tiny 0
    		\nodepart{four}\tiny 0
    		\nodepart{five}\tiny  0xXXXXXXXX
    		\nodepart{six}{$\vdots$}
    		\nodepart{seven}
    		\nodepart{eight}
    		\nodepart{nine}{$\vdots$}
    		\nodepart{ten}
    		\nodepart{eleven}
    		\nodepart{twelve}
    		\nodepart{thirteen}{$\vdots$}
    		\nodepart{fourteen}
    		\nodepart{fifteen}
    		\nodepart{sixteen}
    		\nodepart{seventeen}{$\vdots$}
    		\nodepart{eighteen}
    		\nodepart{nineteen}};
    		
  \end{scope}
  \node at (-2,6.1) {\color{white}0xFFE07600};
  \node at (-1.4,8.1) {\color{red}i};
  \node at (-1.4,7.75) {\color{red}j};
  \node at (-1.4,7.4) {\color{red}file};
  \node at (-1.4,7.1) {\color{red}row};
  \node at (-1.4,6.8) {\color{red}mtz};

  \node at (0,8.6) {\footnotesize\itshape\color{blue} {$ejemplo$ $para$ $matriz$ $(2$ x $3)$}};
  \fill[color=green] (2.8,7.3) -- (6.1,7.3) -- (6.1,6.7) -- (2.8,6.7) -- cycle;

  \draw[white,-](-1.9,0.6)to [out=90,in=-90]node[right,midway]{} ++(0,0.5) ; % para fijar rectangulos
\end{tikzpicture}
						
  \column{0.5\textwidth}
  \lstset{basicstyle=\tiny}
 \begin{lstlisting}
#include<stdio.h>
#include <stdlib.h>

int main (void)
{
  int file=0,row=0,i=0,j=0;
  int **mtz;

  printf("ingrese filas y columnas \n");
  scanf("% d % d",&file,&row);
  mtz=(int **)malloc(file*sizeof(int *));
  for (i=0 ; i<file ; i++){
   mtz[i]=(int *)malloc(row*sizeof(int));
  }
  for (i=0 ; i<file ; i++){
    for (j=0 ; j<row ; j++){
       printf("file% d  rows% d\n",i,j);
       scanf("% d",&mtz[i][j]);
    }
  }
  printf("\n");
  for (i=0 ; i<file ; i++){
    for (j=0 ; j<row ; j++){
       printf("% d \t",mtz[i][j]);
    }
  }
  for (i=0 ; i<file ; i++){
    free(mtz[i]);
  }
  free(mtz);
  return (0);
}
   \end{lstlisting}
 \end{columns}
\end{frame}


\begin{frame}[fragile]
\fontsize{6.5pt}{12pt}\selectfont
 \frametitle{Matriz {\color{yellow}NxN} con {\color{yellow}malloc} en Arquitectura \textcolor{yellow}{X86-32} bits }
 \begin{columns}[c]
  \column{0.4\textwidth}
\lstset{basicstyle=\tiny}
\begin{tikzpicture}
  \begin{scope}[
  	every node/.style={draw, 
  	anchor=text, 
  	rectangle split,
    rectangle split parts=19, 
    rectangle split part fill={green!0,green!0,green!60,green!60,green!0,red!0,red!0,red!0,red!0,red!0,red!0,red!0,blue!0,red!0,red!0,red!0,red!0,red!0,red!0,red!0},
    minimum width=0.6cm}]
    \node (R) at (-0.5,8){ 
    		\nodepart{one}\tiny 0
    		\nodepart{two}\tiny 0
    		\nodepart{three}\tiny {\color{red}2}
    		\nodepart{four}\tiny {\color{red}3}
    		\nodepart{five}\tiny  0xXXXXXXXX
    		\nodepart{six}{$\vdots$}
    		\nodepart{seven}
    		\nodepart{eight}
    		\nodepart{nine}{$\vdots$}
    		\nodepart{ten}
    		\nodepart{eleven}
    		\nodepart{twelve}
    		\nodepart{thirteen}{$\vdots$}
    		\nodepart{fourteen}
    		\nodepart{fifteen}
    		\nodepart{sixteen}
    		\nodepart{seventeen}{$\vdots$}
    		\nodepart{eighteen}
    		\nodepart{nineteen}};
    		
  \end{scope}
  \node at (-2,6.1) {\color{white}0xFFE07600};
  \node at (-1.4,8.1) {\color{red}i};
  \node at (-1.4,7.75) {\color{red}j};
  \node at (-1.4,7.4) {\color{red}file};
  \node at (-1.4,7.1) {\color{red}row};
  \node at (-1.4,6.8) {\color{red}mtz};

  \node at (0,8.6) {\footnotesize\itshape\color{blue} {$ejemplo$ $para$ $matriz$ $(2$ x $3)$}};
  \fill[color=green] (2.8,6.3) -- (6.3,6.3) -- (6.3,6) -- (2.8,6) -- cycle;

  \draw[white,-](-1.9,0.6)to [out=90,in=-90]node[right,midway]{} ++(0,0.5) ; % para fijar rectangulos
\end{tikzpicture}
						
  \column{0.5\textwidth}
  \lstset{basicstyle=\tiny}
 \begin{lstlisting}
#include<stdio.h>
#include <stdlib.h>

int main (void)
{
  int file=0,row=0,i=0,j=0;
  int **mtz;

  printf("ingrese filas y columnas \n");
  scanf("% d % d",&file,&row);
  mtz=(int **)malloc(file*sizeof(int *));
  for (i=0 ; i<file ; i++){
   mtz[i]=(int *)malloc(row*sizeof(int));
  }
  for (i=0 ; i<file ; i++){
    for (j=0 ; j<row ; j++){
       printf("file% d  rows% d\n",i,j);
       scanf("% d",&mtz[i][j]);
    }
  }
  printf("\n");
  for (i=0 ; i<file ; i++){
    for (j=0 ; j<row ; j++){
       printf("% d \t",mtz[i][j]);
    }
  }
  for (i=0 ; i<file ; i++){
    free(mtz[i]);
  }
  free(mtz);
  return (0);
}
   \end{lstlisting}
 \end{columns}
\end{frame}


\begin{frame}[fragile]
\fontsize{6.5pt}{12pt}\selectfont
 \frametitle{Matriz {\color{yellow}NxN} con {\color{yellow}malloc} en Arquitectura \textcolor{yellow}{X86-32} bits }
 \begin{columns}[c]
  \column{0.4\textwidth}
\lstset{basicstyle=\tiny}
\begin{tikzpicture}
  \begin{scope}[
  	every node/.style={draw, 
  	anchor=text, 
  	rectangle split,
    rectangle split parts=19, 
    rectangle split part fill={green!0,green!0,green!0,green!0,green!60,red!0,blue!30,blue!30,red!0,red!0,red!0,red!0,blue!0,red!0,red!0,red!0,red!0,red!0,red!0,red!0},
    minimum width=0.6cm}]
    \node (R) at (-0.5,8){ 
    		\nodepart{one}\tiny 0
    		\nodepart{two}\tiny 0
    		\nodepart{three}\tiny 2
    		\nodepart{four}\tiny 3
    		\nodepart{five}\tiny {\color{red}0x5E1D4A80}
    		\nodepart{six}{$\vdots$}
    		\nodepart{seven}
    		\nodepart{eight}
    		\nodepart{nine}{$\vdots$}
    		\nodepart{ten}
    		\nodepart{eleven}
    		\nodepart{twelve}
    		\nodepart{thirteen}{$\vdots$}
    		\nodepart{fourteen}
    		\nodepart{fifteen}
    		\nodepart{sixteen}
    		\nodepart{seventeen}{$\vdots$}
    		\nodepart{eighteen}
    		\nodepart{nineteen}};
    		
  \end{scope}
  \node at (-2,5.8) {\color{red}0x5E1D4A80};
  \node at (-2,5.45) {\color{red}0x5E1D4A84};
  \node at (-1.4,8.1) {\color{red}i};
  \node at (-1.4,7.75) {\color{red}j};
  \node at (-1.4,7.4) {\color{red}file};
  \node at (-1.4,7.1) {\color{red}row};
  \node at (-1.4,6.8) {\color{red}mtz};

  \node at (0,8.6) {\footnotesize\itshape\color{blue} {$ejemplo$ $para$ $matriz$ $(2$ x $3)$}};
  \fill[color=green] (2.8,6) -- (7.6,6) -- (7.6,5.75) -- (2.8,5.75) -- cycle;

  \draw[white,-](-1.9,0.6)to [out=90,in=-90]node[right,midway]{} ++(0,0.5) ; % para fijar rectangulos
\end{tikzpicture}
						
  \column{0.5\textwidth}
  \lstset{basicstyle=\tiny}
 \begin{lstlisting}
#include<stdio.h>
#include <stdlib.h>

int main (void)
{
  int file=0,row=0,i=0,j=0;
  int **mtz;

  printf("ingrese filas y columnas \n");
  scanf("% d % d",&file,&row);
  mtz=(int **)malloc(file*sizeof(int *));
  for (i=0 ; i<file ; i++){
   mtz[i]=(int *)malloc(row*sizeof(int));
  }
  for (i=0 ; i<file ; i++){
    for (j=0 ; j<row ; j++){
       printf("file% d  rows% d\n",i,j);
       scanf("% d",&mtz[i][j]);
    }
  }
  printf("\n");
  for (i=0 ; i<file ; i++){
    for (j=0 ; j<row ; j++){
       printf("% d \t",mtz[i][j]);
    }
  }
  for (i=0 ; i<file ; i++){
    free(mtz[i]);
  }
  free(mtz);
  return (0);
}
   \end{lstlisting}
 \end{columns}
\end{frame}


\begin{frame}[fragile]
\fontsize{6.5pt}{12pt}\selectfont
 \frametitle{Matriz {\color{yellow}NxN} con {\color{yellow}malloc} en Arquitectura \textcolor{yellow}{X86-32} bits }
 \begin{columns}[c]
  \column{0.4\textwidth}
\lstset{basicstyle=\tiny}
\begin{tikzpicture}
  \begin{scope}[
  	every node/.style={draw, 
  	anchor=text, 
  	rectangle split,
    rectangle split parts=19, 
    rectangle split part fill={green!30,green!0,green!0,green!0,green!0,red!0,blue!30,blue!30,red!0,red!30,red!30,red!30,blue!0,red!0,red!0,red!0,red!0,red!0,red!0,red!0},
    minimum width=0.6cm}]
    \node (R) at (-0.5,8){ 
    		\nodepart{one}\tiny 0
    		\nodepart{two}\tiny 0
    		\nodepart{three}\tiny 2
    		\nodepart{four}\tiny 3
    		\nodepart{five}\tiny 0x5E1D4A80
    		\nodepart{six}{$\vdots$}
    		\nodepart{seven}\tiny {\color{red}0x5E1D4AA0}
    		\nodepart{eight}
    		\nodepart{nine}{$\vdots$}
    		\nodepart{ten}
    		\nodepart{eleven}
    		\nodepart{twelve}
    		\nodepart{thirteen}{$\vdots$}
    		\nodepart{fourteen}
    		\nodepart{fifteen}
    		\nodepart{sixteen}
    		\nodepart{seventeen}{$\vdots$}
    		\nodepart{eighteen}
    		\nodepart{nineteen}};
    		
  \end{scope}
  \node at (-2,5.8) {\color{red}0x5E1D4A80};
  \node at (0.8,5.8) {\color{red}mtz[0]};
  \node at (-2,5.45) {\color{red}0x5E1D4A84};
  \node at (-2,4.45) {\color{red}0x5E1D4AA0};
  \node at (-1.4,8.1) {\color{red}i};
  \node at (-1.4,7.75) {\color{red}j};
  \node at (-1.4,7.4) {\color{red}file};
  \node at (-1.4,7.1) {\color{red}row};
  \node at (-1.4,6.8) {\color{red}mtz};

  \node at (0,8.6) {\footnotesize\itshape\color{blue} {$ejemplo$ $para$ $matriz$ $(2$ x $3)$}};
  \fill[color=green] (2.8,5.55) -- (7.6,5.55) -- (7.6,5.25) -- (2.8,5.25) -- cycle;

  \draw[white,-](-1.9,0.6)to [out=90,in=-90]node[right,midway]{} ++(0,0.5) ; % para fijar rectangulos
\end{tikzpicture}
						
  \column{0.5\textwidth}
  \lstset{basicstyle=\tiny}
 \begin{lstlisting}
#include<stdio.h>
#include <stdlib.h>

int main (void)
{
  int file=0,row=0,i=0,j=0;
  int **mtz;

  printf("ingrese filas y columnas \n");
  scanf("% d % d",&file,&row);
  mtz=(int **)malloc(file*sizeof(int *));
  for (i=0 ; i<file ; i++){
   mtz[i]=(int *)malloc(row*sizeof(int));
  }
  for (i=0 ; i<file ; i++){
    for (j=0 ; j<row ; j++){
       printf("file% d  rows% d\n",i,j);
       scanf("% d",&mtz[i][j]);
    }
  }
  printf("\n");
  for (i=0 ; i<file ; i++){
    for (j=0 ; j<row ; j++){
       printf("% d \t",mtz[i][j]);
    }
  }
  for (i=0 ; i<file ; i++){
    free(mtz[i]);
  }
  free(mtz);
  return (0);
}
   \end{lstlisting}
 \end{columns}
\end{frame}


\begin{frame}[fragile]
\fontsize{6.5pt}{12pt}\selectfont
 \frametitle{Matriz {\color{yellow}NxN} con {\color{yellow}malloc} en Arquitectura \textcolor{yellow}{X86-32} bits }
 \begin{columns}[c]
  \column{0.4\textwidth}
\lstset{basicstyle=\tiny}
\begin{tikzpicture}
  \begin{scope}[
  	every node/.style={draw, 
  	anchor=text, 
  	rectangle split,
    rectangle split parts=19, 
    rectangle split part fill={green!30,green!0,green!0,green!0,green!0,red!0,blue!30,blue!30,red!0,red!30,red!30,red!30,blue!0,red!30,red!30,red!30,red!0,red!0,red!0,red!0},
    minimum width=0.6cm}]
    \node (R) at (-0.5,8){ 
    		\nodepart{one}\tiny 1
    		\nodepart{two}\tiny 0
    		\nodepart{three}\tiny 2
    		\nodepart{four}\tiny 3
    		\nodepart{five}\tiny 0x5E1D4A80
    		\nodepart{six}{$\vdots$}
    		\nodepart{seven}\tiny {\color{red}0x5E1D4AA0}
    		\nodepart{eight}\tiny {\color{red}0x5E1D4AC0}
    		\nodepart{nine}{$\vdots$}
    		\nodepart{ten}
    		\nodepart{eleven}
    		\nodepart{twelve}
    		\nodepart{thirteen}{$\vdots$}
    		\nodepart{fourteen}
    		\nodepart{fifteen}
    		\nodepart{sixteen}
    		\nodepart{seventeen}{$\vdots$}
    		\nodepart{eighteen}
    		\nodepart{nineteen}};
    		
  \end{scope}
  \node at (-2,5.8) {\color{red}0x5E1D4A80};
  \node at (0.8,5.8) {\color{red}mtz[0]};
  \node at (0.8,5.45) {\color{red}mtz[1]};
  \node at (-2,5.45) {\color{red}0x5E1D4A84};
  \node at (-2,4.45) {\color{red}0x5E1D4AA0};
  \node at (-2,2.8) {\color{red}0x5E1D4AC0};
    \node at (-1.4,8.1) {\color{red}i};
  \node at (-1.4,7.75) {\color{red}j};
  \node at (-1.4,7.4) {\color{red}file};
  \node at (-1.4,7.1) {\color{red}row};
  \node at (-1.4,6.8) {\color{red}mtz};

  \node at (0,8.6) {\footnotesize\itshape\color{blue} {$ejemplo$ $para$ $matriz$ $(2$ x $3)$}};
  \fill[color=green] (2.8,5.55) -- (7.6,5.55) -- (7.6,5.25) -- (2.8,5.25) -- cycle;

  \draw[white,-](-1.9,0.6)to [out=90,in=-90]node[right,midway]{} ++(0,0.5) ; % para fijar rectangulos
\end{tikzpicture}
						
  \column{0.5\textwidth}
  \lstset{basicstyle=\tiny}
 \begin{lstlisting}
#include<stdio.h>
#include <stdlib.h>

int main (void)
{
  int file=0,row=0,i=0,j=0;
  int **mtz;

  printf("ingrese filas y columnas \n");
  scanf("% d % d",&file,&row);
  mtz=(int **)malloc(file*sizeof(int *));
  for (i=0 ; i<file ; i++){
   mtz[i]=(int *)malloc(row*sizeof(int));
  }
  for (i=0 ; i<file ; i++){
    for (j=0 ; j<row ; j++){
       printf("file% d  rows% d\n",i,j);
       scanf("% d",&mtz[i][j]);
    }
  }
  printf("\n");
  for (i=0 ; i<file ; i++){
    for (j=0 ; j<row ; j++){
       printf("% d \t",mtz[i][j]);
    }
  }
  for (i=0 ; i<file ; i++){
    free(mtz[i]);
  }
  free(mtz);
  return (0);
}
   \end{lstlisting}
 \end{columns}
\end{frame}


\begin{frame}[fragile]
\fontsize{6.5pt}{12pt}\selectfont
 \frametitle{Matriz {\color{yellow}NxN} con {\color{yellow}malloc} en Arquitectura \textcolor{yellow}{X86-32} bits }
 \begin{columns}[c]
  \column{0.4\textwidth}
\lstset{basicstyle=\tiny}
\begin{tikzpicture}
  \begin{scope}[
  	every node/.style={draw, 
  	anchor=text, 
  	rectangle split,
    rectangle split parts=19, 
    rectangle split part fill={green!30,green!0,green!0,green!0,green!0,red!0,blue!30,blue!30,red!0,red!30,red!30,red!30,blue!0,red!30,red!30,red!30,red!0,red!0,red!0,red!0},
    minimum width=0.6cm}]
    \node (R) at (-0.5,8){ 
    		\nodepart{one}\tiny 0
    		\nodepart{two}\tiny 0
    		\nodepart{three}\tiny 2
    		\nodepart{four}\tiny 3
    		\nodepart{five}\tiny 0x5E1D4A80
    		\nodepart{six}{$\vdots$}
    		\nodepart{seven}\tiny 0x5E1D4AA0
    		\nodepart{eight}\tiny 0x5E1D4AC0
    		\nodepart{nine}{$\vdots$}
    		\nodepart{ten}
    		\nodepart{eleven}
    		\nodepart{twelve}
    		\nodepart{thirteen}{$\vdots$}
    		\nodepart{fourteen}
    		\nodepart{fifteen}
    		\nodepart{sixteen}
    		\nodepart{seventeen}{$\vdots$}
    		\nodepart{eighteen}
    		\nodepart{nineteen}};
    		
  \end{scope}
  \node at (-2,5.8) {\color{red}0x5E1D4A80};
  \node at (0.8,5.8) {\color{red}mtz[0]};
  \node at (0.8,5.45) {\color{red}mtz[1]};
  \node at (-2,5.45) {\color{red}0x5E1D4A84};
  \node at (-2,4.45) {\color{red}0x5E1D4AA0};
  \node at (-2,2.8) {\color{red}0x5E1D4AC0};
    \node at (-1.4,8.1) {\color{red}i};
  \node at (-1.4,7.75) {\color{red}j};
  \node at (-1.4,7.4) {\color{red}file};
  \node at (-1.4,7.1) {\color{red}row};
  \node at (-1.4,6.8) {\color{red}mtz};

  \node at (0,8.6) {\footnotesize\itshape\color{blue} {$ejemplo$ $para$ $matriz$ $(2$ x $3)$}};
  \fill[color=green] (2.8,5.1) -- (4,5.1) -- (4,4.8) -- (2.8,4.8) -- cycle;

  \draw[white,-](-1.9,0.6)to [out=90,in=-90]node[right,midway]{} ++(0,0.5) ; % para fijar rectangulos
\end{tikzpicture}
						
  \column{0.5\textwidth}
  \lstset{basicstyle=\tiny}
 \begin{lstlisting}
#include<stdio.h>
#include <stdlib.h>

int main (void)
{
  int file=0,row=0,i=0,j=0;
  int **mtz;

  printf("ingrese filas y columnas \n");
  scanf("% d % d",&file,&row);
  mtz=(int **)malloc(file*sizeof(int *));
  for (i=0 ; i<file ; i++){
   mtz[i]=(int *)malloc(row*sizeof(int));
  }
  for (i=0 ; i<file ; i++){
    for (j=0 ; j<row ; j++){
       printf("file% d  rows% d\n",i,j);
       scanf("% d",&mtz[i][j]);
    }
  }
  printf("\n");
  for (i=0 ; i<file ; i++){
    for (j=0 ; j<row ; j++){
       printf("% d \t",mtz[i][j]);
    }
  }
  for (i=0 ; i<file ; i++){
    free(mtz[i]);
  }
  free(mtz);
  return (0);
}
   \end{lstlisting}
 \end{columns}
\end{frame}


\begin{frame}[fragile]
\fontsize{6.5pt}{12pt}\selectfont
 \frametitle{Matriz {\color{yellow}NxN} con {\color{yellow}malloc} en Arquitectura \textcolor{yellow}{X86-32} bits }
 \begin{columns}[c]
  \column{0.4\textwidth}
\lstset{basicstyle=\tiny}
\begin{tikzpicture}
  \begin{scope}[
  	every node/.style={draw, 
  	anchor=text, 
  	rectangle split,
    rectangle split parts=19, 
    rectangle split part fill={green!30,green!0,green!30,green!0,green!0,red!0,blue!30,blue!30,red!0,red!30,red!30,red!30,blue!0,red!30,red!30,red!30,red!0,red!0,red!0,red!0},
    minimum width=0.6cm}]
    \node (R) at (-0.5,8){ 
    		\nodepart{one}\tiny 0
    		\nodepart{two}\tiny 0
    		\nodepart{three}\tiny 2
    		\nodepart{four}\tiny 3
    		\nodepart{five}\tiny 0x5E1D4A80
    		\nodepart{six}{$\vdots$}
    		\nodepart{seven}\tiny 0x5E1D4AA0
    		\nodepart{eight}\tiny 0x5E1D4AC0
    		\nodepart{nine}{$\vdots$}
    		\nodepart{ten}
    		\nodepart{eleven}
    		\nodepart{twelve}
    		\nodepart{thirteen}{$\vdots$}
    		\nodepart{fourteen}
    		\nodepart{fifteen}
    		\nodepart{sixteen}
    		\nodepart{seventeen}{$\vdots$}
    		\nodepart{eighteen}
    		\nodepart{nineteen}};
    		
  \end{scope}
  \node at (-2,5.8) {\color{red}0x5E1D4A80};
  \node at (0.8,5.8) {\color{red}mtz[0]};
  \node at (0.8,5.45) {\color{red}mtz[1]};
  \node at (-2,5.45) {\color{red}0x5E1D4A84};
  \node at (-2,4.45) {\color{red}0x5E1D4AA0};
  \node at (-2,2.8) {\color{red}0x5E1D4AC0};
    \node at (-1.4,8.1) {\color{red}i};
  \node at (-1.4,7.75) {\color{red}j};
  \node at (-1.4,7.4) {\color{red}file};
  \node at (-1.4,7.1) {\color{red}row};
  \node at (-1.4,6.8) {\color{red}mtz};

  \node at (0,8.6) {\footnotesize\itshape\color{blue} {$ejemplo$ $para$ $matriz$ $(2$ x $3)$}};
  \fill[color=green] (4.2,5.1) -- (5,5.1) -- (5,4.8) -- (4.2,4.8) -- cycle;

  \draw[white,-](-1.9,0.6)to [out=90,in=-90]node[right,midway]{} ++(0,0.5) ; % para fijar rectangulos
\end{tikzpicture}
						
  \column{0.5\textwidth}
  \lstset{basicstyle=\tiny}
 \begin{lstlisting}
#include<stdio.h>
#include <stdlib.h>

int main (void)
{
  int file=0,row=0,i=0,j=0;
  int **mtz;

  printf("ingrese filas y columnas \n");
  scanf("% d % d",&file,&row);
  mtz=(int **)malloc(file*sizeof(int *));
  for (i=0 ; i<file ; i++){
   mtz[i]=(int *)malloc(row*sizeof(int));
  }
  for (i=0 ; i<file ; i++){
    for (j=0 ; j<row ; j++){
       printf("file% d  rows% d\n",i,j);
       scanf("% d",&mtz[i][j]);
    }
  }
  printf("\n");
  for (i=0 ; i<file ; i++){
    for (j=0 ; j<row ; j++){
       printf("% d \t",mtz[i][j]);
    }
  }
  for (i=0 ; i<file ; i++){
    free(mtz[i]);
  }
  free(mtz);
  return (0);
}
   \end{lstlisting}
 \end{columns}
\end{frame}


\begin{frame}[fragile]
\fontsize{6.5pt}{12pt}\selectfont
 \frametitle{Matriz {\color{yellow}NxN} con {\color{yellow}malloc} en Arquitectura \textcolor{yellow}{X86-32} bits }
 \begin{columns}[c]
  \column{0.4\textwidth}
\lstset{basicstyle=\tiny}
\begin{tikzpicture}
  \begin{scope}[
  	every node/.style={draw, 
  	anchor=text, 
  	rectangle split,
    rectangle split parts=19, 
    rectangle split part fill={green!0,green!30,green!0,green!0,green!0,red!0,blue!30,blue!30,red!0,red!30,red!30,red!30,blue!0,red!30,red!30,red!30,red!0,red!0,red!0,red!0},
    minimum width=0.6cm}]
    \node (R) at (-0.5,8){ 
    		\nodepart{one}\tiny 0
    		\nodepart{two}\tiny 0
    		\nodepart{three}\tiny 2
    		\nodepart{four}\tiny 3
    		\nodepart{five}\tiny 0x5E1D4A80
    		\nodepart{six}{$\vdots$}
    		\nodepart{seven}\tiny 0x5E1D4AA0
    		\nodepart{eight}\tiny 0x5E1D4AC0
    		\nodepart{nine}{$\vdots$}
    		\nodepart{ten}
    		\nodepart{eleven}
    		\nodepart{twelve}
    		\nodepart{thirteen}{$\vdots$}
    		\nodepart{fourteen}
    		\nodepart{fifteen}
    		\nodepart{sixteen}
    		\nodepart{seventeen}{$\vdots$}
    		\nodepart{eighteen}
    		\nodepart{nineteen}};
    		
  \end{scope}
  \node at (-2,5.8) {\color{red}0x5E1D4A80};
  \node at (0.8,5.8) {\color{red}mtz[0]};
  \node at (0.8,5.45) {\color{red}mtz[1]};
  \node at (-2,5.45) {\color{red}0x5E1D4A84};
  \node at (-2,4.45) {\color{red}0x5E1D4AA0};
  \node at (-2,2.8) {\color{red}0x5E1D4AC0};
    \node at (-1.4,8.1) {\color{red}i};
  \node at (-1.4,7.75) {\color{red}j};
  \node at (-1.4,7.4) {\color{red}file};
  \node at (-1.4,7.1) {\color{red}row};
  \node at (-1.4,6.8) {\color{red}mtz};

  \node at (0,8.6) {\footnotesize\itshape\color{blue} {$ejemplo$ $para$ $matriz$ $(2$ x $3)$}};
  \fill[color=green] (3.6,4.8) -- (4.4,4.8) -- (4.4,4.55) -- (3.6,4.55) -- cycle;

  \draw[white,-](-1.9,0.6)to [out=90,in=-90]node[right,midway]{} ++(0,0.5) ; % para fijar rectangulos
\end{tikzpicture}
						
  \column{0.5\textwidth}
  \lstset{basicstyle=\tiny}
 \begin{lstlisting}
#include<stdio.h>
#include <stdlib.h>

int main (void)
{
  int file=0,row=0,i=0,j=0;
  int **mtz;

  printf("ingrese filas y columnas \n");
  scanf("% d % d",&file,&row);
  mtz=(int **)malloc(file*sizeof(int *));
  for (i=0 ; i<file ; i++){
   mtz[i]=(int *)malloc(row*sizeof(int));
  }
  for (i=0 ; i<file ; i++){
    for (j=0 ; j<row ; j++){
       printf("file% d  rows% d\n",i,j);
       scanf("% d",&mtz[i][j]);
    }
  }
  printf("\n");
  for (i=0 ; i<file ; i++){
    for (j=0 ; j<row ; j++){
       printf("% d \t",mtz[i][j]);
    }
  }
  for (i=0 ; i<file ; i++){
    free(mtz[i]);
  }
  free(mtz);
  return (0);
}
   \end{lstlisting}
 \end{columns}
\end{frame}


\begin{frame}[fragile]
\fontsize{6.5pt}{12pt}\selectfont
 \frametitle{Matriz {\color{yellow}NxN} con {\color{yellow}malloc} en Arquitectura \textcolor{yellow}{X86-32} bits }
 \begin{columns}[c]
  \column{0.4\textwidth}
\lstset{basicstyle=\tiny}
\begin{tikzpicture}
  \begin{scope}[
  	every node/.style={draw, 
  	anchor=text, 
  	rectangle split,
    rectangle split parts=19, 
    rectangle split part fill={green!0,green!30,green!0,green!30,green!0,red!0,blue!30,blue!30,red!0,red!30,red!30,red!30,blue!0,red!30,red!30,red!30,red!0,red!0,red!0,red!0},
    minimum width=0.6cm}]
    \node (R) at (-0.5,8){ 
    		\nodepart{one}\tiny 0
    		\nodepart{two}\tiny 0
    		\nodepart{three}\tiny 2
    		\nodepart{four}\tiny 3
    		\nodepart{five}\tiny 0x5E1D4A80
    		\nodepart{six}{$\vdots$}
    		\nodepart{seven}\tiny 0x5E1D4AA0
    		\nodepart{eight}\tiny 0x5E1D4AC0
    		\nodepart{nine}{$\vdots$}
    		\nodepart{ten}
    		\nodepart{eleven}
    		\nodepart{twelve}
    		\nodepart{thirteen}{$\vdots$}
    		\nodepart{fourteen}
    		\nodepart{fifteen}
    		\nodepart{sixteen}
    		\nodepart{seventeen}{$\vdots$}
    		\nodepart{eighteen}
    		\nodepart{nineteen}};
    		
  \end{scope}
  \node at (-2,5.8) {\color{red}0x5E1D4A80};
  \node at (0.8,5.8) {\color{red}mtz[0]};
  \node at (0.8,5.45) {\color{red}mtz[1]};
  \node at (-2,5.45) {\color{red}0x5E1D4A84};
  \node at (-2,4.45) {\color{red}0x5E1D4AA0};
  \node at (-2,2.8) {\color{red}0x5E1D4AC0};
    \node at (-1.4,8.1) {\color{red}i};
  \node at (-1.4,7.75) {\color{red}j};
  \node at (-1.4,7.4) {\color{red}file};
  \node at (-1.4,7.1) {\color{red}row};
  \node at (-1.4,6.8) {\color{red}mtz};

  \node at (0,8.6) {\footnotesize\itshape\color{blue} {$ejemplo$ $para$ $matriz$ $(2$ x $3)$}};
  \fill[color=green] (4.4,4.8) -- (5.4,4.8) -- (5.4,4.55) -- (4.4,4.55) -- cycle;

  \draw[white,-](-1.9,0.6)to [out=90,in=-90]node[right,midway]{} ++(0,0.5) ; % para fijar rectangulos
\end{tikzpicture}
						
  \column{0.5\textwidth}
  \lstset{basicstyle=\tiny}
 \begin{lstlisting}
#include<stdio.h>
#include <stdlib.h>

int main (void)
{
  int file=0,row=0,i=0,j=0;
  int **mtz;

  printf("ingrese filas y columnas \n");
  scanf("% d % d",&file,&row);
  mtz=(int **)malloc(file*sizeof(int *));
  for (i=0 ; i<file ; i++){
   mtz[i]=(int *)malloc(row*sizeof(int));
  }
  for (i=0 ; i<file ; i++){
    for (j=0 ; j<row ; j++){
       printf("file% d  rows% d\n",i,j);
       scanf("% d",&mtz[i][j]);
    }
  }
  printf("\n");
  for (i=0 ; i<file ; i++){
    for (j=0 ; j<row ; j++){
       printf("% d \t",mtz[i][j]);
    }
  }
  for (i=0 ; i<file ; i++){
    free(mtz[i]);
  }
  free(mtz);
  return (0);
}
   \end{lstlisting}
 \end{columns}
\end{frame}


\begin{frame}[fragile]
\fontsize{6.5pt}{12pt}\selectfont
 \frametitle{Matriz {\color{yellow}NxN} con {\color{yellow}malloc} en Arquitectura \textcolor{yellow}{X86-32} bits }
 \begin{columns}[c]
  \column{0.4\textwidth}
\lstset{basicstyle=\tiny}
\begin{tikzpicture}
  \begin{scope}[
  	every node/.style={draw, 
  	anchor=text, 
  	rectangle split,
    rectangle split parts=19, 
    rectangle split part fill={green!30,green!30,green!0,green!0,green!0,red!0,blue!30,blue!30,red!0,red!30,red!30,red!30,blue!0,red!30,red!30,red!30,red!0,red!0,red!0,red!0},
    minimum width=0.6cm}]
    \node (R) at (-0.5,8){ 
    		\nodepart{one}\tiny 0
    		\nodepart{two}\tiny 0
    		\nodepart{three}\tiny 2
    		\nodepart{four}\tiny 3
    		\nodepart{five}\tiny 0x5E1D4A80
    		\nodepart{six}{$\vdots$}
    		\nodepart{seven}\tiny 0x5E1D4AA0
    		\nodepart{eight}\tiny 0x5E1D4AC0
    		\nodepart{nine}{$\vdots$}
    		\nodepart{ten}\tiny {\color{red}11}
    		\nodepart{eleven}
    		\nodepart{twelve}
    		\nodepart{thirteen}{$\vdots$}
    		\nodepart{fourteen}
    		\nodepart{fifteen}
    		\nodepart{sixteen}
    		\nodepart{seventeen}{$\vdots$}
    		\nodepart{eighteen}
    		\nodepart{nineteen}};
    		
  \end{scope}
  \node at (-2,5.8) {\color{red}0x5E1D4A80};
  \node at (0.8,5.8) {\color{red}mtz[0]};
  \node at (0.8,5.45) {\color{red}mtz[1]};
  \node at (-2,5.45) {\color{red}0x5E1D4A84};
  \node at (-2,4.45) {\color{red}0x5E1D4AA0};
  \node at (0.8,4.45) {\color{red}mtz[0][0]};
  \node at (-2,2.8) {\color{red}0x5E1D4AC0};
    \node at (-1.4,8.1) {\color{red}i};
  \node at (-1.4,7.75) {\color{red}j};
  \node at (-1.4,7.4) {\color{red}file};
  \node at (-1.4,7.1) {\color{red}row};
  \node at (-1.4,6.8) {\color{red}mtz};

  \node at (0,8.6) {\footnotesize\itshape\color{blue} {$ejemplo$ $para$ $matriz$ $(2$ x $3)$}};
  \fill[color=green] (3.2,4.3) -- (6.4,4.3) -- (6.4,4.0) -- (3.2,4.0) -- cycle;

  \draw[white,-](-1.9,0.6)to [out=90,in=-90]node[right,midway]{} ++(0,0.5) ; % para fijar rectangulos
\end{tikzpicture}
						
  \column{0.5\textwidth}
  \lstset{basicstyle=\tiny}
 \begin{lstlisting}
#include<stdio.h>
#include <stdlib.h>

int main (void)
{
  int file=0,row=0,i=0,j=0;
  int **mtz;

  printf("ingrese filas y columnas \n");
  scanf("% d % d",&file,&row);
  mtz=(int **)malloc(file*sizeof(int *));
  for (i=0 ; i<file ; i++){
   mtz[i]=(int *)malloc(row*sizeof(int));
  }
  for (i=0 ; i<file ; i++){
    for (j=0 ; j<row ; j++){
       printf("file% d  rows% d\n",i,j);
       scanf("% d",&mtz[i][j]);
    }
  }
  printf("\n");
  for (i=0 ; i<file ; i++){
    for (j=0 ; j<row ; j++){
       printf("% d \t",mtz[i][j]);
    }
  }
  for (i=0 ; i<file ; i++){
    free(mtz[i]);
  }
  free(mtz);
  return (0);
}
   \end{lstlisting}
 \end{columns}
\end{frame}


\begin{frame}[fragile]
\fontsize{6.5pt}{12pt}\selectfont
 \frametitle{Matriz {\color{yellow}NxN} con {\color{yellow}malloc} en Arquitectura \textcolor{yellow}{X86-32} bits }
 \begin{columns}[c]
  \column{0.4\textwidth}
\lstset{basicstyle=\tiny}
\begin{tikzpicture}
  \begin{scope}[
  	every node/.style={draw, 
  	anchor=text, 
  	rectangle split,
    rectangle split parts=19, 
    rectangle split part fill={green!30,green!30,green!0,green!0,green!0,red!0,blue!30,blue!30,red!0,red!30,red!30,red!30,blue!0,red!30,red!30,red!30,red!0,red!0,red!0,red!0},
    minimum width=0.6cm}]
    \node (R) at (-0.5,8){ 
    		\nodepart{one}\tiny 0
    		\nodepart{two}\tiny {\color{red}1}
    		\nodepart{three}\tiny 2
    		\nodepart{four}\tiny 3
    		\nodepart{five}\tiny 0x5E1D4A80
    		\nodepart{six}{$\vdots$}
    		\nodepart{seven}\tiny 0x5E1D4AA0
    		\nodepart{eight}\tiny 0x5E1D4AC0
    		\nodepart{nine}{$\vdots$}
    		\nodepart{ten}\tiny 11
    		\nodepart{eleven}
    		\nodepart{twelve}
    		\nodepart{thirteen}{$\vdots$}
    		\nodepart{fourteen}
    		\nodepart{fifteen}
    		\nodepart{sixteen}
    		\nodepart{seventeen}{$\vdots$}
    		\nodepart{eighteen}
    		\nodepart{nineteen}};
    		
  \end{scope}
  \node at (-2,5.8) {\color{red}0x5E1D4A80};
  \node at (0.8,5.8) {\color{red}mtz[0]};
  \node at (0.8,5.45) {\color{red}mtz[1]};
  \node at (-2,5.45) {\color{red}0x5E1D4A84};
  \node at (-2,4.45) {\color{red}0x5E1D4AA0};
  \node at (0.8,4.45) {\color{red}mtz[0][0]};
  \node at (-2,2.8) {\color{red}0x5E1D4AC0};
    \node at (-1.4,8.1) {\color{red}i};
  \node at (-1.4,7.75) {\color{red}j};
  \node at (-1.4,7.4) {\color{red}file};
  \node at (-1.4,7.1) {\color{red}row};
  \node at (-1.4,6.8) {\color{red}mtz};

  \node at (0,8.6) {\footnotesize\itshape\color{blue} {$ejemplo$ $para$ $matriz$ $(2$ x $3)$}};
  \fill[color=green] (5.4,4.8) -- (5.9,4.8) -- (5.9,4.55) -- (5.4,4.55) -- cycle;

  \draw[white,-](-1.9,0.6)to [out=90,in=-90]node[right,midway]{} ++(0,0.5) ; % para fijar rectangulos
\end{tikzpicture}
						
  \column{0.5\textwidth}
  \lstset{basicstyle=\tiny}
 \begin{lstlisting}
#include<stdio.h>
#include <stdlib.h>

int main (void)
{
  int file=0,row=0,i=0,j=0;
  int **mtz;

  printf("ingrese filas y columnas \n");
  scanf("% d % d",&file,&row);
  mtz=(int **)malloc(file*sizeof(int *));
  for (i=0 ; i<file ; i++){
   mtz[i]=(int *)malloc(row*sizeof(int));
  }
  for (i=0 ; i<file ; i++){
    for (j=0 ; j<row ; j++){
       printf("file% d  rows% d\n",i,j);
       scanf("% d",&mtz[i][j]);
    }
  }
  printf("\n");
  for (i=0 ; i<file ; i++){
    for (j=0 ; j<row ; j++){
       printf("% d \t",mtz[i][j]);
    }
  }
  for (i=0 ; i<file ; i++){
    free(mtz[i]);
  }
  free(mtz);
  return (0);
}
   \end{lstlisting}
 \end{columns}
\end{frame}


\begin{frame}[fragile]
\fontsize{6.5pt}{12pt}\selectfont
 \frametitle{Matriz {\color{yellow}NxN} con {\color{yellow}malloc} en Arquitectura \textcolor{yellow}{X86-32} bits }
 \begin{columns}[c]
  \column{0.4\textwidth}
\lstset{basicstyle=\tiny}
\begin{tikzpicture}
  \begin{scope}[
  	every node/.style={draw, 
  	anchor=text, 
  	rectangle split,
    rectangle split parts=19, 
    rectangle split part fill={green!30,green!30,green!0,green!30,green!0,red!0,blue!30,blue!30,red!0,red!30,red!30,red!30,blue!0,red!30,red!30,red!30,red!0,red!0,red!0,red!0},
    minimum width=0.6cm}]
    \node (R) at (-0.5,8){ 
    		\nodepart{one}\tiny 0
    		\nodepart{two}\tiny 1
    		\nodepart{three}\tiny 2
    		\nodepart{four}\tiny 3
    		\nodepart{five}\tiny 0x5E1D4A80
    		\nodepart{six}{$\vdots$}
    		\nodepart{seven}\tiny 0x5E1D4AA0
    		\nodepart{eight}\tiny 0x5E1D4AC0
    		\nodepart{nine}{$\vdots$}
    		\nodepart{ten}\tiny 11
    		\nodepart{eleven}
    		\nodepart{twelve}
    		\nodepart{thirteen}{$\vdots$}
    		\nodepart{fourteen}
    		\nodepart{fifteen}
    		\nodepart{sixteen}
    		\nodepart{seventeen}{$\vdots$}
    		\nodepart{eighteen}
    		\nodepart{nineteen}};
    		
  \end{scope}
  \node at (-2,5.8) {\color{red}0x5E1D4A80};
  \node at (0.8,5.8) {\color{red}mtz[0]};
  \node at (0.8,5.45) {\color{red}mtz[1]};
  \node at (-2,5.45) {\color{red}0x5E1D4A84};
  \node at (-2,4.45) {\color{red}0x5E1D4AA0};
  \node at (0.8,4.45) {\color{red}mtz[0][0]};
  \node at (-2,2.8) {\color{red}0x5E1D4AC0};
    \node at (-1.4,8.1) {\color{red}i};
  \node at (-1.4,7.75) {\color{red}j};
  \node at (-1.4,7.4) {\color{red}file};
  \node at (-1.4,7.1) {\color{red}row};
  \node at (-1.4,6.8) {\color{red}mtz};

  \node at (0,8.6) {\footnotesize\itshape\color{blue} {$ejemplo$ $para$ $matriz$ $(2$ x $3)$}};
  \fill[color=green] (4.4,4.8) -- (5.4,4.8) -- (5.4,4.55) -- (4.4,4.55) -- cycle;

  \draw[white,-](-1.9,0.6)to [out=90,in=-90]node[right,midway]{} ++(0,0.5) ; % para fijar rectangulos
\end{tikzpicture}
						
  \column{0.5\textwidth}
  \lstset{basicstyle=\tiny}
 \begin{lstlisting}
#include<stdio.h>
#include <stdlib.h>

int main (void)
{
  int file=0,row=0,i=0,j=0;
  int **mtz;

  printf("ingrese filas y columnas \n");
  scanf("% d % d",&file,&row);
  mtz=(int **)malloc(file*sizeof(int *));
  for (i=0 ; i<file ; i++){
   mtz[i]=(int *)malloc(row*sizeof(int));
  }
  for (i=0 ; i<file ; i++){
    for (j=0 ; j<row ; j++){
       printf("file% d  rows% d\n",i,j);
       scanf("% d",&mtz[i][j]);
    }
  }
  printf("\n");
  for (i=0 ; i<file ; i++){
    for (j=0 ; j<row ; j++){
       printf("% d \t",mtz[i][j]);
    }
  }
  for (i=0 ; i<file ; i++){
    free(mtz[i]);
  }
  free(mtz);
  return (0);
}
   \end{lstlisting}
 \end{columns}
\end{frame}




















\begin{frame}[fragile]
\fontsize{6.5pt}{12pt}\selectfont
 \frametitle{Matriz {\color{yellow}NxN} con {\color{yellow}malloc} en Arquitectura \textcolor{yellow}{X86-32} bits }
 \begin{columns}[c]
  \column{0.4\textwidth}
\lstset{basicstyle=\tiny}
\begin{tikzpicture}
  \begin{scope}[
  	every node/.style={draw, 
  	anchor=text, 
  	rectangle split,
    rectangle split parts=19, 
    rectangle split part fill={green!30,green!30,green!0,green!0,green!0,red!0,blue!30,blue!30,red!0,red!30,red!30,red!30,blue!0,red!30,red!30,red!30,red!0,red!0,red!0,red!0},
    minimum width=0.6cm}]
    \node (R) at (-0.5,8){ 
    		\nodepart{one}\tiny 0
    		\nodepart{two}\tiny 1
    		\nodepart{three}\tiny 2
    		\nodepart{four}\tiny 3
    		\nodepart{five}\tiny 0x5E1D4A80
    		\nodepart{six}{$\vdots$}
    		\nodepart{seven}\tiny 0x5E1D4AA0
    		\nodepart{eight}\tiny 0x5E1D4AC0
    		\nodepart{nine}{$\vdots$}
    		\nodepart{ten}\tiny 11
    		\nodepart{eleven}\tiny {\color{red}12}
    		\nodepart{twelve}
    		\nodepart{thirteen}{$\vdots$}
    		\nodepart{fourteen}
    		\nodepart{fifteen}
    		\nodepart{sixteen}
    		\nodepart{seventeen}{$\vdots$}
    		\nodepart{eighteen}
    		\nodepart{nineteen}};
    		
  \end{scope}
  \node at (-2,5.8) {\color{red}0x5E1D4A80};
  \node at (0.8,5.8) {\color{red}mtz[0]};
  \node at (0.8,5.45) {\color{red}mtz[1]};
  \node at (-2,5.45) {\color{red}0x5E1D4A84};
  \node at (-2,4.45) {\color{red}0x5E1D4AA0};
  \node at (0.8,4.45) {\color{red}mtz[0][0]};
  \node at (0.8,4.1) {\color{red}mtz[0][1]};
  \node at (-2,2.8) {\color{red}0x5E1D4AC0};
    \node at (-1.4,8.1) {\color{red}i};
  \node at (-1.4,7.75) {\color{red}j};
  \node at (-1.4,7.4) {\color{red}file};
  \node at (-1.4,7.1) {\color{red}row};
  \node at (-1.4,6.8) {\color{red}mtz};

  \node at (0,8.6) {\footnotesize\itshape\color{blue} {$ejemplo$ $para$ $matriz$ $(2$ x $3)$}};
  \fill[color=green] (3.2,4.3) -- (6.4,4.3) -- (6.4,4.0) -- (3.2,4.0) -- cycle;

  \draw[white,-](-1.9,0.6)to [out=90,in=-90]node[right,midway]{} ++(0,0.5) ; % para fijar rectangulos
\end{tikzpicture}
						
  \column{0.5\textwidth}
  \lstset{basicstyle=\tiny}
 \begin{lstlisting}
#include<stdio.h>
#include <stdlib.h>

int main (void)
{
  int file=0,row=0,i=0,j=0;
  int **mtz;

  printf("ingrese filas y columnas \n");
  scanf("% d % d",&file,&row);
  mtz=(int **)malloc(file*sizeof(int *));
  for (i=0 ; i<file ; i++){
   mtz[i]=(int *)malloc(row*sizeof(int));
  }
  for (i=0 ; i<file ; i++){
    for (j=0 ; j<row ; j++){
       printf("file% d  rows% d\n",i,j);
       scanf("% d",&mtz[i][j]);
    }
  }
  printf("\n");
  for (i=0 ; i<file ; i++){
    for (j=0 ; j<row ; j++){
       printf("% d \t",mtz[i][j]);
    }
  }
  for (i=0 ; i<file ; i++){
    free(mtz[i]);
  }
  free(mtz);
  return (0);
}
   \end{lstlisting}
 \end{columns}
\end{frame}


\begin{frame}[fragile]
\fontsize{6.5pt}{12pt}\selectfont
 \frametitle{Matriz {\color{yellow}NxN} con {\color{yellow}malloc} en Arquitectura \textcolor{yellow}{X86-32} bits }
 \begin{columns}[c]
  \column{0.4\textwidth}
\lstset{basicstyle=\tiny}
\begin{tikzpicture}
  \begin{scope}[
  	every node/.style={draw, 
  	anchor=text, 
  	rectangle split,
    rectangle split parts=19, 
    rectangle split part fill={green!30,green!30,green!0,green!0,green!0,red!0,blue!30,blue!30,red!0,red!30,red!30,red!30,blue!0,red!30,red!30,red!30,red!0,red!0,red!0,red!0},
    minimum width=0.6cm}]
    \node (R) at (-0.5,8){ 
    		\nodepart{one}\tiny 0
    		\nodepart{two}\tiny {\color{red}2}
    		\nodepart{three}\tiny 2
    		\nodepart{four}\tiny 3
    		\nodepart{five}\tiny 0x5E1D4A80
    		\nodepart{six}{$\vdots$}
    		\nodepart{seven}\tiny 0x5E1D4AA0
    		\nodepart{eight}\tiny 0x5E1D4AC0
    		\nodepart{nine}{$\vdots$}
    		\nodepart{ten}\tiny 11
    		\nodepart{eleven}\tiny 12
    		\nodepart{twelve}
    		\nodepart{thirteen}{$\vdots$}
    		\nodepart{fourteen}
    		\nodepart{fifteen}
    		\nodepart{sixteen}
    		\nodepart{seventeen}{$\vdots$}
    		\nodepart{eighteen}
    		\nodepart{nineteen}};
    		
  \end{scope}
  \node at (-2,5.8) {\color{red}0x5E1D4A80};
  \node at (0.8,5.8) {\color{red}mtz[0]};
  \node at (0.8,5.45) {\color{red}mtz[1]};
  \node at (-2,5.45) {\color{red}0x5E1D4A84};
  \node at (-2,4.45) {\color{red}0x5E1D4AA0};
  \node at (0.8,4.45) {\color{red}mtz[0][0]};
  \node at (0.8,4.1) {\color{red}mtz[0][1]};
  \node at (-2,2.8) {\color{red}0x5E1D4AC0};
    \node at (-1.4,8.1) {\color{red}i};
  \node at (-1.4,7.75) {\color{red}j};
  \node at (-1.4,7.4) {\color{red}file};
  \node at (-1.4,7.1) {\color{red}row};
  \node at (-1.4,6.8) {\color{red}mtz};

  \node at (0,8.6) {\footnotesize\itshape\color{blue} {$ejemplo$ $para$ $matriz$ $(2$ x $3)$}};
  \fill[color=green] (5.4,4.8) -- (5.9,4.8) -- (5.9,4.55) -- (5.4,4.55) -- cycle;

  \draw[white,-](-1.9,0.6)to [out=90,in=-90]node[right,midway]{} ++(0,0.5) ; % para fijar rectangulos
\end{tikzpicture}
						
  \column{0.5\textwidth}
  \lstset{basicstyle=\tiny}
 \begin{lstlisting}
#include<stdio.h>
#include <stdlib.h>

int main (void)
{
  int file=0,row=0,i=0,j=0;
  int **mtz;

  printf("ingrese filas y columnas \n");
  scanf("% d % d",&file,&row);
  mtz=(int **)malloc(file*sizeof(int *));
  for (i=0 ; i<file ; i++){
   mtz[i]=(int *)malloc(row*sizeof(int));
  }
  for (i=0 ; i<file ; i++){
    for (j=0 ; j<row ; j++){
       printf("file% d  rows% d\n",i,j);
       scanf("% d",&mtz[i][j]);
    }
  }
  printf("\n");
  for (i=0 ; i<file ; i++){
    for (j=0 ; j<row ; j++){
       printf("% d \t",mtz[i][j]);
    }
  }
  for (i=0 ; i<file ; i++){
    free(mtz[i]);
  }
  free(mtz);
  return (0);
}
   \end{lstlisting}
 \end{columns}
\end{frame}

\begin{frame}[fragile]
\fontsize{6.5pt}{12pt}\selectfont
 \frametitle{Matriz {\color{yellow}NxN} con {\color{yellow}malloc} en Arquitectura \textcolor{yellow}{X86-32} bits }
 \begin{columns}[c]
  \column{0.4\textwidth}
\lstset{basicstyle=\tiny}
\begin{tikzpicture}
  \begin{scope}[
  	every node/.style={draw, 
  	anchor=text, 
  	rectangle split,
    rectangle split parts=19, 
    rectangle split part fill={green!30,green!30,green!0,green!30,green!0,red!0,blue!30,blue!30,red!0,red!30,red!30,red!30,blue!0,red!30,red!30,red!30,red!0,red!0,red!0,red!0},
    minimum width=0.6cm}]
    \node (R) at (-0.5,8){ 
    		\nodepart{one}\tiny 0
    		\nodepart{two}\tiny {\color{red}2}
    		\nodepart{three}\tiny 2
    		\nodepart{four}\tiny 3
    		\nodepart{five}\tiny 0x5E1D4A80
    		\nodepart{six}{$\vdots$}
    		\nodepart{seven}\tiny 0x5E1D4AA0
    		\nodepart{eight}\tiny 0x5E1D4AC0
    		\nodepart{nine}{$\vdots$}
    		\nodepart{ten}\tiny 11
    		\nodepart{eleven}\tiny 12
    		\nodepart{twelve}
    		\nodepart{thirteen}{$\vdots$}
    		\nodepart{fourteen}
    		\nodepart{fifteen}
    		\nodepart{sixteen}
    		\nodepart{seventeen}{$\vdots$}
    		\nodepart{eighteen}
    		\nodepart{nineteen}};
    		
  \end{scope}
  \node at (-2,5.8) {\color{red}0x5E1D4A80};
  \node at (0.8,5.8) {\color{red}mtz[0]};
  \node at (0.8,5.45) {\color{red}mtz[1]};
  \node at (-2,5.45) {\color{red}0x5E1D4A84};
  \node at (-2,4.45) {\color{red}0x5E1D4AA0};
  \node at (0.8,4.45) {\color{red}mtz[0][0]};
  \node at (0.8,4.1) {\color{red}mtz[0][1]};
  \node at (-2,2.8) {\color{red}0x5E1D4AC0};
    \node at (-1.4,8.1) {\color{red}i};
  \node at (-1.4,7.75) {\color{red}j};
  \node at (-1.4,7.4) {\color{red}file};
  \node at (-1.4,7.1) {\color{red}row};
  \node at (-1.4,6.8) {\color{red}mtz};

  \node at (0,8.6) {\footnotesize\itshape\color{blue} {$ejemplo$ $para$ $matriz$ $(2$ x $3)$}};
  \fill[color=green] (4.4,4.8) -- (5.4,4.8) -- (5.4,4.55) -- (4.4,4.55) -- cycle;

  \draw[white,-](-1.9,0.6)to [out=90,in=-90]node[right,midway]{} ++(0,0.5) ; % para fijar rectangulos
\end{tikzpicture}
						
  \column{0.5\textwidth}
  \lstset{basicstyle=\tiny}
 \begin{lstlisting}
#include<stdio.h>
#include <stdlib.h>

int main (void)
{
  int file=0,row=0,i=0,j=0;
  int **mtz;

  printf("ingrese filas y columnas \n");
  scanf("% d % d",&file,&row);
  mtz=(int **)malloc(file*sizeof(int *));
  for (i=0 ; i<file ; i++){
   mtz[i]=(int *)malloc(row*sizeof(int));
  }
  for (i=0 ; i<file ; i++){
    for (j=0 ; j<row ; j++){
       printf("file% d  rows% d\n",i,j);
       scanf("% d",&mtz[i][j]);
    }
  }
  printf("\n");
  for (i=0 ; i<file ; i++){
    for (j=0 ; j<row ; j++){
       printf("% d \t",mtz[i][j]);
    }
  }
  for (i=0 ; i<file ; i++){
    free(mtz[i]);
  }
  free(mtz);
  return (0);
}
   \end{lstlisting}
 \end{columns}
\end{frame}






















\begin{frame}[fragile]
\fontsize{6.5pt}{12pt}\selectfont
 \frametitle{Matriz {\color{yellow}NxN} con {\color{yellow}malloc} en Arquitectura \textcolor{yellow}{X86-32} bits }
 \begin{columns}[c]
  \column{0.4\textwidth}
\lstset{basicstyle=\tiny}
\begin{tikzpicture}
  \begin{scope}[
  	every node/.style={draw, 
  	anchor=text, 
  	rectangle split,
    rectangle split parts=19, 
    rectangle split part fill={green!30,green!30,green!0,green!0,green!0,red!0,blue!30,blue!30,red!0,red!30,red!30,red!30,blue!0,red!30,red!30,red!30,red!0,red!0,red!0,red!0},
    minimum width=0.6cm}]
    \node (R) at (-0.5,8){ 
    		\nodepart{one}\tiny 0
    		\nodepart{two}\tiny 2
    		\nodepart{three}\tiny 2
    		\nodepart{four}\tiny 3
    		\nodepart{five}\tiny 0x5E1D4A80
    		\nodepart{six}{$\vdots$}
    		\nodepart{seven}\tiny 0x5E1D4AA0
    		\nodepart{eight}\tiny 0x5E1D4AC0
    		\nodepart{nine}{$\vdots$}
    		\nodepart{ten}\tiny 11
    		\nodepart{eleven}\tiny 12
    		\nodepart{twelve}\tiny {\color{red}13}
    		\nodepart{thirteen}{$\vdots$}
    		\nodepart{fourteen}
    		\nodepart{fifteen}
    		\nodepart{sixteen}
    		\nodepart{seventeen}{$\vdots$}
    		\nodepart{eighteen}
    		\nodepart{nineteen}};
    		
  \end{scope}
  \node at (-2,5.8) {\color{red}0x5E1D4A80};
  \node at (0.8,5.8) {\color{red}mtz[0]};
  \node at (0.8,5.45) {\color{red}mtz[1]};
  \node at (-2,5.45) {\color{red}0x5E1D4A84};
  \node at (-2,4.45) {\color{red}0x5E1D4AA0};
  \node at (0.8,4.45) {\color{red}mtz[0][0]};
  \node at (0.8,4.1) {\color{red}mtz[0][1]};
  \node at (0.8,3.75) {\color{red}mtz[0][2]};
  \node at (-2,2.8) {\color{red}0x5E1D4AC0};
    \node at (-1.4,8.1) {\color{red}i};
  \node at (-1.4,7.75) {\color{red}j};
  \node at (-1.4,7.4) {\color{red}file};
  \node at (-1.4,7.1) {\color{red}row};
  \node at (-1.4,6.8) {\color{red}mtz};

  \node at (0,8.6) {\footnotesize\itshape\color{blue} {$ejemplo$ $para$ $matriz$ $(2$ x $3)$}};
  \fill[color=green] (3.2,4.3) -- (6.4,4.3) -- (6.4,4.0) -- (3.2,4.0) -- cycle;

  \draw[white,-](-1.9,0.6)to [out=90,in=-90]node[right,midway]{} ++(0,0.5) ; % para fijar rectangulos
\end{tikzpicture}
						
  \column{0.5\textwidth}
  \lstset{basicstyle=\tiny}
 \begin{lstlisting}
#include<stdio.h>
#include <stdlib.h>

int main (void)
{
  int file=0,row=0,i=0,j=0;
  int **mtz;

  printf("ingrese filas y columnas \n");
  scanf("% d % d",&file,&row);
  mtz=(int **)malloc(file*sizeof(int *));
  for (i=0 ; i<file ; i++){
   mtz[i]=(int *)malloc(row*sizeof(int));
  }
  for (i=0 ; i<file ; i++){
    for (j=0 ; j<row ; j++){
       printf("file% d  rows% d\n",i,j);
       scanf("% d",&mtz[i][j]);
    }
  }
  printf("\n");
  for (i=0 ; i<file ; i++){
    for (j=0 ; j<row ; j++){
       printf("% d \t",mtz[i][j]);
    }
  }
  for (i=0 ; i<file ; i++){
    free(mtz[i]);
  }
  free(mtz);
  return (0);
}
   \end{lstlisting}
 \end{columns}
\end{frame}


\begin{frame}[fragile]
\fontsize{6.5pt}{12pt}\selectfont
 \frametitle{Matriz {\color{yellow}NxN} con {\color{yellow}malloc} en Arquitectura \textcolor{yellow}{X86-32} bits }
 \begin{columns}[c]
  \column{0.4\textwidth}
\lstset{basicstyle=\tiny}
\begin{tikzpicture}
  \begin{scope}[
  	every node/.style={draw, 
  	anchor=text, 
  	rectangle split,
    rectangle split parts=19, 
    rectangle split part fill={green!30,green!30,green!0,green!0,green!0,red!0,blue!30,blue!30,red!0,red!30,red!30,red!30,blue!0,red!30,red!30,red!30,red!0,red!0,red!0,red!0},
    minimum width=0.6cm}]
    \node (R) at (-0.5,8){ 
    		\nodepart{one}\tiny 0
    		\nodepart{two}\tiny {\color{red}3}
    		\nodepart{three}\tiny 2
    		\nodepart{four}\tiny 3
    		\nodepart{five}\tiny 0x5E1D4A80
    		\nodepart{six}{$\vdots$}
    		\nodepart{seven}\tiny 0x5E1D4AA0
    		\nodepart{eight}\tiny 0x5E1D4AC0
    		\nodepart{nine}{$\vdots$}
    		\nodepart{ten}\tiny 11
    		\nodepart{eleven}\tiny 12
    		\nodepart{twelve}\tiny 13
    		\nodepart{thirteen}{$\vdots$}
    		\nodepart{fourteen}
    		\nodepart{fifteen}
    		\nodepart{sixteen}
    		\nodepart{seventeen}{$\vdots$}
    		\nodepart{eighteen}
    		\nodepart{nineteen}};
    		
  \end{scope}
  \node at (-2,5.8) {\color{red}0x5E1D4A80};
  \node at (0.8,5.8) {\color{red}mtz[0]};
  \node at (0.8,5.45) {\color{red}mtz[1]};
  \node at (-2,5.45) {\color{red}0x5E1D4A84};
  \node at (-2,4.45) {\color{red}0x5E1D4AA0};
  \node at (0.8,4.45) {\color{red}mtz[0][0]};
  \node at (0.8,4.1) {\color{red}mtz[0][1]};
  \node at (0.8,3.75) {\color{red}mtz[0][2]};
  \node at (-2,2.8) {\color{red}0x5E1D4AC0};
    \node at (-1.4,8.1) {\color{red}i};
  \node at (-1.4,7.75) {\color{red}j};
  \node at (-1.4,7.4) {\color{red}file};
  \node at (-1.4,7.1) {\color{red}row};
  \node at (-1.4,6.8) {\color{red}mtz};

  \node at (0,8.6) {\footnotesize\itshape\color{blue} {$ejemplo$ $para$ $matriz$ $(2$ x $3)$}};
  \fill[color=green] (5.4,4.8) -- (5.9,4.8) -- (5.9,4.55) -- (5.4,4.55) -- cycle;

  \draw[white,-](-1.9,0.6)to [out=90,in=-90]node[right,midway]{} ++(0,0.5) ; % para fijar rectangulos
\end{tikzpicture}
						
  \column{0.5\textwidth}
  \lstset{basicstyle=\tiny}
 \begin{lstlisting}
#include<stdio.h>
#include <stdlib.h>

int main (void)
{
  int file=0,row=0,i=0,j=0;
  int **mtz;

  printf("ingrese filas y columnas \n");
  scanf("% d % d",&file,&row);
  mtz=(int **)malloc(file*sizeof(int *));
  for (i=0 ; i<file ; i++){
   mtz[i]=(int *)malloc(row*sizeof(int));
  }
  for (i=0 ; i<file ; i++){
    for (j=0 ; j<row ; j++){
       printf("file% d  rows% d\n",i,j);
       scanf("% d",&mtz[i][j]);
    }
  }
  printf("\n");
  for (i=0 ; i<file ; i++){
    for (j=0 ; j<row ; j++){
       printf("% d \t",mtz[i][j]);
    }
  }
  for (i=0 ; i<file ; i++){
    free(mtz[i]);
  }
  free(mtz);
  return (0);
}
   \end{lstlisting}
 \end{columns}
\end{frame}


\begin{frame}[fragile]
\fontsize{6.5pt}{12pt}\selectfont
 \frametitle{Matriz {\color{yellow}NxN} con {\color{yellow}malloc} en Arquitectura \textcolor{yellow}{X86-32} bits }
 \begin{columns}[c]
  \column{0.4\textwidth}
\lstset{basicstyle=\tiny}
\begin{tikzpicture}
  \begin{scope}[
  	every node/.style={draw, 
  	anchor=text, 
  	rectangle split,
    rectangle split parts=19, 
    rectangle split part fill={green!30,green!30,green!0,green!30,green!0,red!0,blue!30,blue!30,red!0,red!30,red!30,red!30,blue!0,red!30,red!30,red!30,red!0,red!0,red!0,red!0},
    minimum width=0.6cm}]
    \node (R) at (-0.5,8){ 
    		\nodepart{one}\tiny 0
    		\nodepart{two}\tiny {\color{red}3}
    		\nodepart{three}\tiny 2
    		\nodepart{four}\tiny {\color{red}3}
    		\nodepart{five}\tiny 0x5E1D4A80
    		\nodepart{six}{$\vdots$}
    		\nodepart{seven}\tiny 0x5E1D4AA0
    		\nodepart{eight}\tiny 0x5E1D4AC0
    		\nodepart{nine}{$\vdots$}
    		\nodepart{ten}\tiny 11
    		\nodepart{eleven}\tiny 12
    		\nodepart{twelve}\tiny 13
    		\nodepart{thirteen}{$\vdots$}
    		\nodepart{fourteen}
    		\nodepart{fifteen}
    		\nodepart{sixteen}
    		\nodepart{seventeen}{$\vdots$}
    		\nodepart{eighteen}
    		\nodepart{nineteen}};
    		
  \end{scope}
  \node at (-2,5.8) {\color{red}0x5E1D4A80};
  \node at (0.8,5.8) {\color{red}mtz[0]};
  \node at (0.8,5.45) {\color{red}mtz[1]};
  \node at (-2,5.45) {\color{red}0x5E1D4A84};
  \node at (-2,4.45) {\color{red}0x5E1D4AA0};
  \node at (0.8,4.45) {\color{red}mtz[0][0]};
  \node at (0.8,4.1) {\color{red}mtz[0][1]};
  \node at (0.8,3.75) {\color{red}mtz[0][2]};
  \node at (-2,2.8) {\color{red}0x5E1D4AC0};
    \node at (-1.4,8.1) {\color{red}i};
  \node at (-1.4,7.75) {\color{red}j};
  \node at (-1.4,7.4) {\color{red}file};
  \node at (-1.4,7.1) {\color{red}row};
  \node at (-1.4,6.8) {\color{red}mtz};

  \node at (0,8.6) {\footnotesize\itshape\color{blue} {$ejemplo$ $para$ $matriz$ $(2$ x $3)$}};
  \fill[color=green] (4.4,4.8) -- (5.4,4.8) -- (5.4,4.55) -- (4.4,4.55) -- cycle;

  \node [fill=green] at (0.2,4.8) {\fbox{
						\begin{minipage}{3cm}
							\fontsize{8pt}{8pt}\selectfont
		 {No cumple la condicion} \\
		  \\
		 {y sale del ciclo for} \\
						\end{minipage}}};						  







  \draw[white,-](-1.9,0.6)to [out=90,in=-90]node[right,midway]{} ++(0,0.5) ; % para fijar rectangulos
\end{tikzpicture}
						
  \column{0.5\textwidth}
  \lstset{basicstyle=\tiny}
 \begin{lstlisting}
#include<stdio.h>
#include <stdlib.h>

int main (void)
{
  int file=0,row=0,i=0,j=0;
  int **mtz;

  printf("ingrese filas y columnas \n");
  scanf("% d % d",&file,&row);
  mtz=(int **)malloc(file*sizeof(int *));
  for (i=0 ; i<file ; i++){
   mtz[i]=(int *)malloc(row*sizeof(int));
  }
  for (i=0 ; i<file ; i++){
    for (j=0 ; j<row ; j++){
       printf("file% d  rows% d\n",i,j);
       scanf("% d",&mtz[i][j]);
    }
  }
  printf("\n");
  for (i=0 ; i<file ; i++){
    for (j=0 ; j<row ; j++){
       printf("% d \t",mtz[i][j]);
    }
  }
  for (i=0 ; i<file ; i++){
    free(mtz[i]);
  }
  free(mtz);
  return (0);
}
   \end{lstlisting}
 \end{columns}
\end{frame}


\begin{frame}[fragile]
\fontsize{6.5pt}{12pt}\selectfont
 \frametitle{Matriz {\color{yellow}NxN} con {\color{yellow}malloc} en Arquitectura \textcolor{yellow}{X86-32} bits }
 \begin{columns}[c]
  \column{0.4\textwidth}
\lstset{basicstyle=\tiny}
\begin{tikzpicture}
  \begin{scope}[
  	every node/.style={draw, 
  	anchor=text, 
  	rectangle split,
    rectangle split parts=19, 
    rectangle split part fill={green!30,green!0,green!0,green!0,green!0,red!0,blue!30,blue!30,red!0,red!30,red!30,red!30,blue!0,red!30,red!30,red!30,red!0,red!0,red!0,red!0},
    minimum width=0.6cm}]
    \node (R) at (-0.5,8){ 
    		\nodepart{one}\tiny {\color{red}1}
    		\nodepart{two}\tiny 3
    		\nodepart{three}\tiny 2
    		\nodepart{four}\tiny 3
    		\nodepart{five}\tiny 0x5E1D4A80
    		\nodepart{six}{$\vdots$}
    		\nodepart{seven}\tiny 0x5E1D4AA0
    		\nodepart{eight}\tiny 0x5E1D4AC0
    		\nodepart{nine}{$\vdots$}
    		\nodepart{ten}\tiny 11
    		\nodepart{eleven}\tiny 12
    		\nodepart{twelve}\tiny 13
    		\nodepart{thirteen}{$\vdots$}
    		\nodepart{fourteen}
    		\nodepart{fifteen}
    		\nodepart{sixteen}
    		\nodepart{seventeen}{$\vdots$}
    		\nodepart{eighteen}
    		\nodepart{nineteen}};
    		
  \end{scope}
  \node at (-2,5.8) {\color{red}0x5E1D4A80};
  \node at (0.8,5.8) {\color{red}mtz[0]};
  \node at (0.8,5.45) {\color{red}mtz[1]};
  \node at (-2,5.45) {\color{red}0x5E1D4A84};
  \node at (-2,4.45) {\color{red}0x5E1D4AA0};
  \node at (0.8,4.45) {\color{red}mtz[0][0]};
  \node at (0.8,4.1) {\color{red}mtz[0][1]};
  \node at (0.8,3.75) {\color{red}mtz[0][2]};
  \node at (-2,2.8) {\color{red}0x5E1D4AC0};
    \node at (-1.4,8.1) {\color{red}i};
  \node at (-1.4,7.75) {\color{red}j};
  \node at (-1.4,7.4) {\color{red}file};
  \node at (-1.4,7.1) {\color{red}row};
  \node at (-1.4,6.8) {\color{red}mtz};

  \node at (0,8.6) {\footnotesize\itshape\color{blue} {$ejemplo$ $para$ $matriz$ $(2$ x $3)$}};
  \fill[color=green] (5.3,5.1) -- (5.8,5.1) -- (5.8,4.8) -- (5.3,4.8) -- cycle;

  \draw[white,-](-1.9,0.6)to [out=90,in=-90]node[right,midway]{} ++(0,0.5) ; % para fijar rectangulos
\end{tikzpicture}
						
  \column{0.5\textwidth}
  \lstset{basicstyle=\tiny}
 \begin{lstlisting}
#include<stdio.h>
#include <stdlib.h>

int main (void)
{
  int file=0,row=0,i=0,j=0;
  int **mtz;

  printf("ingrese filas y columnas \n");
  scanf("% d % d",&file,&row);
  mtz=(int **)malloc(file*sizeof(int *));
  for (i=0 ; i<file ; i++){
   mtz[i]=(int *)malloc(row*sizeof(int));
  }
  for (i=0 ; i<file ; i++){
    for (j=0 ; j<row ; j++){
       printf("file% d  rows% d\n",i,j);
       scanf("% d",&mtz[i][j]);
    }
  }
  printf("\n");
  for (i=0 ; i<file ; i++){
    for (j=0 ; j<row ; j++){
       printf("% d \t",mtz[i][j]);
    }
  }
  for (i=0 ; i<file ; i++){
    free(mtz[i]);
  }
  free(mtz);
  return (0);
}
   \end{lstlisting}
 \end{columns}
\end{frame}


\begin{frame}[fragile]
\fontsize{6.5pt}{12pt}\selectfont
 \frametitle{Matriz {\color{yellow}NxN} con {\color{yellow}malloc} en Arquitectura \textcolor{yellow}{X86-32} bits }
 \begin{columns}[c]
  \column{0.4\textwidth}
\lstset{basicstyle=\tiny}
\begin{tikzpicture}
  \begin{scope}[
  	every node/.style={draw, 
  	anchor=text, 
  	rectangle split,
    rectangle split parts=19, 
    rectangle split part fill={green!30,green!0,green!30,green!0,green!0,red!0,blue!30,blue!30,red!0,red!30,red!30,red!30,blue!0,red!30,red!30,red!30,red!0,red!0,red!0,red!0},
    minimum width=0.6cm}]
    \node (R) at (-0.5,8){ 
    		\nodepart{one}\tiny 1
    		\nodepart{two}\tiny 3
    		\nodepart{three}\tiny 2
    		\nodepart{four}\tiny 3
    		\nodepart{five}\tiny 0x5E1D4A80
    		\nodepart{six}{$\vdots$}
    		\nodepart{seven}\tiny 0x5E1D4AA0
    		\nodepart{eight}\tiny 0x5E1D4AC0
    		\nodepart{nine}{$\vdots$}
    		\nodepart{ten}\tiny 11
    		\nodepart{eleven}\tiny 12
    		\nodepart{twelve}\tiny 13
    		\nodepart{thirteen}{$\vdots$}
    		\nodepart{fourteen}
    		\nodepart{fifteen}
    		\nodepart{sixteen}
    		\nodepart{seventeen}{$\vdots$}
    		\nodepart{eighteen}
    		\nodepart{nineteen}};
    		
  \end{scope}
  \node at (-2,5.8) {\color{red}0x5E1D4A80};
  \node at (0.8,5.8) {\color{red}mtz[0]};
  \node at (0.8,5.45) {\color{red}mtz[1]};
  \node at (-2,5.45) {\color{red}0x5E1D4A84};
  \node at (-2,4.45) {\color{red}0x5E1D4AA0};
  \node at (0.8,4.45) {\color{red}mtz[0][0]};
  \node at (0.8,4.1) {\color{red}mtz[0][1]};
  \node at (0.8,3.75) {\color{red}mtz[0][2]};
  \node at (-2,2.8) {\color{red}0x5E1D4AC0};
    \node at (-1.4,8.1) {\color{red}i};
  \node at (-1.4,7.75) {\color{red}j};
  \node at (-1.4,7.4) {\color{red}file};
  \node at (-1.4,7.1) {\color{red}row};
  \node at (-1.4,6.8) {\color{red}mtz};

  \node at (0,8.6) {\footnotesize\itshape\color{blue} {$ejemplo$ $para$ $matriz$ $(2$ x $3)$}};
  \fill[color=green] (4.2,5.1) -- (5,5.1) -- (5,4.8) -- (4.2,4.8) -- cycle;

  \draw[white,-](-1.9,0.6)to [out=90,in=-90]node[right,midway]{} ++(0,0.5) ; % para fijar rectangulos
\end{tikzpicture}
						
  \column{0.5\textwidth}
  \lstset{basicstyle=\tiny}
 \begin{lstlisting}
#include<stdio.h>
#include <stdlib.h>

int main (void)
{
  int file=0,row=0,i=0,j=0;
  int **mtz;

  printf("ingrese filas y columnas \n");
  scanf("% d % d",&file,&row);
  mtz=(int **)malloc(file*sizeof(int *));
  for (i=0 ; i<file ; i++){
   mtz[i]=(int *)malloc(row*sizeof(int));
  }
  for (i=0 ; i<file ; i++){
    for (j=0 ; j<row ; j++){
       printf("file% d  rows% d\n",i,j);
       scanf("% d",&mtz[i][j]);
    }
  }
  printf("\n");
  for (i=0 ; i<file ; i++){
    for (j=0 ; j<row ; j++){
       printf("% d \t",mtz[i][j]);
    }
  }
  for (i=0 ; i<file ; i++){
    free(mtz[i]);
  }
  free(mtz);
  return (0);
}
   \end{lstlisting}
 \end{columns}
\end{frame}


\begin{frame}[fragile]
\fontsize{6.5pt}{12pt}\selectfont
 \frametitle{Matriz {\color{yellow}NxN} con {\color{yellow}malloc} en Arquitectura \textcolor{yellow}{X86-32} bits }
 \begin{columns}[c]
  \column{0.4\textwidth}
\lstset{basicstyle=\tiny}
\begin{tikzpicture}
  \begin{scope}[
  	every node/.style={draw, 
  	anchor=text, 
  	rectangle split,
    rectangle split parts=19, 
    rectangle split part fill={green!0,green!30,green!0,green!0,green!0,red!0,blue!30,blue!30,red!0,red!30,red!30,red!30,blue!0,red!30,red!30,red!30,red!0,red!0,red!0,red!0},
    minimum width=0.6cm}]
    \node (R) at (-0.5,8){ 
    		\nodepart{one}\tiny 1
    		\nodepart{two}\tiny {\color{red}0}
    		\nodepart{three}\tiny 2
    		\nodepart{four}\tiny 3
    		\nodepart{five}\tiny 0x5E1D4A80
    		\nodepart{six}{$\vdots$}
    		\nodepart{seven}\tiny 0x5E1D4AA0
    		\nodepart{eight}\tiny 0x5E1D4AC0
    		\nodepart{nine}{$\vdots$}
    		\nodepart{ten}\tiny 11
    		\nodepart{eleven}\tiny 12
    		\nodepart{twelve}\tiny 13
    		\nodepart{thirteen}{$\vdots$}
    		\nodepart{fourteen}
    		\nodepart{fifteen}
    		\nodepart{sixteen}
    		\nodepart{seventeen}{$\vdots$}
    		\nodepart{eighteen}
    		\nodepart{nineteen}};
    		
  \end{scope}
  \node at (-2,5.8) {\color{red}0x5E1D4A80};
  \node at (0.8,5.8) {\color{red}mtz[0]};
  \node at (0.8,5.45) {\color{red}mtz[1]};
  \node at (-2,5.45) {\color{red}0x5E1D4A84};
  \node at (-2,4.45) {\color{red}0x5E1D4AA0};
  \node at (0.8,4.45) {\color{red}mtz[0][0]};
  \node at (0.8,4.1) {\color{red}mtz[0][1]};
  \node at (0.8,3.75) {\color{red}mtz[0][2]};
  \node at (-2,2.8) {\color{red}0x5E1D4AC0};
    \node at (-1.4,8.1) {\color{red}i};
  \node at (-1.4,7.75) {\color{red}j};
  \node at (-1.4,7.4) {\color{red}file};
  \node at (-1.4,7.1) {\color{red}row};
  \node at (-1.4,6.8) {\color{red}mtz};

  \node at (0,8.6) {\footnotesize\itshape\color{blue} {$ejemplo$ $para$ $matriz$ $(2$ x $3)$}};
  \fill[color=green] (3.6,4.8) -- (4.4,4.8) -- (4.4,4.55) -- (3.6,4.55) -- cycle;

  \draw[white,-](-1.9,0.6)to [out=90,in=-90]node[right,midway]{} ++(0,0.5) ; % para fijar rectangulos
\end{tikzpicture}
						
  \column{0.5\textwidth}
  \lstset{basicstyle=\tiny}
 \begin{lstlisting}
#include<stdio.h>
#include <stdlib.h>

int main (void)
{
  int file=0,row=0,i=0,j=0;
  int **mtz;

  printf("ingrese filas y columnas \n");
  scanf("% d % d",&file,&row);
  mtz=(int **)malloc(file*sizeof(int *));
  for (i=0 ; i<file ; i++){
   mtz[i]=(int *)malloc(row*sizeof(int));
  }
  for (i=0 ; i<file ; i++){
    for (j=0 ; j<row ; j++){
       printf("file% d  rows% d\n",i,j);
       scanf("% d",&mtz[i][j]);
    }
  }
  printf("\n");
  for (i=0 ; i<file ; i++){
    for (j=0 ; j<row ; j++){
       printf("% d \t",mtz[i][j]);
    }
  }
  for (i=0 ; i<file ; i++){
    free(mtz[i]);
  }
  free(mtz);
  return (0);
}
   \end{lstlisting}
 \end{columns}
\end{frame}


\begin{frame}[fragile]
\fontsize{6.5pt}{12pt}\selectfont
 \frametitle{Matriz {\color{yellow}NxN} con {\color{yellow}malloc} en Arquitectura \textcolor{yellow}{X86-32} bits }
 \begin{columns}[c]
  \column{0.4\textwidth}
\lstset{basicstyle=\tiny}
\begin{tikzpicture}
  \begin{scope}[
  	every node/.style={draw, 
  	anchor=text, 
  	rectangle split,
    rectangle split parts=19, 
    rectangle split part fill={green!0,green!30,green!0,green!30,green!0,red!0,blue!30,blue!30,red!0,red!30,red!30,red!30,blue!0,red!30,red!30,red!30,red!0,red!0,red!0,red!0},
    minimum width=0.6cm}]
    \node (R) at (-0.5,8){ 
    		\nodepart{one}\tiny 1
    		\nodepart{two}\tiny 0
    		\nodepart{three}\tiny 2
    		\nodepart{four}\tiny 3
    		\nodepart{five}\tiny 0x5E1D4A80
    		\nodepart{six}{$\vdots$}
    		\nodepart{seven}\tiny 0x5E1D4AA0
    		\nodepart{eight}\tiny 0x5E1D4AC0
    		\nodepart{nine}{$\vdots$}
    		\nodepart{ten}\tiny 11
    		\nodepart{eleven}\tiny 12
    		\nodepart{twelve}\tiny 13
    		\nodepart{thirteen}{$\vdots$}
    		\nodepart{fourteen}
    		\nodepart{fifteen}
    		\nodepart{sixteen}
    		\nodepart{seventeen}{$\vdots$}
    		\nodepart{eighteen}
    		\nodepart{nineteen}};
    		
  \end{scope}
  \node at (-2,5.8) {\color{red}0x5E1D4A80};
  \node at (0.8,5.8) {\color{red}mtz[0]};
  \node at (0.8,5.45) {\color{red}mtz[1]};
  \node at (-2,5.45) {\color{red}0x5E1D4A84};
  \node at (-2,4.45) {\color{red}0x5E1D4AA0};
  \node at (0.8,4.45) {\color{red}mtz[0][0]};
  \node at (0.8,4.1) {\color{red}mtz[0][1]};
  \node at (0.8,3.75) {\color{red}mtz[0][2]};
  \node at (-2,2.8) {\color{red}0x5E1D4AC0};
    \node at (-1.4,8.1) {\color{red}i};
  \node at (-1.4,7.75) {\color{red}j};
  \node at (-1.4,7.4) {\color{red}file};
  \node at (-1.4,7.1) {\color{red}row};
  \node at (-1.4,6.8) {\color{red}mtz};

  \node at (0,8.6) {\footnotesize\itshape\color{blue} {$ejemplo$ $para$ $matriz$ $(2$ x $3)$}};
  \fill[color=green] (4.4,4.8) -- (5.4,4.8) -- (5.4,4.55) -- (4.4,4.55) -- cycle;

  \draw[white,-](-1.9,0.6)to [out=90,in=-90]node[right,midway]{} ++(0,0.5) ; % para fijar rectangulos
\end{tikzpicture}
						
  \column{0.5\textwidth}
  \lstset{basicstyle=\tiny}
 \begin{lstlisting}
#include<stdio.h>
#include <stdlib.h>

int main (void)
{
  int file=0,row=0,i=0,j=0;
  int **mtz;

  printf("ingrese filas y columnas \n");
  scanf("% d % d",&file,&row);
  mtz=(int **)malloc(file*sizeof(int *));
  for (i=0 ; i<file ; i++){
   mtz[i]=(int *)malloc(row*sizeof(int));
  }
  for (i=0 ; i<file ; i++){
    for (j=0 ; j<row ; j++){
       printf("file% d  rows% d\n",i,j);
       scanf("% d",&mtz[i][j]);
    }
  }
  printf("\n");
  for (i=0 ; i<file ; i++){
    for (j=0 ; j<row ; j++){
       printf("% d \t",mtz[i][j]);
    }
  }
  for (i=0 ; i<file ; i++){
    free(mtz[i]);
  }
  free(mtz);
  return (0);
}
   \end{lstlisting}
 \end{columns}
\end{frame}


\begin{frame}[fragile]
\fontsize{6.5pt}{12pt}\selectfont
 \frametitle{Matriz {\color{yellow}NxN} con {\color{yellow}malloc} en Arquitectura \textcolor{yellow}{X86-32} bits }
 \begin{columns}[c]
  \column{0.4\textwidth}
\lstset{basicstyle=\tiny}
\begin{tikzpicture}
  \begin{scope}[
  	every node/.style={draw, 
  	anchor=text, 
  	rectangle split,
    rectangle split parts=19, 
    rectangle split part fill={green!0,green!30,green!0,green!30,green!0,red!0,blue!30,blue!30,red!0,red!30,red!30,red!30,blue!0,red!30,red!30,red!30,red!0,red!0,red!0,red!0},
    minimum width=0.6cm}]
    \node (R) at (-0.5,8){ 
    		\nodepart{one}\tiny 1
    		\nodepart{two}\tiny 0
    		\nodepart{three}\tiny 2
    		\nodepart{four}\tiny 3
    		\nodepart{five}\tiny 0x5E1D4A80
    		\nodepart{six}{$\vdots$}
    		\nodepart{seven}\tiny 0x5E1D4AA0
    		\nodepart{eight}\tiny 0x5E1D4AC0
    		\nodepart{nine}{$\vdots$}
    		\nodepart{ten}\tiny 11
    		\nodepart{eleven}\tiny 12
    		\nodepart{twelve}\tiny 13
    		\nodepart{thirteen}{$\vdots$}
    		\nodepart{fourteen}\tiny {\color{red}21}
    		\nodepart{fifteen}
    		\nodepart{sixteen}
    		\nodepart{seventeen}{$\vdots$}
    		\nodepart{eighteen}
    		\nodepart{nineteen}};
    		
  \end{scope}
  \node at (-2,5.8) {\color{red}0x5E1D4A80};
  \node at (0.8,5.8) {\color{red}mtz[0]};
  \node at (0.8,5.45) {\color{red}mtz[1]};
  \node at (-2,5.45) {\color{red}0x5E1D4A84};
  \node at (-2,4.45) {\color{red}0x5E1D4AA0};
  \node at (0.8,4.45) {\color{red}mtz[0][0]};
  \node at (0.8,4.1) {\color{red}mtz[0][1]};
  \node at (0.8,3.75) {\color{red}mtz[0][2]};
  \node at (0.8,2.8) {\color{red}mtz[1][0]};
  \node at (-2,2.8) {\color{red}0x5E1D4AC0};
    \node at (-1.4,8.1) {\color{red}i};
  \node at (-1.4,7.75) {\color{red}j};
  \node at (-1.4,7.4) {\color{red}file};
  \node at (-1.4,7.1) {\color{red}row};
  \node at (-1.4,6.8) {\color{red}mtz};

  \node at (0,8.6) {\footnotesize\itshape\color{blue} {$ejemplo$ $para$ $matriz$ $(2$ x $3)$}};
  \fill[color=green] (3.2,4.3) -- (6.4,4.3) -- (6.4,4.0) -- (3.2,4.0) -- cycle;

  \draw[white,-](-1.9,0.6)to [out=90,in=-90]node[right,midway]{} ++(0,0.5) ; % para fijar rectangulos
\end{tikzpicture}
						
  \column{0.5\textwidth}
  \lstset{basicstyle=\tiny}
 \begin{lstlisting}
#include<stdio.h>
#include <stdlib.h>

int main (void)
{
  int file=0,row=0,i=0,j=0;
  int **mtz;

  printf("ingrese filas y columnas \n");
  scanf("% d % d",&file,&row);
  mtz=(int **)malloc(file*sizeof(int *));
  for (i=0 ; i<file ; i++){
   mtz[i]=(int *)malloc(row*sizeof(int));
  }
  for (i=0 ; i<file ; i++){
    for (j=0 ; j<row ; j++){
       printf("file% d  rows% d\n",i,j);
       scanf("% d",&mtz[i][j]);
    }
  }
  printf("\n");
  for (i=0 ; i<file ; i++){
    for (j=0 ; j<row ; j++){
       printf("% d \t",mtz[i][j]);
    }
  }
  for (i=0 ; i<file ; i++){
    free(mtz[i]);
  }
  free(mtz);
  return (0);
}
   \end{lstlisting}
 \end{columns}
\end{frame}


\begin{frame}[fragile]
\fontsize{6.5pt}{12pt}\selectfont
 \frametitle{Matriz {\color{yellow}NxN} con {\color{yellow}malloc} en Arquitectura \textcolor{yellow}{X86-32} bits }
 \begin{columns}[c]
  \column{0.4\textwidth}
\lstset{basicstyle=\tiny}
\begin{tikzpicture}
  \begin{scope}[
  	every node/.style={draw, 
  	anchor=text, 
  	rectangle split,
    rectangle split parts=19, 
    rectangle split part fill={green!0,green!30,green!0,green!0,green!0,red!0,blue!30,blue!30,red!0,red!30,red!30,red!30,blue!0,red!30,red!30,red!30,red!0,red!0,red!0,red!0},
    minimum width=0.6cm}]
    \node (R) at (-0.5,8){ 
    		\nodepart{one}\tiny 1
    		\nodepart{two}\tiny {\color{red}1}
    		\nodepart{three}\tiny 2
    		\nodepart{four}\tiny 3
    		\nodepart{five}\tiny 0x5E1D4A80
    		\nodepart{six}{$\vdots$}
    		\nodepart{seven}\tiny 0x5E1D4AA0
    		\nodepart{eight}\tiny 0x5E1D4AC0
    		\nodepart{nine}{$\vdots$}
    		\nodepart{ten}\tiny 11
    		\nodepart{eleven}\tiny 12
    		\nodepart{twelve}\tiny 13
    		\nodepart{thirteen}{$\vdots$}
    		\nodepart{fourteen}\tiny 21
    		\nodepart{fifteen}
    		\nodepart{sixteen}
    		\nodepart{seventeen}{$\vdots$}
    		\nodepart{eighteen}
    		\nodepart{nineteen}};
    		
  \end{scope}
  \node at (-2,5.8) {\color{red}0x5E1D4A80};
  \node at (0.8,5.8) {\color{red}mtz[0]};
  \node at (0.8,5.45) {\color{red}mtz[1]};
  \node at (-2,5.45) {\color{red}0x5E1D4A84};
  \node at (-2,4.45) {\color{red}0x5E1D4AA0};
  \node at (0.8,4.45) {\color{red}mtz[0][0]};
  \node at (0.8,4.1) {\color{red}mtz[0][1]};
  \node at (0.8,3.75) {\color{red}mtz[0][2]};
  \node at (0.8,2.8) {\color{red}mtz[1][0]};
  \node at (-2,2.8) {\color{red}0x5E1D4AC0};
    \node at (-1.4,8.1) {\color{red}i};
  \node at (-1.4,7.75) {\color{red}j};
  \node at (-1.4,7.4) {\color{red}file};
  \node at (-1.4,7.1) {\color{red}row};
  \node at (-1.4,6.8) {\color{red}mtz};

  \node at (0,8.6) {\footnotesize\itshape\color{blue} {$ejemplo$ $para$ $matriz$ $(2$ x $3)$}};
  \fill[color=green] (5.4,4.8) -- (5.9,4.8) -- (5.9,4.55) -- (5.4,4.55) -- cycle;

  \draw[white,-](-1.9,0.6)to [out=90,in=-90]node[right,midway]{} ++(0,0.5) ; % para fijar rectangulos
\end{tikzpicture}
						
  \column{0.5\textwidth}
  \lstset{basicstyle=\tiny}
 \begin{lstlisting}
#include<stdio.h>
#include <stdlib.h>

int main (void)
{
  int file=0,row=0,i=0,j=0;
  int **mtz;

  printf("ingrese filas y columnas \n");
  scanf("% d % d",&file,&row);
  mtz=(int **)malloc(file*sizeof(int *));
  for (i=0 ; i<file ; i++){
   mtz[i]=(int *)malloc(row*sizeof(int));
  }
  for (i=0 ; i<file ; i++){
    for (j=0 ; j<row ; j++){
       printf("file% d  rows% d\n",i,j);
       scanf("% d",&mtz[i][j]);
    }
  }
  printf("\n");
  for (i=0 ; i<file ; i++){
    for (j=0 ; j<row ; j++){
       printf("% d \t",mtz[i][j]);
    }
  }
  for (i=0 ; i<file ; i++){
    free(mtz[i]);
  }
  free(mtz);
  return (0);
}
   \end{lstlisting}
 \end{columns}
\end{frame}


\begin{frame}[fragile]
\fontsize{6.5pt}{12pt}\selectfont
 \frametitle{Matriz {\color{yellow}NxN} con {\color{yellow}malloc} en Arquitectura \textcolor{yellow}{X86-32} bits }
 \begin{columns}[c]
  \column{0.4\textwidth}
\lstset{basicstyle=\tiny}
\begin{tikzpicture}
  \begin{scope}[
  	every node/.style={draw, 
  	anchor=text, 
  	rectangle split,
    rectangle split parts=19, 
    rectangle split part fill={green!0,green!30,green!0,green!30,green!0,red!0,blue!30,blue!30,red!0,red!30,red!30,red!30,blue!0,red!30,red!30,red!30,red!0,red!0,red!0,red!0},
    minimum width=0.6cm}]
    \node (R) at (-0.5,8){ 
    		\nodepart{one}\tiny 1
    		\nodepart{two}\tiny 1
    		\nodepart{three}\tiny 2
    		\nodepart{four}\tiny 3
    		\nodepart{five}\tiny 0x5E1D4A80
    		\nodepart{six}{$\vdots$}
    		\nodepart{seven}\tiny 0x5E1D4AA0
    		\nodepart{eight}\tiny 0x5E1D4AC0
    		\nodepart{nine}{$\vdots$}
    		\nodepart{ten}\tiny 11
    		\nodepart{eleven}\tiny 12
    		\nodepart{twelve}\tiny 13
    		\nodepart{thirteen}{$\vdots$}
    		\nodepart{fourteen}\tiny 21
    		\nodepart{fifteen}
    		\nodepart{sixteen}
    		\nodepart{seventeen}{$\vdots$}
    		\nodepart{eighteen}
    		\nodepart{nineteen}};
    		
  \end{scope}
  \node at (-2,5.8) {\color{red}0x5E1D4A80};
  \node at (0.8,5.8) {\color{red}mtz[0]};
  \node at (0.8,5.45) {\color{red}mtz[1]};
  \node at (-2,5.45) {\color{red}0x5E1D4A84};
  \node at (-2,4.45) {\color{red}0x5E1D4AA0};
  \node at (0.8,4.45) {\color{red}mtz[0][0]};
  \node at (0.8,4.1) {\color{red}mtz[0][1]};
  \node at (0.8,3.75) {\color{red}mtz[0][2]};
  \node at (0.8,2.8) {\color{red}mtz[1][0]};
  \node at (-2,2.8) {\color{red}0x5E1D4AC0};
    \node at (-1.4,8.1) {\color{red}i};
  \node at (-1.4,7.75) {\color{red}j};
  \node at (-1.4,7.4) {\color{red}file};
  \node at (-1.4,7.1) {\color{red}row};
  \node at (-1.4,6.8) {\color{red}mtz};

  \node at (0,8.6) {\footnotesize\itshape\color{blue} {$ejemplo$ $para$ $matriz$ $(2$ x $3)$}};
  \fill[color=green] (4.4,4.8) -- (5.4,4.8) -- (5.4,4.55) -- (4.4,4.55) -- cycle;

  \draw[white,-](-1.9,0.6)to [out=90,in=-90]node[right,midway]{} ++(0,0.5) ; % para fijar rectangulos
\end{tikzpicture}
						
  \column{0.5\textwidth}
  \lstset{basicstyle=\tiny}
 \begin{lstlisting}
#include<stdio.h>
#include <stdlib.h>

int main (void)
{
  int file=0,row=0,i=0,j=0;
  int **mtz;

  printf("ingrese filas y columnas \n");
  scanf("% d % d",&file,&row);
  mtz=(int **)malloc(file*sizeof(int *));
  for (i=0 ; i<file ; i++){
   mtz[i]=(int *)malloc(row*sizeof(int));
  }
  for (i=0 ; i<file ; i++){
    for (j=0 ; j<row ; j++){
       printf("file% d  rows% d\n",i,j);
       scanf("% d",&mtz[i][j]);
    }
  }
  printf("\n");
  for (i=0 ; i<file ; i++){
    for (j=0 ; j<row ; j++){
       printf("% d \t",mtz[i][j]);
    }
  }
  for (i=0 ; i<file ; i++){
    free(mtz[i]);
  }
  free(mtz);
  return (0);
}
   \end{lstlisting}
 \end{columns}
\end{frame}


\begin{frame}[fragile]
\fontsize{6.5pt}{12pt}\selectfont
 \frametitle{Matriz {\color{yellow}NxN} con {\color{yellow}malloc} en Arquitectura \textcolor{yellow}{X86-32} bits }
 \begin{columns}[c]
  \column{0.4\textwidth}
\lstset{basicstyle=\tiny}
\begin{tikzpicture}
  \begin{scope}[
  	every node/.style={draw, 
  	anchor=text, 
  	rectangle split,
    rectangle split parts=19, 
    rectangle split part fill={green!0,green!0,green!0,green!0,green!0,red!0,blue!30,blue!30,red!0,red!30,red!30,red!30,blue!0,red!30,red!30,red!30,red!0,red!0,red!0,red!0},
    minimum width=0.6cm}]
    \node (R) at (-0.5,8){ 
    		\nodepart{one}\tiny 1
    		\nodepart{two}\tiny 1
    		\nodepart{three}\tiny 2
    		\nodepart{four}\tiny 3
    		\nodepart{five}\tiny 0x5E1D4A80
    		\nodepart{six}{$\vdots$}
    		\nodepart{seven}\tiny 0x5E1D4AA0
    		\nodepart{eight}\tiny 0x5E1D4AC0
    		\nodepart{nine}{$\vdots$}
    		\nodepart{ten}\tiny 11
    		\nodepart{eleven}\tiny 12
    		\nodepart{twelve}\tiny 13
    		\nodepart{thirteen}{$\vdots$}
    		\nodepart{fourteen}\tiny 21
    		\nodepart{fifteen}\tiny {\color{red}22}
    		\nodepart{sixteen}
    		\nodepart{seventeen}{$\vdots$}
    		\nodepart{eighteen}
    		\nodepart{nineteen}};
    		
  \end{scope}
  \node at (-2,5.8) {\color{red}0x5E1D4A80};
  \node at (0.8,5.8) {\color{red}mtz[0]};
  \node at (0.8,5.45) {\color{red}mtz[1]};
  \node at (-2,5.45) {\color{red}0x5E1D4A84};
  \node at (-2,4.45) {\color{red}0x5E1D4AA0};
  \node at (0.8,4.45) {\color{red}mtz[0][0]};
  \node at (0.8,4.1) {\color{red}mtz[0][1]};
  \node at (0.8,3.75) {\color{red}mtz[0][2]};
  \node at (0.8,2.8) {\color{red}mtz[1][0]};
  \node at (0.8,2.5) {\color{red}mtz[1][1]};
  \node at (-2,2.8) {\color{red}0x5E1D4AC0};
    \node at (-1.4,8.1) {\color{red}i};
  \node at (-1.4,7.75) {\color{red}j};
  \node at (-1.4,7.4) {\color{red}file};
  \node at (-1.4,7.1) {\color{red}row};
  \node at (-1.4,6.8) {\color{red}mtz};

  \node at (0,8.6) {\footnotesize\itshape\color{blue} {$ejemplo$ $para$ $matriz$ $(2$ x $3)$}};
  \fill[color=green] (3.2,4.3) -- (6.4,4.3) -- (6.4,4.0) -- (3.2,4.0) -- cycle;

  \draw[white,-](-1.9,0.6)to [out=90,in=-90]node[right,midway]{} ++(0,0.5) ; % para fijar rectangulos
\end{tikzpicture}
						
  \column{0.5\textwidth}
  \lstset{basicstyle=\tiny}
 \begin{lstlisting}
#include<stdio.h>
#include <stdlib.h>

int main (void)
{
  int file=0,row=0,i=0,j=0;
  int **mtz;

  printf("ingrese filas y columnas \n");
  scanf("% d % d",&file,&row);
  mtz=(int **)malloc(file*sizeof(int *));
  for (i=0 ; i<file ; i++){
   mtz[i]=(int *)malloc(row*sizeof(int));
  }
  for (i=0 ; i<file ; i++){
    for (j=0 ; j<row ; j++){
       printf("file% d  rows% d\n",i,j);
       scanf("% d",&mtz[i][j]);
    }
  }
  printf("\n");
  for (i=0 ; i<file ; i++){
    for (j=0 ; j<row ; j++){
       printf("% d \t",mtz[i][j]);
    }
  }
  for (i=0 ; i<file ; i++){
    free(mtz[i]);
  }
  free(mtz);
  return (0);
}
   \end{lstlisting}
 \end{columns}
\end{frame}


\begin{frame}[fragile]
\fontsize{6.5pt}{12pt}\selectfont
 \frametitle{Matriz {\color{yellow}NxN} con {\color{yellow}malloc} en Arquitectura \textcolor{yellow}{X86-32} bits }
 \begin{columns}[c]
  \column{0.4\textwidth}
\lstset{basicstyle=\tiny}
\begin{tikzpicture}
  \begin{scope}[
  	every node/.style={draw, 
  	anchor=text, 
  	rectangle split,
    rectangle split parts=19, 
    rectangle split part fill={green!0,green!30,green!0,green!0,green!0,red!0,blue!30,blue!30,red!0,red!30,red!30,red!30,blue!0,red!30,red!30,red!30,red!0,red!0,red!0,red!0},
    minimum width=0.6cm}]
    \node (R) at (-0.5,8){ 
    		\nodepart{one}\tiny 1
    		\nodepart{two}\tiny {\color{red}2}
    		\nodepart{three}\tiny 2
    		\nodepart{four}\tiny 3
    		\nodepart{five}\tiny 0x5E1D4A80
    		\nodepart{six}{$\vdots$}
    		\nodepart{seven}\tiny 0x5E1D4AA0
    		\nodepart{eight}\tiny 0x5E1D4AC0
    		\nodepart{nine}{$\vdots$}
    		\nodepart{ten}\tiny 11
    		\nodepart{eleven}\tiny 12
    		\nodepart{twelve}\tiny 13
    		\nodepart{thirteen}{$\vdots$}
    		\nodepart{fourteen}\tiny 21
    		\nodepart{fifteen}\tiny 22
    		\nodepart{sixteen}
    		\nodepart{seventeen}{$\vdots$}
    		\nodepart{eighteen}
    		\nodepart{nineteen}};
    		
  \end{scope}
  \node at (-2,5.8) {\color{red}0x5E1D4A80};
  \node at (0.8,5.8) {\color{red}mtz[0]};
  \node at (0.8,5.45) {\color{red}mtz[1]};
  \node at (-2,5.45) {\color{red}0x5E1D4A84};
  \node at (-2,4.45) {\color{red}0x5E1D4AA0};
  \node at (0.8,4.45) {\color{red}mtz[0][0]};
  \node at (0.8,4.1) {\color{red}mtz[0][1]};
  \node at (0.8,3.75) {\color{red}mtz[0][2]};
  \node at (0.8,2.8) {\color{red}mtz[1][0]};
  \node at (0.8,2.5) {\color{red}mtz[1][1]};
  \node at (-2,2.8) {\color{red}0x5E1D4AC0};
    \node at (-1.4,8.1) {\color{red}i};
  \node at (-1.4,7.75) {\color{red}j};
  \node at (-1.4,7.4) {\color{red}file};
  \node at (-1.4,7.1) {\color{red}row};
  \node at (-1.4,6.8) {\color{red}mtz};

  \node at (0,8.6) {\footnotesize\itshape\color{blue} {$ejemplo$ $para$ $matriz$ $(2$ x $3)$}};
  \fill[color=green] (5.4,4.8) -- (5.9,4.8) -- (5.9,4.55) -- (5.4,4.55) -- cycle;

  \draw[white,-](-1.9,0.6)to [out=90,in=-90]node[right,midway]{} ++(0,0.5) ; % para fijar rectangulos
\end{tikzpicture}
						
  \column{0.5\textwidth}
  \lstset{basicstyle=\tiny}
 \begin{lstlisting}
#include<stdio.h>
#include <stdlib.h>

int main (void)
{
  int file=0,row=0,i=0,j=0;
  int **mtz;

  printf("ingrese filas y columnas \n");
  scanf("% d % d",&file,&row);
  mtz=(int **)malloc(file*sizeof(int *));
  for (i=0 ; i<file ; i++){
   mtz[i]=(int *)malloc(row*sizeof(int));
  }
  for (i=0 ; i<file ; i++){
    for (j=0 ; j<row ; j++){
       printf("file% d  rows% d\n",i,j);
       scanf("% d",&mtz[i][j]);
    }
  }
  printf("\n");
  for (i=0 ; i<file ; i++){
    for (j=0 ; j<row ; j++){
       printf("% d \t",mtz[i][j]);
    }
  }
  for (i=0 ; i<file ; i++){
    free(mtz[i]);
  }
  free(mtz);
  return (0);
}
   \end{lstlisting}
 \end{columns}
\end{frame}


\begin{frame}[fragile]
\fontsize{6.5pt}{12pt}\selectfont
 \frametitle{Matriz {\color{yellow}NxN} con {\color{yellow}malloc} en Arquitectura \textcolor{yellow}{X86-32} bits }
 \begin{columns}[c]
  \column{0.4\textwidth}
\lstset{basicstyle=\tiny}
\begin{tikzpicture}
  \begin{scope}[
  	every node/.style={draw, 
  	anchor=text, 
  	rectangle split,
    rectangle split parts=19, 
    rectangle split part fill={green!0,green!30,green!0,green!30,green!0,red!0,blue!30,blue!30,red!0,red!30,red!30,red!30,blue!0,red!30,red!30,red!30,red!0,red!0,red!0,red!0},
    minimum width=0.6cm}]
    \node (R) at (-0.5,8){ 
    		\nodepart{one}\tiny 1
    		\nodepart{two}\tiny 2
    		\nodepart{three}\tiny 2
    		\nodepart{four}\tiny 3
    		\nodepart{five}\tiny 0x5E1D4A80
    		\nodepart{six}{$\vdots$}
    		\nodepart{seven}\tiny 0x5E1D4AA0
    		\nodepart{eight}\tiny 0x5E1D4AC0
    		\nodepart{nine}{$\vdots$}
    		\nodepart{ten}\tiny 11
    		\nodepart{eleven}\tiny 12
    		\nodepart{twelve}\tiny 13
    		\nodepart{thirteen}{$\vdots$}
    		\nodepart{fourteen}\tiny 21
    		\nodepart{fifteen}\tiny 22
    		\nodepart{sixteen}
    		\nodepart{seventeen}{$\vdots$}
    		\nodepart{eighteen}
    		\nodepart{nineteen}};
    		
  \end{scope}
  \node at (-2,5.8) {\color{red}0x5E1D4A80};
  \node at (0.8,5.8) {\color{red}mtz[0]};
  \node at (0.8,5.45) {\color{red}mtz[1]};
  \node at (-2,5.45) {\color{red}0x5E1D4A84};
  \node at (-2,4.45) {\color{red}0x5E1D4AA0};
  \node at (0.8,4.45) {\color{red}mtz[0][0]};
  \node at (0.8,4.1) {\color{red}mtz[0][1]};
  \node at (0.8,3.75) {\color{red}mtz[0][2]};
  \node at (0.8,2.8) {\color{red}mtz[1][0]};
  \node at (0.8,2.5) {\color{red}mtz[1][1]};
  \node at (-2,2.8) {\color{red}0x5E1D4AC0};
    \node at (-1.4,8.1) {\color{red}i};
  \node at (-1.4,7.75) {\color{red}j};
  \node at (-1.4,7.4) {\color{red}file};
  \node at (-1.4,7.1) {\color{red}row};
  \node at (-1.4,6.8) {\color{red}mtz};

  \node at (0,8.6) {\footnotesize\itshape\color{blue} {$ejemplo$ $para$ $matriz$ $(2$ x $3)$}};
  \fill[color=green] (4.4,4.8) -- (5.4,4.8) -- (5.4,4.55) -- (4.4,4.55) -- cycle;

  \draw[white,-](-1.9,0.6)to [out=90,in=-90]node[right,midway]{} ++(0,0.5) ; % para fijar rectangulos
\end{tikzpicture}
						
  \column{0.5\textwidth}
  \lstset{basicstyle=\tiny}
 \begin{lstlisting}
#include<stdio.h>
#include <stdlib.h>

int main (void)
{
  int file=0,row=0,i=0,j=0;
  int **mtz;

  printf("ingrese filas y columnas \n");
  scanf("% d % d",&file,&row);
  mtz=(int **)malloc(file*sizeof(int *));
  for (i=0 ; i<file ; i++){
   mtz[i]=(int *)malloc(row*sizeof(int));
  }
  for (i=0 ; i<file ; i++){
    for (j=0 ; j<row ; j++){
       printf("file% d  rows% d\n",i,j);
       scanf("% d",&mtz[i][j]);
    }
  }
  printf("\n");
  for (i=0 ; i<file ; i++){
    for (j=0 ; j<row ; j++){
       printf("% d \t",mtz[i][j]);
    }
  }
  for (i=0 ; i<file ; i++){
    free(mtz[i]);
  }
  free(mtz);
  return (0);
}
   \end{lstlisting}
 \end{columns}
\end{frame}


\begin{frame}[fragile]
\fontsize{6.5pt}{12pt}\selectfont
 \frametitle{Matriz {\color{yellow}NxN} con {\color{yellow}malloc} en Arquitectura \textcolor{yellow}{X86-32} bits }
 \begin{columns}[c]
  \column{0.4\textwidth}
\lstset{basicstyle=\tiny}
\begin{tikzpicture}
  \begin{scope}[
  	every node/.style={draw, 
  	anchor=text, 
  	rectangle split,
    rectangle split parts=19, 
    rectangle split part fill={green!0,green!0,green!0,green!0,green!0,red!0,blue!30,blue!30,red!0,red!30,red!30,red!30,blue!0,red!30,red!30,red!30,red!0,red!0,red!0,red!0},
    minimum width=0.6cm}]
    \node (R) at (-0.5,8){ 
    		\nodepart{one}\tiny 1
    		\nodepart{two}\tiny 2
    		\nodepart{three}\tiny 2
    		\nodepart{four}\tiny 3
    		\nodepart{five}\tiny 0x5E1D4A80
    		\nodepart{six}{$\vdots$}
    		\nodepart{seven}\tiny 0x5E1D4AA0
    		\nodepart{eight}\tiny 0x5E1D4AC0
    		\nodepart{nine}{$\vdots$}
    		\nodepart{ten}\tiny 11
    		\nodepart{eleven}\tiny 12
    		\nodepart{twelve}\tiny 13
    		\nodepart{thirteen}{$\vdots$}
    		\nodepart{fourteen}\tiny 21
    		\nodepart{fifteen}\tiny 22
    		\nodepart{sixteen}\tiny {\color{red}23}
    		\nodepart{seventeen}{$\vdots$}
    		\nodepart{eighteen}
    		\nodepart{nineteen}};
    		
  \end{scope}
  \node at (-2,5.8) {\color{red}0x5E1D4A80};
  \node at (0.8,5.8) {\color{red}mtz[0]};
  \node at (0.8,5.45) {\color{red}mtz[1]};
  \node at (-2,5.45) {\color{red}0x5E1D4A84};
  \node at (-2,4.45) {\color{red}0x5E1D4AA0};
  \node at (0.8,4.45) {\color{red}mtz[0][0]};
  \node at (0.8,4.1) {\color{red}mtz[0][1]};
  \node at (0.8,3.75) {\color{red}mtz[0][2]};
  \node at (0.8,2.8) {\color{red}mtz[1][0]};
  \node at (0.8,2.5) {\color{red}mtz[1][1]};
  \node at (0.8,2.2) {\color{red}mtz[1][2]};
  \node at (-2,2.8) {\color{red}0x5E1D4AC0};
    \node at (-1.4,8.1) {\color{red}i};
  \node at (-1.4,7.75) {\color{red}j};
  \node at (-1.4,7.4) {\color{red}file};
  \node at (-1.4,7.1) {\color{red}row};
  \node at (-1.4,6.8) {\color{red}mtz};

  \node at (0,8.6) {\footnotesize\itshape\color{blue} {$ejemplo$ $para$ $matriz$ $(2$ x $3)$}};
  \fill[color=green] (3.2,4.3) -- (6.4,4.3) -- (6.4,4.0) -- (3.2,4.0) -- cycle;

  \draw[white,-](-1.9,0.6)to [out=90,in=-90]node[right,midway]{} ++(0,0.5) ; % para fijar rectangulos
\end{tikzpicture}
						
  \column{0.5\textwidth}
  \lstset{basicstyle=\tiny}
 \begin{lstlisting}
#include<stdio.h>
#include <stdlib.h>

int main (void)
{
  int file=0,row=0,i=0,j=0;
  int **mtz;

  printf("ingrese filas y columnas \n");
  scanf("% d % d",&file,&row);
  mtz=(int **)malloc(file*sizeof(int *));
  for (i=0 ; i<file ; i++){
   mtz[i]=(int *)malloc(row*sizeof(int));
  }
  for (i=0 ; i<file ; i++){
    for (j=0 ; j<row ; j++){
       printf("file% d  rows% d\n",i,j);
       scanf("% d",&mtz[i][j]);
    }
  }
  printf("\n");
  for (i=0 ; i<file ; i++){
    for (j=0 ; j<row ; j++){
       printf("% d \t",mtz[i][j]);
    }
  }
  for (i=0 ; i<file ; i++){
    free(mtz[i]);
  }
  free(mtz);
  return (0);
}
   \end{lstlisting}
 \end{columns}
\end{frame}


\begin{frame}[fragile]
\fontsize{6.5pt}{12pt}\selectfont
 \frametitle{Matriz {\color{yellow}NxN} con {\color{yellow}malloc} en Arquitectura \textcolor{yellow}{X86-32} bits }
 \begin{columns}[c]
  \column{0.4\textwidth}
\lstset{basicstyle=\tiny}
\begin{tikzpicture}
  \begin{scope}[
  	every node/.style={draw, 
  	anchor=text, 
  	rectangle split,
    rectangle split parts=19, 
    rectangle split part fill={green!0,green!0,green!0,green!0,green!0,red!0,blue!30,blue!30,red!0,red!30,red!30,red!30,blue!0,red!30,red!30,red!30,red!0,red!0,red!0,red!0},
    minimum width=0.6cm}]
    \node (R) at (-0.5,8){ 
    		\nodepart{one}\tiny 1
    		\nodepart{two}\tiny {\color{red}3}
    		\nodepart{three}\tiny 2
    		\nodepart{four}\tiny 3
    		\nodepart{five}\tiny 0x5E1D4A80
    		\nodepart{six}{$\vdots$}
    		\nodepart{seven}\tiny 0x5E1D4AA0
    		\nodepart{eight}\tiny 0x5E1D4AC0
    		\nodepart{nine}{$\vdots$}
    		\nodepart{ten}\tiny 11
    		\nodepart{eleven}\tiny 12
    		\nodepart{twelve}\tiny 13
    		\nodepart{thirteen}{$\vdots$}
    		\nodepart{fourteen}\tiny 21
    		\nodepart{fifteen}\tiny 22
    		\nodepart{sixteen}\tiny 23
    		\nodepart{seventeen}{$\vdots$}
    		\nodepart{eighteen}
    		\nodepart{nineteen}};
    		
  \end{scope}
  \node at (-2,5.8) {\color{red}0x5E1D4A80};
  \node at (0.8,5.8) {\color{red}mtz[0]};
  \node at (0.8,5.45) {\color{red}mtz[1]};
  \node at (-2,5.45) {\color{red}0x5E1D4A84};
  \node at (-2,4.45) {\color{red}0x5E1D4AA0};
  \node at (0.8,4.45) {\color{red}mtz[0][0]};
  \node at (0.8,4.1) {\color{red}mtz[0][1]};
  \node at (0.8,3.75) {\color{red}mtz[0][2]};
  \node at (0.8,2.8) {\color{red}mtz[1][0]};
  \node at (0.8,2.5) {\color{red}mtz[1][1]};
  \node at (0.8,2.2) {\color{red}mtz[1][2]};
  \node at (-2,2.8) {\color{red}0x5E1D4AC0};
    \node at (-1.4,8.1) {\color{red}i};
  \node at (-1.4,7.75) {\color{red}j};
  \node at (-1.4,7.4) {\color{red}file};
  \node at (-1.4,7.1) {\color{red}row};
  \node at (-1.4,6.8) {\color{red}mtz};

  \node at (0,8.6) {\footnotesize\itshape\color{blue} {$ejemplo$ $para$ $matriz$ $(2$ x $3)$}};
  \fill[color=green] (5.4,4.8) -- (5.9,4.8) -- (5.9,4.55) -- (5.4,4.55) -- cycle;

  \draw[white,-](-1.9,0.6)to [out=90,in=-90]node[right,midway]{} ++(0,0.5) ; % para fijar rectangulos
\end{tikzpicture}
						
  \column{0.5\textwidth}
  \lstset{basicstyle=\tiny}
 \begin{lstlisting}
#include<stdio.h>
#include <stdlib.h>

int main (void)
{
  int file=0,row=0,i=0,j=0;
  int **mtz;

  printf("ingrese filas y columnas \n");
  scanf("% d % d",&file,&row);
  mtz=(int **)malloc(file*sizeof(int *));
  for (i=0 ; i<file ; i++){
   mtz[i]=(int *)malloc(row*sizeof(int));
  }
  for (i=0 ; i<file ; i++){
    for (j=0 ; j<row ; j++){
       printf("file% d  rows% d\n",i,j);
       scanf("% d",&mtz[i][j]);
    }
  }
  printf("\n");
  for (i=0 ; i<file ; i++){
    for (j=0 ; j<row ; j++){
       printf("% d \t",mtz[i][j]);
    }
  }
  for (i=0 ; i<file ; i++){
    free(mtz[i]);
  }
  free(mtz);
  return (0);
}
   \end{lstlisting}
 \end{columns}
\end{frame}


\begin{frame}[fragile]
\fontsize{6.5pt}{12pt}\selectfont
 \frametitle{Matriz {\color{yellow}NxN} con {\color{yellow}malloc} en Arquitectura \textcolor{yellow}{X86-32} bits }
 \begin{columns}[c]
  \column{0.4\textwidth}
\lstset{basicstyle=\tiny}
\begin{tikzpicture}
  \begin{scope}[
  	every node/.style={draw, 
  	anchor=text, 
  	rectangle split,
    rectangle split parts=19, 
    rectangle split part fill={green!0,green!30,green!0,green!30,green!0,red!0,blue!30,blue!30,red!0,red!30,red!30,red!30,blue!0,red!30,red!30,red!30,red!0,red!0,red!0,red!0},
    minimum width=0.6cm}]
    \node (R) at (-0.5,8){ 
    		\nodepart{one}\tiny 1
    		\nodepart{two}\tiny 3
    		\nodepart{three}\tiny 2
    		\nodepart{four}\tiny 3
    		\nodepart{five}\tiny 0x5E1D4A80
    		\nodepart{six}{$\vdots$}
    		\nodepart{seven}\tiny 0x5E1D4AA0
    		\nodepart{eight}\tiny 0x5E1D4AC0
    		\nodepart{nine}{$\vdots$}
    		\nodepart{ten}\tiny 11
    		\nodepart{eleven}\tiny 12
    		\nodepart{twelve}\tiny 13
    		\nodepart{thirteen}{$\vdots$}
    		\nodepart{fourteen}\tiny 21
    		\nodepart{fifteen}\tiny 22
    		\nodepart{sixteen}\tiny 23
    		\nodepart{seventeen}{$\vdots$}
    		\nodepart{eighteen}
    		\nodepart{nineteen}};
    		
  \end{scope}
  \node at (-2,5.8) {\color{red}0x5E1D4A80};
  \node at (0.8,5.8) {\color{red}mtz[0]};
  \node at (0.8,5.45) {\color{red}mtz[1]};
  \node at (-2,5.45) {\color{red}0x5E1D4A84};
  \node at (-2,4.45) {\color{red}0x5E1D4AA0};
  \node at (0.8,4.45) {\color{red}mtz[0][0]};
  \node at (0.8,4.1) {\color{red}mtz[0][1]};
  \node at (0.8,3.75) {\color{red}mtz[0][2]};
  \node at (0.8,2.8) {\color{red}mtz[1][0]};
  \node at (0.8,2.5) {\color{red}mtz[1][1]};
  \node at (0.8,2.2) {\color{red}mtz[1][2]};
  \node at (-2,2.8) {\color{red}0x5E1D4AC0};
    \node at (-1.4,8.1) {\color{red}i};
  \node at (-1.4,7.75) {\color{red}j};
  \node at (-1.4,7.4) {\color{red}file};
  \node at (-1.4,7.1) {\color{red}row};
  \node at (-1.4,6.8) {\color{red}mtz};

  \node at (0,8.6) {\footnotesize\itshape\color{blue} {$ejemplo$ $para$ $matriz$ $(2$ x $3)$}};
  \fill[color=green] (4.4,4.8) -- (5.4,4.8) -- (5.4,4.55) -- (4.4,4.55) -- cycle;


  \node [fill=green] at (0.2,4.8) {\fbox{
						\begin{minipage}{3cm}
							\fontsize{8pt}{8pt}\selectfont
		 {No cumple la condicion} \\
		  \\
		 {y sale del ciclo for} \\
						\end{minipage}}};						  

  \draw[white,-](-1.9,0.6)to [out=90,in=-90]node[right,midway]{} ++(0,0.5) ; % para fijar rectangulos
\end{tikzpicture}
						
  \column{0.5\textwidth}
  \lstset{basicstyle=\tiny}
 \begin{lstlisting}
#include<stdio.h>
#include <stdlib.h>

int main (void)
{
  int file=0,row=0,i=0,j=0;
  int **mtz;

  printf("ingrese filas y columnas \n");
  scanf("% d % d",&file,&row);
  mtz=(int **)malloc(file*sizeof(int *));
  for (i=0 ; i<file ; i++){
   mtz[i]=(int *)malloc(row*sizeof(int));
  }
  for (i=0 ; i<file ; i++){
    for (j=0 ; j<row ; j++){
       printf("file% d  rows% d\n",i,j);
       scanf("% d",&mtz[i][j]);
    }
  }
  printf("\n");
  for (i=0 ; i<file ; i++){
    for (j=0 ; j<row ; j++){
       printf("% d \t",mtz[i][j]);
    }
  }
  for (i=0 ; i<file ; i++){
    free(mtz[i]);
  }
  free(mtz);
  return (0);
}
   \end{lstlisting}
 \end{columns}
\end{frame}


\begin{frame}[fragile]
\fontsize{6.5pt}{12pt}\selectfont
 \frametitle{Matriz {\color{yellow}NxN} con {\color{yellow}malloc} en Arquitectura \textcolor{yellow}{X86-32} bits }
 \begin{columns}[c]
  \column{0.4\textwidth}
\lstset{basicstyle=\tiny}
\begin{tikzpicture}
  \begin{scope}[
  	every node/.style={draw, 
  	anchor=text, 
  	rectangle split,
    rectangle split parts=19, 
    rectangle split part fill={green!30,green!0,green!0,green!0,green!0,red!0,blue!30,blue!30,red!0,red!30,red!30,red!30,blue!0,red!30,red!30,red!30,red!0,red!0,red!0,red!0},
    minimum width=0.6cm}]
    \node (R) at (-0.5,8){ 
    		\nodepart{one}\tiny {\color{red}2}
    		\nodepart{two}\tiny 3
    		\nodepart{three}\tiny 2
    		\nodepart{four}\tiny 3
    		\nodepart{five}\tiny 0x5E1D4A80
    		\nodepart{six}{$\vdots$}
    		\nodepart{seven}\tiny 0x5E1D4AA0
    		\nodepart{eight}\tiny 0x5E1D4AC0
    		\nodepart{nine}{$\vdots$}
    		\nodepart{ten}\tiny 11
    		\nodepart{eleven}\tiny 12
    		\nodepart{twelve}\tiny 13
    		\nodepart{thirteen}{$\vdots$}
    		\nodepart{fourteen}\tiny 21
    		\nodepart{fifteen}\tiny 22
    		\nodepart{sixteen}\tiny 23
    		\nodepart{seventeen}{$\vdots$}
    		\nodepart{eighteen}
    		\nodepart{nineteen}};
    		
  \end{scope}
  \node at (-2,5.8) {\color{red}0x5E1D4A80};
  \node at (0.8,5.8) {\color{red}mtz[0]};
  \node at (0.8,5.45) {\color{red}mtz[1]};
  \node at (-2,5.45) {\color{red}0x5E1D4A84};
  \node at (-2,4.45) {\color{red}0x5E1D4AA0};
  \node at (0.8,4.45) {\color{red}mtz[0][0]};
  \node at (0.8,4.1) {\color{red}mtz[0][1]};
  \node at (0.8,3.75) {\color{red}mtz[0][2]};
  \node at (0.8,2.8) {\color{red}mtz[1][0]};
  \node at (0.8,2.5) {\color{red}mtz[1][1]};
  \node at (0.8,2.2) {\color{red}mtz[1][2]};
  \node at (-2,2.8) {\color{red}0x5E1D4AC0};
    \node at (-1.4,8.1) {\color{red}i};
  \node at (-1.4,7.75) {\color{red}j};
  \node at (-1.4,7.4) {\color{red}file};
  \node at (-1.4,7.1) {\color{red}row};
  \node at (-1.4,6.8) {\color{red}mtz};

  \node at (0,8.6) {\footnotesize\itshape\color{blue} {$ejemplo$ $para$ $matriz$ $(2$ x $3)$}};
  \fill[color=green] (5.3,5.1) -- (5.8,5.1) -- (5.8,4.8) -- (5.3,4.8) -- cycle;



  \draw[white,-](-1.9,0.6)to [out=90,in=-90]node[right,midway]{} ++(0,0.5) ; % para fijar rectangulos
\end{tikzpicture}
						
  \column{0.5\textwidth}
  \lstset{basicstyle=\tiny}
 \begin{lstlisting}
#include<stdio.h>
#include <stdlib.h>

int main (void)
{
  int file=0,row=0,i=0,j=0;
  int **mtz;

  printf("ingrese filas y columnas \n");
  scanf("% d % d",&file,&row);
  mtz=(int **)malloc(file*sizeof(int *));
  for (i=0 ; i<file ; i++){
   mtz[i]=(int *)malloc(row*sizeof(int));
  }
  for (i=0 ; i<file ; i++){
    for (j=0 ; j<row ; j++){
       printf("file% d  rows% d\n",i,j);
       scanf("% d",&mtz[i][j]);
    }
  }
  printf("\n");
  for (i=0 ; i<file ; i++){
    for (j=0 ; j<row ; j++){
       printf("% d \t",mtz[i][j]);
    }
  }
  for (i=0 ; i<file ; i++){
    free(mtz[i]);
  }
  free(mtz);
  return (0);
}
   \end{lstlisting}
 \end{columns}
\end{frame}


\begin{frame}[fragile]
\fontsize{6.5pt}{12pt}\selectfont
 \frametitle{Matriz {\color{yellow}NxN} con {\color{yellow}malloc} en Arquitectura \textcolor{yellow}{X86-32} bits }
 \begin{columns}[c]
  \column{0.4\textwidth}
\lstset{basicstyle=\tiny}
\begin{tikzpicture}
  \begin{scope}[
  	every node/.style={draw, 
  	anchor=text, 
  	rectangle split,
    rectangle split parts=19, 
    rectangle split part fill={green!30,green!0,green!30,green!0,green!0,red!0,blue!30,blue!30,red!0,red!30,red!30,red!30,blue!0,red!30,red!30,red!30,red!0,red!0,red!0,red!0},
    minimum width=0.6cm}]
    \node (R) at (-0.5,8){ 
    		\nodepart{one}\tiny 2
    		\nodepart{two}\tiny 3
    		\nodepart{three}\tiny 2
    		\nodepart{four}\tiny 3
    		\nodepart{five}\tiny 0x5E1D4A80
    		\nodepart{six}{$\vdots$}
    		\nodepart{seven}\tiny 0x5E1D4AA0
    		\nodepart{eight}\tiny 0x5E1D4AC0
    		\nodepart{nine}{$\vdots$}
    		\nodepart{ten}\tiny 11
    		\nodepart{eleven}\tiny 12
    		\nodepart{twelve}\tiny 13
    		\nodepart{thirteen}{$\vdots$}
    		\nodepart{fourteen}\tiny 21
    		\nodepart{fifteen}\tiny 22
    		\nodepart{sixteen}\tiny 23
    		\nodepart{seventeen}{$\vdots$}
    		\nodepart{eighteen}
    		\nodepart{nineteen}};
    		
  \end{scope}
  \node at (-2,5.8) {\color{red}0x5E1D4A80};
  \node at (0.8,5.8) {\color{red}mtz[0]};
  \node at (0.8,5.45) {\color{red}mtz[1]};
  \node at (-2,5.45) {\color{red}0x5E1D4A84};
  \node at (-2,4.45) {\color{red}0x5E1D4AA0};
  \node at (0.8,4.45) {\color{red}mtz[0][0]};
  \node at (0.8,4.1) {\color{red}mtz[0][1]};
  \node at (0.8,3.75) {\color{red}mtz[0][2]};
  \node at (0.8,2.8) {\color{red}mtz[1][0]};
  \node at (0.8,2.5) {\color{red}mtz[1][1]};
  \node at (0.8,2.2) {\color{red}mtz[1][2]};
  \node at (-2,2.8) {\color{red}0x5E1D4AC0};
    \node at (-1.4,8.1) {\color{red}i};
  \node at (-1.4,7.75) {\color{red}j};
  \node at (-1.4,7.4) {\color{red}file};
  \node at (-1.4,7.1) {\color{red}row};
  \node at (-1.4,6.8) {\color{red}mtz};

  \node at (0,8.6) {\footnotesize\itshape\color{blue} {$ejemplo$ $para$ $matriz$ $(2$ x $3)$}};
  \fill[color=green] (4.2,5.1) -- (5,5.1) -- (5,4.8) -- (4.2,4.8) -- cycle;

  \node [fill=green] at (0.2,4.8) {\fbox{
						\begin{minipage}{3.3cm}
							\fontsize{8pt}{8pt}\selectfont
		 {No cumple la condicion y} \\
		  \\
		 {sale del ciclo for de {\color{blue}i}} \\
						\end{minipage}}};						  



  \draw[white,-](-1.9,0.6)to [out=90,in=-90]node[right,midway]{} ++(0,0.5) ; % para fijar rectangulos
\end{tikzpicture}
						
  \column{0.5\textwidth}
  \lstset{basicstyle=\tiny}
 \begin{lstlisting}
#include<stdio.h>
#include <stdlib.h>

int main (void)
{
  int file=0,row=0,i=0,j=0;
  int **mtz;

  printf("ingrese filas y columnas \n");
  scanf("% d % d",&file,&row);
  mtz=(int **)malloc(file*sizeof(int *));
  for (i=0 ; i<file ; i++){
   mtz[i]=(int *)malloc(row*sizeof(int));
  }
  for (i=0 ; i<file ; i++){
    for (j=0 ; j<row ; j++){
       printf("file% d  rows% d\n",i,j);
       scanf("% d",&mtz[i][j]);
    }
  }
  printf("\n");
  for (i=0 ; i<file ; i++){
    for (j=0 ; j<row ; j++){
       printf("% d \t",mtz[i][j]);
    }
  }
  for (i=0 ; i<file ; i++){
    free(mtz[i]);
  }
  free(mtz);
  return (0);
}
   \end{lstlisting}
 \end{columns}
\end{frame}


\begin{frame}[fragile]
\fontsize{6.5pt}{12pt}\selectfont
 \frametitle{Matriz {\color{yellow}NxN} con {\color{yellow}malloc} en Arquitectura \textcolor{yellow}{X86-32} bits }
 \begin{columns}[c]
  \column{0.4\textwidth}
\lstset{basicstyle=\tiny}
\begin{tikzpicture}
  \begin{scope}[
  	every node/.style={draw, 
  	anchor=text, 
  	rectangle split,
    rectangle split parts=19, 
    rectangle split part fill={green!30,green!0,green!30,green!0,green!0,red!0,blue!30,blue!30,red!0,red!30,red!30,red!30,blue!0,red!30,red!30,red!30,red!0,red!0,red!0,red!0},
    minimum width=0.6cm}]
    \node (R) at (-0.5,8){ 
    		\nodepart{one}\tiny 2
    		\nodepart{two}\tiny 3
    		\nodepart{three}\tiny 2
    		\nodepart{four}\tiny 3
    		\nodepart{five}\tiny 0x5E1D4A80
    		\nodepart{six}{$\vdots$}
    		\nodepart{seven}\tiny 0x5E1D4AA0
    		\nodepart{eight}\tiny 0x5E1D4AC0
    		\nodepart{nine}{$\vdots$}
    		\nodepart{ten}\tiny 11
    		\nodepart{eleven}\tiny 12
    		\nodepart{twelve}\tiny 13
    		\nodepart{thirteen}{$\vdots$}
    		\nodepart{fourteen}\tiny 21
    		\nodepart{fifteen}\tiny 22
    		\nodepart{sixteen}\tiny 23
    		\nodepart{seventeen}{$\vdots$}
    		\nodepart{eighteen}
    		\nodepart{nineteen}};
    		
  \end{scope}
  \node at (-2,5.8) {\color{red}0x5E1D4A80};
  \node at (0.8,5.8) {\color{red}mtz[0]};
  \node at (0.8,5.45) {\color{red}mtz[1]};
  \node at (-2,5.45) {\color{red}0x5E1D4A84};
  \node at (-2,4.45) {\color{red}0x5E1D4AA0};
  \node at (0.8,4.45) {\color{red}mtz[0][0]};
  \node at (0.8,4.1) {\color{red}mtz[0][1]};
  \node at (0.8,3.75) {\color{red}mtz[0][2]};
  \node at (0.8,2.8) {\color{red}mtz[1][0]};
  \node at (0.8,2.5) {\color{red}mtz[1][1]};
  \node at (0.8,2.2) {\color{red}mtz[1][2]};
  \node at (-2,2.8) {\color{red}0x5E1D4AC0};
    \node at (-1.4,8.1) {\color{red}i};
  \node at (-1.4,7.75) {\color{red}j};
  \node at (-1.4,7.4) {\color{red}file};
  \node at (-1.4,7.1) {\color{red}row};
  \node at (-1.4,6.8) {\color{red}mtz};

  \node at (0,8.6) {\footnotesize\itshape\color{blue} {$ejemplo$ $para$ $matriz$ $(2$ x $3)$}};
  \fill[color=green] (2.7,3.4) -- (7,3.4) -- (7,2.2) -- (2.7,2.2) -- cycle;

  \node [fill=green] at (0.2,1.4) {\fbox{
						\begin{minipage}{2.4cm}
							\fontsize{8pt}{8pt}\selectfont
\begin{equation*}
\begin{bmatrix}
\color{blue}11  & \color{blue}12 & \color{blue}13 \\
\color{blue}21 & \color{blue}22 & \color{blue}23 
\end{bmatrix} 
\end{equation*}
						\end{minipage}}};						  



  \draw[white,-](-1.9,0.6)to [out=90,in=-90]node[right,midway]{} ++(0,0.5) ; % para fijar rectangulos
\end{tikzpicture}
						
  \column{0.5\textwidth}
  \lstset{basicstyle=\tiny}
 \begin{lstlisting}
#include<stdio.h>
#include <stdlib.h>

int main (void)
{
  int file=0,row=0,i=0,j=0;
  int **mtz;

  printf("ingrese filas y columnas \n");
  scanf("% d % d",&file,&row);
  mtz=(int **)malloc(file*sizeof(int *));
  for (i=0 ; i<file ; i++){
   mtz[i]=(int *)malloc(row*sizeof(int));
  }
  for (i=0 ; i<file ; i++){
    for (j=0 ; j<row ; j++){
       printf("file% d  rows% d\n",i,j);
       scanf("% d",&mtz[i][j]);
    }
  }
  printf("\n");
  for (i=0 ; i<file ; i++){
    for (j=0 ; j<row ; j++){
       printf("% d \t",mtz[i][j]);
    }
  }
  for (i=0 ; i<file ; i++){
    free(mtz[i]);
  }
  free(mtz);
  return (0);
}
   \end{lstlisting}
 \end{columns}
\end{frame}


\begin{frame}[fragile]
\fontsize{6.5pt}{12pt}\selectfont
 \frametitle{Matriz {\color{yellow}NxN} con {\color{yellow}malloc} en Arquitectura \textcolor{yellow}{X86-32} bits }
 \begin{columns}[c]
  \column{0.4\textwidth}
\lstset{basicstyle=\tiny}
\begin{tikzpicture}
  \begin{scope}[
  	every node/.style={draw, 
  	anchor=text, 
  	rectangle split,
    rectangle split parts=19, 
    rectangle split part fill={green!0,green!0,green!0,green!0,green!0,red!0,blue!0,blue!0,red!0,red!0,red!0,red!0,blue!0,red!0,red!0,red!0,red!0,red!0,red!0,red!0},
    minimum width=0.6cm}]
    \node (R) at (-0.5,8){ 
    		\nodepart{one}\tiny 2
    		\nodepart{two}\tiny 3
    		\nodepart{three}\tiny 2
    		\nodepart{four}\tiny 3
    		\nodepart{five}\tiny 0xXXXXXXXX
    		\nodepart{six}{$\vdots$}
    		\nodepart{seven}\tiny 
    		\nodepart{eight}\tiny 
    		\nodepart{nine}{$\vdots$}
    		\nodepart{ten}\tiny 11
    		\nodepart{eleven}\tiny 12
    		\nodepart{twelve}\tiny 13
    		\nodepart{thirteen}{$\vdots$}
    		\nodepart{fourteen}\tiny 21
    		\nodepart{fifteen}\tiny 22
    		\nodepart{sixteen}\tiny 23
    		\nodepart{seventeen}{$\vdots$}
    		\nodepart{eighteen}
    		\nodepart{nineteen}};
    		
  \end{scope}
    \node at (-1.4,8.1) {\color{red}i};
  \node at (-1.4,7.75) {\color{red}j};
  \node at (-1.4,7.4) {\color{red}file};
  \node at (-1.4,7.1) {\color{red}row};
  \node at (-1.4,6.8) {\color{red}mtz};

  \node at (0,8.6) {\footnotesize\itshape\color{blue} {$ejemplo$ $para$ $matriz$ $(2$ x $3)$}};
  \fill[color=green] (2.7,2.15) -- (6,2.15) -- (6,1.15) -- (2.7,1.15) -- cycle;

  \node [fill=green] at (0.2,1.4) {\fbox{
						\begin{minipage}{3cm}
							\fontsize{8pt}{8pt}\selectfont
		 {Libero toda la memoria} \\
		  \\
		 {pedida con malloc} \\
						\end{minipage}}};						  



  \draw[white,-](-1.9,0.6)to [out=90,in=-90]node[right,midway]{} ++(0,0.5) ; % para fijar rectangulos
\end{tikzpicture}
						
  \column{0.5\textwidth}
  \lstset{basicstyle=\tiny}
 \begin{lstlisting}
#include<stdio.h>
#include <stdlib.h>

int main (void)
{
  int file=0,row=0,i=0,j=0;
  int **mtz;

  printf("ingrese filas y columnas \n");
  scanf("% d % d",&file,&row);
  mtz=(int **)malloc(file*sizeof(int *));
  for (i=0 ; i<file ; i++){
   mtz[i]=(int *)malloc(row*sizeof(int));
  }
  for (i=0 ; i<file ; i++){
    for (j=0 ; j<row ; j++){
       printf("file% d  rows% d\n",i,j);
       scanf("% d",&mtz[i][j]);
    }
  }
  printf("\n");
  for (i=0 ; i<file ; i++){
    for (j=0 ; j<row ; j++){
       printf("% d \t",mtz[i][j]);
    }
  }
  for (i=0 ; i<file ; i++){
    free(mtz[i]);
  }
  free(mtz);
  return (0);
}
   \end{lstlisting}
 \end{columns}
\end{frame}


\end{document}

i++
  \fill[color=green] (5.3,5.1) -- (5.8,5.1) -- (5.8,4.8) -- (5.3,4.8) -- cycle;


compara i
  \fill[color=green] (4.2,5.1) -- (5,5.1) -- (5,4.8) -- (4.2,4.8) -- cycle;

j=0
  \fill[color=green] (3.6,4.8) -- (4.4,4.8) -- (4.4,4.55) -- (3.6,4.55) -- cycle;

compara j
  \fill[color=green] (4.4,4.8) -- (5.4,4.8) -- (5.4,4.55) -- (4.4,4.55) -- cycle;
j++
  \fill[color=green] (5.4,4.8) -- (5.9,4.8) -- (5.9,4.55) -- (5.4,4.55) -- cycle;
scanf

  \fill[color=green] (3.2,4.3) -- (6.4,4.3) -- (6.4,4.0) -- (3.2,4.0) -- cycle;


  \fill[color=green] (4.4,4.8) -- (5.4,4.8) -- (5.4,4.55) -- (4.4,4.55) -- cycle;



no cumple condicion
  \node [fill=green] at (0.2,4.8) {\fbox{
						\begin{minipage}{3cm}
							\fontsize{8pt}{8pt}\selectfont
		 {No cumple la condicion} \\
		  \\
		 {y sale del ciclo for} \\
						\end{minipage}}};						  


\begin{equation*}
\begin{bmatrix}
\color{blue}11  & \color{blue}12 & \color{blue}13 \\
\color{blue}21 & \color{blue}22 & \color{blue}23 
\end{bmatrix} 
\end{equation*}
